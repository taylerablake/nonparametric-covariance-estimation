\documentclass[../chapter-1-introduction.tex]{subfiles}
\begin{document}

An estimate of the covariance matrix or its inverse is required for nearly all statistical procedures in classical multivariate data analysis, time series analysis, spatial statistics and, more recently, the growing field of statistical learning. Covariance estimates play a critical role in the performance of techniques for clustering and classification such as linear discriminant analysis (LDA), quadratic discriminant analysis (QDA), factor analysis, and principal components analysis (PCA), analysis of conditional independence through graphical models, classical multivariate regression, prediction, and Kriging. Covariance estimation with high dimensional data has has recently gained growing interest; it is generally recognized that there are two primary hurdles responsible for the difficulty in covariance estimation: the instability of sample covariance matrices in high dimensions and a positive-definite constraint we wish estimates to obey.

\bigskip

Prevalent technological advances in industry and many areas of science make high dimensional longitudinal and functional data a common occurrence, arising in numerous areas including medicine, public health, biology, and environmental science with specific applications including fMRI, spectroscopic imaging, gene microarrays among many others, presenting a need for effective covariance estimation in the challenging situation where parameter dimensionality $p$ is possibly much larger than the number of observations, $n$. 

\bigskip

We consider two types of potentially high dimensional data: the first is the case of functional data or times series data, where each observation corresponds to a curve sampled densely at a fine grid of time points; in this case, it is typical that the number of time points is larger than the number of observations. The second is the case of sparse longitudinal data where measurement times may be almost unique yet sparsely distributed within the observed time range for each individual in the study. In this case, the nature of the high dimensionality may not be a consequence of having more measurements per subject than the number of subjects themselves, but rather because when pooled across subjects, the total number of unique observed time points is greater than the number of individuals.  Several approaches have been taken in effort to overcome the issue of high dimensionality in covariance estimation. Regularization improves stability of covariance estimates in high dimensions, particularly in the case where the parameter dimensionality $p$ is much larger than the number of observations $n$. Regularization of the covariance matrix and its Cholesky decomposition has been explored extensively through various approaches including banding, tapering, kernel smoothing, penalized likelihood, and penalized regression; see citet{pourahmadi2011covariance} for a comprehensive overview. 

\bigskip

To overcome the hurdle of enforcing covariance estimates to be positive definite, several have considered modeling various matrix decompositions including variance-correlation decomposition, spectral decomposition, and Cholesky decomposition. The Cholesky decomposition has received particular attention, as it which allows for a statistically meaningful interpretation as well as an unconstrained parameterization of elements of the covariance matrix. This parameterization allows for estimation to be accomplished as simply as in least squares regression. If we assume that the data follow an autoregressive process with (possibly) heteroskedastic errors, then the two matrices comprising the Cholesky decomposition, the Cholesky factor (which diagonalizes the covariance matrix) and diagonal matrix itself, hold the autoregressive coefficients and the error variances, respectively.

\bigskip

In longitudinal studies, the measurement schedule could consist of targeted time points or could consist of completely arbitrary (random) time points. If either the measurement schedule has targeted time points which are not necessarily equally spaced or if there is missing data, then we have what is considered incomplete and unbalanced data. If the measurement schedule has arbitrary or almost unique time points for every individual so that at a given time point there could be very few or even only a single measurement, we must consider how to handle what we consider as sparse longitudinal data. We view the response as a stochastic process with corresponding continuous covariance function and the generalized autoregressive parameters as the evaluation of a continuous bivariate function at the pairs of observed time points rather than specifying a finite set of observations to be multivariate normal and estimating the covariance matrix. This is advantageous because it is unlikely that we are only interested in the covariance between pairs of observed design points, so it is reasonable to approach covariance estimation in a way that allows us to obtain an estimate of the covariance between two measurements at any pair of time points within the time interval of interest. 

\bigskip


Through the Cholesky decomposition, we formulate covariance estimation as a penalized regression problem and propose novel covariance penalties designed to yield natural null models presented in the literature. By transforming the axes of the design points, we express these penalties in terms of two directions: the lag component and the additive component and characterize the solution coefficient function in terms of a functional ANOVA decomposition. Some have side-stepped the issue of high dimensionality by prescribing simple parametric models for the elements of the Cholesky decomposition.  \citet{chen2011efficient}, \citet{pourahmadi1999joint}, and \citet{pourahmadi2002dynamic} have elicited stationary parametric models for the generalized autoregressive coefficients, letting the GARPs depend only on the distance between two time points. To induce the structural simplicity of such stationary models with the flexibility of a nonparametric approach, we penalize all functional components but that corresponding to the lag component so that the set of null models is comprised of stationary models.  \citet{huang2007estimation} follow the hueristic argument presented in \citet{pourahmadi1999joint} that the generalized autoregressive parameters are monotone decreasing in as lag increases and set off-diagonal elements of either the covariance matrix or the Cholesky factor corresponding to large lags to zero. Rather than shrinking element of the Cholesky factor to zero after particular value of $l$, we choose to enforce structure of the Cholesky factor such that the null models coincide with parsimonious models commonly used in time series analysis and with simple parametric models proposed in the nonparametric covariance estimation literature.

\bigskip

The remainder of the chapter serves as a brief survey of developments in covariance estimation.  We will highlight a number of approaches to parsimonious covariance modeling, but our attention will be delegated to recent progress in parsimonious covariance models for longitudinal data. The review will conclude with the presentation of matrix factorizations for reparameterizing elements of the covariance matrix, translating covariance estimation into a generalized linear modeling problem.	


\end{document}