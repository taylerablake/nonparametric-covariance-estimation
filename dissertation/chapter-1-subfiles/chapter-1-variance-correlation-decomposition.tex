\documentclass[../chapter-1-introduction.tex]{subfiles}
\begin{document}


The variance-correlation decomposition of $\Sigma$ is perhaps the most familiar of the following three parameterizations, which parameterizes the covariance matrix according to

\begin{equation}\label{eq:variance-correlation-decomposition}
\Sigma = DRD,
\end{equation}
\noindent
where $D = \mbox{diag}\left(\sqrt{\sigma_{11}},\dots , \sqrt{\sigma_{MM}}\right)$ denotes the diagonal matrix with diagonal entries equal to tje square-roots of those of $\Sigma$, and $R$ is the corresponding correlation matrix. This parameterization enjoys attractive practicality because the standard deviations are on the same scale as the responses, and because the estimation of $D$ and $R$ can be separated by eithering iteratively fixing one sequence of parameters to estimate the other. Moreover, one set of parameters may be more important than the others in some applications; the dynamic correlation model presented in  Engle’s (2002) is actually motivated by the fact that variances (volatilities) of individual assets are more important than their time-varying correlations.
\bigskip

While the diagonal entries of $D$ are constrained to be nonnegative, their logarithms are unconstrained. However, the correlation matrix $R$ is positive-definite constrained to have unit diagonal entires and off-diagonal entries to be less than or equal to 1 in absolute value. Because of these constraints, the variance-correlation decomposition does not lend to modeling its components with the use of covariates.
\end{document}
