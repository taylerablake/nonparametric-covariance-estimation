%  The dissertation abstract can only be 500 words.


\begin{quote}

With high dimensional longitudinal and functional data becoming much more common, there is a strong need for methods of estimating large covariance matrices. Estimation is made difficult by the instability of sample covariance matrices in high dimensions and a positive- definite constraint we desire to impose on estimates. A Cholesky decomposition of the co- variance matrix allows for parameter estimation via unconstrained optimization as well as a statistically meaningful interpretation of the parameter estimates. Regularization improves stability of covariance estimates in high dimensions, as well as in the case where functional data are sparse and individual curves are sampled at different and possibly unequally spaced time points. By viewing the entries of the covariance matrix as the evaluation of a continuous bivariate function at the pairs of observed time points, we treat covariance estimation as bivariate smoothing.
Within regularization framework, we propose novel covariance penalties which are designed to yield natural null models presented in the literature for stationarity or short-term dependence. These penalties are expressed in terms of variation in continuous time lag and its orthogonal complement. We present numerical results and data analysis to illustrate the utility of the proposed method.
\end{quote}


