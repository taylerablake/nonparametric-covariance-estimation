
\chapter{Concluding remarks and future work}\label{concluding-remarks-chapter}


%%%%%%%%%%%%%%%%%%%%%%%%%%%%%%%%%%%%%%%%%%%%%%%%%%%


Our formulation of covariance estimation supplies a flexible framework which free of the impediment presented by the positive definite constraint and a statistically intuitive interpretation of the elements of a covariance matrix. Modeling the Cholesky decomposition rather than the covariance matrix itself allows us to reframe covariance estimation as a regression problem. By estimating the parameters of the corresponding regression model using bivariate smoothing, we naturally accommodate irregularly-spaced longitudinal data of varying within-subject sample sizes by without the need for data imputation methods. 

\bigskip
%%%%%%%%%%%%%%%%%%%%%%%%%%%%%%%%%%%%%%%%%%%%%%%%%%%

We propose two representations of $\left(\phi, \log\sigma^2\right)$ the functional components of the Cholesky decomposition. The first leverages reproducing kernel Hilbert space methods to model the functional component corresponding to the generalized autoregressive

%%%%%%%%%%%%%%%%%%%%%%%%%%%%%%%%%%%%%%%%%%%%%%%%%%%

\begin{itemize}
\item functional component selection via the non-negative garrote
\item In the case with P-splines, explore the performance of generalized additive models analogous to the SSANOVA models without an interaction term.
	\begin{itemize}
	\item Under a tensor-product model $B = B_l \otimes B_m$, because the support of $\phi$ lies on the triangle $0 \le s < t \le 1$ rather than on a rectangle, the tensor product basis must be trimmed to omit any pairs of knots outside the domain. Without doing this, estimation of the basis coefficients becomes very unstable. But trimming the basis removes components necessary for each of the marginal bases to possess the properties of a B-spline basis, so functional components corresponding to $l$ and $m$ are unidentifiable. 
	\item Omission of the interaction term permits estimation by restricted maximum likelihood, where the tuning parameters can be interpreted as variances of random effects.
	\item Three-dimensional B-splines for smoothing over non-rectangular domains has been utilized in the field of graphics and computer vision, but hasn't received much attention in nonparametric statistical modeling - presumably due to how nasty the associated math is. 
	\end{itemize}
\end{itemize}

