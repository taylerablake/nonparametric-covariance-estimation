

%
%Review of generalized thresholding estimators, including the soft thresholding estimator is presented in in \ref{subsubsection:chapter-1-sss-1-3-4}. Recall that  $S^\lambda$ can be written as the solution to the optimization problem
%
%\begin{equation} \label{eq:soft-thresholding-objective-function}
%\mathpzc{s}_\lambda\left( z \right)  = \argmin{\sigma} \left[ \frac{1}{2} \left(\sigma - z\right)^2 + J_\lambda\left(\sigma \right)\right],
%\end{equation}
%\noindent
%so that estimation of the covariance matrix can be accomplished by solving multiple univariate Lasso-penalized least squares problems. 
%
%\bigskip
%
%Under certain conditions pertaining to the ration of sample sizes of the training and validation datasets, the $K$-fold cross validation criterion is a consistent estimator of the Frobenius norm risk. It is defined 
%
%\begin{equation} \label{eq:K-fold-matrix--cv}
%\mbox{CV}_F\left(\lambda \right) = \argmin{\lambda} K^{-1} \sum_{k = 1}^K  \vert \vert\hat{\Sigma}^{\left(-k\right)} - \tilde{\Sigma}^{\left(k\right)}  \vert \vert_F^2, 
%\end{equation}
%\noindent
%where $\tilde{\Sigma}^{\left(k\right)}$ is the unregularized estimator based on based on $\mathcal{D}_k$, and $\hat{\Sigma}^{\left(-k\right)}$ is the regularized estimator under consideration based on the data after holding $\mathcal{D}_k$ out.  Using this approach, the size of the training data set is approximately $\left(K - 1 \right)N/K$, and the size of the validation set is approximately $N/K$ (though these quantities are only relevant when subjects have equal numbers of observations). For linear models, it has been shown that cross validation is asymptotically consistent is the ratio of the validation data set size over the training set size goes to 1. See \citet{shao1993linear}. This result motivates the reverse cross validation criterion, which is defined as follows:
%
%\begin{equation} \label{eq:K-fold-matrix-reverse-cv}
%\mbox{rCV}_F\left(\lambda \right) = \argmin{\lambda} K^{-1} \sum_{k = 1}^K  \vert \vert\hat{\Sigma}^{\left(k\right)} - \tilde{\Sigma}^{\left(-k\right)}  \vert \vert_F^2, 
%\end{equation}
%\noindent
%where $\tilde{\Sigma}^{\left(-k\right)}$ is the unregularized estimator based on based on the data after holding out $\mathcal{D}_k$, and $\hat{\Sigma}^{\left(k\right)}$ is the regularized estimator under consideration based on $\mathcal{D}_k$. 
%There is little established about the optimal method for tuning parameter selection in for the class of estimators based on element-wise shrinkage of the sample covariance matrix.  However, based on the results of an extensive simulation study presented in \citet{fang2016tuning}, we use $K = 10$-fold cross validation to select the tuning parameters for both the tapering estimator $S^\omega$ and the soft thresholding estimator $S^{\lambda}$. They authors implement cross validation for a number of element-wise shrinkage estimators for covariance matrices in the \citet{CVTuningCov} R package, which was used to calculate the risk estimates for $S^{\omega}$ and $S^{\lambda}$. 
%
%\bigskip
%
%Element-wise shrinkage estimators of the covariance matrix, including the soft thresholding estimator, are not guaranteed to be positive definite, though \citet{rothman2009generalized} established that in the limit, soft thresholding produces a positive definite estimator with probability tending to 1.  We observed simulations runs which yielded a soft thresholding estimator that was indeed not positive definite.   In this case, the estimate has at least one eigenvalue less than or equal to zero, and the evaluation of the entropy loss \ref{eq:entropy-loss} is undefined. To enable the evaluation of the entropy loss, we coerced these estimates to the ``nearest'' positive definite estimate via application of the technique presented in \citet{cheng1998modified}.  For a symmetric matrix $A$, which is not positive definite,  a modified Cholesky algorithm produces a symmetric perturbation matrix $E$ such that $A + E$ is positive definite.
%
%\bigskip
%
%\citet{pan2003modelling} present an iterative procedure for estimating coefficient vectors $\lambda$, $\gamma$ of the polynomial model \ref{eq:GARP-IV-parametric-model}. Their algorithm uses a quasi-Newton step for computing the MLE under the multivariate normal likelihood. Their work is  is implemented in the JMCM package for \textsf{R}, which we used to compute the polynomial MCD estimates.  For implementation details, see \citet{pan2017jmcm}. 	 
%
%{\needsparaphrased{We need to decide which tables will be included in the non-appendix numerical results. In the actual dissertation, section 6 will not immediate follow section 5.}}
%
\bigskip
