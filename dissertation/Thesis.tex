\documentclass{osudissert96}

\usepackage{graphicx,psfrag,amsfonts,float,mathbbol,xcolor,cleveref}
\usepackage{arydshln}
\usepackage{amsmath}
\usepackage{tikz}
\usepackage[mathscr]{euscript}
\usepackage{enumitem}
\usepackage{accents}
\usepackage{framed}
\usepackage[utf8]{inputenc}
\usepackage{natbib}

\usepackage{caption}
\usepackage{subcaption}
%\usepackage{subfig}
\usepackage{algorithm}
\usepackage{algpseudocode}
\usepackage{etoolbox}
\usepackage{lscape}
\usepackage{nomencl}
\usepackage{setspace}
\usepackage{natbib}
\usepackage{mathtools}
\usepackage{IEEEtrantools}
\usepackage{times}
\usepackage{arydshln}
\usepackage{multirow}
\usepackage{amsthm}
\usepackage{booktabs}
\usepackage{rotating}
%\usepackage{algorithm}
%\usepackage{chicago}
\usepackage{lscape}
\usepackage{etoolbox}
\appto\normalsize{\belowdisplayshortskip=\belowdisplayskip}
% the following two are not strictly necessary

\usepackage[letterpaper, left=1in, top=1in, right=1in, bottom=1in,nohead,includefoot, verbose, ignoremp]{geometry}
\newcommand\numberthis{\addtocounter{equation}{1}\tag{\theequation}}
\newcommand*\needsparaphrased{\color{red}}
\newcommand*\needscited{\color{orange}}
\newcommand*\needsproof{\color{blue}}
\newcommand*\outlineskeleton{\color{green}}
\newcommand{\ms}{\scriptscriptstyle}
\newcommand{\hilbert}{\mathcal{H}}
\newcommand{\hilbertl}{\mathcal{H}_{\langle l \rangle}}
\newcommand{\hilbertm}{\mathcal{H}_{\langle m \rangle}}
\newcommand{\hilbertlnull}{\mathcal{H}_{0\langle l \rangle}}
\newcommand{\hilbertmnull}{\mathcal{H}_{0\langle m \rangle}}
\newcommand{\hilbertlpen}{\mathcal{H}_{1\langle l \rangle}}
\newcommand{\hilbertmpen}{\mathcal{H}_{1\langle m \rangle}}
\newcommand{\PP}{\mathcal{P}}
\newcommand{\vphistar}{\mbox{\boldmath $\phi$}}
\newcommand{\vsigmasq}{\mbox{\boldmath $\sigma^2$}}
\newcommand{\bfeps}{\mbox{\boldmath $\epsilon$}}
\newcommand{\bfgamma}{\mbox{\boldmath $\gamma$}}
\newcommand{\bflam}{\mbox{\boldmath $\lambda$}}
\newcommand{\bfphi}{\mbox{\boldmath $\phi$}}
\newcommand{\bfsigma}{\mbox{\boldmath $\sigma$}}
\newcommand{\bfkappa}{\mbox{\boldmath $\kappa$}}
\newcommand{\bfbeta}{\mbox{\boldmath $\beta$}}
\newcommand{\bfalpha}{\mbox{\boldmath $\alpha$}}
\newcommand{\bftheta}{\mbox{\boldmath $\theta$}}
\newcommand{\bfe}{\mbox{\boldmath $e$}}
\newcommand{\bft}{\mbox{\boldmath $t$}}
\newcommand{\bfg}{\mbox{\boldmath $g$}}
\newcommand{\bfv}{\mbox{\boldmath $v$}}
\newcommand{\bfu}{\mbox{\boldmath $u$}}
\newcommand{\bfx}{\mbox{\boldmath $x$}}
\newcommand{\bfy}{\mbox{\boldmath $y$}}
\newcommand{\bff}{\mbox{\boldmath $f$}}
\newcommand{\bfone}{\mbox{\boldmath $1$}}
\newcommand{\bfo}{\mbox{\boldmath $0$}}
\newcommand{\bfO}{\mbox{\boldmath $O$}}
\newcommand{\bfX}{\mbox{\boldmath $X$}}
\newcommand{\bfz}{\mbox{\boldmath $z$}}
\newcommand{\tildeY}{\tilde{Y}}
\newcommand{\tildey}{\tilde{y}}
\newcommand{\tildeQ}{\tilde{Q}}
\newcommand{\tildeK}{\tilde{K}}
\newcommand{\tildeR}{\tilde{R}}
\newcommand{\tildeA}{\tilde{A}}
\newcommand{\tildeepsilon}{\tilde{\epsilon}}
\newcommand{\bfepsilon}{\mbox{\boldmath $\epsilon$}}
\newcommand{\tildeS}{\tilde{S}}

\newcommand{\bfm}{\mbox{\boldmath $m}}
\newcommand{\bfa}{\mbox{\boldmath $a$}}
\newcommand{\bfb}{\mbox{\boldmath $b$}}
\newcommand{\bfY}{\mbox{\boldmath $Y$}}
\newcommand{\bfB}{\mbox{\boldmath $B$}}
\newcommand{\bfZ}{\mbox{\boldmath $Z$}}
\newcommand{\tildeB}{\tilde{B}}

\newcommand{\cardT}{\vert \mathcal{T} \vert}
%\newenvironment{theorem}[1][Theorem]{\begin{trivlist}
%\item[\hskip \labelsep {\bfseries #1}]}{\end{trivlist}}
%\newenvironment{corollary}[1][Corollary]{\begin{trivlist}
%\item[\hskip \labelsep {\bfseries #1}]}{\end{trivlist}}
%\newenvironment{proposition}[1][Proposition]{\begin{trivlist}
%\item[\hskip \labelsep {\bfseries #1}]}{\end{trivlist}}
%\newenvironment{definition}[1][Definition]{\begin{trivlist}
%\item[\hskip \labelsep {\bfseries #1}]}{\end{trivlist}}
\newcommand{\argmin}[1]{\underset{#1}{\operatorname{arg}\,\operatorname{min}}\;}
\newtheorem{theorem}{Theorem}[section]
\newtheorem{lemma}[theorem]{Lemma}
\newtheorem{proposition}[theorem]{Proposition}
\newtheorem{corollary}[theorem]{Corollary}

\theoremstyle{definition}
\newtheorem{definition}{Definition}[section]
\newtheorem{example}{Example}[]
\def\bL{\mathbf{L}}

\begingroup\lccode`~=`_
\lowercase{\endgroup\def~}#1{_{\scriptscriptstyle#1}}
\AtBeginDocument{\mathcode`_="8000 \catcode`_=12 }
\DeclareMathAlphabet{\mathpzc}{OT1}{pzc}{m}{it}

\captionsetup[figure]{font=small,skip=1pt}

\makeatletter
\renewcommand{\theenumi}{\Roman{enumi}}
\renewcommand{\labelenumi}{\theenumi.}
\renewcommand{\theenumii}{\Alph{enumii}}
\renewcommand{\labelenumii}{\theenumii.}
\renewcommand{\p@enumii}{\theenumi.}
\makeatother


% BibTeX from the BiBTeX Documentation
\def\BibTeX{{\rm B\kern-.05em{\sc i\kern-.025em b}\kern-.08em
    T\kern-.1667em\lower.7ex\hbox{E}\kern-.125emX}}

\renewcommand\typesetChapterTitle[1]{#1}

\setcounter{MaxMatrixCols}{20}

\begin{document}
%\bibliographystyle{plainnat}
%
% First, declare the parts of your title page 
%

\author{Tayler A. Blake}
\title{Nonparametric Covariance Estimation for Longitudinal Data}
%\authordegrees{B.A., M.S.}  % Degrees thus far, not including this one.
\unit{Department of Statistics}

\advisorname{Yoonkyung Lee}
\member{Katherine A. Calder}
\member{Sebastian Kurtek}
%\member{Yet another dude}      % Normally you will have advisor + 2 members

\maketitle

% Next, EITHER a copyright or BLANK page.
%
%   The following creates a page used to copyright your dissertation
%
%   BACKGROUND: Even without this copyright page, your dissertation will
%               carry a common-law copyright. However, if your
%               dissertation ends up seeing wide distribution, your
%               common-law copyright is at risk of being expunged.
%               Adding this copyright page prevents that from happening.
%
%               There are NO DOWNSIDES to including a copyright page as
%               your document is automatically copyright by law anyway.
%               However, this copyright page is OPTIONAL. If you get rid
%               of it, uncomment the \blankpage that follows it so that
%               there is a blank page here. The graduate school requires
%               a page here that is either blank or carries the
%               copyright.
%
%   IMPORTANT NOTE: The graduate school requires either a copyright page
%                   here or a BLANK PAGE here. If you get rid of the
%                   copyright, uncomment the \blankpage that follows it.
%                   You should NOT have BOTH uncommented.
%

% If you get rid of \disscopyright, restore the \blankpage line after it
\disscopyright
%\blankpage

%
% Abstract goes here.
%

\begin{abstract}
  %  The dissertation abstract can only be 500 words.


\begin{quote}

With high dimensional longitudinal and functional data becoming much more common, there is a strong need for methods of estimating large covariance matrices. Estimation is made difficult by the instability of sample covariance matrices in high dimensions and a positive- definite constraint we desire to impose on estimates. A Cholesky decomposition of the co- variance matrix allows for parameter estimation via unconstrained optimization as well as a statistically meaningful interpretation of the parameter estimates. Regularization improves stability of covariance estimates in high dimensions, as well as in the case where functional data are sparse and individual curves are sampled at different and possibly unequally spaced time points. By viewing the entries of the covariance matrix as the evaluation of a continuous bivariate function at the pairs of observed time points, we treat covariance estimation as bivariate smoothing.
Within regularization framework, we propose novel covariance penalties which are designed to yield natural null models presented in the literature for stationarity or short-term dependence. These penalties are expressed in terms of variation in continuous time lag and its orthogonal complement. We present numerical results and data analysis to illustrate the utility of the proposed method.
\end{quote}



\end{abstract}


%
% UPDATED TEXT (2010):
%  The graduate school does not require an external abstract. If this
%  changes, follow the old instructions below.
%
% HISTORICAL TEXT (1996):
%  Uncomment the three lines below to generate the external abstract.  Two
%  copies of this must be turned in to the graduate school.  These lines can
%  be placed pretty much anywhere, since the page numbering should be
%  independent of the rest of the thesis
%

% \begin{externalabstract}
%   %  The dissertation abstract can only be 500 words.


\begin{quote}

With high dimensional longitudinal and functional data becoming much more common, there is a strong need for methods of estimating large covariance matrices. Estimation is made difficult by the instability of sample covariance matrices in high dimensions and a positive- definite constraint we desire to impose on estimates. A Cholesky decomposition of the co- variance matrix allows for parameter estimation via unconstrained optimization as well as a statistically meaningful interpretation of the parameter estimates. Regularization improves stability of covariance estimates in high dimensions, as well as in the case where functional data are sparse and individual curves are sampled at different and possibly unequally spaced time points. By viewing the entries of the covariance matrix as the evaluation of a continuous bivariate function at the pairs of observed time points, we treat covariance estimation as bivariate smoothing.
Within regularization framework, we propose novel covariance penalties which are designed to yield natural null models presented in the literature for stationarity or short-term dependence. These penalties are expressed in terms of variation in continuous time lag and its orthogonal complement. We present numerical results and data analysis to illustrate the utility of the proposed method.
\end{quote}



% \end{externalabstract}

%
%  My Dedication
%

\dedication{PLACEHOLDER}


%
% Bring in Acknowledgement and Vita from separate files named ``ack.tex''
% and ``vita.tex''.
%

\begin{acknowledgements}

Far more people have had a contributing impact on this dissertation than I could possibly acknowledge in a few pages. The research that follows is as much the culmination of the oblique but persistent influence of the supporting cast of characters as it is the culmination of the core protagonists' focused efforts. Despite the incomplete list to follow, I must express my deepest gratitude for the village that has shaped me as a person and as a thinker over the years during which this work was completed. 

\bigskip

I want to thank my family and Todd for their patience and support throughout the duration of this excursion. Everyone should be as lucky as I am to have people in her life who empower her to forge her own path, and to be different if different is what she needs to be. 

\bigskip
 
I thank my advisor, Yoon Lee, for her both exceptional and patient academic guidance. Her ability to balance succinct, economical expression of thought with acute attention to detail holds my true admiration. I owe much of the clarity in the exposition of this work to her. I also want to extend my gratitude to the other members of my committee, Kate Calder and Sebastian Kurtek for their time and thoughtful feedback. I owe much of the conclusion of this work to Kate, whose transparent and honest advice provided perspective at a time when I most needed it. 

\bigskip

I have learned as much from my fellow graduate students as anyone else at Ohio State. I feel incredibly lucky to have been a part of a program in which the sense of competition between the students is replaced with a sense of community and teamwork. There is a warm, fuzzy spot in my heart for you guys and the slurry of Cockins Hall memories, wonderful and horrible and everything in between. 

\bigskip

My most sincere appreciation goes to Chris Holloman and my team at ICC. It is a rare opportunity to work with a group of individuals who make work as much of an opportunity to have fun it is an opportunity for growth. They have made simultaneously working while finishing this paper as easy as it could possibly be. You guys are the best.

\bigskip

This research is supported in part by the National Science Foundation under grants DMS-15-13566 and DMS-16-13110. 

\end{acknowledgements}


\begin{vita}

\dateitem{January 0, 1800}{Born - Cowtown, USA}

\dateitem{1900}{B.S. Cow Science}

\dateitem{1950}{M.S. Cow-Dairy  Science}

\dateitem{1985-present}{Graduate Teaching Associate,\\
			 Holstein University.}


\begin{publist}

%% UPDATE FOR 2010:
%  Grad school only wants research publications, and it only wants those
%  research pubs that are actually published. Accepted or ``to appear''
%  publications don't count. If they look closely, they'll tell you to
%  remove any publications that aren't in print. Haivng said that, they
%  probably won't look that closely unless you put a really long list
%  here. You're tempting fate if you add instructional publications
%  though.

\researchpubs

\pubitem{B.~Simpson
\newblock ``Milking a Cow''.
\newblock {\em Journal of Dairy Science}, 00(2):277--287, Feb. 1900.}

% \instructpubs
%
% \pubitem{B.~Simpson, ed.,
% \newblock ``Lab notes for Cow Science 101'', 1909.}

\end{publist}



\begin{fieldsstudy}

% The \majorfield* uses the unit specified in the \unit command used
% earlier in your document. If you want to use a different unit, use the
% second form shown here
\majorfield*
% \majorfield{Cow and Dairy Science}

%%
%% Note:  If there were only one field of study, the following list 
%%        would best be done using the following command:
%%
%%  \onestudy{Only Topic}{Only Professor}
%%

% \begin{studieslist}
% \studyitem{Topic 1}{Prof.\ Big Dude}
% \studyitem{Topic 2}{Prof.\ Other Dude}
% \studyitem{Topic 3}{Prof.\ Another Dude}
% \end{studieslist}

\end{fieldsstudy}

\end{vita}




\tableofcontents
\listoftables
\listoffigures

\documentclass[12pt]{article}
\usepackage{graphicx,psfrag,amsfonts,float,mathbbol,xcolor,cleveref}
\usepackage{arydshln}
\usepackage{amsmath}
\usepackage{tikz}
\usepackage[mathscr]{euscript}
\usepackage{enumitem}
\usepackage{subfiles}
\usepackage{accents}
\usepackage{framed}
\usepackage{subcaption}
\usepackage{subfiles}
\usepackage{natbib}
\usepackage{mathtools}
\usepackage{IEEEtrantools}
\usepackage{times}
\usepackage{cite}
\usepackage{rotating}
\usepackage{amsthm}
\usepackage[letterpaper, left=1in, top=1in, right=1in, bottom=1in,nohead,includefoot, verbose, ignoremp]{geometry}
\usepackage{booktabs}
\newcommand{\ra}[1]{\renewcommand{\arraystretch}{#1}}
\newcommand*\needsparaphrased{\color{red}}
\newcommand*\needscited{\color{orange}}
\newcommand*\needsproof{\color{blue}}
\newcommand*\outlineskeleton{\color{green}}
\newcommand{\PP}{\mathcal{P}}
\newcommand{\bfeps}{\mbox{\boldmath $\epsilon$}}
\newcommand{\bfgamma}{\mbox{\boldmath $\gamma$}}
\newcommand{\bflam}{\mbox{\boldmath $\lambda$}}
\newcommand{\bfphi}{\mbox{\boldmath $\phi$}}
\newcommand{\bfsigma}{\mbox{\boldmath $\sigma$}}
\newcommand{\bfbeta}{\mbox{\boldmath $\beta$}}
\newcommand{\bfalpha}{\mbox{\boldmath $\alpha$}}
\newcommand{\bfe}{\mbox{\boldmath $e$}}
\newcommand{\bff}{\mbox{\boldmath $f$}}
\newcommand{\bfone}{\mbox{\boldmath $1$}}
\newcommand{\bft}{\mbox{\boldmath $t$}}
\newcommand{\bfo}{\mbox{\boldmath $0$}}
\newcommand{\bfO}{\mbox{\boldmath $O$}}
\newcommand{\bfx}{\mbox{\boldmath $x$}}
\newcommand{\bfX}{\mbox{\boldmath $X$}}
\newcommand{\bfz}{\mbox{\boldmath $z$}}
\newcommand\numberthis{\addtocounter{equation}{1}\tag{\theequation}}

\newcommand{\argmin}[1]{\underset{#1}{\operatorname{arg}\,\operatorname{min}}\;}
\newcommand{\bfm}{\mbox{\boldmath $m}}
\newcommand{\bfy}{\mbox{\boldmath $y$}}
\newcommand{\bfa}{\mbox{\boldmath $a$}}
\newcommand{\bfb}{\mbox{\boldmath $b$}}
\newcommand{\bfY}{\mbox{\boldmath $Y$}}
\newcommand{\bfS}{\mbox{\boldmath $S$}}
\newcommand{\bfZ}{\mbox{\boldmath $Z$}}
\newcommand{\cardT}{\vert \mathcal{T} \vert}
%\newenvironment{theorem}[1][Theorem]{\begin{trivlist}
%\item[\hskip \labelsep {\bfseries #1}]}{\end{trivlist}}
%\newenvironment{corollary}[1][Corollary]{\begin{trivlist}
%\item[\hskip \labelsep {\bfseries #1}]}{\end{trivlist}}
%\newenvironment{proposition}[1][Proposition]{\begin{trivlist}
%\item[\hskip \labelsep {\bfseries #1}]}{\end{trivlist}}
%\newenvironment{definition}[1][Definition]{\begin{trivlist}
%\item[\hskip \labelsep {\bfseries #1}]}{\end{trivlist}}

\newtheorem{theorem}{Theorem}[section]
\newtheorem{lemma}[theorem]{Lemma}
\newtheorem{proposition}[theorem]{Proposition}
\newtheorem{corollary}[theorem]{Corollary}

\theoremstyle{definition}
\newtheorem{definition}{Definition}[section]
\newtheorem{example}{Example}[section]
\def\bL{\mathbf{L}}

\begingroup\lccode`~=`_
\lowercase{\endgroup\def~}#1{_{\scriptscriptstyle#1}}
\AtBeginDocument{\mathcode`_="8000 \catcode`_=12 }

\DeclareMathAlphabet{\mathpzc}{OT1}{pzc}{m}{it}

\makeatletter
\renewcommand{\theenumi}{\Roman{enumi}}
\renewcommand{\labelenumi}{\theenumi.}
\renewcommand{\theenumii}{\Alph{enumii}}
\renewcommand{\labelenumii}{\theenumii.}
\renewcommand{\p@enumii}{\theenumi.}
\makeatother

\begin{document}

%\nocite{*}
\def\bL{\mathbf{L}}
%\usepackage{mathtime}

%%UNCOMMENT following line if you have package

\title{ Nonparametric Covariance Estimation for Longitudinal Data via Penalized Tensor Product Splines}

\author{Tayler A. Blake\thanks{The Ohio State University, 1958 Neil Avenue, Columbus, OH 43201} \and  Yoonkyung Lee\thanks{The Ohio State University, 1958 Neil Avenue, Columbus, OH 43201}}

\bibliographystyle{plainnat}
\maketitle

\begin{abstract}
With high dimensional longitudinal and functional data becoming much more common, there is a strong need for methods of estimating large covariance matrices. Estimation is made difficult  by the instability of sample covariance matrices in high dimensions and a positive-definite constraint we desire to impose on estimates. A Cholesky decomposition of the covariance matrix allows for parameter estimation via unconstrained optimization as well as a statistically meaningful interpretation of the parameter estimates. Regularization improves stability of covariance estimates in high dimensions, as well as in the case where functional data are sparse and individual curves are sampled at different and possibly unequally spaced time points. By viewing the entries of the covariance matrix as the evaluation of a continuous bivariate function at the pairs of observed time points, we treat covariance estimation as bivariate smoothing. 

\bigskip

Within regularization framework, we propose novel covariance penalties which are designed to yield natural null models presented in the literature for stationarity or short-term dependence. These penalties are expressed in terms of variation in continuous time lag and its orthogonal complement. We present numerical results and data analysis to illustrate the utility of the proposed method. \\
\\
%\begin{keywords}
{\bf keywords:} non-parametric, covariance, longitudinal data, functional data, splines, reproducing kernel Hilbert space
%\end{keywords}
\end{abstract}


\section{Introduction}

\indent

\subfile{chapter-1-subfiles/chapter-1-introduction}

\section{Covariance estimation: a review}


\bigskip

Estimation of the covariance matrix is fundamental to the analysis of multivariate data, and the most commonly used estimator is the sample covariance matrix, $S$. While it is both positive-definite and an unbiased estimator of $\Sigma$, it is unstable large dimension $M$. Approaches rooted in decision theory yield stable estimators which are scalar multiples of the sample covariance matrix; these estimators distort the eigenstructure of $\Sigma$ unless the sample size is greater than the dimension, $N >> M$ (\citet{dempster1972covariance}.)  There is a vast body of work which addresses the efficient estimation of the covariance matrix of a normal distribution by correcting the eigenstructure distortion or reducing the number of parameters to be estimated. See \citet{stein1975estimation}, \citet{lin1985monte}, \citet{yang1994estimation}, \citet{daniels1999nonconjugate}, \citet{champion2003empirical} 

\bigskip

The sample covariance matrix $S$, which is used in virtually all multivariate techniques, is both unbiased and positive-definite. The flexible estimator is also computationally convenient, however it is neither parsimonious nor, in high dimensions, a stable estimator. Given a sample of size $N$ $Y_1,\dots , Y_N$, from an $M$-dimensional Normal distribution with mean $\mu$ and covariance matrix $\Sigma$, the sample covariance matrix

\begin{equation} \label{eq:sample-covariance-matrix}
S = \left(N-1\right)^{-1} \sum_{i = 1}^N \left(Y_i - \bar{Y}\right)\left(Y_i - \bar{Y}\right)'
\end{equation}
\noindent
is a straightforward estimator of the $\frac{M\left(M+1\right)}{2}$ parameters of the unstructured covariance matrix $\Sigma$. The number of parameters of $\Sigma = \left( \sigma_{ij} \right)$ grows quadratically in the dimension $M$, and the parameters must satisfy the positive-definiteness constraint

\begin{equation} \label{eq:positive-definite-constraint} 
v'\Sigma v = \sum_{i,j = 1}^M v_i v_j \sigma_{ij} \ge 0 
\end{equation}
\noindent
for all $v \in \Re^M$. The challenge presented by these hurdles have motivated a growing body of research in statistics and its areas of application aimed at effectively estimating covariance matrices.

\bigskip

Our review of work in this area focuses on developments made from two connected perspectives: regularization or sparsity in covariance matrices for high-dimensional data, and generalized linear models (GLM) or parsimony and use of covariates in low dimensions. A recurring technique in both perspectives is the reduction of covariance estimation to estimating a single of sequence of regression. The generalized linear model (GLM) framework \citet{McCullagh1989} merges numerous seemingly disconnected approaches to model the mean of a distribution, and can accommodate many types of including normal, probit, logistic and Poisson regressions, survival data, and log-linear models for contingency tables. The key to the power of the GLM paradigm is the use of a link function to induce unconstrained reparameterization for the mean of a distribution, and hence the ability to reduce the dimension of the parameter space via modeling the covariate effect additively by increasing the number of parameters gradually one at a time corresponding to inclusion of each covariate. The extension of the GLM has lead to large class of models including nonparametric and generalized additive models, Bayesian GLM, and generalized linear mixed models. See \citet{hastie1990generalized},  \citet{dey2000generalized},  \citet{mcculloch2001generalized}. An analogous framework for modeling covariance matrices facilitates further developments in covariance estimation from the Bayesian, nonparametric and other paradigms.


%%---------------------------------------------------------------------------------------------------------------------------------------------------------------------------------------------------------------------------------------------------

\subsection{Structured parametric covariances}


\subfile{chapter-1-subfiles/chapter-1-parametric-covariance-models}

%%---------------------------------------------------------------------------------------------------------------------------------------------------------------------------------------------------------------------------------------------------
\subsection{Shrinkage estimators based on the sample covariance matrix}

%\subfile{chapter-1-subfiles/chapter-1-S-based-shrinkage-estimators}

\subsubsection{Shrinking the spectrum and the correlation matrix}

\subfile{chapter-1-subfiles/chapter-1-spectrum-shrinkage}

\subsubsection{Ledoit-Wolf shrinkage estimator}

\subfile{chapter-1-subfiles/chapter-1-ledoit-wolf-estimator}


\subsubsection{Elementwise shrinkage} \label{subsubsection:chapter-1-sss-1-3-4}
A broad class of estimators that aim to stabilize the sample covariance matrix do so by applying shrinkage elementwise to the same covariance matrix. Shrinking the elements of the sample covariance matrix has been approached in a multitude of ways, including banding, tapering, and thresholding. These estimators are computationally inexpensive, with the exception of cross validation necessary for smoothing parameter selection. The tradeoff accompanying the ease of computation is that, because transformations of sample estimates are elementwise, the resulting estimators are not guaranteed to be positive definite.

\subsubsection{Tapering and banding the sample covariance matrix} \label{chapter-1-banding-tapering-estimators}
 
\subfile{chapter-1-subfiles/chapter-1-banding-tapering-estimators}

\subsubsection{thresholding the sample covariance matrix}

\subfile{chapter-1-subfiles/chapter-1-thresholding-estimators}

\bigskip

Alternately, for estimating the covariance of a random vector which is assumed to have a natural (time) ordering, several have proposed applying kernel smoothing methods directly to elements of the sample covariance matrix or a function of the sample covariance matrix. \citet{zeger1994semiparametric} introduced a nonparametric estimator obtained by kernel smoothing the sample variogram and squared residuals.  \citet{yao2005functional} applied a local linear smoother to the sample covariance matrix in the direction of the diagonal and a local quadratic smoother in the direction orthogonal to the diagonal to account for the presence of additional variation due to measurement error. The latter work is one of the few nonparametric methods utilizing smoothing in both dimensions of the covariance matrix, which was an inspiration of sorts for the work we present in Chapter 2. Like other elementwise shrinkage estimators, however, their proposed estimator is not guaranteed to be positive definite. 

\subsubsection{Tuning parameter selection for element-wise shrinkage estimators}

The performance of any regularized estimator depends heavily on the quality of tuning parameter selection. The Frobenius is a natural measure of the accuracy of an estimator; it quantifies the sum over the unique elements of $\Sigma$ of the the first term in \ref{eq:general-thresholding-objective-function}, 

\begin{equation} \label{eq:forbenius-norm}
\vert \vert  \hat{\Sigma}^\lambda - \Sigma \vert \vert^2 = \left(\sum_{i,j} \left(\hat{\sigma}^\lambda_{ij} - \sigma_{ij} \right)^2\right)^{1/2}
\end{equation}
\noindent
If $\Sigma$ were available, one would choose the value of the tuning parameter $\lambda$ which minimizes \ref{eq:frobenius-norm}. In practice, one tries to first approximate the risk, or 
\[
E_\Sigma\left[\vert \vert  \hat{\Sigma}^\lambda - \Sigma \vert \vert^2 \right],
\]
\noindent
and then choose the optimal value of $\lambda$.  As in regression methods, cross validation and a number of its variants have become popular choices for tuning parameter selection in covariance estimation, though unanimous agreement on which precise procedure is optimal is fleeting.  $K$-fold cross validation requires first splitting the data into folds $\mathcal{D}_1, \mathcal{D}_2, \dots, \mathcal{D}_K$. The value of the tuning parameter is selected to minimize

\begin{equation} \label{eq:K-fold-matrix--cv}
\mbox{CV}_F\left(\lambda \right) = \argmin{\lambda} K^{-1} \sum_{k = 1}^K  \vert \vert\hat{\Sigma}^{\left(-k\right)} - \tilde{\Sigma}^{\left(k\right)}  \vert \vert_F^2, 
\end{equation}
\noindent
where $\tilde{\Sigma}^{\left(k\right)}$ is the unregularized estimator based on based on $\mathcal{D}_k$, and $\hat{\Sigma}^{\left(-k\right)}$ is the regularized estimator under consideration based on the data after holding $\mathcal{D}_k$ out.  Using this approach, the size of the training data set is approximately $\left(K - 1 \right)N/K$, and the size of the validation set is approximately $N/K$ (though these quantities are only relevant when subjects have equal numbers of observations). For linear models, it has been shown that cross validation is asymptotically consistent is the ratio of the validation data set size over the training set size goes to 1. See \citet{shao1993linear}. This result motivates the reverse cross validation criterion, which is defined as follows:

\begin{equation} \label{eq:K-fold-matrix-reverse-cv}
\mbox{rCV}_F\left(\lambda \right) = \argmin{\lambda} K^{-1} \sum_{k = 1}^K  \vert \vert\hat{\Sigma}^{\left(k\right)} - \tilde{\Sigma}^{\left(-k\right)}  \vert \vert_F^2, 
\end{equation}
\noindent
where $\tilde{\Sigma}^{\left(-k\right)}$ is the unregularized estimator based on based on the data after holding out $\mathcal{D}_k$, and $\hat{\Sigma}^{\left(k\right)}$ is the regularized estimator under consideration based on $\mathcal{D}_k$. 


%%---------------------------------------------------------------------------------------------------------------------------------------------------------------------------------------------------------------------------------------------------

\subsection{Matrix decompositions} \label{chapter-1-matrix-decompositions}
The most methodic and successful approaches to covariance modeling is to decompose the covariance matrix into its variance and dependence components. The following section demonstrates the role of multiple matrix parameterizations in removing the positive definite constraint that poses a challenge in most covariance estimation settings.

\bigskip
\subsubsection{The variance-correlation decomposition}
\subfile{chapter-1-subfiles/chapter-1-variance-correlation-decomposition}

\bigskip

\subsubsection{Gaussian graphical models} 
\subfile{chapter-1-subfiles/chapter-1-inverse-covariance-decomposition}

\subsubsection{The spectral decomposition}
\subfile{chapter-1-subfiles/chapter-1-spectral-decomposition}

\bigskip

\subsubsection{The Cholesky decomposition} \label{chapter-1-cholesky-decomposition}
\subfile{chapter-1-subfiles/chapter-1-cholesky-decomposition}



%%---------------------------------------------------------------------------------------------------------------------------------------------------------------------------------------------------------------------------------------------------
\subsection{Generalized linear models for covariances}
\subfile{chapter-1-subfiles/chapter-1-glm-covariance}

\subfile{chapter-1-subfiles/chapter-1-mcd-as-glm}

%%---------------------------------------------------------------------------------------------------------------------------------------------------------------------------------------------------------------------------------------------------

\bigskip
%We adopt the approach based on the Cholesky decomposition. The modified Cholesky decomposition (MCD) has received much attention in the covariance estimation literature, as it ensures positive-definite covariance estimates, and, unlike the spectral decomposition whose parameters follow an orthogonality constraint, the Cholesky decomposition are unconstrained and have an attractive statistical interpretation as particular regression coefficients and variances.  
%{\needsparaphrased{The Cholesky decomposition is similar to the spectral decomposition in that  is diagonalized by a lower triangular matrix T: 
%
%\[
%T \Sigma T' = D,
%\]
%where the nonredundant entries of T are unconstrained and more meaningful statistically than those of the orthogonal matrix of the spectral decomposition. The matrix T is constructed from the regression coefficients when yt is regressed on its predecessors:
%
%\begin{equation}
%y_t = \sum_{j = 1}^{t-1} \phi_{t,j} y_j + \epsilon_t,
%\end{equation}
%\noindent
%where the $\left(t, j\right)$ entry of $T$ is $\phi_{tj}$ , the negatives of the regression coefficients and the $(t, t)$ entry of $D$ is $\sigma_t^2 = var\left(\epsilon_t\right)$, the innovation variance. A schematic view of the components of a covariance matrix obtained through successive regressions (Gram-Schmidt orthogonalization procedure) is given in Table 2. Since the $\phi_{ij}$s are regression coefficients, it is evident that for any unstructured covariance matrix these and the log innovation variances are unconstrained, in the sequel they are referred to as the generalized autoregressive parameters (GARP) and innovation variances (IV) of Y or ? (Pourahmadi, 1999, 2000). Interestingly, this regression approach reveals the equivalence of modeling a covariance matrix to that of dealing with a sequence of $p - 1$ varying-coefficient and varying-order regression models. Consequently, one can bring the entire regression machinery to the service of the unintuitive task of modeling covariance matrices. Stated differently, the framework above is similar to that of using increasing order autoregressive models in approximating the covariance matrix or the spectrum of a stationary time series.}}
%
%The covariance matrix $\Sigma$ of a zero-mean random vector $Y = \left(y_1, \dots , y_m\right)'$ has the following unique modified Cholesky decomposition (Newton, 1988)
%
%\begin{equation} \label{eq:cholesky-matrix-decomposition}
%T \Sigma T' = D, 
%\end{equation}
%
%where $T$ is a lower triangular matrix with $1$?s as its diagonal entries and $D = \mbox{diag}\left(\sigma_1^2, \dots , \sigma_m^2\right)$ is a diagonal matrix. An attractive feature of this decomposition is that unlike the entries of $\Sigma$, the subdiagonal entries of $T$ and the log of the diagonal elements of $D$, $\log\left( \sigma_m^2 \right)$, $t = 1, \dots , m$, are not constrained. This permits one to impose structures on the unconstrained parameters without worrying about the resulting estimator not satisfying the positive-definiteness constraint. Denote estimators of $T$ and $D$ in \ref{eq:T-Sigma-Ttrans-equals-D} by  $\hat{T}$ and $\hat{D}$, which may be obtained by fitting linear models or some other structural models; then an estimator of $\Sigma$ given by $\Sigma  = \hat{T}^{-T} \hat{D} \hat{T}^{-T}$ is guaranteed to be positive-definite.  From this perspective, covariance modeling can be considered an extension of generalized linear models \citet{McCullagh1989}. Factoring $\Sigma$ as in \ref{eq:cholesky-matrix-decomposition} provides a link function $g\left(\Sigma\right) = \left(T, \log\left(D\right)\right)$ where $\log\left(D\right) = \mbox{diag}\left( \log\left(\sigma_1^2\right),\dots , \log\left(\sigma_m^2 \right) \right)$. Parametric, nonparametric, or  Bayesian models may then be applied to  the unconstrained entries of $T$ and $\log\left(D\right)$.  Whereas other decompositions are permutation-invariant, the interpretation of  the regression model induced by the MCD assumes a natural (time) ordering among the variables in $Y$.
%
%\bigskip
%
%{\needsparaphrased{immediately leads to the modified Cholesky decomposition \ref{eq:cholesky-matrix-decomposition}. It also can be used to clarify the close relation between the decomposition (2) and the time series ARMA models in that the latter is means to diagonalize a Toeplitz covariance matrix, for details see Pourahmadi (2001, Sec. 4.2.5).
%
%
%
%\needsparaphrased{In sharp contrast, the fact that the lower triangular matrix $T$ in the Cholesky decomposition of a covariance matrix $\Sigma$ is unconstrained makes it ideal for nonparametric estimation.
%Wu and Pourahmadi (2003) have used local polynomial estimators to smooth the subdiagonals of $T$. For the moment, denoting such estimators of $T$ and $D$ in (2) by $T$ and $D$, an
%estimator of $\Sigma$ given by $\Sigma = \hat{T}^{-1}D{\hat{T}^{-1}}^{\prime}$ is guaranteed to be positive-definite. Although one could smooth rows and columns of $T$,  the idea of smoothing along its subdiagonals is motivated by the similarity of the regressions in (3) to the varying-coefficients autoregressions (Kitagawa and Gersch, 1985, 1996; Dahlhaus, 1997): Xm
%
%Xm
%j=0
%\begin{equation}
%f_{j,p}\left(t/p\right)y_{t_j} = \sigma_p\left(t/p\right)\epsilon_t, \quad t = 0, 1, 2, \dots, M,
%\end{equation}
%\noindent
%where $f_{0,p}\left(�\right) = 1$, $f_{j,p}\left(�\right)$, 1 ? j ? m, and ?p(�) are continuous functions on $\left[0, 1\right]$ and {?t}
%30 is a sequence of independent random variables each with mean zero and variance one. This analogy and comparison with the matrix $T$ for stationary autoregressions having constant
%entries along subdiagonals suggest taking the subdiagonals of $T$ to be realizations of some smooth univariate functions:
%
%\begin{equation*}
%\phi_{t,t-j} = f_{j,M}\left(t/M\right),\quad \sigma_t + \sigma_M \left(t/M\right). 
%\end{equation*}
%
%The details of smoothing and selection of the order $m$ of the autoregression and a simulation study comparing performance of the sample covariance matrix to smoothed estimators are given in Wu and Pourahmadi (2003). Due to the closer connection between entries of $T$ and the family of regression (3), it is conceivable that some of the entries of $T$ could be zero or close to it. Smith and Kohn (2002) have used a prior that allows for zero entries in $T$ and have obtained a parsimonious model for $\Sigma$ without assuming a parametric structure. Similar results are reported in Huang, Liu and Pourahmadi (2004) using penalized likelihood with $L_1$-penalty to estimate $T$ for Gaussian data.}
% A commonly utilized approach in previous work is to model $\phi_{ijk} = z_{ijk}^T \gamma$ where $z_{ijk}$ is a vector of powers of time differences and $\gamma$ is a vector of unknown ``dependence'' parameters to be estimated from the data. \citet{chen2011efficient}, \citet{lin2009robust}, \citet{pan2003modelling},  and \citet{pourahmadi1999joint} define
%
%\begin{equation}
%z_{ijk}^T = \left(1, t_{ij} - t_{ik},\left( t_{ij} - t_{ik} \right)^2, \dots, \left(t_{ij} - t_{ik}\right)^{q-1} \right) \label{covmodel}
%\end{equation}
%
%Modeling the covariance in such a way is reduces a potentially high dimensional problem to something much more computationally feasible; if one models the innovation variances $\sigma^2\left(t\right)$ similarly using a $d$-dimensional vector of covariates, the problem reduces to estimating $q+d$ unconstrained parameters, where much of the dimensionality reduction is a result of characterizing the GARPs in terms of only the difference between pairs of observed time points, and not the time points themselves.  Modeling $\phi$ in such a way is equivalent to specifying a Toeplitz structure for $\Sigma$. A $p \times p$ Toeplitz matrix $M$ is a matrix with elements $m_{ij}$ such that $m_{ij} = m_{\vert i-j \vert}$ i.e. a matrix of the form
%
%
%\bigskip
%
%The estimated covariance matrix may be considerably biased when the specified parametric model is far from the truth.  To avoid model misspecification that potentially accompanies parametric analysis, many have alternatively  proposed nonparametric and semiparametric techniques approaches to estimation.  While these estimators can be very flexible and thus exhibit low bias, this advantage can be offset with high variance.  To balance the tradeoff between bias and variance, shrinkage or regularization may be applied to estimates to improve stability of estimators. \citet{diggle1998nonparametric} proposed nonparametric estimation of the covariance matrix of longitudinal data by smoothing raw sample variogram ordinates and squared residuals.  [DISCUSS THE NONPARAMETRIC SMOOTHER OF HANS GEORG MULLER HERE]  However, neither of these methods ensure that the resulting estimates are positive-definite.  
%
%\bigskip
%Several others have proposed methods for covariance estimation within the same paradigm of a smooth, continuous function underlying a discretized covariance matrix associated with the observed data.   \citet{pourahmadi1999joint} employ the Cholesky decomposition to guarantee positive-definiteness and imposed structure on the elements of the Cholesky decomposition and heuristically argue that $\phi_{t,t-l}$ should be monotonically decreasing in $l$. That is, the effect of $y_{t-l}$ on $y_t$ through the autoregressive parameterization should decrease as the distance in time between the two measurements increases. In similar spirit, others including \citet{bickel2008regularized} and \citet{levina2008sparse} enforce such structure by setting $\phi_{t,t-l}$ equal to zero for $l$ large enough, or equivalently, setting all subdiagonals of $T$ to zero beyond the $K^{th}$ off-diagonal. The tuning parameter $K$ is chosen using a model selection criterion such as Akaike information criterion, Bayesian information criterion, or cross validation or a variant thereof.  In terms of the autoregressive model corresponding to the Cholesky decomposition, this form of regularization, known as ``banding'' the Cholesky factor $T$, is equivalent to regressing $y_t$ on only its $K$ immediate predecessors, setting $\phi_{tj} = 0$ for $t-j>K$. 
%
%\bigskip
%
%From this perspective, it is apparent that the presentation of covariance estimation as a least squares regression problem suggests that the familiar ideas of model regularization for least-squares regression can be used for estimating covariances.  . \citet{huang2007estimation} 
%
%however, their two-step method did not utilize the information that many of the subdiagonals of T are essentially zeros at the first step. Inefficient estimation may result because of ignoring regularization structure in constructing the raw estimator. 
%
%\bigskip
%
%Several have applied these approaches to covariance estimation; 
%\bigskip
%
%Alternatively, one can view $T$ as a bivariate function,
%
%Several others have considered this approach to covariance estimation; \citet{kaufman2008covariance} assume a stationary process, restricting covariance estimates to a specific class of functions.  They as well as  Huang, Liu, and Liu \citet{huang2007estimation} follow the hueristic argument presented by \citet{pourahmadi1999joint} that $\phi_{t,t-l}$ is monotone decreasing in $l$ and set off-diagonal elements of either the covariance matrix or the Cholesky factor corresponding to large lags to zero.   As in \citet{huang2007estimation}, \citet{kaufman2008covariance}, and \citet{yao2005functional}, we treat covariance estimation as a function estimation problem where the covariance matrix is viewed as the evaluation of a smooth function at particular design points. 
%
%including \citet{bickel2008regularized} and \citet{huang2006covariance}  have proposed nonparametric estimators of a specific covariance matrix (or its inverse) rather than the parameters of a covariance function. 
%
%\bigskip
%
%\citet{yao2005functional} do not utilize the Cholesky parameterization, and their estimates are not guaranteed to be positive definite.  We combine the advantages of bivariate smoothing as in \citet{yao2005functional} with the added utility of the Cholesky parameterization in \citet{huang2007estimation}; in doing so, we present a flexible and coherent approach to covariance estimation, while simultaneously we ensuring positive definiteness of estimates.Rather than shrinking element of the Cholesky factor to zero after a particular value of $l$, we choose to softly enforce monotonicity in $l$ by using a hinge penalty as in the work of \citet{tibshirani2011nearly}. 
%
%\section{The Cholesky Decomposition and the MLE for $\Sigma$}
%
%Let $Y = \left( y_{1}, y_{2}, \dots, y_{m} \right)'$ denote a mean zero random vector with variance-covariance matrix $\Sigma$, which we can think of as the time-ordered measurements on one subject in a longitudinal study. To present a comprehensive overview our estimation procedure, we begin with the representation of the covariance matrix, $\Sigma$, in terms of its Cholesky decomposition. Decomposing $\Sigma$ in such a way allows for both an unconstrained parameterization and statistically meaningful interpretation of covariance parameters. For any positive definite matrix $\Sigma$, there exists a unique lower triangular matrix $T$ with diagonal entries equal to $1$ which diagonalizes $\Sigma$:
%
%\begin{equation} \label{eq:T-Sigma-Ttrans-equals-D}
% T \Sigma T^T = D
%\end{equation}
%\noindent
%
%The convenient statistical interpretation of the parameters of the covariance matrix then comes if we consider, for $t = 2, \dots, m$, regressing $y_t$ on its predecessors $y_1,\dots, y_{t-1}$, letting
%\begin{equation} 
%{y}_{i}  = \sum_{j=1}^{i-1} \phi_{ij} y_{j} + \sigma_{i}\epsilon_{i} \label{eq:discrete-evenly-spaced-ar-model},
%\end{equation}
%\noindent
%where $\mbox{var}\left( \epsilon_i \right) = \sigma_i^2$. If we take the $i$-$j^{th}$ element $T$ to be $-\phi_{ij}$ for $j < i$, and take the $i^{th}$ diagonal entry of $D$ to be $\mbox{var}\left( \epsilon_i \right) = \sigma_i^2$, a vectorized expression for Model~\ref{eq:discrete-evenly-spaced-ar-model} is given by
%
%\begin{equation}
%\bfeps = T Y \label{eq:vectorized-ar-model}.
%\end{equation}
%\noindent
%and taking covariances on both sides of \eqref{epsilon}, we see that $T$ and $D$ satisfy \ref{eq:T-Sigma-Ttrans-equals-D}. Immediately, we have that $\Sigma^{-1} = T' D^{-1} T$. The regression coefficients $\lbrace \phi_{ij} \rbrace$ are referred to as the \emph{generalized autoregressive parameters} (GARPs), and the $\lbrace \sigma_{ij} \rbrace$ are referred to as the \emph{innovation variances} (IVs.) 
%\bigskip
%Assuming that $Y$ follows a multivariate normal distribution, the loglikelihood function $\ell \left( Y, \Sigma \right)$ satisfies
%
%\begin{equation} \label{eq:loglik-general-form}
%-2\ell\left( Y, \Sigma \right) = \log \vert \Sigma \vert + Y' \Sigma Y
%\end{equation}
%\noindent
%From \ref{eq:T-Sigma-Ttrans-equals-D}, we have that 
%\[
%\vert \Sigma\vert = \vert D \vert = \prod_{i = 1}^m \sigma_i^2
%\]
%and 
%\[
%\Sigma^{-1} = T' D^{-1} T.
%\]
%Thus, \ref{eq:loglik-general-form} can be written in terms of the prediction errors and their variances of the non-redundant entries of $\left(T , D\right)$:
%
%\begin{align}
%\begin{split} \label{eq:loglik-cholesky-form}
%-2\ell\left( Y, \Sigma \right) &= \log \vert D \vert + Y' T' D^{-1} T Y \\
%&= \sum_{i = 1}^m \log \sigma_i^2  + \sum_{i = 1}^m \frac {\epsilon_i^2}{\sigma_i^2},
%\end{split}
%\end{align}
%\noindent
%where $\epsilon_1 = y_1$ and $\epsilon_i = y_i - \sum_{j = 1}^{i-1} \phi_{ij} y_j$. Maximum likelihood estimation or any of its penalized variants may then be employed to obtain estimates of $T$ and $D$.
%
%\bigskip
%Unlike many of those before who have used the Cholesky decomposition as a means of modeling $\Sigma$, we allow observed time points to be individual-specific and not necessarily regularly spaced.  Let $Y_1, \dots, Y_N$ denote a random sample of mean zero vectors of longitudinal measurements taken on $N$ subjects having common covariance structure $\Sigma$.  We allow subject $i$ to have observation vector $y_i = \left(y_{i1} ,\dots , y_{i,m_i}\right)'$ with corresponding vector of observation times $\left(t_{i1} ,\dots , t_{i,m_i}\right)'$.  Accommodating the subject-specific sample sizes and measurement times requires merely adding a subscript, and Model \ref{eq:discrete-evenly-spaced-ar-model} becomes 
%
%\begin{equation}
%{y}_{ij}  = \sum_{k=1}^{j-1} \phi_{ijk} y_{ik} + \sigma_{ij}\epsilon_{ij}, \label{eq:discrete-unevenly-spaced-ar-model}
%\end{equation}
%\noindent
%where $\phi_{ijk}$ is the autoregressive coefficient corresponding to the pair of measurements observed at time $t_{ij}$ and $t_{ik}$. A vectorized representation of Model~\ref{eq:discrete-unevenly-spaced-ar-model} can be obtained as before by adding the necessary parameters to $T$ and $D$.
\bigskip

Modeling $\phi_{ij} = \phi\left(t_i, t_j\right)$ as a smooth bivariate function, we cast the problem of estimating a covariance matrix as the estimation of a functional varying coefficient model. The existing body of literature surrounding these models is an extensive one; see \citet{csenturk2008generalized}, \citet{csenturk2013modeling}, and \citet{noh2010sparse}. This class of models is both flexible and interpretable, making them a pragmatic modeling choice when understanding the underlying data generating mechanism is of as much importance as strong predictive capability. We employ two representations of the GARPs, which we refer to as the \textit{generalized autoregressive coefficient function} within this frame. Chapter 2 presents a reproducing kernel Hilbert space framework for the estimation of both $\phi$ and $\sigma^2$. In Chapter 3, we discuss an alternative representation the varying coefficient function using the penalized B-splines of \citet{eilers1996flexible}. We properties of the P-splines that establish their connection to the usual spline penalty on the second derivative and demonstrate how their simple construction allows for extremely flexible regularization.



\bibliography{../Master}
\end{document}
     


\chapter{Covariance Estimation: A Review} \label{background-review-chapter}

\indent

Estimation of a  covariance matrix $\Sigma$ is fundamental to the analysis of multivariate data. The two primary challenges in fulfilling this prerequisite are due to the total number of parameters to be estimated in relation to the dimension, and the structural constraints that the elements of a covariance matrix should satisfy. The number of parameters grows quadratically in the dimension, and these parameters must satisfy the positive-definiteness constraint. That is, the elements of a $p \times p$ covariance matrix $\Sigma = \left[ \sigma_{ij} \right]$ for $p$ variables, satisfy the constraint that 

\begin{equation} \label{eq:positive-definite-constraint} 
c'\Sigma c = \sum_{i,j = 1}^p c_i c_j \sigma_{ij} \ge 0
\end{equation} 

\noindent
for all $c = \left(c_1,\dots, c_p \right)' \in \Re^p$. These challenges have motivated a growing body of research aimed at effectively estimating covariance matrices. Given a sample of random vectors $Y_1,\dots, Y_N$ from a distribution with covariance matrix $\Sigma$, a common starting point in the pursuit of an estimate of this matrix is the sample covariance matrix $S$:

\begin{equation} \label{eq:sample-covariance-matrix}
S = \left(N-1\right)^{-1} \sum_{i = 1}^N \left(Y_i - \bar{Y}\right)\left(Y_i - \bar{Y}\right)'
\end{equation}

\noindent
where $\bar{Y} = N^{-1}\sum_{i=1}^N Y_i$ denotes the sample mean vector. The sample covariance matrix is both a straightforward and flexible estimator of the $\frac{p\left(p+1\right)}{2}$ parameters of the unstructured covariance matrix $\Sigma$, and it is unbiased for $\Sigma$. Its construction produces a positive definite estimate, so that the constraint in \eqref{eq:positive-definite-constraint} is satisfied. 

\bigskip

Despite these merits, it has been well established that the empirical covariance matrix is unstable in high dimensions; see \cite{lin1985monte} or \cite{johnstone2001distribution}, for example. The sample covariance is not parsimonious, making it unsatisfactory when it is suspected that the true underlying covariance matrix is sparse, or has many of its elements equal to zero. Moreover, it is not uncommon to encounter practical situations in which the data do not permit the straightforward construction in \eqref{eq:sample-covariance-matrix}. Specifically, we are interested in estimating the covariance matrix associated with a vector of repeated measurements generated from longitudinal studies in which the measurements on the $i^{th}$ subject $Y_i = \left(y_{i1}, y_{i2}, \dots, y_{ip}\right)'$ are associated with measurement times $t_i = \left(t_{i1}, t_{i2}, \dots, t_{ip}\right)'$. In this setting, the sample covariance matrix is not necessarily an optimal estimator of the covariance matrix because it does not naturally incorporate the temporal structure of the data. Moreover, construction of the sample covariance matrix requires rectangular, or balanced, data. Table~\ref{table:ideal-repeated-measurements} shows the ideal shape of a (rectangular) longitudinal data set. Unfortunately, longitudinal studies frequently produce non-rectangular data, where trajectories are potentially sparsely observed at times which are not common across all subjects. In the case, construction of the sample covariance matrix as defined in \eqref{eq:sample-covariance-matrix} is infeasible. 

\bigskip

\begin{table}[H] 
\centering
\caption{\textit{Ideal shape of repeated measurements.}}
\begin{tabular}{cc|cccccc}
\multicolumn{8}{c}{Time}\\
& & $1$&$2$ &  $\dots$ & $t$ & $\dots$ & $p$ \\ \midrule
& 1 & $y_{11}$&$y_{12}$ &$\dots$ & $y_{1t}$ & $\dots$& $y_{1p}$ \\
& 2 & $y_{21}$&$y_{22}$ &$\dots$ & $y_{2t}$ & $\dots$& $y_{2p}$ \\
\begin{rotate}{90}%
\mbox{Unit}\end{rotate} & $\vdots$ &$\vdots$&$\vdots$ & &$\vdots$ & & $\vdots$ \\
& $i$ & $y_{i1}$&$y_{i2}$ &$\dots$ & $y_{it}$ & $\dots$& $y_{ip}$ \\
 & $\vdots$ &$\vdots$&$\vdots$ & &$\vdots$ & & $\vdots$ \\
 & $N$ & $y_{N1}$&$y_{N2}$ &$\dots$ & $y_{Nt}$ & $\dots$& $y_{Np}$ \\
\end{tabular} \label{table:ideal-repeated-measurements}
\end{table}

These drawbacks have accelerated numerous initiatives detouring the pitfalls on the most obvious route to a covariance estimate toward and deliberate modeling of structured covariance matrices for longitudinal data. These methods employ a number of approaches to reducing the dimension of the parameter space to balance flexibility and stability of estimators. In this chapter, we present a review of existing methods for covariance estimation, focusing on those developed specifically for the application to longitudinal data. Our review is by no means exhaustive and focuses on developments made in covariance estimation from two connected perspectives: regularized covariance matrices, and parsimonious models, including the use of covariates in low dimensions through generalized linear models (GLM) for covariance. We examine three general classes of estimators: structured covariance models, the sample covariance matrix and its regularized variants, and models for reparameterizations of the covariance matrix. To promote clarity in the discussion of covariance estimation, for the remainder of this dissertation, we assume that a random vector $Y = \left(y_1, \dots, y_p \right)'$ is centered to have mean zero, unless explicitly indicated.

\bigskip

%The generalized linear modeling framework \cite{McCullagh1989} merges numerous seemingly disconnected approaches to model the mean of a distribution, and can accommodate many types of including normal, probit, logistic and Poisson regressions, survival data, and log-linear models for contingency tables. The key to the power of the GLM paradigm is the use of a link function to induce unconstrained reparameterization for the mean of a distribution. This is particularly attractive for longitudinal data or spatial data, where the variables exhibit a natural ordering, as it permits the ability to reduce the dimension of the parameter space by increasing the number of parameters one at a time by including additional covariates. The extension of the GLM has lead to large class of models including nonparametric and generalized additive models, Bayesian GLM, and generalized linear mixed models. See \cite{hastie1990generalized},  \cite{dey2000generalized},  \cite{mcculloch2001generalized}. An analogous framework for modeling covariance matrices facilitates further developments in covariance estimation from the Bayesian, nonparametric and other paradigms.
%




\bigskip




%%---------------------------------------------------------------------------------------------------------------------------------------------------------------------------------------------------------------------------------------------------



\section{Structured Parametric Covariances} \label{chapter-1-parametric-covariance-models}


In the applied statistics literature, particularly for repeated measures data, it is quite common to pick a stationary covariance matrix for the covariance structure. These parametric structures were an attractive alternative to the common approaches that had, at the time, historically been used to estimate the covariance of a multivariate random vector. These approaches included the univariate ANOVA and repeated measures ANOVA. In the univariate ANOVA setting, a separate conventional analysis of variance is performed on the data from each distinct measurement time. Repeated measures ANOVA entails performing ANOVA as if the data were from a split-plot experiment with time of measurement being factor defining the split-plots.  The ensuing parametric covariance models sought to exploit the temporal structure in longitudinal data, while the ANOVA-based approaches fail to explicitly make sure of this information. The parsimony of these parametric structures make their computational requirements modest, and software packages implementing fitting procedures for a growing number of simple models are readily accessible. In this section, we discuss some of of the parametric models most commonly encountered in the covariance estimation literature. For comprehensive discussion of parametric models for repeated measures data, see \cite{jennrich1986unbalanced}, for example. 

%\bigskip
\subsubsection{The Compound Symmetric Model}

At one time, the compound symmetric model was a very popular choice for parametric covariance structure. It specifies constant variance and constant correlation between all pairs of variables, where the elements of the covariance matrix are given by
\begin{equation}\label{eq:compound-symmetric-model}
\sigma_{ij} = \left\{ \begin{array}{lr}
\sigma^2, & i = j, \\
\rho, & i \ne j,
\end{array}\right.
\end{equation}
\noindent
where $\sigma_{ij}$ denotes the $\left(i,j\right)$ element of $\Sigma$. The parsimony of this model is a primary reason for its attractiveness, having only two parameters to be estimated. However, with the development of models allowing for heterogeneous variances and non-constant correlation, it has received less attention as of late, particularly in the longitudinal statistics literature. 

%\bigskip

\subsubsection{Autoregressive Models}

Low order autoregressive models are among the most frequently used models for time series and repeated measures data. The first order autoregressive model for response variable $y_t$ associated with measurement time $t$ specifies
\begin{equation}\label{eq:ar-1-model}
y_{t} = \left\{ \begin{array}{lr}
\mu_t + \epsilon_t, & t = 1,\\
& \\
\mu_t + \rho\left(y_{t-1} - \mu_{t-1}\right) + \epsilon_t, & t = 2,\dots, p,
\end{array}\right.
\end{equation}
\noindent 
where $\vert \rho \vert < 1$, and the innovations $\left\{\epsilon_t\right\}$ are independently distributed according to $N\left(0,\sigma_t^2\right)$ with $\sigma_1^2 = \sigma^2/\left(1-\rho^2\right)$, and $\sigma_t^2 = \sigma^2$ for $t = 2, \dots, p$. The corresponding elements of the covariance matrix are monotonically decreasing in $l = \vert i-j \vert$; specifically,
\begin{equation}\label{eq:compound-symmetric-model}
\sigma_{ij} = \left\{ \begin{array}{lr}
\sigma^2, & i = j, \\
& \\
\rho^{\vert i - j \vert}, & i \ne j,\\
\end{array}\right.
\end{equation}
\noindent
The AR(1) model generalizes to any arbitrary order $k$ by simply adding additional predecessors to the covariates in the linear model for $y_t$:
\begin{equation*}
y_{t} = \left\{ \begin{array}{lr}
\epsilon_t, & t = 1,\\
& \\
\sum\limits_{j = 1}^{k} \phi_j\left(y_{t-j} - \mu_{t-j}\right) + \epsilon_t, & t = 2, \dots, p,
\end{array}\right.
\end{equation*}
\noindent
where $k = \min\left(p,t-1\right)$, and the $\left\{\epsilon_t\right\}$ are independent mean zero Normal random variables. The variance of $\left\{\epsilon_t\right\}$ is constant for $t > k$, and for $t \le k$, the variance is specified so as to ensure that the variance is constant across all responses $y_t$ and the covariance between $y_i$ and $y_j$ depends only on $\vert i - j\vert$. 

%\bigskip
\subsubsection{Moving Average Models}

Equally as common as the autoregressive model is the moving average model. The response specification for $q^{th}$ order moving average model  is given by 
\begin{equation}\label{eq:ma-q-model}
y_{t} = \sum_{j = 0}^{q} \theta_j \epsilon_{t-j},
\end{equation}
\noindent
where the $\left\{\epsilon_t\right\}$ are independently and identically distributed mean zero Normal random variables with variance $\sigma^2$. This model corresponds to covariance matrix having elements defined as follows:
\begin{equation*}
\sigma_{ij} = \left\{ \begin{array}{ll}
\left(\theta_{i-j} + \theta_{1}\theta_{i-j +1} + \dots + \theta_{q-i+j}\theta_{q}\right)/\left(1 + \sum_{j = 1}^q \theta_j^2\right), & \vert i-j\vert \le q,\\ 
& \\
& \\
0, &  \vert i-j\vert > q, \\
& \\
\sigma^2 \sum\limits_{j = 0}^q \theta_j^2, & i = j.\\
\end{array}\right.
\end{equation*}
\noindent
Thus, variances are constant and correlations between $y_t$ and $y_{t-l}$ vanish beyond a finite, constant lag $l$. Here $\rho_1,\dots, \rho_q$ are arbitrary parameters subject only to positive definiteness constraints. 

\bigskip

This model generalizes to a $q^{th}$-order Toeplitz model, which specifies
\begin{equation} \label{eq:toeplitz-covariance-model}
\sigma_{ij} = \left\{ \begin{array}{ll}
\rho_{i-j} & \vert i - j \vert\le q, \\ 
&\\
0 & \vert i - j \vert >  q, \\ 
& \\
\sigma^2  & i = j,\\
\end{array}\right.
\end{equation}
\noindent
or covariance matrix of the form
\begin{equation} \label{eq:toeplitz-covariance-matrix}
\begin{bmatrix} m_0 & m_1 & m_2 & \dots & m_{p-1}\\ m_1 & m_0 & m_1 & \dots & m_{p-2}\\m_2 & m_1 & m_0 & \dots & m_{p-3}\\ \vdots & \vdots & \vdots & \ddots & \vdots\\  m_{p-1} & m_{p-2} & m_{p-3} & \dots & m_0 \end{bmatrix}, 
\end{equation}
\noindent
where $m_j = 0$ for all $j > q$. %Imposing further model parsimony can be achieved by setting elements on some subdiagonals equal to zero, resulting in a banded estimator, which we will more thoroughly discuss in Section~\ref{chapter-1-shrinking-the-sample-cov}. 

\bigskip

The aforementioned models are stationary, specifying constant variance and with equal same-lag correlations among responses when the data are observed on a regular grid. Heterogeneous extensions of these models specify the same form of the correlation but allow time-dependent response variances. Completely general time dependence (subject to positive definiteness constraints) requires the covariance structure to be characterized by $O\left(p\right)$ parameters, while specifying linear or quadratic dependence on time leads to more parsimonious heterogeneous models. 

%\bigskip
\subsubsection{ARIMA Models}

An ARIMA($p,d,q$) model generalizes a stationary autoregressive moving average (ARMA) model by postulating that not the observations themselves, but rather the $d^{th}$-order differences among consecutive measurements follow a stationary ARMA($p,q$) model. A special case is the ARIMA($0,1,0$) model - the random walk:
\begin{equation}
y_{j} - \mu_{j} =   \sum_{k = 1}^j \epsilon_{k}, \quad j = 1, \dots, p,
\end{equation}
\noindent
where the $\epsilon_{k}$ are independent mean zero Normal random variables with variance $\sigma^2$. The variance of the process $Var\left(y_j\right) = j\sigma^2$ increases linearly in time. The the correlation between $y_{j}$ and $y_{k}$ also increases, but nonlinearly, in time:
\[
Corr\left(y_{j},y_{k}\right) = \sqrt{\frac{{j}}{{k}}}, \quad j > k.
\]
\noindent
This model is applicable to longitudinal data which are observed on a regular grid, however, its continuous time analogue permits this restriction to be relaxed. An important special case is the continuous time analogue to the random walk, the Weiner process, which has covariance function $Cov\left(y\left(t_i\right), y\left(t_j\right)\right) = \sigma^2 \min\left(t_i, t_j\right)$.

%\bigskip
\subsubsection{Random Coefficient Models}

Random coefficient models are a broad class of models often used for clustered or longitudinal data. They offer reasonable flexibility for characterizing dependency structure but remain parsimonious because the number of model parameters is unrelated to the number of repeated measurements and can be applied to non-rectangular data.  The formulation of the covariance structure for these models is most usually a consideration of regressions that vary across subjects rather than a consideration of within-subject similarity, which is why they are most often considered distinct from parametric covariance models. Still, they yield parametric covariance structures that generally have non-constant variances and non-stationary correlations.  A general form of the random coefficient model for $p_i \times 1$ vector of measurements on subject $i$ is given by 
\begin{equation}
Y_i = X_i\beta + Z_i \gamma_i + \epsilon_i, \quad i = 1, \dots, N,
\end{equation}
\noindent
where $Y_i = \left(y_{i1}, \dots, y_{ip_i}\right)'$ are measurements taken at equally-spaced times $t_{i1},\dots,  t_{ip_i}$, the $Z_i$ are specified matrices, the $\gamma_i$ are vectors of random coefficients distributed independently as $N \left(0, G_i\right)$, the $G_i$ are positive definite but otherwise unstructured matrices, and the $\epsilon_i$ are distributed independently (of the $\gamma_i$ and of each other) as $N \left(0, \sigma^2 \mathrm{I}_{p_i}\right)$. The $G_i$ are usually assumed to be equal across subjects, so the covariance matrix of $Y_i$ is taken to be $\Sigma_i = Z_i GZ'_i + \sigma^2 \mathrm{I}_{p_i}$. Special cases include the linear random coefficients (RCL) and quadratic random coefficients (RCQ) models. In the linear case, $Z_i = \left[1_{p_i} , \left(t_{i1},\dots,t_{i, p_i}\right)'\right]$ and 
\begin{equation*}
G = \begin{bmatrix}
\sigma_{00} & \sigma_{01} \\
\sigma_{10} & \sigma_{11} 
\end{bmatrix}
\end{equation*}
\noindent
In the quadratic case, $Z_i =\begin{bmatrix}1_{p_i}, \left(t_{i1}, \dots, t_{i,p_i}\right)', \left(t^2_{i1}, \dots, t^2_{i,p_i}\right)'\end{bmatrix}$. It is worth noting that when $Z_i = 1_{p_i}$, the random coefficient model corresponds to the compound symmetric model \eqref{eq:compound-symmetric-model}. The covariance structure for a subject having measurements taken at measurement times $t_1 = 1, \dots,t_{p_i} = p_i$ is given by  
\begin{equation}
\sigma_{ij} = \left\{ \begin{array}{ll}
\frac{\sigma_{00} + \sigma_{01}\left(i + j\right) + \sigma_{11} ij}{\sqrt{\sigma^2 + \sigma_{00} + 2i\sigma_{01} + \sigma_{11}i^2\sqrt{\sigma^2 + \sigma_{00} + 2j\sigma_{01} +j^2\sigma_{11}} }} &  i \ne j \\ 
& \\
\sigma^2 + \sigma_{00} + 2\sigma_{01}j + \sigma_{11}j^2 &  i= j, \\
\end{array}\right.
\end{equation}

These models permit variance and covariances exhibiting several kinds of time dependency, including increasing or decreasing variances and correlations of which some are negative while others are positive. However, this model does not permit variances which are concave-down in time, and it precludes the variances from being constant if the same-lag correlations are different.
%
%\bigskip
%
%The previous list highlights a number of major parametric covariance specifications, but it is far from an exhaustive list of parametric covariance structures - we will later reference structures which we have not discussed here, such as antedependence models. For additional models for repeated measures data, see\cite{jennrich1986unbalanced}, for example. 



%%---------------------------------------------------------------------------------------------------------------------------------------------------------------------------------------------------------------------------------------------------
\section{Shrinking the Sample Covariance Matrix} \label{chapter-1-shrinking-the-sample-cov}

The simple structure of parametric models is typically accompanied by straightforward interpretation of model coefficients and minimal computational issues. While the choices for parametric model structure are seemingly unlimited, specifying the appropriate parametric covariance structure is challenging even for the experts, and model misspecification can lead to considerably biased estimates. From this standpoint, it is prudent to allow the data to drive the formulation of the dependency structure. Estimates of parametric models are one extreme, typically exhibiting low variance but potentially high bias. The sample covariance matrix could characterize the other extreme. An unbiased estimator for the $p\left(p+1\right)/2$ parameters of an unstructured covariance matrix, it trades stability for flexibility. Between these poles lies a broad class of estimators which seeks to balance these two objectives.

\bigskip

Approaches rooted in decision theory yield stable estimators which are scalar multiples of the sample covariance matrix; these estimators distort the eigenstructure of $\Sigma$ unless the sample size is much greater than the dimension, $N \gg p$ \citep{dempster1972covariance}.  There is a vast body of work which addresses the efficient estimation of the covariance matrix of a normal distribution by correcting the eigenstructure distortion or reducing the number of parameters to be estimated. See \cite{stein1975estimation}, \cite{lin1985monte}, \cite{yang1994estimation}, \cite{daniels1999nonconjugate}, and \cite{champion2003empirical}. 

\subsubsection{Stein's Estimator}

\cite{stein1975estimation} observed that the sample covariance matrix systematically distorts the eigenstructure of $\Sigma$, especially when $p$ is large. His work spurred efforts in the improvement of $S$, which he did by simply shrinking its eigenvalues. Given the spectral decomposition of the sample covariance matrix
\[
S = \hat{P} \hat{\Lambda} \hat{P}' = \sum_{i = 1}^p \hat{\lambda}_i \hat{e}_i \hat{e}'_i,
\]
\noindent
he considered estimators of the form
\begin{equation}\label{eq:stein-eigen-estimator}
\hat{\Sigma} = \hat{P} \Phi\left(\hat{\lambda}\right) \hat{P}',
\end{equation}
\noindent
where $\hat{\lambda} = \left(\hat{\lambda}_1, \dots, \hat{\lambda}_p\right)'$, $\hat{\lambda}_1 > \dots > \hat{\lambda}_p$ are the ordered eigenvalues of $S$, $\hat{P}$ is the orthogonal matrix whose $i^{th}$ column is the normalized eigenvector of $S$ corresponding to $\hat{\lambda}_i$, and $\Phi\left(\hat{\lambda}\right) = diag\left(\phi_1,\dots, \phi_p \right)$ is the diagonal matrix where $\phi_j\left(\hat{\lambda} \right)$ is an estimate of the $j^{th}$ largest eigenvalue of $\Sigma$. Letting $\phi_j\left(\hat{\lambda} \right) = \hat{\lambda}_j$ corresponds to the usual unbiased estimator $S$. It is known that $\hat{\lambda}_1$ and $\hat{\lambda}_p$ are biased low and high, respectively, so Stein specified $\Phi\left(\hat{\lambda}\right)$ to shrink the eigenvalues toward central values to counteract the biases of the sample eigenvalues. The modified estimators of the eigenvalues of $\Sigma$ are given by $\phi_j = \frac{N \hat{\lambda_j}}{\alpha_j}$, where
\begin{equation}\label{eq:stein-eigen-estimator}
\alpha_j\left(\lambda\right) = N - p + 2\hat{\lambda}_j \sum_{i \ne j} \frac{1}{\hat{\lambda}_j - \hat{\lambda}_i}.
\end{equation}
\noindent
The Stein estimators $\phi_j$ differ from the sample eigenvalues when are nearly equal and $N/p$ is not small. The work of \cite{lin1985monte} includes an algorithm to modify any $\phi_j$'s which are negative and or do not satisfy $\phi_1 > \dots > \phi_p$.

%\bigskip
\subsubsection{Ledoit and Wolf's Estimator}

The estimator proposed by \cite{ledoit2004well} is motivated by the fact that the sample covariance matrix is unbiased but has high variance - the risk associated with $S$ is considerable when $p \gg N$, and even in cases when the dimension is close to the sample size. In contrast, very little estimation error is associated with a highly structured estimator of a covariance matrix, like those presented in Section~\ref{chapter-1-parametric-covariance-models}, but when the model is misspecified, these can exhibit severe bias. A natural inclination is to define an estimator as a linear combination of the two extremes, letting

\begin{equation} \label{eq:ledoit-wolf-estimator}
\hat{\Sigma} = \alpha_1 I + \alpha_2 S,
\end{equation}
\noindent
where $\alpha_1$, $\alpha_2$ are chosen to minimize 
\[
\frac{1}{p} \vert \vert\hat{\Sigma}-\Sigma   \vert \vert_{F}^2 = \frac{1}{p} \mbox{tr}\left[ \left(\hat{\Sigma}-\Sigma \right)^2\right].
\] 
\noindent
They show that the optimal $\alpha_i$ depend on only four characteristics of the true covariance matrix:
\begin{align}
\begin{split}
\mu &= \mbox{tr}\left(\Sigma\right)/p, \\
\alpha^2 &= \vert\vert \Sigma - \mu I\vert\vert^2, \\
\beta^2 &= \vert\vert S - \Sigma  \vert\vert^2, \\
\delta^2 &= \vert\vert S - \mu I\vert\vert^2.
\end{split}
\end{align}
\noindent
\cite{ledoit2004well} give consistent estimators of these quantities, so that substitution of these in $\hat{\Sigma}$ produces a positive definite estimator of $\Sigma$. They demonstrate the superiority of their estimator to several others including the sample covariance matrix and the empirical Bayes estimator \citep{haff1980empirical}.


%\subsubsection{Elementwise shrinkage} \label{elementwise-shrinkage-estimators}
\bigskip


A broad class of estimators aim to stabilize the sample covariance matrix by applying shrinkage, elementwise, to each of its entries. Many have explored the use of thresholding, banding, and tapering to stabilize the covariance matrix, resulting in estimators that are computationally inexpensive due to their convenient construction. This convenience, however, comes with a tradeoff: because the estimators are constructed by elementwise transformations of the sample covariance, they are not guaranteed to be positive definite.  Nonetheless, certain types of elementwise shrinkage estimators enjoy attractive asymptotic properties \citep{bickel2008regularized} which, in addition to their straightforwardness, perhaps  offset their finite sample shortcomings. 


\subsubsection{Banding the Sample Covariance Matrix}
 
Setting certain entries of the sample covariance matrix to zero is one approach to stabilize the estimator by reducing the dimension of the parameter space. Time series analysis is an example of the classic situation in which $p \gg N$. One typically observes a sample size of $N = 1$, with the data being a single, long realization of the random vector, which severely necessitates a reduction in the dimension of the parameter space. One way to do this is to assuming stationarity of the process, which reduces the number of distinct parameters of the $p \times p$ covariance matrix $\Sigma$ from $p\left(p + 1\right)/2$ to $p$, which could be still be large. Moving average and autoregressive models reduce the number of parameters in the same way as banding a covariance or inverse covariance matrix \citep{bickel2008regularized,wu2009banding}.  For a given sample covariance matrix $S = \left[ s_{ij} \right]$ and integer $k$, $0 < k < p$, the $k$-banded sample covariance matrix is given by
\begin{equation} \label{eq:general-banded-estimator} 
B_k\left(S\right) = \begin{bmatrix} s_{ij} 1\left(\vert i-j \vert \le k\right) \end{bmatrix}.
\end{equation}
\noindent
This kind of regularization is ideal when the indices have been arranged so that
\[
\vert i -  j\vert > k \Rightarrow  \sigma_{ij} = 0.
\]
Such structure often implies that variables far apart in with respect to time ordering are only weakly correlated, such as when, for example, $y_t$, $t = 1, \dots,p$ follow a finite heterogeneous moving average process
\begin{equation*} 
y_t = \sum_{j = 1}^k \theta_{t, t-j} \epsilon_j,
\end{equation*}
\noindent
where the $\epsilon_j$'s are iid mean zero errors having finite variance. Banding estimators are a special case of tapering estimators, which have the form
\begin{equation} \label{eq:general-tapering-estimator} 
\hat{\Sigma} = R \ast S, 
\end{equation}
\noindent
where $R$ is a positive definite tapering matrix, and the $\left( \ast \right)$ operator denotes the Schur matrix multiplication (the element-wise matrix product). The Schur product of two positive definite matrices is also guaranteed to be positive definite, so the tapering estimator's positive definiteness is dependent on the choice of tapering matrix $R$. Banding the sample covariance matrix is equivalent to premultiplying $S$ by 
\[
R = \left[r_{ij}\right] = \left[ 1\left(\vert i-j \vert \le k\right)\right],
\] 
\noindent
which is not positive definite. %However, several have used the same concept on the lower triangular matrix of the Cholesky decomposition of $\Sigma^{-1}$, including \cite{wu2003nonparametric}, \cite{huang2006covariance}, \cite{levina2008sparse}. Banding the Cholesky factor mitigates the need for the tapering matrix to be positive definite, since the parameters of the reparameterization are completely free while still guaranteeing that the estimate is positive definite. Detailed discussion follows in Section~\ref{chapter-1-cholesky-decomposition}. 

\bigskip

Asymptotic analysis of banding estimators is available when $N$, $p$, and $k$ are large. \cite{bickel2008regularized} establish consistency of the banded estimator in the operator norm, and uniform consistency over the class of ``approximately bandable'' matrices under a normal likelihood. Convergence requires that $\log p/ N \rightarrow 0$, and they derive an explicit rate of convergence which depends on the rate at which $k$ grows. \cite{cai2010optimal} proposed the following tapering estimator of the sample covariance matrix:
\begin{equation} \label{eq:cai-tapering-estimator}
S^{\omega} =  \begin{bmatrix} \omega_{ij}^k s_{ij} \end{bmatrix},
\end{equation}
\noindent
where the $\omega_{ij}^k$ are given by 
\begin{equation*}
\omega^k_{ij} = k_h^{-1} \left[ \left( k - \vert i-j\vert\right)_+ - \left(k_h - \vert i-j\vert\right)_+ \right].
\end{equation*}
\noindent
The weights $\omega^k_{ij}$ are indexed with superscript to indicate that they  are controlled by a tuning parameter, $k$,  which can take integer values between 0 and $p$, the dimension of the covariance matrix.  If $k_h = k/2$ is even, then the weights may be rewritten as
\begin{align*}
\omega_{ij} = \left\{\begin{array}{ll} 1, & \vert i -j  \vert \le k_h \\
                             2 - \frac{i - j}{k_h}, & k_h < \vert i -j  \vert \le k, \\
                             0, & \mbox{otherwise}.  \end{array} \right.
\end{align*}
\noindent
This expression indicates how the selection of $k$ controls the amount of shrinkage applied to a particular element of the sample covariance matrix. Elements of $S$ belonging to the subdiagonals closest to the main diagonal are left unregularized. The shrinkage applied to elements increases as we move away from the diagonal: a multiplicative shrinkage factor of $2 - \frac{i - j}{k_h}$ is applied to elements belonging to subdiagonals $k_h,\dots,k-1,k$, and elements further than $k$ subdiagonals from the main diagonal are shrunk to zero. \cite{cai2010optimal} derived optimal rates of convergence under the operator norm for their estimator and presented simulations demonstrating that it nearly uniformly outperforms the banding estimator of \cite{bickel2008regularized}.  


\bigskip
\subsubsection{Thresholding the Sample Covariance Matrix} \label{chapter-1-thresholding-estimators}


When both $N$ and $p$ are large, it is reasonable to assume that $\Sigma$ is sparse, so that many elements of the covariance matrix are equal to 0. In this case, setting certain elements of the sample estimate to zero can improve the quality of the estimator. Thresholding was originally a method developed in nonparametric function estimation, but recently \cite{bickel2008covariance} and \cite{rothman2009generalized} have utilized thresholding for estimating large covariance matrices.  Shrinkage and thresholding estimators can be viewed as the solution to the problem of minimizing a penalized quadratic loss function, and since the thresholding operator is applied elementwise to the sample covariance $S$,  these optimization problems are univariate. \cite{rothman2009generalized} presented a class of generalized thresholding estimators constructed by applying a thresholding operator to each element of the sample covariance matrix. This class includes the soft-thresholding estimator given by
\[
S^{\lambda}=   \begin{bmatrix} \mbox{sign}\left(s_{ij}\right) \left(s_{ij} - \lambda\right)_+ \end{bmatrix},
\]
\noindent 
where $s_{ij}$ denotes the $i$-$j^{th}$ entry of the sample covariance matrix, and $\lambda$ is a penalty parameter controlling the amount of shrinkage applied to $S$. 
%Their generalized thresholding estimator $\mathcal{s}_\lambda\left( z \right)$ is the solution to
%\begin{equation} \label{eq:general-thresholding-objective-function}
%\mathcal{s}_\lambda\left( z \right)  = \argmin{\sigma} \left[ \frac{1}{2} \left(\sigma - z\right)^2 + J\left(\sigma \right)\right],
%\end{equation}
%\noindent
%where $J$ penalizes the size of the elements of the estimated matrix. Soft thresholding results from minimizing \eqref{eq:general-thresholding-objective-function} using the lasso penalty, $J_\lambda = \lambda \vert \sigma \vert$, which corresponds to thresholding rule
%\begin{equation} 
%\mathcal{s}_\lambda\left( \sigma \right) = \textup{sign}\left(\sigma\right) \left(\sigma  - \lambda\right)_+.
%\end{equation}
%\noindent
%For detailed discussion of the connection between penalty functions and the resulting thresholding rules, see \cite{antoniadis2001regularization}. 

These estimators are simple to compute compared to competitor estimates, like the $L_1$-penalized likelihood estimator, but they suffer from the lack of guaranteed positive definiteness. However, similar to the result for banded estimators, \cite{bickel2008covariance} have established the consistency of the threshold estimator in the operator norm, uniformly over the class of matrices that satisfy a certain sparsity requirement. Soft thresholding can result in zeros irregularly placed in the resulting estimator, which may not be an optimal choice for sparsity pattern when there is a natural ordering of the variables as with longitudinal data.

\bigskip

Alternately, for estimating the covariance of a random vector which is assumed to have a natural (time) ordering, several have proposed applying kernel smoothing methods directly to elements of the sample covariance matrix or a function of the sample covariance matrix. \cite{zeger1994semiparametric} introduced a nonparametric estimator obtained by kernel smoothing the sample variogram and squared residuals.  \cite{yao2005functional} applied a local linear smoother to the sample covariance matrix in the direction of the diagonal and a local quadratic smoother in the direction orthogonal to the diagonal to account for the presence of additional variation due to measurement error. The latter work is one of the few nonparametric methods utilizing smoothing in both dimensions of the covariance matrix, which was an inspiration of sorts for the work we present in Chapter~\ref{SSANOVA-chapter}. Like other elementwise shrinkage estimators, however, their proposed estimator is not guaranteed to be positive definite. 

\bigskip

%The performance of any regularized estimator depends heavily on the quality of tuning parameter selection. The Frobenius norm is a natural way to quantify the discrepancy between an estimator $\hat{\Sigma}_\lambda$ and the true covariance matrix $\Sigma$, where the loss associated with $\hat{\Sigma}_\lambda$ is given by  
%
%\begin{equation} \label{eq:frobenius-norm}
%\vert \vert  \hat{\Sigma}^\lambda - \Sigma \vert \vert^2 = \left(\sum_{i,j} \left(\hat{\sigma}^\lambda_{ij} - \sigma_{ij} \right)^2\right)^{1/2}
%\end{equation}
%\noindent
%
%If $\Sigma$ were available, one would choose the value of the tuning parameter $\lambda$ which minimizes \eqref{eq:frobenius-norm}. In practice, one tries to first approximate the risk, or 
%\[
%E_\Sigma\left[\vert \vert  \hat{\Sigma}^\lambda - \Sigma \vert \vert^2 \right],
%\]
%\noindent
%and then choose the optimal value of $\lambda$.  As in regression methods, cross validation and a number of its variants have become popular choices for tuning parameter selection in covariance estimation, though unanimous agreement on which precise procedure is optimal is fleeting.  $K$-fold cross validation requires first splitting the data into folds $\mathcal{D}_1, \mathcal{D}_2, \dots, \mathcal{D}_K$. The value of the tuning parameter is selected to minimize
%\begin{equation} \label{eq:K-fold-matrix--cv}
%\mbox{CV}_F\left(\lambda \right) = \argmin{\lambda} K^{-1} \sum_{k = 1}^K  \vert \vert\hat{\Sigma}^{\left(-k\right)} - \tilde{\Sigma}^{\left(k\right)}  \vert \vert_F^2, 
%\end{equation}
%\noindent
%where $\tilde{\Sigma}^{\left(k\right)}$ is the unregularized estimator based on based on $\mathcal{D}_k$, and $\hat{\Sigma}^{\left(-k\right)}$ is the regularized estimator under consideration based on the data after holding $\mathcal{D}_k$ out.  Using this approach, the size of the training data set is approximately $\left(K - 1 \right)N/K$, and the size of the validation set is approximately $N/K$ (though these quantities are only relevant when subjects have equal numbers of observations). For linear models, it has been shown that cross validation is asymptotically consistent is the ratio of the validation data set size over the training set size goes to 1. See \cite{shao1993linear}. This result motivates the reverse cross validation criterion, which is defined as follows:
%\begin{equation} \label{eq:K-fold-matrix-reverse-cv}
%\mbox{rCV}_F\left(\lambda \right) = \argmin{\lambda} K^{-1} \sum_{k = 1}^K  \vert \vert\hat{\Sigma}^{\left(k\right)} - \tilde{\Sigma}^{\left(-k\right)}  \vert \vert_F^2, 
%\end{equation}
%\noindent
%where $\tilde{\Sigma}^{\left(-k\right)}$ is the unregularized estimator based on based on the data after holding out $\mathcal{D}_k$, and $\hat{\Sigma}^{\left(k\right)}$ is the regularized estimator under consideration based on $\mathcal{D}_k$. 
%

%%---------------------------------------------------------------------------------------------------------------------------------------------------------------------------------------------------------------------------------------------------


\section{Matrix Decompositions} \label{chapter-1-matrix-decompositions}


The positive definite constraint poses a challenge in most covariance estimation settings. In this section, we demonstrate the role of matrix decompositions in removing it from the estimation procedure altogether. These approaches decompose the covariance matrix into its variance and dependence components, and are closely connected to the use of generalized linear models for covariance estimation. In this light, this overview serves as a prerequisite to Section~\ref{covariance-glms} which will discuss covariance estimation from the generalized linear modeling perspective. 


\subsection{The Variance-Correlation Decomposition}

The variance-correlation decomposition of $\Sigma$  parameterizes the covariance matrix according to
\begin{equation}\label{eq:variance-correlation-decomposition}
\Sigma = DRD,
\end{equation}
\noindent
where $D = \mbox{diag}\left(\sqrt{\sigma_{11}},\dots , \sqrt{\sigma_{pp}}\right)$ denotes the diagonal matrix with diagonal entries equal to the square-roots of those of $\Sigma$, and $R$ is the corresponding correlation matrix. This parameterization enjoys attractive practicality because the standard deviations are on the same scale as the responses, and because the estimation of $D$ and $R$ can be separated by iteratively fixing one sequence of parameters to estimate the other. In some applications, one set of parameters may be more important than the others; the dynamic correlation model presented in \cite{engle2002dynamic} is actually motivated by the fact that variances (volatilities) of individual assets are more important than their time-varying correlations.

\bigskip

While the natural log of the diagonal entries of $D$ are unconstrained, the correlation matrix $R$ is constrained to have unit diagonal entires and off-diagonal entries to be less than or equal to 1 in absolute value. Consequently, the variance-correlation decomposition does not lend to modeling its components with the use of covariates. In the literature of longitudinal data analysis and other areas of application which frequently handle correlated data, preferred models for the variance-correlation decomposition typically involve structured correlation matrices with a few parameters, in the interest of parsimony and ensuring positive definiteness \citep{zimmerman1997structured}.


%\subsection{Gaussian graphical models} 
%
%The marginal (pairwise) dependence among the entries of a random vector are captured by the off-diagonal entries of $\Sigma$ or the entries of the correlation matrix $R = \left(\rho_{ij}\right)$. However, the conditional dependencies can be found in the off-diagonal entries of the precision matrix $\Sigma^{-1} = \left[ \sigma^{ij} \right]$. More precisely, for $Y$ a mean zero normal random vector with a positive-definite covariance matrix, if the $\left(i,j\right)$ component of the precision matrix is zero, then given the other variables, $y_i$ and $y_j$ are conditionally independent \citep{Anderson84a}. 
%
%\bigskip
%
%Gaussian graphical models are a common way of representing the conditional independence structure in a $p$-dimensional random vector $Y$, with the nodes of the graph corresponding to variables. The absence of an edge between variables $i$ and $j$, or a zero in the $\left(i,j\right)$ position of the inverse covariance matrix indicates that the two variables are conditionally independent. The entries of the variance-correlation decomposition of the precision matrix 
%\begin{equation} \label{eq:inverse-covariance-decomposition}
%\Sigma^{-1} = \left( \sigma^{ij}\right) = \tilde{D} \tilde{R} \tilde{D} 
%\end{equation}
%\noindent
%can be interpreted as certain coefficients of a regression model, which assumes no natural ordering of the $p$ variables corresponding to the columns of the covariance matrix. This lack of assumed structure among the dimensions of the matrix make it a less natural choice for modeling the covariance of longitudinal data. However, we've included it as part of this discussion because these models share a number of similarities to the Cholesky decomposition, which plays a central role in our contribution to the work in this area. 
%
%\bigskip
%
%A number of regression-based approaches to modeling the precision structure have spawned from the work of \cite{Meinshausen2006highDimGraphs}. Their method is based on solving $p$ separate LASSO regression problems. The entries of $\left(\tilde{R}, \tilde{D}\right)$ have direct statistical interpretations in terms of partial correlations, and variance of predicting a variable given the rest. Regression calculations can be used to show that the partial correlation coefficient between $y_i$ and $y_j$ after removing the linear effect of the $p - 2$ remaining variables is given by 
%\begin{equation} \label{eq:partial-correlation}
%\tilde{\rho}_{ij}= -\frac{\sigma^{ij}}{\sqrt{\sigma^{ii}\sigma^{jj}}}.
%\end{equation}
%\noindent
%The partial variance of $y_i$ after removing the linear effect of the remaining $p-$ variables is given by 
%\begin{equation} \label{eq:partial-variance}
%\tilde{d}^2_{ii}= \frac{1}{\sigma^{ii}}.
%\end{equation}
%To connect these parameters to those of a regression model, consider partitioning random vector $Y = \left(y_1,\dots, y_p\right)'$ into two components $\left(Y'_1,Y'_2\right)'$ of dimensions $p_1$ and $p_2$, and similarly partitioning its covariance and precision matrices:
%\begin{equation} \label{eq:partitioned-covariance-matrix}
%\Sigma = \begin{bmatrix} \Sigma_{11} & \Sigma_{12} \\ \Sigma_{21} & \Sigma_{22} \\  
%\end{bmatrix}, \quad \Sigma = \begin{bmatrix} \Sigma_{11} & \Sigma_{12} \\ \Sigma_{21} & \Sigma_{22} \\  
%\end{bmatrix},
%\end{equation}
%\noindent
%Let $\Phi_{2\vert 1}$ denote the $p_2 \times p_1$ matrix of regression coefficients resulting from the least squares regression of $Y_2$ on $Y_1$, and let $e_{2\vert 1} = Y_2 - \Phi_{2\vert 1} Y_1$ denote the corresponding vector of residuals. The regression coefficients $\Phi_{2\vert 1}$ and residuals $e_{2\vert 1}$ are obtained from restricting $e_{2\vert 1}$ to be uncorrelated with $Y_1$:
%\begin{align}
% \begin{split} \label{eq:conditional-coef-y2-given-y1}
% \Phi_{2\vert 1} &= \Sigma_{21}  \Sigma_{11}^{-1}  \\
% &= -\left( \Sigma^{22}\right)^{-1} \Sigma^{21} 
% \end{split}
% \end{align}
%\begin{align}
% \begin{split} \label{eq:conditional-cov-y2-given-y1}
%Cov\left(e_{2\vert 1}\right) &=  \Sigma_{22} - \Sigma_{21}\Sigma_{11}^{-1}\Sigma_{12}\\
%&=  \Sigma_{22\vert 1}  = \left(\Sigma^{22} \right)^{-1}. 
% \end{split}
%\end{align}
%If we let $p_2 = 1$, then one can establish the relationship between elements of the inverse covariance matrix and these regression coefficients and conditional covariances. When $Y_1 = Y_{-\left(i\right)} = \left( y_1, \dots, y_{i-1}, y_{i+1},\dots, y_p \right)'$ and $Y_2$ corresponds to a single $y_i$, $\Sigma_{22\vert 1}$, a scalar, is referred to as the \textit{partial variance} of $y_i$ given the other variables.  Denote the linear least squares predictor of $y_i$ based on $Y_{-\left(i\right)}$ by $y^*_i$ and $\epsilon^*_i = y_i - y^*_i$ with prediction variance $Var\left(\epsilon^*_i \right) = {d^*}^2_i$. Then
%\[
%y_i = \sum_{j \ne i} \beta_{ij} y_j + \epsilon^*_i,
%\] 
%\noindent
%where \eqref{eq:conditional-cov-y2-given-y1} and \eqref{eq:conditional-coef-y2-given-y1} give 
%\begin{align}
% \begin{split} \label{eq:conditional-coef-y2-given-y1}
%\beta_{ij} &= -\frac{\sigma^{ij}}{\sigma^{ii}}, \quad j \ne i \\
%{d^*}_i^2 &= Var\left(y_i \vert y_j\right) =  \frac{1}{\sigma_{ii}},\quad j \ne i, \;\; i = 1,\dots, p
% \end{split}
%\end{align}
%\noindent
%Thus, the unconstrained regression coefficient of the $j^{th}$ variable when we regressing $y_i$ on the rest of the variables is given by the $\left(i,j\right)$ entry of the inverse covariance matrix. The partial correlation between $y_i$ and $y_j$ can be defined if we consider the case where $p_2 = 2$. Letting $Y_2 = \left(y_i, y_j\right)'$, $i \ne j$ and $Y_1 = Y_{-\left(ij\right)}$ contain the remaining $p - 2$ variables, the covariance of $\left(y_i, y_j\right)$ after removing the linear effects of $\left\{ y_k : k \ne i,j\right\}$ is given by 
%\begin{align*}
%\Sigma_{22 \vert 1} &= \begin{bmatrix} \sigma^{ii} & \sigma^{ij} \\ \sigma^{ji} & \sigma^{jj} \end{bmatrix}^{-1} \\
%&= \frac{1}{\sigma^{ii}\sigma^{jj} - \left(\sigma^{ij}\right)^2}\begin{bmatrix} \sigma^{jj} & -\sigma^{ij} \\ -\sigma^{ij} & \sigma^{ii}\end{bmatrix}
%\end{align*}
%\noindent
%The regression coefficients \eqref{eq:conditional-coef-y2-given-y1} can be written in terms of the partial correlation between $y_i$ and $y_j$:
%\begin{equation} \label{eq:partial-correlation-coefficient}
%\rho^*_{ij} = -\frac{\sigma^{ij}}{\sqrt{\sigma^{ii}}\sigma^{ij}}.
%\end{equation}
%\noindent
%Rewriting the $\beta_{ij}$, we have
%\begin{equation} \label{eq:partial-correlation-coefficient}
%\beta_{ij} = \rho^*_{ij} \sqrt{\frac{\sigma^{jj}}{\sigma^{ii}}},
%\end{equation}
%\noindent
%which shows that the sparsity of the inverse covariance matrix mirrors that of the matrix of partial correlations. This parallel motivates estimation of the inverse covariance matrix by fitting a sequence of penalized regression models, notably the  approach taken by \cite{peng2012partial} which imposes a Lasso penalty on the off-diagonal elements of the partial correlation matrix. 


\subsection{The Spectral Decomposition}

The spectral decomposition is the basis of several methods in multivariate statistics, including principal component analysis and factor analysis \citep{Anderson84a,hotelling1933analysis}. The spectral decomposition of a covariance matrix $\Sigma$ is given by
\begin{equation} \label{eq:spectral-decomposition}
\Sigma = P \Lambda P' = \sum_{i = 1}^p \lambda_i e_i e'_i,
\end{equation}
\noindent
where $\Lambda$ is a diagonal matrix of eigenvalues $\lambda_1,\dots, \lambda_p$, and $P$ is the orthogonal matrix of normalized eigenvectors, having  $e_i$ as its $i^{th}$ column. The entries of $\Lambda$ and $P$ can be interpreted as the variances and coefficients of the $p$ principal components. The matrix $P$ is constrained by its orthogonality, so modeling it within the framework to reduce parameter dimension is inconvenient. In spite of this, \cite{chiu1996matrix} proposed an new unconstrained reparameterization of a covariance matrix using the spectral decomposition, modeling the matrix logarithm:
\begin{equation} \label{eq:spectral-decomposition}
\log \Sigma = P \left(\log\Lambda\right) P' = \sum_{i = 1}^p \log\left(\lambda_i \right)e_i e'_i,
\end{equation}
\noindent
The components $\log \lambda_i$ are free but lack any relevant statistical interpretability. Interestingly, this highlights the tradeoff between the requirements for unconstrained parameterization of covariance matrices and the statistical interpretability of the new parameters. We further discuss the log-linear GLM for covariance matrices in Section~\ref{log-linear-glms}.


\subsection{The Cholesky Decomposition} \label{chapter-1-cholesky-decomposition}

The Cholesky decomposition has received a lot of attention in recent developments in covariance estimation. Unlike the spectral decomposition, it offers an unconstrained parameterization without sacrificing the interpretability of the components of the decomposition. The Cholesky decomposition of a positive-definite matrix is given by
\begin{equation}\label{eq:standard-cholesky-decomposition}
\Sigma = CC',
\end{equation}
\noindent
where $C = \left[c_{ij} \right]$ is a unique lower-triangular matrix with positive diagonal entries. This factorization is frequently encountered in optimization techniques and matrix computation \citep{golub2012matrix}. It is difficult to attach any statistical interpretation to the entries of $C$ in this form \citep{pinheiro1996unconstrained}. However, statistical interpretation of the diagonal entries of $C$ and the resulting unit lower-triangular matrix is available by transforming $C$ to a unit lower-triangular matrix, dividing the $i^{th}$ column of $C$ by its $i^{th}$ diagonal element $c_{ii}$. Letting $D^{1/2} = diag\left( c_{11},\dots, c_{pp} \right)$, the standard Cholesky decomposition \eqref{eq:standard-cholesky-decomposition} can be written
\begin{equation}\label{eq:standard-cholesky-decomposition-transform}
\Sigma = CD^{-1/2}DD^{-1/2}C' = L D L',
\end{equation}
\noindent
where $L = D^{-1/2}C$. This is commonly referred to as the modified Cholesky decomposition (MCD) of $\Sigma$. It is common to write \eqref{eq:standard-cholesky-decomposition-transform} in terms of the lower triangular matrix that diagonalizes $\Sigma$:
\begin{equation}\label{eq:modified-cholesky-decomposition}
D = T\Sigma T',
\end{equation}
 \noindent
where $T = L^{-1}$. Like the orthogonal matrix $P$ in the spectral decomposition, the lower triangular matrix $T$ diagonalizes $\Sigma$, however the entries of $T$ can be written as the coefficients of a particular regression model, and are therefore unconstrained. The elements of the diagonal matrix $D$ can also be interpreted as parameters associated with the same model. Let $Y = \left( y_1,\dots, y_p \right)'$ denote a mean zero random vector with positive definite covariance matrix $\Sigma$, and consider regressing $y_t$ on its predecessors $y_1, \dots, y_{t-1}$. Let $\hat{y}_t$ be the linear least-squares predictor of $y_t$ based on previous measurements $y_{t-1}, \dots , y_1$. Standard regression machinery gives us that there exist unique scalars $\phi_{tj}$ so that
\begin{equation} \label{eq:mcd-ar-model}
y_t = \left\{ \begin{array}{ll} \epsilon_t, & t = 1\\
\sum_{j = 1}^{t-1} \phi_{tj} y_j + \epsilon_t, & t = 2, \dots, p,
\end{array}\right.
\end{equation}
\noindent
and the mean zero prediction errors are independently distributed. Denote the variance of the prediction errors by $Var\left(\epsilon_t\right) = \sigma_t^2 $. The connection between the Cholesky decomposition and the autoregressive model \eqref{eq:mcd-ar-model} is established by noting that the Cholesky factor contains the negatives of the regression coefficients and the prediction error variances are the diagonal elements of $D$.  Let $\epsilon = \left(\epsilon_1, \dots, \epsilon_p\right)'$ denote the vector of uncorrelated prediction residuals with
\[
Cov\left(\epsilon\right) = D = diag\left(\sigma_1^2,\dots, \sigma_p^2\right).
\]
\noindent
Then model \eqref{eq:mcd-ar-model} can be written 
\begin{equation} \label{eq:e-equals-T-Y}
\epsilon = TY,
\end{equation}
\noindent
where the $\left(t, j\right)$ entry of $T$ is $-\phi_{tj}$ , and the $(t, t)$ entry of $D$ is the variance of the $t^{th}$ prediction residual: $\sigma_t^2 = Var\left(\epsilon_t\right)$. 
\begin{align}
\begin{bmatrix}
1&&&&\\
-\phi_{21}&1&&&\\
-\phi_{31}&-\phi_{32}&1&&\\
\vdots &&&\ddots& \\
-\phi_{p1}&-\phi_{p2}& \dots & -\phi_{p,p-1}&1\\
\end{bmatrix}
\begin{bmatrix}
y_1 \\
y_2 \\ \vdots \\ y_p
\end{bmatrix} = \begin{bmatrix}
\epsilon_1 \\
\epsilon_2 \\ \vdots \\ \epsilon_p
\end{bmatrix}
\end{align}


Table~\ref{table:cholesky-decomposition-successive-regressions} illustrates how the components of a covariance matrix are obtained through successive regressions. Specifically, this representation demonstrates how modeling a covariance matrix is equivalent to fitting a sequence of $p - 1$ varying-coefficient and varying-order regression models. Since the $\phi_{tj}$ are regression coefficients, for any unstructured covariance matrix, these and the log variances are unconstrained. The regression coefficients of the model in \eqref{eq:mcd-ar-model} are referred to as the \textit{generalized autoregressive parameters} (GARP) and \textit{innovation variances} (IV) \citep{pourahmadi1999joint,pourahmadi2000maximum}. The powerful implication of the parallel regression framework of decomposition \eqref{eq:modified-cholesky-decomposition} is the accessibility of the entire portfolio of regression methods for the service of modeling covariance matrices. Moreover, the estimator $\hat{\Sigma}^{-1} = \hat{T}' \hat{D}^{-1} {T}$ constructed from the unconstrained parameters $\phi_{tj}$, $\sigma_j^2$ is guaranteed to be positive definite. 
\bigskip

\begin{table}[H]
\centering
\caption{\textit{Autoregressive coefficients and prediction error variances of successive regressions.}}
\begin{tabular}{cccccc}
 $y_{1}$&$y_{2}$ & $y_{3}$ & $\dots$ &$y_{p-1}$& $y_{p}$\\ \midrule
 $1$& &&&&\\
$\phi_{21}$& 1 &&&& \\
$\phi_{31}$& $\phi_{32}$& 1 &&& \\ 
$\vdots$ & $\vdots$ & & $\ddots$&& \\
$\vdots$ & $\vdots$ & && $\ddots$& \\
$\phi_{p1}$& $\phi_{p2}$&$\dots$ &$\dots$ &$\phi_{p,p-1}$ & 1\\ \midrule
$\sigma_1^2$ & $\sigma_2^2$ & $\dots$&$\dots$ &$\sigma_{p-1}^2$ &$\sigma_p^2$
\end{tabular} \label{table:cholesky-decomposition-successive-regressions}
\end{table}

%\bigskip
%
%immediately leads to the modified Cholesky decomposition \eqref{eq:cholesky-matrix-decomposition}. It also can be used to clarify the close relation between the decomposition (2) and the time series ARMA models in that the latter is means to diagonalize a Toeplitz covariance matrix, for details see Pourahmadi (2001, Sec. 4.2.5).
%
%
%
%\needsparaphrased{In sharp contrast, the fact that the lower triangular matrix $T$ in the Cholesky decomposition of a covariance matrix $\Sigma$ is unconstrained makes it ideal for nonparametric estimation.
%Wu and Pourahmadi (2003) have used local polynomial estimators to smooth the subdiagonals of $T$. For the moment, denoting such estimators of $T$ and $D$ in (2) by $T$ and $D$, an
%estimator of $\Sigma$ given by $\Sigma = \hat{T}^{-1}D{\hat{T}^{-1}}^{\prime}$ is guaranteed to be positive-definite. Although one could smooth rows and columns of $T$,  the idea of smoothing along its subdiagonals is motivated by the similarity of the regressions in (3) to the varying-coefficients autoregressions (Kitagawa and Gersch, 1985, 1996; Dahlhaus, 1997): Xm
%
%Xm
%j=0
%\begin{equation}
%f_{j,p}\left(t/p\right)y_{t_j} = \sigma_p\left(t/p\right)\epsilon_t, \quad t = 0, 1, 2, \dots, p,
%\end{equation}
%\noindent
%where $f_{0,p}\left(�\right) = 1$, $f_{j,p}\left(�\right)$, 1 ? j ? m, and ?p(�) are continuous functions on $\left[0, 1\right]$ and 
%30 is a sequence of independent random variables each with mean zero and variance one. This analogy and comparison with the matrix $T$ for stationary autoregressions having constant
%entries along subdiagonals suggest taking the subdiagonals of $T$ to be realizations of some smooth univariate functions:
%
%\begin{equation*}
%\phi_{t,t-j} = f_{j,p}\left(t/p\right),\quad \sigma_t + \sigma_p \left(t/p\right). 
%\end{equation*}
%\begin{equation}
%z_{ijk}^T = \left(1, t_{ij} - t_{ik},\left( t_{ij} - t_{ik} \right)^2, \dots, \left(t_{ij} - t_{ik}\right)^{q-1} \right) \label{covmodel}
%\end{equation}




\bigskip

%
%From this perspective, it is apparent that the presentation of covariance estimation as a least squares regression problem suggests that the familiar ideas of model regularization for least-squares regression can be used for estimating covariances.  . \cite{huang2007estimation} 
%
%however, their two-step method did not utilize the information that many of the subdiagonals of T are essentially zeros at the first step. Inefficient estimation may result because of ignoring regularization structure in constructing the raw estimator. 
%
%\bigskip
%
%Several have applied these approaches to covariance estimation; 
%\bigskip
%
%Alternatively, one can view $T$ as a bivariate function,
%
%Several others have considered this approach to covariance estimation; \cite{kaufman2008covariance} assume a stationary process, restricting covariance estimates to a specific class of functions.  They as well as  Huang, Liu, and Liu \cite{huang2007estimation} follow the hueristic argument presented by \cite{pourahmadi1999joint} that $\phi_{t,t-l}$ is monotone decreasing in $l$ and set off-diagonal elements of either the covariance matrix or the Cholesky factor corresponding to large lags to zero.   As in \cite{huang2007estimation}, \cite{kaufman2008covariance}, and \cite{yao2005functional}, we treat covariance estimation as a function estimation problem where the covariance matrix is viewed as the evaluation of a smooth function at particular design points. 
%
%including \cite{bickel2008regularized} and \cite{huang2006covariance}  have proposed nonparametric estimators of a specific covariance matrix (or its inverse) rather than the parameters of a covariance function. 
%
%\bigskip
%
%\cite{yao2005functional} do not utilize the Cholesky parameterization, and their estimates are not guaranteed to be positive definite.  We combine the advantages of bivariate smoothing as in \cite{yao2005functional} with the added utility of the Cholesky parameterization in \cite{huang2007estimation}; in doing so, we present a flexible and coherent approach to covariance estimation, while simultaneously we ensuring positive definiteness of estimates.Rather than shrinking element of the Cholesky factor to zero after a particular value of $l$, we choose to softly enforce monotonicity in $l$ by using a hinge penalty as in the work of \cite{tibshirani2011nearly}. 

%%---------------------------------------------------------------------------------------------------------------------------------------------------------------------------------------------------------------------------------------------------
\section{Generalized Linear Models for Covariances} \label{covariance-glms}


The positive-definiteness constraint and parameter space dimensionality are the major hurdles plaguing covariance estimation. However, within the context of regression analysis for modeling the mean vector $\mu$ of a random vector $Y = \left(y_1, \dots , y_p\right)'$, similar challenges have been handled successfully through the use of generalized linear models (GLM). The GLM framework \cite{McCullagh1989} merges numerous seemingly disconnected approaches for modeling the mean of a distribution. Much of the success of the GLM is due to the use of a link function $g\left(\cdot\right)$ and a linear predictor $g\left(\cdot\right) = X\beta$, where $X$ is a design matrix containing covariates which characterize the behaviour of the response. The link function and linear predictor together induce an unconstrained parameterization and reduce the parameter space dimension simultaneously.  The covariance matrix, which is defined $\Sigma = E\left(Y - \mu\right)\left(Y - \mu\right)'$, can be viewed a mean-like parameter, so it is a natural inclination to exploit the idea of the GLM for covariance estimation. In the GLM setting, simply applying a link function componentwise to the constrained mean vector $\mu$ permits its unconstrained estimation. Unfortunately, employing the same approach to covariance matrices isn't viable since positive-definiteness is a simultaneous constraint on all entries of a matrix. 

\bigskip

In addition to providing an avenue for sidestepping the positive definite constraint, the use of the GLM allows for the explicit use of covariates for estimating a covariance matrix, which is particularly attractive for longitudinal data or spatial data, where the variables exhibit a natural ordering. Extensions of the GLM to large classes of models include nonparametric and generalized additive models, Bayesian GLM, and generalized linear mixed models; see \cite{hastie1990generalized},  \cite{dey2000generalized}, and \cite{mcculloch2001generalized}. An analogous framework for modeling covariance matrices facilitates further developments in covariance estimation from the Bayesian, nonparametric and other paradigms. Successfully employing a link function for unconstrained estimation of a general covariance matrix necessitates decomposing a covariance matrix into its ``variance'' and ``dependence'' components. In the previous section, we discussed  the variance-correlation decomposition, the spectral decomposition, and the Cholesky decomposition, which factor $\Sigma$ in such a way, and described the advantages that the Cholesky decomposition enjoys over the other two.  

\bigskip




%Approaches to modeling covariances with the explicit use covariates has been extensively explored in the time series literature, while the implicit use of covariates for covariance modeling has been the focus of many in the areas of variance components; see \cite{klein1997statistical} and \cite{searle2009variance}. Time series techniques based on spectral and Cholesky decompositions provide the necessary tools for handling the positive definiteness constraint on a stationary covariance matrix or covariance function. 

\bigskip
%%---------------------------------------------------------------------------------------------------------------------------------------------------------------------------------------------------------------------------------------------------

\subsection{Linear Models for Covariance}
\cite{gabriel1962ante} was among the first to implicitly parameterize a multivariate normal distribution in terms of entries of the precision matrix $\Sigma^{-1}$.  \cite{dempster1972covariance} recognized the entries of $\Sigma^{-1} = \left[\sigma^{ij} \right]$ as the canonical parameters of the exponential family of normal distributions with mean zero and unknown covariance matrix $\Sigma$:
\[
\log f\left(Y, \Sigma^{-1}\right) = -\frac{1}{2}\mbox{tr}\Sigma^{-1} \left(Y'Y\right) + \log\vert \Sigma \vert^{-1/2} - p \log\sqrt{\pi}
\]
Soon thereafter, the simple structures of time series and variance components models motivated \cite{anderson1973asymptotically} to define the class of linear covariance models:
\begin{equation}\label{eq:linear-covariance-model}
\Sigma = \sum_{i = 1}^q \alpha_qU_q,
\end{equation}
\noindent
where the $U_i$s are known symmetric matrices and the $\alpha_i$s are unknown parameters, restricted to ensure that $\Sigma$ is positive definite. This class of models is general enough to include all linear mixed effects models as well as certain time series and graphical models. In, for $q$ large enough, any covariance matrix admits representation of the form \eqref{eq:linear-covariance-model}, since one can decompose every covariance matrix as 
\begin{equation} \label{eq:linear-covariance-model-2}	
\Sigma = \sum_{i = 1}^p \sum_{j = 1}^p \sigma_{ij} U_{ij},
\end{equation}
\noindent
where $U_{ij}$ is an $p \times p$ matrix with a 1 in the $\left(i,j\right)$ position, and zeros everywhere else. The linear model \eqref{eq:linear-covariance-model} can be viewed as modeling the link-transformed covariance $g\left(\Sigma\right) =\sum_{i = 1}^q \alpha_qU_q$, where $g\left(\cdot\right)$ is the identity link. Despite the convenient parameterization, the positive definite constraint \eqref{eq:positive-definite-constraint} makes estimation an arduous task. 

\bigskip

Inducing sparsity by setting certain elements of the covariance matrix or its inverse to zero is a common approach to reducing the dimensionality of a covariance structure. Inspection of model \eqref{eq:linear-covariance-model} and the covariance parameterization given in \eqref{eq:linear-covariance-model-2} makes it easy to see that this can be achieved by eliminating certain $U_{ij}$ from the covariates in the linear covariance model. On the extreme end of the sparsity spectrum is the case of independent observations and $\Sigma$ is diagonal, eliminating all $U_{ij}$ from the linear model covariates for $i \ne j$. Connection between the linear covariance model and other models for covariance discussed in previous sections can be established if we consider intermediary cases, such as classes of stationary moving average (MA) and autoregressive (AR) models introduced in the early times series literature. The $MA(q)$ model corresponds to a banded covariance matrix, setting 
\begin{equation}  \label{eq:ar-p-elementwise-shrinkage}
\sigma_{ij} = 0 \quad \mbox{for }\vert i - j \vert > q, 
\end{equation}
\noindent
while the $AR(p)$ model corresponds to a banded inverse:
\begin{equation} \label{eq:ar-p-elementwise-shrinkage}
\sigma^{ij} = 0 \quad \mbox{for }\vert i - j \vert > p. 
\end{equation}
Of course, there are the nonstationary analogues to these classes of models, some of which were discussed in Section~\ref{chapter-1-parametric-covariance-models}. We will review others which are related to antedependence models and Gaussian graphical models. Random variables $y_1, \dots, y_p$, which correspond to observation times $t_1,\dots, t_p$, with multivariate normal joint distribution said to be $p^{th}$-order antedependent or $AD(p)$ \citep{gabriel1962ante} if $y_t$ and $y_{t+s+1}$ are independent given the intervening values $y_{t+1}, \dots , y_{t+s}$ for $t = 1, \dots , p - s - 1$ and all $s \ge p$. A random vector $Y = \left(y_1, \dots , y_p\right)$ is $AD(p)$ if and only if its covariance matrix satisfies \eqref{eq:ar-p-elementwise-shrinkage}. Closely connected are the classes of variable order $AD$ models and varying order, varying coefficient autoregressive models \citep{kitagawa1985smoothness} in which the coefficients and order of antedependence depend on time. 


%%---------------------------------------------------------------------------------------------------------------------------------------------------------------------------------------------------------------------------------------------------


\subsection{Log-Linear Covariance Models} \label{log-linear-glms}

The constraint on the $\alpha_i$s in \eqref{eq:linear-covariance-model} was eliminated with the introduction of log-linear covariance models (\cite{chiu1996matrix},  \cite{pinheiro1996unconstrained}). For a general covariance matrix having spectral decomposition $\Sigma = P \Lambda P'$ its matrix logarithm, $\log\Sigma$, defined 
\[
\log \Sigma = P\left( \log\Lambda \right)P'
\]
\noindent
is a symmetric matrix with unconstrained entries taking values in $\Re$. Application of the log-link function leads to the log-linear model for $\Sigma$:
\begin{equation} \label{eq:log-linear-covariance-model}
g\left(\Sigma\right)  = \log\Sigma  = \sum_{i = 1}^q \alpha_i U_i, 
\end{equation}
\noindent
where the $U_i$s are as before in \eqref{eq:linear-covariance-model} and the $\alpha_i$s are now unconstrained. The $\alpha_i$s, however, now lack statistical interpretation since $g\left(A\right) = \log A$ is a highly nonlinear operation. But for diagonal $\Sigma$, $\log \Sigma = \mbox{diag}\left(\sigma_{11},\dots, \sigma_{pp}\right)$, and model \eqref{eq:log-linear-covariance-model} reduces to modeling of heterogeneous variances, which has been extensively studied. Detailed presentation is given in \cite{carroll1988transformation}, \cite{verbyla1993modelling} and in references therein. 

\bigskip

\cite{rice1991estimating} were the first to pursue nonparametric estimation of the spectral decomposition for functional data, which arise from experiments which produce observed responses in the form of curves. See \cite{ramsay2006functional}, \cite{ramsay2007applied}. The covariance structure is estimated via functional principal component analysis (fPCA); principal components of functional data are estimated using penalized least squares of the normalized eigenvectors, subject to the orthogonality constraint. Additionally, \cite{boente2000kernel} proposeds kernel-based PCA, but maintaining orthogonality of the smooth principal components remains a major computational challenge in both approaches.

\subsection{The Cholesky Decomposition as a Generalized Linear Model}

The log link resolves the issued presented by the constrained parameter space associated with the identity link, leading to unconstrained parameterization of a covariance matrix. However, the parameters of the matrix logarithm lack any meaningful statistical interpretation. %The hybrid link constructed from the modified Cholesky decomposition of $\Sigma^{-1}$ given in \eqref{eq:cholesky-decompostion-link-function} combines ideas in \cite{edgeworth1892xxii}, \cite{gabriel1962ante}, \cite{anderson1973asymptotically}, \cite{dempster1972covariance}, \cite{chiu1996matrix}, and \cite{zimmerman1997structured}. 
The Cholesky decomposition leads to unconstrained and statistically meaningful reparameterization of the covariance matrix so that the ensuing GLM overcomes most of the shortcomings of the linear and log-linear models.  %For an unstructured covariance matrix $\Sigma$, the nonredundant entries of the components $\left(T, \log D\right)$ of the modified Cholesky decompostion~\eqref{eq:modified-cholesky-decomposition} can be written as the entries of 

%\begin{equation}\label{eq:cholesky-decompostion-link-function}
%g\left( \Sigma \right) = 2I - T - T' + \log D.
%\end{equation}
%
%\noindent
The nonredundant entries of $\left(T, \log D\right)$ are unconstrained, allowing them to be modeled using any desired technique, including parametric, semi- and nonparametric, and Bayesian approaches. For a random sample of mean zero $p$-dimensional vectors $Y_1,\dots , Y_N$  from a normal density with covariance matrix $\Sigma$, the form of the likelihood allows for relatively simple computation of the MLE of the parameters. Up to a constant, the log likelihood satisfies
\begin{align}
\begin{split} \label{eq:regular-cholesky-log-likelihood}
-2\ell\left(\Sigma \vert Y_1,\dots, Y_N\right) &= \sum_{i = 1}^N \left( \log \vert \Sigma \vert  + Y'_i \Sigma^{-1}Y'_i\right) \\
&= N \log \vert D \vert + N \mbox{tr}\left(\Sigma^{-1}S\right) \\
& = N \log \vert D \vert + N \mbox{tr}\left(D^{-1}TST'\right), 
\end{split}
\end{align}
\noindent
where $S = N^{-1}\sum_{i=1}^N Y_iY'_i$. The negative log likelihood \eqref{eq:regular-cholesky-log-likelihood} is quadratic in $T$ for fixed $D$, so the MLE for the $\phi_{tj}$ has closed form. Similarly, the MLE for $D$ for fixed $T$ has closed form. See \cite{pourahmadi2000maximum}. 

\bigskip

While the MLE is flexible under a saturated model, this advantage can be offset with high variance. Many have attempted to balance the tradeoff between bias and variance by reducing the dimension of the parameter space under model \eqref{eq:mcd-ar-model} in a number of ways. Because the Cholesky decomposition can be viewed as a link function corresponding to a GLM for the covariance matrix, this can be done in a straightforward way with the use of covariates to  elicit parametric models for $\phi_{jk}$ and $\log\sigma_j^2$.  For example, the entries of $T$ and $\log D$ can be modeled as follows:
\begin{align}
\begin{split} \label{eq:linear-models-for-GARPs-IVs}
\phi_{jk} &= x'_{jk} \beta,\\
\log\sigma_j^2 &= z'_j \gamma,
\end{split}
\end{align}
\noindent
where $x_{tj}$ and $z_{t}$ denote $q \times 1$ and $d \times 1$ vectors of known covariates, and $\beta = \left(\beta_1,\dots, \beta_q \right)'$ and $\gamma = \left(\gamma_1,\dots, \gamma_d \right)'$ are the parameters relating these covariates to the innovation variances and the dependence among the elements of $Y$. Covariates most frequently used in the analysis of real longitudinal data sets are low order polynomials of lag and time. \cite{pourahmadi1999joint}, \cite{pourahmadi2000maximum}, and \cite{pan2003modelling} parameterize $\phi_{tj}$ and $\log \sigma^2_t$ using covariates
\begin{align}
\begin{split}  \label{eq:GARP-IV-parametric-model}
x'_{jk} &= \left(1, t_j - t_k, \left(t_j - t_k\right)^2,\dots, \left(t_j - t_k\right)^{d-1}\right)' \\
z'_{j}  &= \left(1, t_j, \dots, t_j^{q-1}\right)'
\end{split}
\end{align}

They prescribe methods for identifying models of the form \eqref{eq:linear-models-for-GARPs-IVs} using model selection criteria such as AIC and regressograms, which are a nonstationary analogue of the correlelogram one typically encounters in the time series literature. \cite{pan2003modelling} jointly estimate the mean and covariance of longitudinal data using maximum likelihood, iterating between estimation of the mean vector $\mu$, the log innovation variances $\log \sigma_{t}^2$, and the generalized autoregressive parameters $\phi_{tj}$. Score functions can be computed by direct differentiation of the normal log likelihood. Optimization is carried out by solving the score functions via iterative quasi-Newton method. 

\bigskip

Modeling the covariance in such a way reduces a potentially high dimensional problem to something much more computationally feasible; if one models the innovation variances $\sigma_t^2$ similarly using a $d$-dimensional vector of covariates, the problem reduces to estimating $\left(q+d\right)$ unconstrained parameters, where much of the dimensionality reduction is a result of characterizing the GARPs in terms of only the difference between pairs of observed time points, and not the time points themselves.  This model specification of $\phi$ is equivalent to specifying a Toeplitz structure for $\Sigma$.
% \cite{chen2011efficient}, \cite{lin2009robust}, \cite{pan2003modelling},  and \cite{pourahmadi1999joint} define
\bigskip

With the entries of $T$ unconstrained, the Cholesky decomposition is ideal for nonparametric estimation and regularization methods. Many have alternatively proposed nonparametric and semiparametric techniques  to reduce dimensionality without the risk of model misspecification often accompanying parametric models.  \cite{wu2003nonparametric} proposed local polynomial smoothers to individually estimate the subdiagonals of $T$. The idea of smoothing along the subdiagonals rather than down the rows or columns, or viewing $T$ as a bivariate function is analogous to the successive regressions in \eqref{eq:mcd-ar-model}. A similar procedure by \cite{dahlhaus1997fitting} uses varying coefficient regression models for each subdiagonal of $T$:
\begin{equation} \label{eq:one-dimensional-mcd-vc-model}
y_t = \sum_{j = 1}^{t-1} f_{j}\left( t \right) y_{t-j} + \sigma^2\left(t\right)
\end{equation}
\cite{wu2003nonparametric} give details of smoothing and selection of the order $k$ of the autoregression under the assumption that the $N$ subjects share common observation times.  In the first step, they derive a raw estimate of the covariance matrix and the estimated covariance matrix is subject to the modified Cholesky decomposition. In the second step, they apply local polynomial smoothing to the diagonal elements of $D$ and the subdiagonals of $T$.  

\bigskip

The connection between the entries of $T$ and the family of regression models \eqref{eq:mcd-ar-model} makes it conceivable that $T$  exhibits sparsity, having some of its entries could be zero or close to zero. \cite{smith2002parsimonious} propose a prior distribution that allows for zero entries in $T$ and have obtained a parsimonious model for $\Sigma$ without assuming a parametric structure. Similar results are reported in \cite{huang2006covariance} using penalized likelihood with $L_1$-penalty to estimate $T$ for Gaussian data. Similar in spirit to the tapering estimators based on the sample covariance matrix (Section~\ref{chapter-1-shrinking-the-sample-cov}), several have proposed imposing sparsity by banding the Cholesky factor, including \cite{wu2003nonparametric} and \cite{huang2006covariance}. \cite{levina2008sparse} adaptively band the Cholesky factor using penalized maximum likelihood estimation. Their novel `nested Lasso' penalty produces an estimator with an adaptive bandwidth for each row of the Cholesky factor. This structure has more flexibility than regular banding, but, unlike regular Lasso applied to the entries of the Cholesky factor, results in a sparse estimator for the inverse of the covariance matrix.

\bigskip

\subsubsection{Incoherence of Generalized Autoregressive Parameters with Unbalanced Data}

The aforementioned methods require balanced longitudinal data; it is unclear how they can be applied directly to irregular or incomplete data. In most longitudinal studies, the functional trajectories of the involved smooth random processes are not directly observable, and often, the observed data are sparse and irregularly spaced measurements of these trajectories. In the case that there is no fixed number of measurements and set of associated observation times, there is no applicable notion of a discrete lag, as in the usual formulation of autoregressive models. To handle data collected in such a manner requires methods which are formulated in terms of continuous measurements.
 
 \bigskip 
 
Alternatively, the framework within which the data are generated may assume that a fixed number of measurements are to be collected at a common set of times for all subjects. In this case, unbalanced longitudinal data arises as a result of missing observations. To our knowledge, \cite{huang2012cautionary} was the first to explicitly discuss the problems presented by unbalanced data within this framework, in the context of model \eqref{eq:mcd-ar-model}. These issues are closely related to the ambiguity surrounding the definition of a discrete lag when there is no notion of a regular measurement grid, which \cite{huang2012cautionary} refers to as incoherence in the autoregressive parameters (as well as the prediction variances). They demonstrate incoherence with a simple example: let $y_{it}$ denote the $t^{th}$ repeated measurement on subject $i$. Consider modeling 
\begin{equation}
y_{it} = \phi y_{i,t-1} + \epsilon_{it},
\end{equation}
\noindent
for $t = 2,3,4$ with $y_{i1} = \epsilon_{i1}$, where $\epsilon_i = \left(\epsilon_{i1}, \dots, \epsilon_{ip_i} \right)'$, $\epsilon_i \sim N\left(0, I\right)$. For a subject with a complete set of observations, the diagonal matrix of innovation variances is given by $D = I_4$, and the corresponding $T$ and $\Sigma$ are given by 
\[
T = \begin{bmatrix}
1& 0 & 0 & 0  \\
\phi & 1& 0 & 0 \\
0 & \phi & 1& 0 \\
0 & 0 & \phi & 1\\
\end{bmatrix}, \quad
\Sigma = \begin{bmatrix} 
1 & \phi & \phi^2 & \phi^3 \\
\phi & 1 + \phi^2  & \phi^2 + \phi^3 &  \phi^3 + \phi^4 \\
\phi^2 & \phi^2 + \phi^3 & 1 + \phi^2 + \phi^4 & \phi + \phi^3 + \phi^5 \\
\phi^3 & \phi^3 + \phi^4 & \phi + \phi^3 + \phi^5 & 1 + \phi^2 + \phi^4 + \phi^6 
\end{bmatrix}
\]
Consider a pair of subjects, with Subject 1 having $p_1 = 3$ measurements at $t = 1, 2, 4$, and Subject 2 having $p_2 = 3$ measurements at $t = 1, 3, 4$. The covariance matrix for Subject 1, $\Sigma_1$, can be obtained by deletion of the third row and column of $\Sigma$, and similarly $\Sigma_2$ can be obtained by deletion of the second row and column of $\Sigma$. The Cholesky decompositions of the subject-specific covariance matrices are given by 
\begin{align*}
T_1 = \begin{bmatrix}
1& 0 & 0  \\
-\phi & 1& 0  \\
0 & -\phi^2 & 1
\end{bmatrix}, \quad
D_1 = \begin{bmatrix} 
1 & 0 & 0  \\
0 & 1 & 0 \\
0 & 0 & 1 + \phi^2
\end{bmatrix}, \\
T_2 = \begin{bmatrix}
1& 0 & 0  \\
-\phi^2 & 1& 0  \\
0 & -\phi & 1
\end{bmatrix}, \quad
D_2 = \begin{bmatrix} 
1 & 0 & 0  \\
0 & 1 + \phi^2 & 0 \\
0 & 0 & 1 
\end{bmatrix}, 
\end{align*}

The parameter $\phi_{ijk}$ denotes the coefficient associated with regressing the $j^{th}$ measurement on the $k^{th}$ measurement taken on subject $i$. For example, $\phi_{i21}$ is interpreted as the coefficient when regressing the second measurement on the first, they take different values for each subject. For Subject 1, the measurement at time 2 is regressed on the measurement at time 1, and for Subject 2, the measurement at time 3 is regressed on the measurement at time 1. This results in a discrepancy between the autoregressive coefficients, which are given by $\phi_{121} = \phi$ and $\phi_{221} = \phi^2$. There is similar discordance between the innovation variances. 

\bigskip
 
This incoherence indicates that a naive approach to estimating the regression model \eqref{eq:mcd-ar-model} is inappropriate when the data are unbalanced. \cite{huang2012cautionary} assume that there is a common set of observation times define a ``grand'' covariance matrix $\Sigma$, which is common to all subjects, the measurements on subject $i$ can be modeled with covariance matrix $\Sigma_i$ which is a principal minor of $\Sigma$. They propose handling data from longitudinal studies with dropouts and intermittent missing values by imputation, using the EM algorithm when the data are missing at random.  \cite{huang2007estimation} employ a similar approach, assuming the same framework surrounding the data generation as \cite{huang2012cautionary}. They jointly model the mean and covariance matrix of longitudinal data using basis function expansions. They treat the subdiagonals of $T$ as smooth functions which they approximate using B-splines, and carry out estimation via maximum (normal) likelihood. They regularize the estimated covariance matrix through the choice of $k$, the number of nonzero subdiagonals, and the total number of basis functions used to approximate the $k$ smoothed diagonals, which are selected using Bayesian information criterion (BIC).  
   

%
%We propose an alternate route for estimating the Cholesky decomposition of a covariance matrix when the data are unbalanced. To begin Chapter~\ref{SSANOVA-chapter}, we present a functional varying coefficient model to extend model \eqref{eq:mcd-ar-model}. The functional coefficient model accommodates unbalanced data without the need for imputation and serves as a flexible alternative to parametric models for the GARPs. We propose a general blueprint for the construction of an estimator of a covariance matrix for longitudinal data by modeling $T$ as smooth two-dimensional surface. Chapter~\ref{SSANOVA-chapter} presents a reproducing kernel Hilbert space framework for estimating the functional components of the Cholesky decomposition. Chapter~\ref{psplines-chapter} demonstrates multidimensional smoothing with penalized B-splines as a flexible and computationally convenient alternative to the Hilbert space methods.
%



%\begin{table}[H]
%\centering
%\caption{\textit{Ideal shape of repeated measurements.}}
%\begin{tabular}{cc|cccccc}
%\multicolumn{8}{c}{Occasion}\\
%& & $1$&$2$ &  $\dots$ & $t$ & $\dots$ & $m$ \\ \midrule
%& 1 & $y_{11}$&$y_{12}$ &$\dots$ & $y_{1t}$ & $\dots$& $y_{1m}$ \\
%& 2 & $y_{21}$&$y_{22}$ &$\dots$ & $y_{2t}$ & $\dots$& $y_{2m}$ \\
%\begin{rotate}{90}%
%\mbox{Unit}\end{rotate} & $\vdots$ &$\vdots$&$\vdots$ & &$\vdots$ & & $\vdots$ \\
%& $i$ & $y_{i1}$&$y_{i2}$ &$\dots$ & $y_{it}$ & $\dots$& $y_{im}$ \\
% & $\vdots$ &$\vdots$&$\vdots$ & &$\vdots$ & & $\vdots$ \\
% & $N$ & $y_{N1}$&$y_{N2}$ &$\dots$ & $y_{Nt}$ & $\dots$& $y_{Nm}$ \\
%\end{tabular} \label{table:ideal-repeated-measurements}
%\end{table}
%
%%---------------------------------------------------------------------------------------------------------------------------------------------------------------------------------------------------------------------------------------------------
%We adopt the approach based on the Cholesky decomposition. The modified Cholesky decomposition (MCD) has received much attention in the covariance estimation literature, as it ensures positive-definite covariance estimates, and, unlike the spectral decomposition whose parameters follow an orthogonality constraint, the Cholesky decomposition are unconstrained and have an attractive statistical interpretation as particular regression coefficients and variances.  
%{\needsparaphrased{The Cholesky decomposition is similar to the spectral decomposition in that  is diagonalized by a lower triangular matrix T: 
%
%\[
%T \Sigma T' = D,
%\]
%where the nonredundant entries of T are unconstrained and more meaningful statistically than those of the orthogonal matrix of the spectral decomposition. The matrix T is constructed from the regression coefficients when yt is regressed on its predecessors:
%
%\begin{equation}
%y_t = \sum_{j = 1}^{t-1} \phi_{tj} y_j + \epsilon_t,
%\end{equation}
%\noindent
%where the $\left(t, j\right)$ entry of $T$ is $\phi_{tj}$ , the negatives of the regression coefficients and the $(t, t)$ entry of $D$ is $\sigma_t^2 = var\left(\epsilon_t\right)$, the innovation variance. A schematic view of the components of a covariance matrix obtained through successive regressions (Gram-Schmidt orthogonalization procedure) is given in Table 2. Since the $\phi_{ij}$s are regression coefficients, it is evident that for any unstructured covariance matrix these and the log innovation variances are unconstrained, in the sequel they are referred to as the generalized autoregressive parameters (GARP) and innovation variances (IV) of Y or ? (Pourahmadi, 1999, 2000). Interestingly, this regression approach reveals the equivalence of modeling a covariance matrix to that of dealing with a sequence of $p - 1$ varying-coefficient and varying-order regression models. Consequently, one can bring the entire regression machinery to the service of the unintuitive task of modeling covariance matrices. Stated differently, the framework above is similar to that of using increasing order autoregressive models in approximating the covariance matrix or the spectrum of a stationary time series.}}
%
%The covariance matrix $\Sigma$ of a zero-mean random vector $Y = \left(y_1, \dots , y_p\right)'$ has the following unique modified Cholesky decomposition (Newton, 1988)
%
%\begin{equation} \label{eq:cholesky-matrix-decomposition}
%T \Sigma T' = D, 
%\end{equation}
%
%where $T$ is a lower triangular matrix with $1$?s as its diagonal entries and $D = \mbox{diag}\left(\sigma_1^2, \dots , \sigma_p^2\right)$ is a diagonal matrix. An attractive feature of this decomposition is that unlike the entries of $\Sigma$, the subdiagonal entries of $T$ and the log of the diagonal elements of $D$, $\log\left( \sigma_p^2 \right)$, $t = 1, \dots , m$, are not constrained. This permits one to impose structures on the unconstrained parameters without worrying about the resulting estimator not satisfying the positive-definiteness constraint. Denote estimators of $T$ and $D$ in \eqref{eq:T-Sigma-Ttrans-equals-D} by  $\hat{T}$ and $\hat{D}$, which may be obtained by fitting linear models or some other structural models; then an estimator of $\Sigma$ given by $\Sigma  = \hat{T}^{-T} \hat{D} \hat{T}^{-T}$ is guaranteed to be positive-definite.  From this perspective, covariance modeling can be considered an extension of generalized linear models \cite{McCullagh1989}. Factoring $\Sigma$ as in \eqref{eq:cholesky-matrix-decomposition} provides a link function $g\left(\Sigma\right) = \left(T, \log\left(D\right)\right)$ where $\log\left(D\right) = \mbox{diag}\left( \log\left(\sigma_1^2\right),\dots , \log\left(\sigma_p^2 \right) \right)$. Parametric, nonparametric, or  Bayesian models may then be applied to  the unconstrained entries of $T$ and $\log\left(D\right)$.  Whereas other decompositions are permutation-invariant, the interpretation of  the regression model induced by the MCD assumes a natural (time) ordering among the variables in $Y$.
%
%\bigskip
%
%{\needsparaphrased{immediately leads to the modified Cholesky decomposition \eqref{eq:cholesky-matrix-decomposition}. It also can be used to clarify the close relation between the decomposition (2) and the time series ARMA models in that the latter is means to diagonalize a Toeplitz covariance matrix, for details see Pourahmadi (2001, Sec. 4.2.5).
%
%
%
%\needsparaphrased{In sharp contrast, the fact that the lower triangular matrix $T$ in the Cholesky decomposition of a covariance matrix $\Sigma$ is unconstrained makes it ideal for nonparametric estimation.
%Wu and Pourahmadi (2003) have used local polynomial estimators to smooth the subdiagonals of $T$. For the moment, denoting such estimators of $T$ and $D$ in (2) by $T$ and $D$, an
%estimator of $\Sigma$ given by $\Sigma = \hat{T}^{-1}D{\hat{T}^{-1}}^{\prime}$ is guaranteed to be positive-definite. Although one could smooth rows and columns of $T$,  the idea of smoothing along its subdiagonals is motivated by the similarity of the regressions in (3) to the varying-coefficients autoregressions (Kitagawa and Gersch, 1985, 1996; Dahlhaus, 1997): Xm
%
%Xm
%j=0
%\begin{equation}
%f_{j,p}\left(t/p\right)y_{t_j} = \sigma_p\left(t/p\right)\epsilon_t, \quad t = 0, 1, 2, \dots, p,
%\end{equation}
%\noindent
%where $f_{0,p}\left(�\right) = 1$, $f_{j,p}\left(�\right)$, 1 ? j ? m, and ?p(�) are continuous functions on $\left[0, 1\right]$ and {?t}
%30 is a sequence of independent random variables each with mean zero and variance one. This analogy and comparison with the matrix $T$ for stationary autoregressions having constant
%entries along subdiagonals suggest taking the subdiagonals of $T$ to be realizations of some smooth univariate functions:
%
%\begin{equation*}
%\phi_{t,t-j} = f_{j,p}\left(t/p\right),\quad \sigma_t + \sigma_p \left(t/p\right). 
%\end{equation*}
%
%The details of smoothing and selection of the order $m$ of the autoregression and a simulation study comparing performance of the sample covariance matrix to smoothed estimators are given in Wu and Pourahmadi (2003). Due to the closer connection between entries of $T$ and the family of regression (3), it is conceivable that some of the entries of $T$ could be zero or close to it. Smith and Kohn (2002) have used a prior that allows for zero entries in $T$ and have obtained a parsimonious model for $\Sigma$ without assuming a parametric structure. Similar results are reported in Huang, Liu and Pourahmadi (2004) using penalized likelihood with $L_1$-penalty to estimate $T$ for Gaussian data.}
% A commonly utilized approach in previous work is to model $\phi_{ijk} = z_{ijk}^T \gamma$ where $z_{ijk}$ is a vector of powers of time differences and $\gamma$ is a vector of unknown ``dependence'' parameters to be estimated from the data. \cite{chen2011efficient}, \cite{lin2009robust}, \cite{pan2003modelling},  and \cite{pourahmadi1999joint} define
%
%\begin{equation}
%z_{ijk}^T = \left(1, t_{ij} - t_{ik},\left( t_{ij} - t_{ik} \right)^2, \dots, \left(t_{ij} - t_{ik}\right)^{q-1} \right) \label{covmodel}
%\end{equation}
%
%Modeling the covariance in such a way is reduces a potentially high dimensional problem to something much more computationally feasible; if one models the innovation variances $\sigma^2\left(t\right)$ similarly using a $d$-dimensional vector of covariates, the problem reduces to estimating $q+d$ unconstrained parameters, where much of the dimensionality reduction is a result of characterizing the GARPs in terms of only the difference between pairs of observed time points, and not the time points themselves.  Modeling $\phi$ in such a way is equivalent to specifying a Toeplitz structure for $\Sigma$. A $p \times p$ Toeplitz matrix $p$ is a matrix with elements $m_{ij}$ such that $m_{ij} = m_{\vert i-j \vert}$ i.e. a matrix of the form
%
%
%\bigskip
%
%The estimated covariance matrix may be considerably biased when the specified parametric model is far from the truth.  To avoid model misspecification that potentially accompanies parametric analysis, many have alternatively  proposed nonparametric and semiparametric techniques approaches to estimation.  While these estimators can be very flexible and thus exhibit low bias, this advantage can be offset with high variance.  To balance the tradeoff between bias and variance, shrinkage or regularization may be applied to estimates to improve stability of estimators. \cite{diggle1998nonparametric} proposed nonparametric estimation of the covariance matrix of longitudinal data by smoothing raw sample variogram ordinates and squared residuals.  [DISCUSS THE NONPARAMETRIC SMOOTHER OF HANS GEORG MULLER HERE]  However, neither of these methods ensure that the resulting estimates are positive-definite.  
%
%\bigskip
%Several others have proposed methods for covariance estimation within the same paradigm of a smooth, continuous function underlying a discretized covariance matrix associated with the observed data.   \cite{pourahmadi1999joint} employ the Cholesky decomposition to guarantee positive-definiteness and imposed structure on the elements of the Cholesky decomposition and heuristically argue that $\phi_{t,t-l}$ should be monotonically decreasing in $l$. That is, the effect of $y_{t-l}$ on $y_t$ through the autoregressive parameterization should decrease as the distance in time between the two measurements increases. In similar spirit, others including \cite{bickel2008regularized} and \cite{levina2008sparse} enforce such structure by setting $\phi_{t,t-l}$ equal to zero for $l$ large enough, or equivalently, setting all subdiagonals of $T$ to zero beyond the $K^{th}$ off-diagonal. The tuning parameter $K$ is chosen using a model selection criterion such as Akaike information criterion, Bayesian information criterion, or cross validation or a variant thereof.  In terms of the autoregressive model corresponding to the Cholesky decomposition, this form of regularization, known as ``banding'' the Cholesky factor $T$, is equivalent to regressing $y_t$ on only its $K$ immediate predecessors, setting $\phi_{tj} = 0$ for $t-j>K$. 
%
%\bigskip
%
%From this perspective, it is apparent that the presentation of covariance estimation as a least squares regression problem suggests that the familiar ideas of model regularization for least-squares regression can be used for estimating covariances.  . \cite{huang2007estimation} 
%
%however, their two-step method did not utilize the information that many of the subdiagonals of T are essentially zeros at the first step. Inefficient estimation may result because of ignoring regularization structure in constructing the raw estimator. 
%
%\bigskip
%
%Several have applied these approaches to covariance estimation; 
%\bigskip
%
%Alternatively, one can view $T$ as a bivariate function,
%
%Several others have considered this approach to covariance estimation; \cite{kaufman2008covariance} assume a stationary process, restricting covariance estimates to a specific class of functions.  They as well as  Huang, Liu, and Liu \cite{huang2007estimation} follow the hueristic argument presented by \cite{pourahmadi1999joint} that $\phi_{t,t-l}$ is monotone decreasing in $l$ and set off-diagonal elements of either the covariance matrix or the Cholesky factor corresponding to large lags to zero.   As in \cite{huang2007estimation}, \cite{kaufman2008covariance}, and \cite{yao2005functional}, we treat covariance estimation as a function estimation problem where the covariance matrix is viewed as the evaluation of a smooth function at particular design points. 
%
%including \cite{bickel2008regularized} and \cite{huang2006covariance}  have proposed nonparametric estimators of a specific covariance matrix (or its inverse) rather than the parameters of a covariance function. 
%
%\bigskip
%
%\cite{yao2005functional} do not utilize the Cholesky parameterization, and their estimates are not guaranteed to be positive definite.  We combine the advantages of bivariate smoothing as in \cite{yao2005functional} with the added utility of the Cholesky parameterization in \cite{huang2007estimation}; in doing so, we present a flexible and coherent approach to covariance estimation, while simultaneously we ensuring positive definiteness of estimates.Rather than shrinking element of the Cholesky factor to zero after a particular value of $l$, we choose to softly enforce monotonicity in $l$ by using a hinge penalty as in the work of \cite{tibshirani2011nearly}. 
%
%\section{The Cholesky Decomposition and the MLE for $\Sigma$}
%
%Let $Y = \left( y_{1}, y_{2}, \dots, y_{m} \right)'$ denote a mean zero random vector with variance-covariance matrix $\Sigma$, which we can think of as the time-ordered measurements on one subject in a longitudinal study. To present a comprehensive overview our estimation procedure, we begin with the representation of the covariance matrix, $\Sigma$, in terms of its Cholesky decomposition. Decomposing $\Sigma$ in such a way allows for both an unconstrained parameterization and statistically meaningful interpretation of covariance parameters. For any positive definite matrix $\Sigma$, there exists a unique lower triangular matrix $T$ with diagonal entries equal to $1$ which diagonalizes $\Sigma$:
%
%\begin{equation} \label{eq:T-Sigma-Ttrans-equals-D}
% T \Sigma T^T = D
%\end{equation}
%\noindent
%
%The convenient statistical interpretation of the parameters of the covariance matrix then comes if we consider, for $t = 2, \dots, m$, regressing $y_t$ on its predecessors $y_1,\dots, y_{t-1}$, letting
%\begin{equation} 
%{y}_{i}  = \sum_{j=1}^{i-1} \phi_{ij} y_{j} + \sigma_{i}\epsilon_{i} \label{eq:discrete-evenly-spaced-ar-model},
%\end{equation}
%\noindent
%where $\mbox{var}\left( \epsilon_i \right) = \sigma_i^2$. If we take the $i$-$j^{th}$ element $T$ to be $-\phi_{ij}$ for $j < i$, and take the $i^{th}$ diagonal entry of $D$ to be $\mbox{var}\left( \epsilon_i \right) = \sigma_i^2$, a vectorized expression for Model~\eqref{eq:discrete-evenly-spaced-ar-model} is given by
%
%\begin{equation}
%\bfeps = T Y \label{eq:vectorized-ar-model}.
%\end{equation}
%\noindent
%and taking covariances on both sides of \eqref{epsilon}, we see that $T$ and $D$ satisfy \eqref{eq:T-Sigma-Ttrans-equals-D}. Immediately, we have that $\Sigma^{-1} = T' D^{-1} T$. The regression coefficients $\lbrace \phi_{ij} \rbrace$ are referred to as the \emph{generalized autoregressive parameters} (GARPs), and the $\lbrace \sigma_{ij} \rbrace$ are referred to as the \emph{innovation variances} (IVs.) 
%\bigskip
%Assuming that $Y$ follows a multivariate normal distribution, the loglikelihood function $\ell \left( Y, \Sigma \right)$ satisfies
%
%\begin{equation} \label{eq:loglik-general-form}
%-2\ell\left( Y, \Sigma \right) = \log \vert \Sigma \vert + Y' \Sigma Y
%\end{equation}
%\noindent
%From \eqref{eq:T-Sigma-Ttrans-equals-D}, we have that 
%\[
%\vert \Sigma\vert = \vert D \vert = \prod_{i = 1}^m \sigma_i^2
%\]
%and 
%\[
%\Sigma^{-1} = T' D^{-1} T.
%\]
%Thus, \eqref{eq:loglik-general-form} can be written in terms of the prediction errors and their variances of the non-redundant entries of $\left(T , D\right)$:
%
%\begin{align}
%\begin{split} \label{eq:loglik-cholesky-form}
%-2\ell\left( Y, \Sigma \right) &= \log \vert D \vert + Y' T' D^{-1} T Y \\
%&= \sum_{i = 1}^m \log \sigma_i^2  + \sum_{i = 1}^m \frac {\epsilon_i^2}{\sigma_i^2},
%\end{split}
%\end{align}
%\noindent
%where $\epsilon_1 = y_1$ and $\epsilon_i = y_i - \sum_{j = 1}^{i-1} \phi_{ij} y_j$. Maximum likelihood estimation or any of its penalized variants may then be employed to obtain estimates of $T$ and $D$.
%
%\bigskip
%Unlike many of those before who have used the Cholesky decomposition as a means of modeling $\Sigma$, we allow observed time points to be individual-specific and not necessarily regularly spaced.  Let $Y_1, \dots, Y_N$ denote a random sample of mean zero vectors of longitudinal measurements taken on $N$ subjects having common covariance structure $\Sigma$.  We allow subject $i$ to have observation vector $y_i = \left(y_{i1} ,\dots , y_{i,m_i}\right)'$ with corresponding vector of observation times $\left(t_{i1} ,\dots , t_{i,m_i}\right)'$.  Accommodating the subject-specific sample sizes and measurement times requires merely adding a subscript, and Model \eqref{eq:discrete-evenly-spaced-ar-model} becomes 
%
%\begin{equation}
%{y}_{ij}  = \sum_{k=1}^{j-1} \phi_{ijk} y_{ik} + \sigma_{ij}\epsilon_{ij}, \label{eq:discrete-unevenly-spaced-ar-model}
%\end{equation}
%\noindent
%where $\phi_{ijk}$ is the autoregressive coefficient corresponding to the pair of measurements observed at time $t_{ij}$ and $t_{ik}$. A vectorized representation of Model~\eqref{eq:discrete-unevenly-spaced-ar-model} can be obtained as before by adding the necessary parameters to $T$ and $D$.



%\bigskip
%
%Modeling $\phi_{ij} = \phi\left(t_i, t_j\right)$ as a smooth bivariate function, we cast the problem of estimating a covariance matrix as the estimation of a functional varying coefficient model. The existing body of literature surrounding these models is an extensive one; see \cite{csenturk2008generalized}, \cite{csenturk2013modeling}, and \cite{noh2010sparse}. This class of models is both flexible and interpretable, making them a pragmatic modeling choice when understanding the underlying data generating mechanism is of as much importance as strong predictive capability. 




%

\chapter{A reproducing kernel Hilbert space estimation framework for covariance estimation} \label{SSANOVA-chapter}
%
%A predominant difficulty in the estimation of covariance matrices is the potentially high dimensionality of the problem, as the number of unknown elements in the covariance matrix grows quadratically with the size of the matrix. It is well-known that the sample covariance matrix can be unstable in high dimensions; ways for controlling the complexity of estimates is highly desirable for improving stability of estimates. In the longitudinal-data literature, it is a common practice to use parametric models for the covariance structure.  Many have specified parsimonious parametric models for $\phi_{ijk}$ to overcome the issue of dimensionality.  
%
%\bigskip
%
%We naturally accommodate irregularly spaced data and unequal sample sizes between subjects by defining the autoregressive parameters as the values of a smooth function evaluated at within-subject pairs of observed time points.  Furthermore, by viewing $\phi\left(t,s\right)$ as a smooth \emph{bivariate} function, we can utilize the information across the subdiagonals of $T$ to inform the fit, rather than treating each subdiagonal separately.  As in the classical nonparametric function estimation setting, we assume $\phi$ to vary in a high-dimensional (possibly infinite) function space. We propose two representations of $\phi\left(\cdot, \cdot\right)$ and $\sigma\left(\cdot, \cdot\right)$: approximation by smoothing splines and approximation by B-spline basis expansion. 
%
%We assume $Y\left(t\right)$ has covariance function $G\left(t,s\right)$ and that $\epsilon\left(t\right)$ follows a zero mean Gaussian white noise process with unit variance. Under mild assumptions regarding the behaviour of $Y$, then $G\left(t,s\right)$ satisfies some smoothness conditions, where smoothness is defined in terms of square integrability of certain derivatives. We view the entries of $\Sigma$ as values of $G$ evaluated at the distinct pairs of within-subject observed time points. 
%\bigskip


If we consider the Cholesky decomposition of $\Sigma$ within such functional context, it is natural to extent the same notion to the elements of $T$ and $D$. We take the GARPs $\lbrace \phi_{tj} \rbrace$ and innovation variances to be the evaluation of the smooth functions $\tilde{\phi}\left(t,s\right)$ and $\sigma^2\left(t\right)$ at observed time points, which we assume  are drawn from some distribution having compact domain $\mathcal{T}$. Without loss of generality, we take $\mathcal{T} = \left[0,1\right]$. Henceforth, we view $\tilde{\phi}$ and $\sigma^2$ as a smooth continuous functions, but for ease of exposition, we let $\tilde{\phi}_{ij}$ denote the varying coefficient function evalutated at $\left(t_i,t_j\right)$: 
\[
\tilde{\phi}_{tj} = \tilde{\phi}\left(t_{i},t_{j}\right). 
\]
Adopting similar notation for the innovation variance function, denote $\sigma_{j}^2 = \sigma^2\left(t_{j}\right)$ where $0 \le t_{j} < t_{i} \le 1$ for $j < i$. This leads to varying coefficient model

\begin{equation}  \label{eq:cholesky-regression-model-1} 
y\left(t_{i} \right)  = \sum_{j=1}^{i-1} \tilde{\phi}\left(t_{i} ,t_{j}\right) y\left(t_{j}\right) + \sigma\left(t_{j}\right)\epsilon\left({t_j}\right) \;\;\;\; i=1,\dots, p, 
\end{equation}
\noindent

Our goal is now to estimate the above model, utilizing bivariate smoothing to estimate $\tilde{\phi}\left(t,s\right)$ for $0 \le s < t \le 1$,  and one-dimensional smoothing to estimate $\sigma\left(t \right)$, $0 \le t \le 1$. Our proposed method for covariance estimation defines a flexible, general framework which makes all of the existing techniques for penalized regression accessible for the seemingly far different task of estimating a covariance matrix.

\bigskip

Our approach to estimation is constructed to provide a fully data-driven methodology for selecting the optimal covariance model (given some optimization criterion) from a expansive class of estimators ranging in complexity from that of the previously aforementioned parametric models to that of completely unstructured estimators, like the sample covariance matrix. We leverage the collection of regularization techniques that are accessible in the usual function estimation setting. By properly specifying the roughness penalty, our optimization procedure results in null models which correspond to the parametric and semiparametric models for $\phi$ and $\sigma^2$ discussed in Chapter~\ref{background-review-chapter}. To facilitate the penalty specification that achieves this, we consider modeling the varying coefficient function which takes inputs

\begin{align} 
\begin{split}\label{eq:l-m-transformation}
l &= t - s \\
m &= \frac{t + s}{2}, \\
\end{split}
\end{align}
\noindent
 where $l$ is the continuous analogue of the usual ``lag'' between time points $t$ and $s$, and $m$ is simply its orthogonal direction. We have discussed many parsimonious covariance structures which model $y\left(t\right)$ as a stationary process with covariance function which depends on time points $t_i$ and $t_j$ only through the Euclidean distance $\vert \vert t_i - t_j \vert \vert$ between them. Covariance functions taking the form $Cov\left(y\left( t_i \right),y\left( t_j \right)\right) =G\left(t_i,t_j\right) = G\left(\vert \vert t_i - t_j \vert \vert \right)$ can then be written as 

\begin{equation*}
Cov\left(y\left( t_i \right),y\left( t_j \right)\right) = G\left( l_{ij}  \right)
\end{equation*}
\noindent
where $l_{ij} =  \vert  t_i - t_j  \vert $. Regularizing the functional components of the Cholesky decomposition so that functions incurring large penalty correspond to functions which vary in only $l$ and are constant in $m$ allows us to model nonstationarity in a fully data-driven way.  Our goal is to estimate

\begin{equation} 
\phi\left(l,m\right) = \phi\left(s-t, \frac{1}{2}\left(s+t\right)\right) = \tilde{\phi}\left(t,s\right).
\end{equation}

\bigskip

While our framework allows for estimation of the autoregressive coefficient function and the innovation variance function via any nonparametric regression setup, we focus on two primary approaches for representing $\phi$ and $\sigma$. First, we assume that $\phi$ belongs to a reproducing kernel Hilbert space, $\mathcal{H}$ and employ the smoothing spline methods of Kimeldorf and Wahba (see \cite{kimeldorf1971some} and \cite{wahba1990spline} for comprehensive presentation.)  To enhance the statistical interpretability of model parameters, we decompose $\phi$ into functional components similar to the notion of the main effect and the interaction terms in classical analysis of variance. We adopt the smoothing spline analogue of the classical ANOVA model proposed by \cite{gu2013smoothing}, and estimation is achieved through similar computational strategies.

\bigskip

Let random vector $Y$ follow a multivariate normal distribution with zero mean vector and covariance $\Sigma$. The loglikelihood function $\ell \left( Y, \Sigma \right)$ satisfies

\begin{equation} \label{eq:loglik-general-form}
-2\ell\left( Y, \Sigma \right) = \log \vert \Sigma \vert + Y' \Sigma Y
\end{equation}
\noindent
Using $T \Sigma T' = D$, we can write 
\[
\vert \Sigma\vert = \vert D \vert = \prod_{i = 1}^m \sigma_i^2
\]
and 
\[
\Sigma^{-1} = T' D^{-1} T.
\]
Writing \ref{eq:loglik-general-form} in terms of the prediction errors and their variances of the non-redundant entries of $\left(T , D\right)$, we have

\begin{align}
\begin{split} \label{eq:loglik-cholesky-form}
-2\ell\left( Y, \Sigma \right) &= \log \vert D \vert + Y' T' D^{-1} T Y \\
&= \sum_{i = 1}^m \log \sigma_i^2  + \sum_{i = 1}^m \frac {\epsilon_i^2}{\sigma_i^2},
\end{split}
\end{align}
\noindent
where 
\begin{equation} \label{eq:loglik-cholesky-form}
\epsilon_i = \left\{\begin{array}{lr}y\left(t_1\right), & i = 1, \\
y\left(t_i\right) - \sum_{j = 1}^{i-1} \phi\left(\bfv_{ij}\right) y_j, & i= 2, \dots, p, \\
\end{array} \right.
\end{equation}
\noindent
where $\phi\left(\bfv_{ij}\right) = \phi\left(l_{ij},m_{ij}\right) = \tilde{\phi}\left(t_i,t_j\right)$.  Accommodating subject-specific sample sizes and measurement times merely requires appending an additional index to observation times. Let  $Y_1, \dots, Y_N$ denote a sample of $N$ independent mean zero random trajectories from a  multivariate normal distribution with common covariance $\Sigma$. We associate with each trajectory $Y_i = \left(y_{i1}, \dots, y_{i,m_i}\right)'$ with a vector of potentially subject-specific observation times $\left(t_{i1}, \dots, t_{i,m_i}\right)'$, so that the $j^{th}$ measurement of trajectory $i$ is modeled

\begin{align}
\begin{split} \label{eq:cholesky-regression-model-2} 
y\left(t_{ij} \right)  &= \sum_{k=1}^{j-1} \tilde{\phi}\left(t_{ij} ,t_{ik}\right) y\left(t_{ik}\right) + \sigma\left(t_{ij}\right)\epsilon\left(t_{ij}\right)  \\
&= \sum_{k=1}^{j-1} \phi\left(\bfv_{ijk}\right) y\left(t_{ik}\right) + \sigma\left(t_{ij}\right)\epsilon\left(t_{ij}\right)
\end{split}
\end{align}
\noindent
for $i = 1,\dots, N$, $j = 2,\dots, m_i$.
\noindent
Making similar ammendments to indexing, the joint log likelihood for the sample $Y_1, \dots, Y_N$ is given by  

\begin{equation} \label{eq:joint-loglik}
-2\ell\left( Y_1,\dots, Y_N, \phi, \sigma^2 \right) = \sum_{i = 1}^N \sum_{j = 1}^{m_i} \log \sigma_{ij}^2  + \sum_{i = 1}^N \sum_{j = 1}^{m_i} \frac {\epsilon_{ij}^2}{\sigma_{ij}^2},
\end{equation}

\bigskip

With this, we can estimate $\phi$ and $\log\sigma^2$ using maximum likelihood or any of its penalized variants by appending a roughness penalty (penalties) to \ref{eq:joint-loglik}. Employing regularization, we take $\phi$, $\sigma^2$ to minimize 

\begin{equation} \label{eq:penalized-joint-loglik}
-2\ell\left( Y_1,\dots, Y_N, \phi, \sigma^2 \right) +    \lambda J\left( \phi \right) +  \breve{\lambda}\breve{J}\left( \sigma^2 \right),
\end{equation}
\noindent
where $J$ and $\breve{J}$ are roughness penalties on $\phi$ and $\sigma^2$, and $\lambda$, $\breve{\lambda}$ are non-negative smoothing parameters.  To jointly estimate the GARP function and the IV function, we adopt an iterative approach in the spirit of \cite{huang2006covariance}, \cite{huang2007estimation}, and \cite{pourahmadi2000maximum}. A procedure for minimizing \ref{eq:joint-loglik} starts with initializing $\left\{\sigma^2_{ij}\right\} = 1$ for $i = 1,\dots, N$, $j = 1,\dots, m_i$.  For fixed $\sigma^2$, the penalized likelihood (as a function of $\phi$) is given by

\begin{equation} \label{eq:penalized-joint-loglik-given-sigma}
-2\ell\left( Y_1,\dots, Y_N, \phi \vert \sigma^2\right) + \lambda J\left(\phi\right) = \sum_{i=1}^N \sum_{j=2}^{m_i} \sigma^{-2}_{ij}\left( y_{ij} - \sum_{k<j} \phi\left(\bfv_{ijk}\right) y_{ik}  \right)^2 + \lambda J\left( \phi \right),
\end{equation}
\noindent
which corresponds to the usual penalized least squares functional encountered  in the nonparametric function estimation literature. The first term, the residual sums of squares, encourages the fitted function's fidelity to the data. The second term penalizes the roughness of $\phi$, and $\lambda$ is a smoothing parameter which controls the tradeoff between the two conflicting concerns. Given $\phi^*$ the minimizer of \ref{eq:penalized-joint-loglik-given-sigma} and setting $\phi = \phi^*$, we update our estimate of $\sigma^2$ by minimizing 

\begin{equation} \label{eq:penalized-joint-loglik-given-phi}
-2\ell\left( Y_1,\dots, Y_N, \sigma^2 \vert \phi \right) + \breve{\lambda} \breve{J}\left(\sigma^2\right) = \sum_{i=1}^N \sum_{j=2}^{m_i} \log \sigma^2_{ij} + \sum_{i=1}^N \sum_{j=1}^{m_i} \sigma_{ij}^{-2} {r_{ij}^*}^2 + \breve{\lambda} \breve{J}\left(\sigma^2 \right),
\end{equation}
where the $\left\{{r_{ij}^*}^2  =\left( y_{ij} - \sum_{k<j} \phi^*\left(\bfv_{ijk}\right) y_{ik}  \right)\right\}$ denote the working residuals based on the current estimate of $\phi$. This process of iteratively updating $\phi^*$ and ${\sigma^2}^*$ is repeated until convergence is achieved. 
\bigskip


%%%%%%%%%%%%%%%%%%%%%%%%%%%%%%%%%%%%%%%%%%%%%%%%%%%%%%%%%%%%%%%%%%%%%%%%%%%%%%%%%%%%%%%%%%
%%%%%%%%%%%%%%%%%%%%%%%%%%%%%%%%%%%%%%%%%%%%%%%%%%%%%%%%%%%%%%%%%%%%%%%%%%%%%%%%%%%%%%%%%%


%%%%%%%%%%%%%%%%%%%%%%%%%%%%%%%%%%%%%%%%%%%%%%%%%%%%%%%%%%%%%%%%%%%%%%%%%%%%%%%%%%%%%%%%%%%
%%%%%%%%%%%%%%%%%%%%%%%%%%%%%%%%%%%%%%%%%%%%%%%%%%%%%%%%%%%%%%%%%%%%%%%%%%%%%%%%%%%%%%%%%%%
%%%%%%%%%%%%%%%%%%%%%%%%%%%%%%%%%%%%%%%%%%%%%%%%%%%%%%%%%%%%%%%%%%%%%%%%%%%%%%%%%%%%%%%%%%%
%%%%%%%%%%%%%%%%%%%%%%%%%%%%%%%%%%%%%%%%%%%%%%%%%%%%%%%%%%%%%%%%%%%%%%%%%%%%%%%%%%%%%%%%%%%
%%%%%%%%%%%%%%%%%%%%%%%%%%%%%%%%%%%%%%%%%%%%%%%%%%%%%%%%%%%%%%%%%%%%%%%%%%%%%%%%%%%%%%%%%%%
%%%%%%%%%%%%%%%%%%%%%%%%%%%%%%%%%%%%%%%%%%%%%%%%%%%%%%%%%%%%%%%%%%%%%%%%%%%%%%%%%%%%%%%%%%%
%%%%%%%%%%%%%%%%%%%%%%%%%%%%%%%%%%%%%%%%%%%%%%%%%%%%%%%%%%%%%%%%%%%%%%%%%%%%%%%%%%%%%%%%%%%

\section{A smoothing spline ANOVA model for the generalized autoregressive coefficients}\label{RKHS-framework-for-phi}

%%%%%%%%%%%%%%%%%%%%%%%%%%%%%%%%%%%%%%%%%%%%%%%%%%%%%%%%%%%%%%%%%%%%%%%%%%%%%%%%%%%%%%%%%%%
%%%%%%%%%%%%%%%%%%%%%%%%%%%%%%%%%%%%%%%%%%%%%%%%%%%%%%%%%%%%%%%%%%%%%%%%%%%%%%%%%%%%%%%%%%%
%%%%%%%%%%%%%%%%%%%%%%%%%%%%%%%%%%%%%%%%%%%%%%%%%%%%%%%%%%%%%%%%%%%%%%%%%%%%%%%%%%%%%%%%%%%
%%%%%%%%%%%%%%%%%%%%%%%%%%%%%%%%%%%%%%%%%%%%%%%%%%%%%%%%%%%%%%%%%%%%%%%%%%%%%%%%%%%%%%%%%%%
%%%%%%%%%%%%%%%%%%%%%%%%%%%%%%%%%%%%%%%%%%%%%%%%%%%%%%%%%%%%%%%%%%%%%%%%%%%%%%%%%%%%%%%%%%%
%%%%%%%%%%%%%%%%%%%%%%%%%%%%%%%%%%%%%%%%%%%%%%%%%%%%%%%%%%%%%%%%%%%%%%%%%%%%%%%%%%%%%%%%%%%
%%%%%%%%%%%%%%%%%%%%%%%%%%%%%%%%%%%%%%%%%%%%%%%%%%%%%%%%%%%%%%%%%%%%%%%%%%%%%%%%%%%%%%%%%%%

%%%%%%%%%%%%%%%%%%%%%%%%%%%%%%%%%%%%%%%%%%%%%%%%%%%%%%%%%%%%%%%%%%%%%%%%%%%%%%%%%%%%%%%%%%%
%%%%%%%%%%%%%%%%%%%%%%%%%%%%%%%%%%%%%%%%%%%%%%%%%%%%%%%%%%%%%%%%%%%%%%%%%%%%%%%%%%%%%%%%%%%
%%%%%%%%%%%%%%%%%%%%%%%%%%%%%%%%%%%%%%%%%%%%%%%%%%%%%%%%%%%%%%%%%%%%%%%%%%%%%%%%%%%%%%%%%%%
%%%%%%%%%%%%%%%%%%%%%%%%%%%%%%%%%%%%%%%%%%%%%%%%%%%%%%%%%%%%%%%%%%%%%%%%%%%%%%%%%%%%%%%%%%%
%%%%%%%%%%%%%%%%%%%%%%%%%%%%%%%%%%%%%%%%%%%%%%%%%%%%%%%%%%%%%%%%%%%%%%%%%%%%%%%%%%%%%%%%%%%
%%%%%%%%%%%%%%%%%%%%%%%%%%%%%%%%%%%%%%%%%%%%%%%%%%%%%%%%%%%%%%%%%%%%%%%%%%%%%%%%%%%%%%%%%%%
%%%%%%%%%%%%%%%%%%%%%%%%%%%%%%%%%%%%%%%%%%%%%%%%%%%%%%%%%%%%%%%%%%%%%%%%%%%%%%%%%%%%%%%%%%%

%\subsection{An RKHS framework for estimating $\phi$, $\sigma^2$} \label{RKHS-framework-for-phi}
%\subfile{chapter-2-subfiles/chapter-2-smoothing-spline-representation}
This section presents a reproducing kernel Hilbert space (RKHS) framework for estimating the generalized autoregressive coefficient function $\phi$, and in later sections we apply the same approach for estimating the innovation variance function, $\sigma^2$. Specifically, we adopt the smoothing spline ANOVA models developed by \cite{gu2002smoothing} to connect fitted models to parsimonious models proposed in the literature for the components of the Cholesky decomposition. The flexibility of this framework permits an entirely data-driven modeling approach through careful penalty specification and the use of already well-developed model selection methods. Though RKHS methods, and in particular smoothing spline ANOVA models, have been studied extensively for nonparametric function estimation (see \cite{aronszajn1950theory}, \cite{wahba1990spline}, and \cite{berlinet2011reproducing} for detailed examinations), to our knowledge they have received little attention in the context of covariance models. To demonstrate our framework, we first must establish some notation and review the relevant mathematical details of reproducing kernel Hilbert spaces. 

\bigskip

A Hilbert space $\hilbert$ of functions on a set $\mathcal{V}$ with inner product $\langle \cdot, \cdot\rangle_\hilbert$ is defined as a complete inner product linear space. A Hilbert space is called a reproducing kernel Hilbert space if the evaluation functional $\left[\bfv\right]f = f\left(\bfv\right)$ is continuous in $\hilbert$ for all $\bfv \in \mathcal{V}$. The Reisz Representation Theorem gives that there exists $Q \in \hilbert$, the representer of the evaluation functional $\left[\bfv\right]\left(\cdot\right)$, such that $\langle Q_\bfv, \phi \rangle_\hilbert = \phi\left(\bfv\right)$ for all $\phi \in \mathcal{H}$. See \cite{gu2013smoothing}, Theorem 2.2.

\bigskip

The symmetric, bivariate function $Q\left(\bfv_1, \bfv_2 \right) = Q_{\bfv_2 }\left(\bfv_1\right) = \langle Q_{\bfv_1}, Q_{\bfv_2} \rangle_\hilbert$ is called the reproducing kernel (RK) of $\hilbert$. The RK satisfies that for every $\bfv \in \mathcal{V}$ and $f \in \mathcal{H}$,

\begin{enumerate}
\item $Q\left(\cdot, \bfv \right) \in \hilbert$ 
\item $f\left(\bfv\right) = \langle f, Q\left(\cdot, v\right)\rangle_\hilbert$\label{rkhs-reproducing-property}
\end{enumerate}
\noindent
The first property is called the reproducing property of $Q$. Every reproducing kernel uniquely determines the RKHS, and in turn, every RKHS has unique reproducing kernel. See \cite{gu2013smoothing}, Theorem 2.3. The kernel satisfies that for any $\left\{\bfv_1,\dots, \bfv_{n_1}\right\}$, $\left\{\breve{\bfv}_1,\dots, \breve{\bfv}_{n_2}\right\} \in \mathcal{V}$ and $\left\{a_1,\dots, a_{n_1}\right\}$, $\left\{a_1,\dots, a'_{n_2}\right\} \in \Re$,

\begin{equation}
 \langle\sum_{i = 1}^{n_1} a_i Q\left(\cdot, \bfv_i\right), \sum_{j = 1}^{n_2} a'_j Q\left(\cdot, \breve{\bfv}_j\right) \rangle_\hilbert.
\end{equation}

\bigskip


Let $\mathcal{N}_J = \left\{ \phi:\; J\left(\phi\right) = 0\right\}$ denote the null space of $J$, and consider the decomposition

\[
\hilbert = \mathcal{N}_J \oplus \hilbert_J.
\]
\noindent
The space $\hilbert_J$ is a RKHS having $J\left(\phi\right)$ as the squared norm. The representer of any bounded linear functional can be obtained from the reproducing kernel $Q$. Let $\psi_{ij}$ denote the representer for the evaluation functional, $L_{ij}$, i.e. $\psi_{ij}$ satisfies

\[
\langle \psi_{ij}, \phi \rangle = L_{ij} \phi, \quad \phi \in \hilbert.
\]
\noindent
Then one may write $\psi\left( \bfv_{ij} \right)$ as the inner product of itself with the reproducing kernel:

\begin{equation} \label{eq:representer-as-inner-product}
\psi_{ij}\left( \bfv \right) = \langle \psi_{ij}, Q_{\bfv} \rangle = L_{ij} Q_{\bfv} = L_{ij\left(\cdot\right)} Q \left(\bfv,\cdot\right)
\end{equation}
 \noindent
 where the notation $L_{ij\left(\cdot\right)}$ indicates that $L_{ij}$ is applied to what immediately follows as a function of $\left( \cdot \right)$, so that one can obtain $\psi_{ij}\left(\bfv\right)$ by applying $L_{ij}$ to $Q\left(\bfv, \bfv^*\right)$, considered as a function of $\bfv^*$. \cite{wahba1990spline} established an explicit form for the minimizer of the penalized sums of squares
 
 \begin{equation} \label{eq:phi-penalized-sums-of-squares}
 -2\ell_\phi + \lambda J\left(\phi\right) = \sum_{i=1}^N \sum_{j=2}^{m_i} \sigma^{-2}_{ij}\left( y_{ij} - \sum_{k<j} \phi\left(\bfv_{ijk}\right) y_{ik}  \right)^2 + \lambda J\left( \phi \right),
 \end{equation}
 \noindent
 which can now be written
 \begin{equation} \label{eq:phi-penalized-sums-of-squares-RK-norm}
-2\ell_\phi + \lambda J\left(\phi\right) = \sum_{i=1}^N \sum_{j=2}^{m_i} \sigma^{-2}_{ij}\left( y_{ij} - \sum_{k<j}\left( L_{ijk}\phi\right) y_{ik}  \right)^2 + \lambda \vert\vert P_J \phi \vert \vert_\hilbert^2, 
\end{equation} 
\noindent
where $P_J$ is the projection operator which projects $\phi$ onto the subspace $\hilbert_J$, and $L_{ijk}$ denotes the evaluation functional $\left[\bfv_{ijk}\right] \phi$. 
 
 \begin{theorem} \label{theorem:finite-dimensional-minimizer}
 Let $\left\{\eta_1,\dots, \eta_{d_0}\right\}$ span the null space of $P_J$, $\hilbert_0$. Let  $V = \bigcup\limits_{i,j,k} \bfv_{ijk} \equiv \left\{ \bfv_1,\dots,\bfv_{\vert V \vert} \right\}$ denote the set of unique within-subject pairs of observation times. Let $B$ denote the $\vert V \vert \times d_0$ matrix having $i^{th}$ column equal to $\eta_i$ evaluated at the vector of observed $\bfv \in V$, and assume that $B$ has full column rank. Then the minimizer $\phi_\lambda$ of \ref{eq:phi-penalized-sums-of-squares-RK-norm} is given by
 
\begin{equation} \label{eq:form-of-the-minimizer-phi}
\phi_\lambda = \sum_{\nu = 1}^{d_0} d_\nu \eta_nu + \sum_{\bfv_i \in V} c_i \xi_i,
\end{equation}
\noindent
where $\xi_i = P_J \psi_i$ is the projection of $L_i$, the representer for the evaluation functional corresponding to the $i^{th}$ element of $V$, onto $\hilbert_J$.
\end{theorem}
\vspace{0.5cm}
\noindent
The proof, which is similar in spirit to the proof of Theorem 1.3.1 in \cite{wahba1990spline} can be found in Appendix~\ref{chapter-2-appendix}.

\bigskip

%\subfile{chapter-2-subfiles/tensor-product-hilbert-space-construction}

Convenient construction of a reproducing kernel Hilbert space on a domain
\[
\mathcal{V} = \mathcal{V}_1 \otimes \mathcal{V}_2
\]
\noindent
which can be written as a product domain, is available through the tensor product of the RKHS for each of the marginal domains $\mathcal{V}_1$ and $\mathcal{V}_2$. Without loss of generality, we can let $l,\;m \in \left[0,1\right] = \mathcal{V}_1 = \mathcal{V}_2$. Given Hilbert space for the domain of $l$, $\hilbert_{\left[1\right]}$ with reproducing kernel $Q_1$ and Hilbert space on the domain of $m$, $\hilbert_{\left[2\right]}$ with reproducing kernel $Q_2$, the reproducing kernel $Q = Q_{\left[1\right]}Q_{\left[2\right]}$ corresponds to that of the tensor product space of $\hilbert_{\left[1\right]}$ and $\hilbert_{\left[2\right]}$, denoted
\[
\hilbert = \hilbert_{\left[1\right]} \otimes \hilbert_{\left[2\right]}.
\]
\noindent
See \cite{gu2002smoothing}, Theorem 2.6. Let $\mathcal{A}_1$, $\mathcal{A}_2$ denote the averaging operators defining ANOVA decompositions on $\hilbert_{\left[1\right]}$, $\hilbert_{\left[2\right]}$, respectively, where $\hilbert_{0\left[i\right]}$ has RK $Q_{0\left[i\right]}$, $i = 1, 2$ and $\hilbert_{1\left[i\right]}$ has RK $Q_{1\left[i\right]}$ satisfying $\mathcal{A}_1Q_{\left[1\right]}\left(l,\cdot\right) = \mathcal{A}_2Q_{\left[2\right]}\left(m,\cdot\right) = 0$. Then the tensor product space $\hilbert$ has tensor sum decomposition

\begin{align} 
\begin{split} \label{eq:tensor-sum-decomposition}
\hilbert &= \left[\hilbert_{0\left[1\right]} \oplus \hilbert_{1\left[1\right]} \right] \otimes \left[\hilbert_{0\left[2\right]} \oplus \hilbert_{1\left[2\right]} \right] \\
&= \left[\hilbert_{0\left[1\right]} \otimes  \hilbert_{0\left[2\right]}\right] \oplus \left[\hilbert_{0\left[1\right]} \otimes \hilbert_{1\left[2\right]}\right] \oplus \left[\hilbert_{1\left[1\right]} \otimes  \hilbert_{0\left[2\right]}\right] \oplus \left[\hilbert_{1\left[1\right]} \otimes  \hilbert_{1\left[2\right]}\right] 
\end{split}
\end{align}
\noindent
If $Q_{0\left[i\right]} \propto 1$ for $i = 1,2$, then $\hilbert$ can be further simplified:
\begin{equation}
\hilbert = \hilbert_1 \oplus \hilbert_2,
\end{equation}
\noindent
which has reproducing kernel $Q = Q_{\left[1\right]}Q_{\left[2\right]}$.

\begin{example}{\textbf {Tensor product cubic spline}}\\
\vspace{0.5cm}
Let the marginal domains of $l$ and $m$ correspond to $\hilbert_1$ and $\hilbert_2$ respectively, where
\[
\hilbert_i = \mathcal{C}^{\left(m_i\right)} = \left\{ \phi: \int \limits_{0}^1 \phi^{\left(m_i\right)}\;dv < \infty  \right\},
\]
\noindent
which are equipped with inner product
\begin{align}
\begin{split}
\langle f,g\rangle &= \langle f,g\rangle_0 + \langle f,g\rangle_1\\
 &= \sum_{\nu=0}^{m_i-1}M_{\nu} f M_{\nu} g + \int_0^1 f^{\left( m_i \right)}\left(v\right)g^{\left( m_i \right)}\left(v\right)dv, \quad i = 1,2
\end{split}
\end{align}
\noindent
where the order $i$ differential operator $M_\nu$ is defined $M_\nu \phi = \int_0^1 \phi^{\left( m \right)}\left(v\right) dv\;,\;\; \nu = 1, \dots, m_i$, $i = 1,2$. Denote the norm corresponding to this inner product by

\[
\vert \vert f \vert \vert^2 = \left< f,f\right> = \left< f,f\right>_0 + \left< f,f\right>_1 = \vert \vert P_0 f \vert \vert^2 + \vert \vert P_1 f \vert \vert^2
\]
\noindent
The reproducing kernel $Q$ can be expressed in terms of the scaled Bernoulli polynomials $\left\{ k_j\left(v\right) = \frac{1}{j!}B_j\left(v\right) \right\}$, $v \in \left[0,1\right]$, where $B_j$ is defined according to:

\begin{align*}
B_0\left(x\right) &= 1\\
\frac{d}{dx} B_j\left(x\right) &= jB_{j-1}\left(x\right), \;j = 1, 2, \dots
\end{align*}
\noindent
One can verify that $\int \limits_0^1 k_\mu^\nu dv = \delta_{\mu,\nu}$ for $\nu, \mu= 0,\dots, m_i -1$, where $\delta_{\mu,\nu}$ is the Kronecker delta. This implies that the $k_\nu$, $\nu = 0,\dots, m_i-1$ for an orthonormal basis for $\hilbert_{0\left[i\right]} = \left\{ \phi: \phi^{\left( m_i \right)} = 0 \right\}$ under the inner product
\[
\langle f,g\rangle_0 =  \sum_{\nu=0}^{m_i-1}M_{\nu} f M_{\nu} g, \quad i = 1, 2, 
\]
\noindent
and that 
\[
Q_{0\left[i\right]}\left(v,v'\right) = \sum_{\nu=0}^{m_i-1}  k_\nu\left(v\right)  k_\nu\left(v'\right) 
\]
\noindent
is the reproducing kernel for $\hilbert_{0\left[i\right]}$. The subspaces of $\hilbert_{\left[i\right]}$ which are orthogonal to $\hilbert_{0\left[i\right]}$ are comprised of functions $\phi$ satisfying 
\[
\hilbert_{1\left[i\right]} = \lbrace \phi: M_\nu f = 0,\;\; \nu = 0,1,\dots, m_i-1,\int\limits_{0}^1 \phi^{\left(m_i\right)}\;dv < \infty \rbrace, \quad i = 1,2.
\]
One can show that the representer for the evaluation functional $\left[v\right] \phi$ in $\hilbert_{1\left[i\right]}$ with squared norm $\langle f,g\rangle_1= \int_0^1 f^{\left(m_i\right)}g^{\left(m_i\right)}\;dv$ is given by the function

\begin{equation}
{Q_{\left[i\right]} }_v'\left(v\right) = k_{m_i}\left(v\right)k_{m_i}\left(v'\right) + \left(-1\right)^{m_i-1}k_{2m_i}\left(v' - v\right)
\end{equation}

\noindent
See \cite{gu2002smoothing} Example 2.3.3 for proof. The tensor product smoothing spline results from letting $m_1 = m_2 = 2$, so that the marginal subspaces can be written

\begin{align} \label{eq:cubic-spline-hilbert-space}
\left\{ \phi: \phi'' \in \mathcal{L}_2\left[0,1\right] \right\} = &\left\{ \phi: \phi \propto 1 \right\} \oplus  \left\{ \phi: \phi \propto k_1 \right\} \oplus \left\{ \phi: \int_0^1 \phi dv = \int_0^1 \phi' dv = 0,\; \phi'' \in \mathcal{L}_2\left[0,1\right]  \right\} \\
&= \hilbert_{00} \oplus \hilbert_{01} \oplus \hilbert_1,
\end{align}
\noindent
where $ \hilbert_{01} \oplus \hilbert_1$ forms the contrast in a one-way ANOVA decomposition with averaging operator $\mathcal{A}\phi = \int_0^1 \phi\;dv$. The corresponding reproducing kernels are
\begin{align} \label{eq:cubic-spline-hilbert-space-rks}
Q_{00}\left(v,v'\right) &= 1\\
Q_{01}\left(v,v'\right) &= k_1\left(v\right)k_1\left(v'\right)\\
Q_{1}\left(v,v'\right) &= k_2\left(v\right)k_2\left(v'\right) - k_4\left(v-v'\right).
\end{align}
\noindent
The tensor product space can be constructed with nine tensor sum terms; the construction of the tensor product space from the terms of the tensor sum. The corresponding reproducing kernels and inner products are given in Table~\ref{table:tensor-product-cubic-spline-RKHS-table} and Table~\ref{table:tensor-product-cubic-spline-RK-table}, respectively.

\begin{table}[H]
\centering % used for centering table
\begin{tabular}{r|c|c|c|} % centered columns (4 columns)
\multicolumn{1}{c}{} & \multicolumn{1}{c}{	$\hilbert_{00\left[2\right]}$}	&	\multicolumn{1}{c}{$\hilbert_{01\left[2\right]}$}	&\multicolumn{1}{c}{ $\hilbert_{1\left[2\right]}$}\\ [1.5ex] 
\cline{2-4}  % inserts single horizontal line\\
$\hilbert_{00\left[1\right]}$		& $\hilbert_{00\left[1\right]}\otimes \hilbert_{00\left[2\right]}$ 	&	$\hilbert_{00\left[1\right]}	\otimes \hilbert_{01\left[2\right]} $	&	$\hilbert_{00\left[1\right]}	\otimes \hilbert_{1\left[2\right]}$   \\ [1.5ex] 
$\hilbert_{01\left[1\right]}$		& $\hilbert_{01\left[1\right]} \otimes \hilbert_{00\left[2\right]}$			& 	$\hilbert_{01\left[1\right]} \otimes \hilbert_{01\left[2\right]}$   &   $\hilbert_{01\left[1\right]} \otimes \hilbert_{1\left[2\right]}$\\ [1.5ex] 
 $\hilbert_{1\left[1\right]}$	& 	 $\hilbert_{1\left[1\right]} \otimes \hilbert_{00\left[2\right]}$	&	$\hilbert_{1\left[1\right]} \otimes \hilbert_{01\left[2\right]}$ 	&	$\hilbert_{1\left[1\right]} \otimes \hilbert_{1\left[2\right]}$ \\ [1.5ex] 
\cline{2-4}
\end{tabular}
\caption{\textit{Construction of the tensor product cubic spline subspace from marginal subspaces $\hilbert_{\left[1\right]}$, $\hilbert_{\left[2\right]}$}} % title of Table
\label{table:tensor-product-cubic-spline-RKHS-table}
\end{table}

\begin{landscape}
\begin{table}[H]
\caption{\textit{Tensor product cubic spline subspace reproducing kernels and inner products}} % title of Table
\centering % used for centering table
\begin{tabular}{lll} % centered columns (4 columns)
\hline \\
\hline %inserts double horizontal lines
Subspace 	& 		Reproducing kernel 		& 	Inner product \\
\hline % inserts single horizontal line
$\hilbert_{00\left[1\right]} \otimes \hilbert_{00\left[2\right]}$ & 	$1$								     & 	$\left( \int_0^1 \int_0^1 f \right) \left( \int_0^1 \int_0^1 g \right)$ \\ [1ex] 
$\hilbert_{01\left[1\right]} \otimes \hilbert_{00\left[2\right]} $& 	$k_1\left(l\right)k_1\left(l'\right)$						     & 	$\left( \int_0^1 \int_0^1 f'_{\left[1\right]} \right) \left( \int_0^1 \int_0^1 g'_{\left[1\right]} \right)$ \\ [1ex] 
$\hilbert_{01\left[1\right]} \otimes \hilbert_{01\left[2\right]}$ & 	$k_1\left(l\right)k_1\left(l'\right)k_1\left(m\right)k_1\left(m'\right)$ & $\left( \int_0^1 \int_0^1 f''_{\left[12\right]} \right) \left( \int_0^1 \int_0^1 g''_{\left[12\right]} \right)$ \\ [1ex] 
$\hilbert_{1\left[1\right]} \otimes \hilbert_{00\left[2\right]}$  	& 	$k_2\left(l\right)k_2\left(l'\right) - k_4\left(l - l'\right)$	      & $\int_0^1 \left( \int_0^1 f''_{\left[12\right]}\;dl' \right) \left(  \int_0^1 g''_{\left[12\right]} \;dl'\right)\;dl $\\ [1ex] 
$\hilbert_{1\left[1\right]} \otimes \hilbert_{01\left[2\right]}$ 	& 	$\left[k_2\left(l\right)k_2\left(l'\right) - k_4\left(l - l'\right)\right]k_1\left(m\right)k_1\left(m'\right)$ & $\int_0^1 \left( \int_0^1 f^{\left(3\right)}_{\left[112\right]}\;dl' \right) \left(  \int_0^1 g^{\left(3\right)}_{\left[112\right]} \;dl'\right)\;dl$ \\ [1ex]  
$\hilbert_{1\left[1\right]} \otimes \hilbert_{1\left[2\right]}$  		& $\left[k_2\left(l\right)k_2\left(l'\right) - k_4\left(l - l'\right)\right]\left[k_2\left(m\right)k_2\left(m'\right) - k_4\left(m - m'\right)\right]$ & $\int_0^1  \int_0^1 f^{\left(4\right)}_{\left[1122\right]}g^{\left(4\right)}_{\left[1122\right]}$ \\ [1ex]  
\hline %inserts single line
\end{tabular}
\label{table:tensor-product-cubic-spline-RK-table}
\end{table}
\end{landscape}
\end{example}

\bigskip

For $\bfv \in V$ where $V$ is a product domain, ANOVA decompositions can be characterized by 
\begin{equation}\label{eq:ssanova-decomposition-of-RKHS}
\hilbert = \bigoplus\limits_{\beta=0}^{g} \hilbert_\beta
\end{equation}
\noindent
and
\begin{equation}\label{eq:ssanova-decomposition-of-penalty}
J\left(\phi\right) = \sum_{\beta=0}^{g} \theta^{-1}_\beta J_\beta \left( \phi_\beta \right),
\end{equation}
\noindent
where $\phi_\beta \in \hilbert_\beta$, $J_\beta$ is the square norm in $\hilbert_\beta$, and $0 < \theta_\beta < \infty$. This gives 

\begin{align*}
\hilbert_0 &= \mathcal{N}_J \\
\hilbert_J &= \bigoplus\limits_{\beta=1}^{g} \hilbert_\beta, \mbox{ and} \\
Q &= \sum_{\beta=1}^g \theta_\beta Q_\beta,
\end{align*}
\noindent
where $Q_\beta$ is the RK in $\hilbert_\beta$. The $\left \{ \theta_\beta \right\}$ are additional smoothing parameters, which are implicit in notation to follow for the sake of concise demonstration. 


\bigskip
\noindent
Let $Y$ denote the vector of length $n_y= \sum_{i} M_i - N$  constructed by stacking the $N$ observed response vectors $Y_1,\dots, Y_N$ less their first element $y_{i1}$ one on top of each other:

\begin{align*}
Y &= \left( Y'_1, Y'_2, \dots, Y'_{N} \right)'\\
 &= \left( y_{12}, y_{13},\dots, y_{1,m_1}, \dots, y_{N,2}, y_{N,3},\dots, y_{N,m_N} \right)'
\end{align*}
\noindent
Define $X_i$ to be the $m_i \times \vert V \vert$ matrix containing the covariates necessary for regressing each measurement $y_{i2}, \dots, y_{i,m_i}$ on its predecessors as in model~\ref{eq:cholesky-regression-model-2}, and stack these on top of one another to obtain

\begin{equation} \label{eq:ar-design-matrix-1}
X = \begin{bmatrix}
X_1 \\
X_2\\
\vdots \\
X_N
\end{bmatrix},
\end{equation}
\noindent
which has dimension $n_y \times \vert V \vert$. Then the solution $\phi_\lambda$ minimizing \ref{eq:phi-penalized-sums-of-squares-RK-norm}  is the solution to the minimization problem

\begin{equation} \label{eq:ar-design-matrix-1}
\vert \vert D^{-1/2}\left( Y - X \left( Bd + Qc \right) \right) \vert \vert^2  + \lambda c^\prime Q c 
\end{equation}
\noindent
where the $\left(i,j\right)$ entry of the $\vert V \vert \times \vert V \vert$ matrix $Q$ is given by $\langle P_1 \xi_i,  P_1 \xi_j \rangle_\hilbert$. The $\vert V \vert \times d_0$ matrix $B$ has $i$-$\nu^{th}$ element equal to $\eta_\nu\left(\bfv_i\right)$, and we assume $B$ to be full column rank.  The diagonal matrix $D$ holds the $n_y \times n_y$  innovation variances $\sigma^2_{ijk}$. 

\bigskip

\begin{example}{Construction of $X_i$ with complete data} \label{example:construction-of-X}

\vspace{.3cm} 

Straightforward construction of the autoregressive design matrix $X_i$ is straight forward in the case that there are an equal number of measurements on each subject at a common set of measurement times $t_1,\dots, t_M$. When complete data are available for measurement times $t_1, \dots, t_M$, 

\begin{equation}
X_i =  \begin{bmatrix} 
y_{i, t_1} & 0 & 0 &0&& \dots & 0 \\
 0 & y_{i, t_1} &  y_{i, t_2}&0 &0& \dots & 0 \\
 \vdots &&&&&&\\
 0 & \dots &0 & \dots& y_{i,t_1} & \dots &  y_{i, t_{p-1}}
\end{bmatrix}
\end{equation}
\noindent
for all $i = 1,\dots, N$. Note that this design matrix specification does not require that measurement times be regularly spaced.  
\end{example}

\begin{example}{Construction of $X_i$ with incomplete data}

\vspace{.3cm} 

We demonstrate the construction of the autoregressive design matrices when subjects do not share a universal set of observation times for $N = 2$; the construction extends naturally for an arbitrary number of trajectories. Let subjects have corresponding sample sizes $m_1 = 4$, $m_2 = 4$, with measurements on subject 1 taken at $t_{11} = 0, t_{12} = 0.2, t_{13} = 0.5, t_{14} = 0.9$ and on subject 2 taken at $t_{21} = 0, t_{22} = 0.1, t_{23} = 0.5, t_{24} = 0.7$.  Then the unique within-subject pairs of observation times $\left(t,s\right)$ such that $0 \le s < t \le 1$ are

\begin{table}[H]
\centering
\begin{tabular}{l|r;{2pt/2pt}r;{2pt/2pt}r;{2pt/2pt}r;{2pt/2pt}r;{2pt/2pt}r;{2pt/2pt}r;{2pt/2pt}r;{2pt/2pt}r;{2pt/2pt}r;{2pt/2pt}r}
t & 0.1 & 0.2 & 0.5 & 0.5 & 0.5 & 0.7 & 0.7 & 0.7 & 0.9 & 0.9 & 0.9 \\ 
 s & 0.0 & 0.0 & 0.0 & 0.1 & 0.2 & 0.0 & 0.1 & 0.5 & 0.0 & 0.2 & 0.5 \\
\end{tabular}
\end{table}
\noindent
This gives that $V =  \left\{\bfv_{121},\dots, \bfv_{143}  \right\} \bigcup \left\{\bfv_{221},\dots, \bfv_{243}  \right\} = \left\{\bfv_1,\dots, \bfv_{11} \right\}$, where the distinct observed $v = \left(l, m\right)$ are 

\begin{table}[H]
\centering
\begin{tabular}{l|r;{2pt/2pt}r;{2pt/2pt}r;{2pt/2pt}r;{2pt/2pt}r;{2pt/2pt}r;{2pt/2pt}r;{2pt/2pt}r;{2pt/2pt}r;{2pt/2pt}r;{2pt/2pt}r}
l & 0.10 & 0.20 & 0.50 & 0.40 & 0.30 & 0.70 & 0.60 & 0.20 & 0.90 & 0.70 & 0.40 \\ 
  m & 0.05 & 0.10 & 0.25 & 0.30 & 0.35 & 0.35 & 0.40 & 0.60 & 0.45 & 0.55 & 0.70 \\ 
\end{tabular}
\end{table}
\noindent
Then a potential construction of the autoregressive design matrix for subject is given by:
\begin{equation}
X_1 =  \begin{bmatrix} 
0   & y_{1, 1}  &	0            &    0   &    0           & 0 & 0 & 0 & 0 & 0  \\
0   &	0  	      &	y_{1, 1}  &    0   & y_{1, 2}   &  0 & 0 & 0 & 0 & 0 \\
 0   &    0         & 0           &    0   &    0          & 0  & 0	&  y_{1, 1}    & y_{1, 2}& y_{1, 3} 
\end{bmatrix}
\end{equation}
\noindent
and similarly, for subject 2:

\begin{equation}
X_2 =  \begin{bmatrix} 
y_{2, 1}  & 	0  &	  0           &    0            &    0   & 0 & 0 & 0 & 0 & 0  \\
0   	      &  	0  &	y_{2, 1}  &    y_{2,2}   &    0   &  0 & 0 & 0 & 0 & 0 \\
 0   	      &        0  &    0           &    0            &  y_{2, 1}    & y_{2, 2}& y_{2, 3} &    0   & 0  & 0
\end{bmatrix}
\end{equation}
\end{example}

\subsubsection{Construction of the solution $\hat{\phi}$}

Differentiating $-2\ell_\phi + \lambda J\left(\phi\right)$ with respect to $c$ and $d$ and setting equal to zero, we have that 

\begin{align}
\frac{\partial}{\partial c}\left[-2\ell_\phi + \lambda J\left(\phi\right)\right] = Q X^\prime D^{-1}\left[ X\left(Bd + Qc\right) - Y  \right] + \lambda Qc &= 0 \nonumber \\
%\Longleftrightarrow    W^\prime D^{-1} W \left( Bd + Kc\right) + \lambda c &= W^\prime D^{-1} Y \\
\iff    X'D^{-1} X \bigg[ Bd + Qc \bigg] + \lambda c  &= X' D^{-1}Y \label{eq:normal-eq-1}
\end{align}

\begin{align}
\frac{\partial}{\partial d}\left[-2\ell_\phi + \lambda J\left(\phi\right)\right] = B^\prime X^\prime D^{-1}\left[ X\left(Bd + Qc\right) - Y  \right] &=0 \nonumber \\
%\Longleftrightarrow    W^\prime D^{-1} W \left( Bd + Kc\right) + \lambda c &= W^\prime D^{-1} Y \\
\iff   - \lambda B' c  &= 0  
\end{align}
\bigskip
\noindent
For fixed smoothing parameter, the solution $\phi$ is obtained by finding $c$ and $d$ which satisfy
\begin{align} 
Y &= X \bigg[ Bd + \left(Q  + \lambda \left(X^\prime D^{-1} X \right)^{-1} \right) c \bigg] \label{eq:ssanova-normal-eq-1} \\
B' c  &= 0  \label{eq:ssanova-normal-eq-2}
\end{align}
\noindent


Letting $\tildeY = D^{-1/2} Y$, $\tildeB = D^{-1/2} X B $, and $\tildeQ = D^{-1/2} X Q$, the penalized log likelihood \ref{eq:penalized-likelihood-vectorized} may be written

\begin{equation}\label{eq:penalized-loglik-tilde-vectorized}
-2\ell_\lambda \left(c, d \right) + \lambda J\left( \phi \right) = \bigg[ \tildeY - \tildeB d - \tildeQ c\bigg]'\bigg[ \tildeY - \tildeB d - \tildeQ c\bigg] + \lambda c'Qc.
\end{equation}
\noindent
Taking partial derivatives with respect to $d$ and $c$ and setting equal to zero yields normal equations 

\begin{align}
\begin{split}
\tildeB'\tildeB d + \tildeB'\tildeQ c &= \tildeB' \tildeY \\
\tildeQ'\tildeB d + \tildeQ'\tildeQ c + \lambda Q c &= \tildeQ' \tildeY, 
\end{split}
\end{align}

\noindent
Some algebra yields that this is equivalent to solving the system

\begin{equation} \label{eq:vectorized-normal-equations}
\begin{bmatrix}
\tildeB'\tildeB & \tildeB'\tildeQ \\
\tildeQ'\tildeB & \tildeQ'\tildeQ + \lambda Q\\
\end{bmatrix}
\begin{bmatrix}
d\\
c\\
\end{bmatrix}
= \begin{bmatrix}
\tildeB'\tildeY \\
 \tildeQ'\tildeY\\
\end{bmatrix}
\end{equation}


Fixing smoothing parameters $\lambda$ and $\theta_\beta$ (hidden in $Q$ and $\tildeQ$ if present), assuming that $\tildeQ$ is full column rank, \ref{eq:vectorized-normal-equations} can be solved by the Cholesky decomposition of the $\left( n + d_0 \right) \times \left( n + d_0 \right)$ matrix followed by forward and backward substitution. See \cite{golub2012matrix}. Singularity of $\tildeQ$ demands special consideration. Write the Cholesky decomposition

\begin{equation} \label{eq:normal-equation-cholesky}
\begin{bmatrix}
\tildeB'\tildeB & \tildeB'\tildeQ \\
\tildeQ'\tildeB & \tildeQ'\tildeQ + \lambda Q\\
\end{bmatrix}
= \begin{bmatrix}
C'_1 & 0 \\
C'_2  & C'_3 
\end{bmatrix}
\begin{bmatrix}
C_1 & C_2 \\
0  & C_3 
\end{bmatrix}
\end{equation}
\noindent
where $\tildeB'\tildeB = C'_1 C_1$, $C_2 = C_1^{-T} \tildeB' \tildeQ$, and $C'_3 C_3 = \lambda Q +  \tildeQ'\left( I - \tildeB\left( \tildeB' \tildeB \right)^{-1} \tildeB' \right)\tildeQ$. Using an exchange of indices known as pivoting, one may write 

\begin{equation*}
C_3 = \begin{bmatrix} H_1 & H_2 \\ 0 & 0 \end{bmatrix} = \begin{bmatrix} H \\  0 \end{bmatrix},
\end{equation*}
\noindent
where $H_1$ is nonsingular. Define
\begin{equation} \label{eq:cholesky-factor-mod}
\tilde{C}_3 = \begin{bmatrix}
H_1 & H_2 \\
0  & \delta I 
\end{bmatrix}, \;\;
\tilde{C} = \begin{bmatrix}
C_1 & C_2 \\
0  & \tilde{C}_3 
\end{bmatrix};
\end{equation}
\noindent
then
\begin{equation} \label{eq:cholesky-factor-mod-inverse}
\tilde{C}^{-1} = \begin{bmatrix}
C_1^{-1} & -C_1^{-1} C_2 \tilde{C}_3^{-1} \\
0  & \tilde{C}_3^{-1}
\end{bmatrix}.
\end{equation}

Premultiplying \ref{eq:normal-equation-cholesky} by $\tilde{C}^{-T}$, straightforward algebra gives 

\begin{equation} \label{eq:vectorized-normal-equations-cholesky}
\begin{bmatrix}
I & 0 \\
0 & \tilde{C}_3^{-T} C_3^{T} C_3 \tilde{C}_3^{-1}\\
\end{bmatrix}
\begin{bmatrix}
\tilde{d}\\
\tilde{c}\\
\end{bmatrix}
= \begin{bmatrix}
C_1^{-T} \tildeB'\tildeY \\
\tilde{C}_3^{-T} \tildeQ'\left( I - \tildeB\left( \tildeB' \tildeB \right)^{-1} \tildeB' \right) \tildeY\\
\end{bmatrix}
\end{equation}
\noindent
where $\left( \tilde{d}'\;\;\tilde{c}' \right)' =  \tilde{C}' \left( d\;\;c \right)'$. Partition $\tilde{C}_3 = \begin{bmatrix} K &  L\end{bmatrix}$; then $HK = I$ and $HL = 0$. So

\begin{align*}
\tilde{C}_3^{-T} C_3^{T} C_3 \tilde{C}_3^{-1} &= \begin{bmatrix} K' \\ L' \end{bmatrix} C'_3C_3 \begin{bmatrix} K &  L\end{bmatrix} \\
&= \begin{bmatrix} K' \\ L' \end{bmatrix} H'H \begin{bmatrix} K &  L\end{bmatrix} \\
&= \begin{bmatrix} I & 0 \\ 0 & 0 \end{bmatrix}.
\end{align*}
\noindent
If $L'C_3^{T} C_3 L = 0$, then $L'\tildeQ'\left( I - \tildeB\left( \tildeB' \tildeB \right)^{-1} \tildeB' \right)\tildeQ L = 0$, so $L'\tildeQ'\left( I - \tildeB\left( \tildeB' \tildeB \right)^{-1} \tildeB' \right) \tildeY = 0$. Thus, the linear system has form

\begin{equation} \label{eq:vectorized-normal-equations-cholesky-2}
\begin{bmatrix}
I & 0 & 0\\
0 & I & 0 \\
0 & 0 & 0 \\
\end{bmatrix}
\begin{bmatrix}
\tilde{d}\\
\tilde{c}_1\\
\tilde{c}_2
\end{bmatrix}
= \begin{bmatrix}
* \\
* \\
0
\end{bmatrix},
\end{equation}
\noindent
which can be solved, but with $c_2$ arbitrary. One may perform the Cholesky decomposition of \ref{eq:vectorized-normal-equations} with pivoting, replace the trailing $0$ with $\delta I$ for appropriate value of $\delta$, and proceed as if $\tildeQ$ were of full rank. 
\bigskip

It follows that

\begin{equation} \label{eq:tildeY-hat-equals-tildeA-tildeY}
\widehat{\tildeY} = \tildeB d + \tildeQ c = \begin{bmatrix} \tildeB & \tildeQ \end{bmatrix} \tilde{C}^{-1} \tilde{C}^{-T} \begin{bmatrix} \tildeB' \\ \tildeQ' \end{bmatrix} \tildeY = \tildeA_{\lambda,\bftheta} \tildeY.
\end{equation} 
\noindent
where
\begin{align}
\begin{split} \label{eq:smoothing-matrix-A-tilde}
\tildeA_{\lambda,\bftheta} =& \begin{bmatrix} \tildeB & \tildeQ \end{bmatrix} \tilde{C}^{-1} \tilde{C}^{-T} \begin{bmatrix} \tildeB' \\ \tildeQ' \end{bmatrix}  \\
&= G + \left(I - G\right) \tildeQ \left[\tildeQ'\left( I - G \right)\tildeQ + \lambda Q\right]^{-1} \tildeQ'\left(I - G\right),
\end{split}
\end{align} 
\noindent
for
\[
G = \tildeB\left(\tildeB' \tildeB \right)^{-1}\tildeB'.
\]



%%%%%%%%%%%%%%%%%%%%%%%%%%%%%%%%%%%%%%%%%%%%%%%%%%%%%%%%%%%%%%%%%%%%%%%%%%%%%%%%%%%%%%%%%%%
%%%%%%%%%%%%%%%%%%%%%%%%%%%%%%%%%%%%%%%%%%%%%%%%%%%%%%%%%%%%%%%%%%%%%%%%%%%%%%%%%%%%%%%%%%%
%%%%%%%%%%%%%%%%%%%%%%%%%%%%%%%%%%%%%%%%%%%%%%%%%%%%%%%%%%%%%%%%%%%%%%%%%%%%%%%%%%%%%%%%%%%
%%%%%%%%%%%%%%%%%%%%%%%%%%%%%%%%%%%%%%%%%%%%%%%%%%%%%%%%%%%%%%%%%%%%%%%%%%%%%%%%%%%%%%%%%%%
%%%%%%%%%%%%%%%%%%%%%%%%%%%%%%%%%%%%%%%%%%%%%%%%%%%%%%%%%%%%%%%%%%%%%%%%%%%%%%%%%%%%%%%%%%%
%%%%%%%%%%%%%%%%%%%%%%%%%%%%%%%%%%%%%%%%%%%%%%%%%%%%%%%%%%%%%%%%%%%%%%%%%%%%%%%%%%%%%%%%%%%
%%%%%%%%%%%%%%%%%%%%%%%%%%%%%%%%%%%%%%%%%%%%%%%%%%%%%%%%%%%%%%%%%%%%%%%%%%%%%%%%%%%%%%%%%%%



%%%%%%%%%%%%%%%%%%%%%%%%%%%%%%%%%%%%%%%%%%%%%%%%%%%%%%%%%%%%%%%%%%%%%%%%%%%%%%%%%%%%%%%%%%%
%%%%%%%%%%%%%%%%%%%%%%%%%%%%%%%%%%%%%%%%%%%%%%%%%%%%%%%%%%%%%%%%%%%%%%%%%%%%%%%%%%%%%%%%%%%
%%%%%%%%%%%%%%%%%%%%%%%%%%%%%%%%%%%%%%%%%%%%%%%%%%%%%%%%%%%%%%%%%%%%%%%%%%%%%%%%%%%%%%%%%%%
%%%%%%%%%%%%%%%%%%%%%%%%%%%%%%%%%%%%%%%%%%%%%%%%%%%%%%%%%%%%%%%%%%%%%%%%%%%%%%%%%%%%%%%%%%%
%%%%%%%%%%%%%%%%%%%%%%%%%%%%%%%%%%%%%%%%%%%%%%%%%%%%%%%%%%%%%%%%%%%%%%%%%%%%%%%%%%%%%%%%%%%
%%%%%%%%%%%%%%%%%%%%%%%%%%%%%%%%%%%%%%%%%%%%%%%%%%%%%%%%%%%%%%%%%%%%%%%%%%%%%%%%%%%%%%%%%%%
%%%%%%%%%%%%%%%%%%%%%%%%%%%%%%%%%%%%%%%%%%%%%%%%%%%%%%%%%%%%%%%%%%%%%%%%%%%%%%%%%%%%%%%%%%%
\section{Smoothing parameter selection} \label{SSANOVA-smoothing-parameter-selection}
%\subfile{chapter-2-subfiles/chapter-2-smoothing-spline-model-selection}
%\subfile{chapter-2-subfiles/chapter-2-smoothing-spline-solution}

By varying smoothing parameters $\lambda$ and $\theta_\beta$, the minimizer $\phi_\lambda$ of \ref{eq:vectorized-normal-equations} defines a family of potential estimates. In practice, we need to choose a specific estimate from the family, which requires effective methods for smoothing parameter selection. We consider two criteria that are commonly used for smoothing parameter selection in the context of smoothing spline models for longitudinal data. The first score is an unbiased estimate of a relative loss and assumes a known variances $\sigma_t^2$. The unbiased risk estimate has attractive asymptotic properties; see \cite{gu2013smoothing} for a comprehensive examination. The second score, the leave-one-subject-out cross validation (LosoCV) score, provides an estimate of the same loss without assuming a known variance function. We review a computationally convenient approximation of the LosoCV score proposed by \cite{xu2012asymptotic}, who demonstrates the shortcut score's asymptotic optimality. To simplify notation for the initial presentation, we only make explicit the dependence of estimates and their components on $\lambda$ and conceal any dependence on $\theta_\beta$. 


\subsubsection{Unbiased risk estimate}

Define  $\tildeY = D^{-1/2} Y$, $\tildeB = D^{-1/2} X B $, and $\tildeQ = D^{-1/2} X Q$ as before. Let $\tildeepsilon = D^{-1/2} \epsilon$ denote the vector of length  $\sum_{i = 1}^Nm_i - N$ containing the standardized prediction errors $\epsilon_{ij} \sim N\left(0,1\right)$, and write the vector of transformed means 

\begin{equation} 
\Phi = D^{-1/2} X \left[ Bd + Qc \right].
\end{equation}
\noindent
We can assess $\hat{\tildeY}_\lambda$, an estimate of the mean of $\tildeY$ based on observed data $y_{ij}$, $i = 1,\dots, N$, $j = 1,\dots, m_i$, using the loss function

\begin{align}
\begin{split}
L\left(\lambda\right) &= \sum_{i = 1}^N \sum_{j = 1}^{m_i} \left(\hat{\tildey}_{ij} - E\left[\tildey_{ij}\right] \right)^2\\
&= \vert \vert \tildeY - \tilde{\mu} \vert \vert^2
\end{split}
\end{align}
\noindent
where $\mu = D^{-1/2}W \Phi^*$ denotes the $\left( \sum \limits_{i} m_i - N\right) \times 1$ with $i^{th}$ element equal to the expected value of the  $i^{th}$ element of $\tildeY$.  Then straightforward algebra yields that 

\begin{align} 
L\left(\lambda\right) = \mu'\left( I - \tildeA_{\lambda,\bftheta} \right)^2\mu - 2\mu'\left( I - \tildeA_{\lambda,\bftheta} \right)^2 \tildeA_{\lambda,\bftheta} \tildeepsilon + \tildeepsilon' \tildeA_{\lambda,\bftheta}^2 \tildeepsilon
\end{align}

Define the unbiased risk estimate
\begin{equation} 
U\left(\lambda\right) = \frac{1}{N}\tildeY'\left( I - \tildeA_{\lambda,\bftheta} \right)^2\tildeY + \frac{2}{N}\mbox{tr}\tildeA_{\lambda,\bftheta}
\end{equation}
 \noindent
Adding and subtracting $\mu$ to the quadratic terms, one can verify with straightforward algebra that

\begin{align}
\begin{split}
U\left(\lambda\right) &= \left( \tildeY - \mu + \mu - \tildeA_{\lambda,\bftheta} \tildeY \right)'\left( \tildeY - \mu + \mu - \tildeA_{\lambda,\bftheta} \tildeY \right) + 2\mbox{tr}\tildeA \\
&= \left(\tildeA \tildeY - \mu \right)'\left( \tildeA \tildeY - \mu \right) + \tildeepsilon'\tildeepsilon + 2\tildeepsilon' \left( I- \tildeA\right)\mu- 2\left( \tildeepsilon'\tildeA \tildeepsilon -  \mbox{tr}\tildeA\right)
\end{split}
\end{align}
\noindent
This gives
\begin{equation} 
U\left(\lambda\right) - L\left(\lambda\right) - \tildeepsilon'\tildeepsilon  =  2\tildeepsilon' \left( I- \tildeA\right)\mu- 2\left( \tildeepsilon'\tildeA \tildeepsilon -  \mbox{tr}\tildeA\right), 
\end{equation}
 \noindent
 which allows one to easily see that $U\left(\lambda\right)$ is unbiased for the relative loss $L\left(\lambda\right) + \tildeepsilon'\tildeepsilon$.  Under mild conditions on the risk function
 
 \[
 R\left(\lambda\right) = E\left[L\left(\lambda\right)\right],
 \]
\noindent
one can establish that $U$ is also a consistent estimator. See \cite{gu2013smoothing}, Chapter 3 for a formal theorem and proof.


\subsubsection{Leave-one-subject-out cross validation}  
The conditions under which the the cross validation and generalized cross validation scores traditionally used for smoothing parameter selection yield desirable properties generally do not hold when the data are clustered or longitudinal in nature. Instead, the leave-one-subject-out (LosoCV) cross validation score has been widely used for smoothing parameter selection for semiparametric and nonparametric models for longitudinal or functional data. The LosoCV criterion is defined as

\begin{equation} \label{eq:LOSOCV}
V_{loso}\left(\lambda\right) = \frac{1}{N}\sum_{i=1}^N \left( \tildeY_i - \widehat{\tilde{\mu}}^{\left[-i\right]}_{i}\right)'\left( \tildeY_i -  \widehat{\tilde{\mu}}^{\left[-i\right]}_{i}\right)
\end{equation}
\noindent
where $\widehat{\tilde{\mu}}^{\left[-i\right]}_{i}$ is the estimate of $E\left[ \tildeY_i \right]$ based on the data when $\tildeY_i$ is omitted. Intuitively, the LosoCV score is appealing because it preserves any within-subject dependence by leaving out all observations from the same subject together in the cross-validation.  However, despite its prevalent use, theoretical justifications for its use have not been established. In their seminal work, \cite{rice1991estimating} were the first to present a heuristic justification of LosoCV by demonstrating that it mimics the mean squared prediction error: consider new observations $\tildeY^*_i = \left(\tilde{y}_{i1}^*, \tilde{y}_{i1}^*, \dots, \tilde{y}_{i, m_i}^*\right)$. We may write the mean squared prediction error for the new observations as follows:  
\bigskip 

\begin{align}
\begin{split}\label{eq:MSPE}
MSPE &= \frac{1}{N}\sum_{i=1}^N E\left[ \vert \vert \tildeY^*_i - \widehat{\tilde{\mu}}_{i} \vert \vert^2 \right]\\
&=  \frac{1}{N}\sum_{i=1}^N E\left[ \vert \vert \tildeY^*_i - D_i^{-1/2}W_i \Phi^* + D_i^{-1/2}W_i \Phi^* - D_i^{-1/2}W_i \hat{\Phi}^*\vert \vert^2 \right]\\
&=  \frac{1}{N}\sum_{i=1}^N \left\{m_i + E\left[ \vert \vert \tilde{\mu}_{i} - \widehat{\tilde{\mu}}^{\left[ -i \right]}_{i} \vert \vert^2 \right] \right\}
\end{split}
\end{align}
\noindent
where $\tilde{\epsilon}_i = \tildeY^*_i - D_i^{-1/2}W_i \Phi^*$. When $\left\{ \sigma^2\left(t\right)\right\}$ is known, $\tilde{\epsilon}_i$ is a mean zero multivariate normal vector with $Cov\left(\tilde{\epsilon}_i\right) = I_{m_i}$, which gives the last equality. Since $\tildeY_i$ and $ \widehat{\tilde{\mu}}_{i} $ are independent, the expected LosoCV score can be written
\begin{equation} \label{eq:MSPE_LOSOCV}
E\left[V_{loso}\left(\lambda\right) \right] =  \frac{1}{N}\sum_{i=1}^N\left\{ m_i +  E\left[ \vert \vert \widehat{\tilde{\mu}}_{i} - \tilde{\mu}_{i} \vert \vert^2 \right] \right\}. 
\end{equation}
\noindent
When $N$ is large, we expect that $\widehat{\tilde{\mu}}_{i}$ should be close to $\widehat{\tilde{\mu}}^{\left[ -i \right]}_{i}$, so $E\left[V_{loso}\left(\lambda\right) \right]$ should be a good approximation to the mean-squared prediction error. For a formal proof of consistency, see \cite{xu2012asymptotic}.


\bigskip

The definition of $V_{loso}$ would lead one to initially believe that calculation of the score requires solving $N$ separate minimization problems, however, \cite{xu2012asymptotic} established a computational shortcut that requires solving only one minimization problem that involves all data. 
%  \subsubsection{Computation of the LosoCV score}
  
  \begin{lemma}[Shortcut formula for LosoCV] \label{lemma:losocv-shortcut}
  The LosoCV score satisfies the following identity:
  \begin{equation*}
 V_{loso}\left( \lambda \right) = \frac{1}{N} \sum_{i = 1}^N \left(\tildeY_i - \widehat{\tildeY_i}\right)' \left(I_{ii} - \tildeA_{ii}\right)^{-T}\left(I_{ii} - \tildeA_{ii}\right)^{-1}\left(\tildeY_i - \widehat{\tildeY_i}\right),
  \end{equation*}
  \noindent
  where $\tildeA_{ii}$ is the diagonal block of smoothing matrix $\tildeA_{\lambda,\bftheta}$ corresponding to the observations on subject $i$, and $I_{ii}$ is a $m_i \times m_i$ identity matrix.
\end{lemma}

A detailed presentation and proof can be found in \cite{xu2012asymptotic} and supplementary materials \cite{xuasymptotic}.  The authors additionally proposed an approximation to the LosoCV score to further reduce the computational cost of evaluating $V_{loso}$, which can be expensive due to the inversion of the $I_{ii} - \tildeA_{ii}$. Using the Taylor expansion of $\left(I_{ii} - \tildeA_{ii}\right)^{-1} \approx I_{ii} + \tildeA_{ii}$, we can use the following to approximate $V_{loso}$:

\begin{equation} \label{eq:approx-losocv}
V_{loso}^*\left( \lambda \right) = \frac{1}{N} \vert \vert \left(I - \tildeA_{\lambda,\bftheta}\right)\tildeY \vert \vert^2 + \frac{2}{N} \sum_{i = 1}^N \hat{\tilde{e}}'_{i}\tildeA_{ii}\hat{\tilde{e}}_i,
\end{equation}

\noindent
where $\hat{\tilde{e}}_i$ is the portion of the vector of prediction errors $\left(I - \tildeA_{\lambda,\bftheta}\right)\tildeY$ corresponding to subject $i$. They show that under mild conditions, and for fixed, nonrandom $\lambda$, the approximate LosoCV score $V_{loso}^*$ and the true LosoCV score $V_{loso}$ are asymptotically equivalent. See Theorem 3.1 of \cite{xu2012asymptotic}.
  
\vspace{0.8in} 


\subsubsection{Selection of multiple smoothing parameters}

With the definition of the unbiased risk estimate and the leave-one-subject-out criteria, the expression of the smoothing matrix in Equation~\ref{eq:smoothing-matrix-A-tilde} permits the straightforward evaluation of both scores $U\left(\lambda, \bftheta \right)$ and $V_{loso}^*\left(\lambda, \bftheta \right)$, where $\bftheta = \left(\theta_1,\dots, \theta_g\right)'$ denotes the vector of smoothing parameters associated with each RK.  In this section, we discuss a algorithm to minimize the unbiased risk estimate $U\left(\lambda, \bftheta\right)$ with respect to $\lambda$ and $\bftheta$ hidden in $Q = \sum_{\beta = 1}^q \theta_\beta Q_\beta$, where the $\left(i,j\right)$ entry of $Q_\beta$ is given by $R_\beta\left(\bfv_i,\bfv_j\right)$.  We present minimization of the unbiased risk estimate explicitly, but the mechanics of the optimization are very similar to those necessary for optimizing the leave-one-subject-out cross validation criterion. The details of a procedue for explicitly minimizing the alternative criterion are presented in \cite{xu2012asymptotic}, which is based on the algorithms of \cite{gu1991minimizing}, \cite{kim2004smoothing} (which is the basis for the algorithm which follows) and \cite{wood2004stable}. The key difference between the minimization of $U$ and the minimization of $V^*_{loso}$ lies in the calculation of the gradient and the Hessian matrix in the Newton update. To minimize the unbiased risk estimate,

\begin{enumerate}
\item Fix $\bftheta$; minimize $U\left(\lambda \vert \bftheta\right)$ with respect to $\lambda$.
\item Update $\bftheta$ using the current estimate of $\lambda$.
\end{enumerate}

\noindent
Executing step 1 follows immediately from the expression for the smoothing matrix. Step 2 requires evaluating the gradient and the Hessian of $U\left( \bftheta \vert \lambda \right)$ with respect to $\bfkappa = \log\left(\bftheta\right)$. Optimizing with respect to $\bfkappa$ rather than on the original scale is motivated by two driving factors: first, $\bfkappa$ is invariant to scale transformations. With examination of $U$ and $V^*$ and \ref{eq:smoothing-matrix-A-tilde}, it is immediate that the $\theta_\beta \tildeQ_\beta$ are what matter in determining the minimum. Multiplying the $\tildeQ_\beta$ by any positive constant leaves the $\theta_\beta$ subject to rescaling, though the problem itself is unchanged by scale transformations. The derivatives of $U\left(\cdot\right)$ with respect to $\bfkappa$ are invariant to such transformations, while the derivatives with respect to $\bftheta$ are not. In addition, optimizing with respect to $\bfkappa$ converts a constrained optimization ($\theta_\beta \ge 0$) problem to an unconstrained one.

\subsubsection{Algorithms}

The following presents the main algorithm for minimizing $U\left(\lambda, \bftheta \right)$ and its key components are presented in the section to follow. The minimization of $U$ is done via two nested loops. Fixing tuning parameters, the outer loop minimizes $U$ with respect to smoothing parameters via quasi-Newton iteration of \cite{dennis1996numerical}, as implemented in the \texttt{nlm} function in \texttt{R}. The inner loop then minimizes $\ell_\lambda$ with fixed tuning parameters via Newton iteration. Fixing the $\theta_\beta$s in $J \left(\phi^*\right) = \sum_\beta \theta^{-1}_\beta J_\beta \left(\phi_\beta^*\right)$, the outer loop with a single $\lambda$ is straightforward. 

\begin{algorithm}[H]
\caption{ }
\begin{algorithmic}
\STATE \textbf{Initialization:} 
	\STATE Set $\Delta \bfkappa := 0$; \;$\bfkappa_{-}:=\bfkappa_{0}$; \;$V_- = \infty$; \;( or $M_- = \infty$)

\STATE \textbf{Iteration:} 
	\WHILE{not converged}
		\STATE For current value $\bfkappa^* = \bfkappa_- + \Delta \bfkappa$, compute $Q^*_\theta = \sum_{\beta = 1}^g \theta^*_\beta Q_\beta$ and scale so that $\mbox{tr}\left(Q_\beta\right)$ is fixed. 
		\STATE Compute $\tildeA_{\lambda,\bftheta}\left(\lambda \vert \bftheta^* \right) = \tildeA_{\lambda,\bftheta}\left(\lambda, \exp\left({\bfkappa^*} \right)\right)$.
		\STATE Minimize $U\left(\lambda \vert \bfkappa^* \right) =  \tildeY'\left( I - \tildeA_{\lambda,\bftheta} \right)^2\tildeY + 2\mbox{tr}\tildeA_{\lambda,\bftheta} $
		\STATE Set $U_* := \min \limits_\lambda Y\left( \lambda \vert \bfkappa^* \right) $
		\IF{$U^* > U_-$ }
		 		\STATE Set $\Delta \bfkappa := \Delta \bfkappa/2$
		 		\STATE Go to (1).
		\ELSE
		\STATE Continue
		\ENDIF
		\STATE Evaluate gradient $\mathbf{g} = \left(\partial /\partial \bfkappa\right) U\left(\bfkappa \vert \lambda\right)$
		\STATE Evaluate Hessian $H = \left(\partial^2 /\partial \bfkappa\partial \bfkappa' \right) U\left(\bfkappa \vert \lambda\right)$.
		\STATE Calculate step $\Delta \bfkappa$:
			\IF{$H$ positive definite}  
				\STATE $\Delta \bfkappa := -H^{-1} \mathbf{g}$
			\ELSE
				\STATE $\Delta \bfkappa := -\tilde{H}^{-1} \mathbf{g}$, where $\tilde{H} = \textup{diag}\left(\bfeps\right)$ is positive definite. \label{ensure-hessian-PD}
			\ENDIF
	\ENDWHILE
\STATE \textbf{Calculate optimal model:} 
	\IF{$\Delta \kappa_\beta < -\gamma$, for $\gamma$ large}
		\STATE Set $\kappa_{*\beta} := -\infty$
	\ENDIF
	\STATE Compute $Q^*_\theta = \sum_{\beta = 1}^g \theta^*{\beta} Q_\beta$;
	\STATE Calculate $\begin{bmatrix} d \\ c \end{bmatrix} = \tilde{C}^{-1} \tilde{C}^{-T} \begin{bmatrix} \tildeB' \\ {\tildeQ_*^\theta}' \end{bmatrix} \tildeY$
\end{algorithmic}
\end{algorithm}

Calculation of the gradient $\bfg$ and Hessian $H$ mirror the details in \cite{gu1991minimizing}, replacing the null basis matrix $B$ and representer matrix $Q$ with $D^{-1}XB$ and $D^{-1}XB$, respectively. They also present details on convergence criteria based on those suggested in \cite{gill1981practical}, who also present detailed discussion of the Newton method based on the Cholesky decomposition necessary for calculating the update direction for $\bfkappa$. The step in \ref{ensure-hessian-PD} returns a descent direction even when $H$ is not positive definite by adding positive mass to the diagonal elements of $H$ if necessary to produce $\tilde{H} = G'G$ where $G$ is upper triangular. See \cite{gill1981practical} 4.4.2.2 for details. 
\bigskip

The unbiased risk estimate $U\left(\lambda, \bftheta\right)$ is fully parameterized by 

\begin{equation}
\left(\lambda_1, \dots, \lambda_q\right) = \left(\lambda \theta^{-1}_1, \dots, \lambda \theta^{-1}_q\right),
\end{equation}
\noindent
so the smoothing parameters $\left(\lambda, \theta_1, \dots, \theta_q\right)$ over-parameterize the score, which is the reason for scaling the trace of $Q_\beta$. The starting values for the $\theta$ quasi-Newton iteration are obtained with two passes of the fixed-$\theta$ outer loop as follows:

\begin{enumerate}
\item Set $\breve{\theta}_\beta^{-1} \propto \mbox{tr}\left( \tildeQ_\beta \right)$, minimize $U\left(\lambda\right)$ with respect to $\lambda$ to obtain $\breve{\phi}$. \label{theta-starting-values-1}
\item Set $\check{\theta}_\beta^{-1} \propto  J_\beta\left(\breve{\phi}_\beta \right)$, minimize $U\left(\lambda\right)$ with respect to $\lambda$ to obtain $\check{\phi}$. \label{theta-starting-values-2}
\end{enumerate}
\noindent
The first pass allows equal opportunity for each penalty to contribute to the GCV score, allowing for arbitrary scaling of $J_\beta \left(\phi_\beta\right)$. The second pass grants greater allowance to terms exhibiting strength in the first pass. The following $\theta$ iteration fixes $\lambda$ and starts from $\check{\theta}_\beta$. These are the starting values adopted by \cite{gu1991minimizing}; the starting values for the first pass loop are arbitrary, but are invariant to scalings of the $\theta_\beta$. The starting values in \ref{theta-starting-values-2} for the second pass of the outer are based on more involved assumptions derived from the background formulation of the smoothing problem: the penalty is of the form

\[
J\left(\right)= \sum_{\beta = 1}^q \theta^{-1}_\beta \langle \phi, \phi\rangle_\beta
\]
\noindent
After the first pass, the initial fit $\breve{\phi}$ reveals where the structure in the true $\phi$ lie in terms of the components of the subspaces $\hilbert_\beta$. Less penalty should be applied to terms exhibiting strong signal.  


%%%%%%%%%%%%%%%%%%%%%%%%%%%%%%%%%%%%%%%%%%%%%%%%%%%%%%%%%%%%%%%%%%%%%%%%%%%%%%%%%%%%%%%%%%%
%%%%%%%%%%%%%%%%%%%%%%%%%%%%%%%%%%%%%%%%%%%%%%%%%%%%%%%%%%%%%%%%%%%%%%%%%%%%%%%%%%%%%%%%%%%
%%%%%%%%%%%%%%%%%%%%%%%%%%%%%%%%%%%%%%%%%%%%%%%%%%%%%%%%%%%%%%%%%%%%%%%%%%%%%%%%%%%%%%%%%%%
%%%%%%%%%%%%%%%%%%%%%%%%%%%%%%%%%%%%%%%%%%%%%%%%%%%%%%%%%%%%%%%%%%%%%%%%%%%%%%%%%%%%%%%%%%%
%%%%%%%%%%%%%%%%%%%%%%%%%%%%%%%%%%%%%%%%%%%%%%%%%%%%%%%%%%%%%%%%%%%%%%%%%%%%%%%%%%%%%%%%%%%
%%%%%%%%%%%%%%%%%%%%%%%%%%%%%%%%%%%%%%%%%%%%%%%%%%%%%%%%%%%%%%%%%%%%%%%%%%%%%%%%%%%%%%%%%%%
%%%%%%%%%%%%%%%%%%%%%%%%%%%%%%%%%%%%%%%%%%%%%%%%%%%%%%%%%%%%%%%%%%%%%%%%%%%%%%%%%%%%%%%%%%%
\section{A smoothing spline model for the innovation variances}
%\subfile{chapter-2-subfiles/chapter-2-iv-smoothing-spline-representation}

Once we have an initial estimate of the generalized autoregressive coefficient function, $\phi$, we can use the model residuals to estimate the innovation variance function $\sigma^2\left(t\right)$. We use the same estimation approach as outlined in Section~\ref{RKHS-framework-for-phi}. Fixing $\phi = \phi^*$ for given estimate $\phi^*$, the negative log likelihood of the data $Y_1,\dots, Y_N$ is satisfies

\begin{equation} \label{eq:penalized-joint-loglik-given-phi-2}
-\ell\left( Y_1,\dots, Y_N, \phi, \sigma^2 \right) =  \frac{1}{2}\sum_{i = 1}^N \sum_{j = 1}^{m_i} \log \sigma^2_{ij}  + \frac{1}{2}\sum_{i = 1}^N \sum_{j = 1}^{m_i} \frac {\epsilon_{ij}^2}{\sigma^2_{ij}};
\end{equation}
\noindent
where $\epsilon_{ij} =  y_{ij} - \sum_{k<j} \phi^*_{ijk} y_{ik}$. Let 

\begin{equation}
\mbox{RSS}\left( t \right) = \sum_{i,j:t_{ij}= t} \left( y_{ij} - \sum_{k<j} \phi_{ijk} y_{ik}\right)^2
\end{equation}
\noindent
denote the squared residuals for the observations $y_{ij}$ having corresponding measurement time $t_{ij} - t$. Then $\mbox{RSS}\left( t \right)/\sigma^2\left(t\right) \sim \chi^2_{df_t}$, where the degrees of freedom $df_{t}$ corresponds to the number of observations $y_{ij}$ having corresponding measurement time $t$. In this light, for fixed $\phi$, the penalized likelihood \ref{eq:penalized-joint-loglik-given-phi-2} is that of a variance model with the $\epsilon_{ij}^2$ serving as the response.  This corresponds to a generalized linear model with gamma errors and known scale parameter equal to 2. Let $z_{ij} = \epsilon_{ij}^2$, and let $Z_{i} = \left(z_{i1},z_{i,m_i} \right)'$ denote the vector of residuals for the $i^{th}$ observed trajectory. The Gamma distribution is parameterized by shape parameter$\alpha$ and scale parameter $\beta$, where the mean of the distribution given by $\mu = \alpha \beta$. Reparameterizing the Gamma likelihood in terms of $\left(\alpha, \mu \right)$ and dropping terms that don't involve $\mu\left(\cdot\right)$ gives  
\begin{align}
-\ell\left(z,\mu, \alpha \right) &\propto \alpha\left[\frac{z}{\mu} + \log \mu\right]  \label{eq:gamma-iv-likelihood} \\ 
&= \alpha\left[ze^{-\eta} + \eta\right],\label{eq:gamma-iv-likelihood-canonical-link}
\end{align}
\noindent
where $\alpha^{-1}$ is the dispersion parameter and $\eta = \log \mu$. Letting $\mu_{ij}$ denote $E\left[ z_{ij} \right] = \sigma_{ij}^2$, the log likelihood of the working residuals becomes 

\begin{equation} \label{eq:penalized-joint-loglik-given-phi-3}
-\ell\left( Z_1,\dots, Z_N, \phi, \sigma^2 \right) =  \sum_{i = 1}^N \sum_{j = 1}^{m_i} \log \mu_{ij}  + \sum_{i = 1}^N \sum_{j = 1}^{m_i} \frac {z_{ij}}{\mu_{ij}},
\end{equation}
\noindent
which we can see coincides with a Gamma dsitribution with scale parameter $\alpha = 2$. Smoothing spline ANOVA models for exponential families have been studied extensively (\cite{wahba1995smoothing}, \cite{wang1997grkpack}, \cite{gu2013smoothing}). Parallel to the penalized sums of squares for $\phi$ (\ref{eq:penalized-least-squares-2}), we can append a smoothness penalty to obtain the penalized likelihood for $\eta\left(t\right) = \log\sigma^2\left(t\right)$:

\begin{equation} \label{eq:penalized-joint-loglik-given-phi-3}
-\ell\left( Z_1,\dots, Z_N, \phi, \sigma^2 \right) + =  \sum_{i = 1}^N \sum_{j = 1}^{m_i} \eta_{ij}  + \sum_{i = 1}^N \sum_{j = 1}^{m_i} z_{ij} e^{-\eta_{ij}} + \lambda J\left(\eta\right),  
\end{equation}
noindent
for $\eta \in \hilbert = \oplus_{\beta = 0}^q \hilbert_\beta$, where the penalty $J$ can be written as a square norm and decomposed as in (\ref{eq:ssanova-decomposition-of-penalty}), with

\begin{equation*} 
J\left(\kappa \right) = \langle \eta,\eta \rangle = \sum_{\beta = 1}^q \theta_\beta^{-1}\langle \eta,\eta \rangle_{\beta}.
\end{equation*}
\noindent 
The $\langle \cdot, \cdot \rangle_{\beta}$ are inner products in $\hilbert_\beta$ having reproducing kernels $Q_\beta\left(t,t'\right)$. The penalty $J\left(\kappa\right)$ is an inner product in $\oplus_{\beta = 0}^q \hilbert_\beta$ with reproducing kernel $\sum_{\beta=1}^q \theta_\beta Q_\beta\left(t, t'\right)$ and null space $\mathcal{N}_J = \hilbert_0$. The first term in (\ref{eq:penalized-joint-loglik-given-phi-3}) serves as a measure of the goodness of fit of $\kappa$ to the data, and only depends on $\kappa$ through the evaluation functional $\left[t_{ij}\right]\kappa$. So the argument justifying the form of the minimizer in (\ref{eq:form-of-smoothing-spline-solution}) applies, and the minimizer of the penalized likelihood has the form 

\begin{equation} \label{eq:form-of-smoothing-spline-solution-kappa}
\eta\left( t \right) = \sum_{\nu = 1}^{d_0} d_\nu\kappa_\nu\left( t \right) + \sum_{i = 1}^{\vert \mathcal{T} \vert} c_i Q_J\left( t, t_i \right),
\end{equation}  

\noindent
where $\mathcal{T} = \bigcup_{j=1}^N\bigcup_{k=1}^{m_i} t_{jk}$ denotes the unique values of the observations times pooled across subjects, where $\left\{\kappa_\nu \right\}_{\nu=1}^{d_0}$ is a basis for the null space $\mathcal{N}_J = \hilbert_0$. 

\bigskip

Standard theory for exponential families gives us that the functional 

\begin{align}
\begin{split}
L\left( \eta \right) &= -\sum_{i=1}^N \sum_{j=1}^{m_i} \left[ z_{ij} \eta\left(t_{ij}\right) - b\left(\eta\left(t_{ij}\right)\right) \right] \\
&= -\sum_{i=1}^N \sum_{j=1}^{m_i} \left[ z_{ij} \eta\left(t_{ij}\right) - b\left(\eta\left(t_{ij}\right)\right) \right]
\end{split} \label{eq:penalized-likelihood-functional}
\end{align}

\noindent
is continuous and convex in $\eta \in \hilbert$. We assume that the $\vert V \vert \times d_0$ matrix $B$ which has $i$-$\nu^{th}$ element $\eta_\nu\left(\bfv_i\right)$ is full column rank, so that $L\left(f\right)$ is strictly convex in $\hilbert$ and the minimizer of (\ref{eq:penalized-joint-loglik-given-phi-3}) uniquely exists. See \cite{wahba1995smoothing}. 

\bigskip

For fixed $\lambda$ and $\theta_\beta$, which may be hidden in $J$, the penalized log likelihood (\ref{eq:penalized-joint-loglik-given-phi-3}) is convex in $\eta$, so that the minimizer can be computed via Newton iteration. Let 

\begin{align*}
u_{ij} = -z_{ij} + b'\left( \tilde{\eta}\left(t_{ij}\right) \right) =  -z_{ij} +  \tilde{\mu}\left(t_{ij}\right), \mbox{ and}\\
\tilde{\omega}_{ij} = b''\left( \tilde{\eta}\left(t_{ij}\right) \right) = \tilde{v}\left(t_{ij}\right).
\end{align*}
The quadratic approximation of $-z_{ij} \eta\left(t_{ij}\right) + b\left(\eta\left(t_{ij}\right)\right)$ at $\tilde{\eta}\left(t_{ij}\right)$ is given by 

\begin{align*}
\begin{split}
-y_{ij}\tilde{\eta}\left(t_{ij}\right) + b\left(\tilde{\eta}\left(t_{ij}\right)\right) + \tilde{u}_{ij} \left[  \eta\left(t_{ij}\right) - \tilde{\eta}\left(t_{ij}\right)  \right] + \frac{1}{2} \tilde{\omega}_{ij} \left[ \eta\left(t_{ij}\right) - \tilde{\eta}\left(t_{ij}\right)  \right]^2 \\
 =  \frac{1}{2} \tilde{\omega}_{ij} \left[ \eta\left(t_{ij}\right) - \tilde{\eta}\left(t_{ij}\right) + \frac{\tilde{u}_{ij}}{\tilde{\omega}_{ij}} \right]^2 + C_{ij},
 \end{split}
\end{align*}

\noindent
where $C_{ij}$ is independent of $\tilde{\eta}\left(t_{ij}\right)$.  The Newton iteration uses the minimizer of the penalized weights sums of squares

\begin{equation} \label{eq:penalized-weighted-sums-of-squares}
\sum_{i=1}^N\sum_{j=1}^{m_i} \tilde{\omega}_{ij}\left(\tilde{y}_{ij} - \eta\left(t_{ij}\right)  \right)^2 + \lambda J\left(\eta\right)
\end{equation}

\noindent
to update $\tilde{\eta}$, where $\tilde{y}_{ij} = \tilde{\eta}\left(t_{ij}\right) - \tilde{u}_{ij}/\tilde{\omega}_{ij}$.


%%%%%%%%%%%%%%%%%%%%%%%%%%%%%%%%%%%%%%%%%%%%%%%%%%%%%%%%%%%%%%%%%%%%%%%%%%%%%%%%%%%%%%%%%%%
%%%%%%%%%%%%%%%%%%%%%%%%%%%%%%%%%%%%%%%%%%%%%%%%%%%%%%%%%%%%%%%%%%%%%%%%%%%%%%%%%%%%%%%%%%%
%%%%%%%%%%%%%%%%%%%%%%%%%%%%%%%%%%%%%%%%%%%%%%%%%%%%%%%%%%%%%%%%%%%%%%%%%%%%%%%%%%%%%%%%%%%
%%%%%%%%%%%%%%%%%%%%%%%%%%%%%%%%%%%%%%%%%%%%%%%%%%%%%%%%%%%%%%%%%%%%%%%%%%%%%%%%%%%%%%%%%%%
%%%%%%%%%%%%%%%%%%%%%%%%%%%%%%%%%%%%%%%%%%%%%%%%%%%%%%%%%%%%%%%%%%%%%%%%%%%%%%%%%%%%%%%%%%%
%%%%%%%%%%%%%%%%%%%%%%%%%%%%%%%%%%%%%%%%%%%%%%%%%%%%%%%%%%%%%%%%%%%%%%%%%%%%%%%%%%%%%%%%%%%
%%%%%%%%%%%%%%%%%%%%%%%%%%%%%%%%%%%%%%%%%%%%%%%%%%%%%%%%%%%%%%%%%%%%%%%%%%%%%%%%%%%%%%%%%%%
\section{Smoothing parameter selection for exponential families}
%%%%%%%%%%%%%%%%%%%%%%%%%%%%%%%%%%%%%%%%%%%%%%%%%%%%%%%%%%%%%%%%%%%%%%%%%%%%%%%%%%%%%%%%%%%
%%%%%%%%%%%%%%%%%%%%%%%%%%%%%%%%%%%%%%%%%%%%%%%%%%%%%%%%%%%%%%%%%%%%%%%%%%%%%%%%%%%%%%%%%%%
%%%%%%%%%%%%%%%%%%%%%%%%%%%%%%%%%%%%%%%%%%%%%%%%%%%%%%%%%%%%%%%%%%%%%%%%%%%%%%%%%%%%%%%%%%%
%%%%%%%%%%%%%%%%%%%%%%%%%%%%%%%%%%%%%%%%%%%%%%%%%%%%%%%%%%%%%%%%%%%%%%%%%%%%%%%%%%%%%%%%%%%
%%%%%%%%%%%%%%%%%%%%%%%%%%%%%%%%%%%%%%%%%%%%%%%%%%%%%%%%%%%%%%%%%%%%%%%%%%%%%%%%%%%%%%%%%%%
%%%%%%%%%%%%%%%%%%%%%%%%%%%%%%%%%%%%%%%%%%%%%%%%%%%%%%%%%%%%%%%%%%%%%%%%%%%%%%%%%%%%%%%%%%%
%%%%%%%%%%%%%%%%%%%%%%%%%%%%%%%%%%%%%%%%%%%%%%%%%%%%%%%%%%%%%%%%%%%%%%%%%%%%%%%%%%%%%%%%%%%

%\subfile{chapter-2-subfiles/chapter-2-iv-smoothing-parameter-selection}

The gamma penalized log likelihood (\ref{eq:form-of-smoothing-spline-solution-kappa}) is non-quadratic, so $\eta_\lambda$ must be computed using iteration even for fixed smoothing parameters. A typical choice for method of smoothing parameter selection when data are generated from a distribution belonging to exponential families is performance-oriented. The follow section provides a brief overview of the the performance-oriented iteration, specifically for selecting the optimal degree of smoothing for $\sigma^2$. This approach is just one of many in the inventory of model selection techniques for penalized regression with exponential families. We refer the reader desiring detailed examination to \cite{zhang2006component}, \cite{xiang1996generalized}, \cite{wahba1995smoothing},  \cite{wood2004stable}, and \cite{wood2017generalized}. 

\bigskip

A measure of the discrepancy between distributions belonging to an exponential family having densities of the form $p\left(z\right) = exp\left\{\left(z \eta - b\left(\eta\right)\right)/a\left(\phi\right) + c\left(z,\phi\right) \right\}$ is the Kullback-Leibler distance

\begin{align}
\begin{split} \label{eq:kl-distance-definition}
\mbox{KL}\left(\eta, \eta_\lambda\right) &= E_\lambda\left[Z \left(\eta - \eta_\lambda \right) - \left(b\left(\eta\right)- b\left(\eta_\lambda\right) \right)\right]/a\left(\phi\right)\\
&=\left[ b'\left(\eta\right) \left(\eta - \eta_\lambda \right) - \left(b\left(\eta\right)- b\left(\eta_\lambda\right) \right)\right]/a\left(\phi\right),
\end{split}
\end{align}
\noindent
For the gamma dsitribution, letting $\eta = \log \mu$, the KL distance simplifies to
\[
-\mu\left( e^{-\eta} - e^{-{\eta_\lambda}}\right) - \left(\eta-{\eta_\lambda}\right).
\]

\noindent
The KL distance is not symmetric, so sometimes people opt for its symmetrized version:

\begin{align}
\begin{split} \label{eq:skl-distance-definition}
\mbox{SKL}\left(\eta, \eta_\lambda\right) &= \mbox{KL}\left(\eta, \eta_\lambda\right) + \mbox{KL}\left(\eta_\lambda, \eta \right)\\
&= \left(b'\left(\eta\right) - b'\left(\eta_\lambda\right) \right)\left( \eta - \eta_\lambda\right)/a\left(\phi\right), \\
&= \left(\mu - \mu_\lambda \right)\left( \eta - \eta_\lambda\right)/a\left(\phi\right),
\end{split}
\end{align}

\noindent
A natural choice of loss function for measuring the performance of an estimator $\eta_\lambda\left(t\right)$ of $\eta \left(t\right)$ is the symmetrized Kullback-Leibler distance averaged over the observed time points $t_{11}, \dots ,  t_{N,m_N}$:

\begin{equation}\label{eq:SKL-loss-function}
L\left( \eta,\eta_\lambda \right) = \frac{1}{N}\sum_{i=1}^N \frac{1}{m_i}\sum_{j=1}^{m_i}  \left(\mu\left(t_{ij}\right) - \mu_\lambda \left(t_{ij}\right)\right)\left( \eta\left(t_{ij}\right) - \eta_\lambda\left(t_{ij}\right)\right),
\end{equation}

\noindent
For the Gamma distribution, this reduces to 

\begin{equation}\label{eq:gamma-SKL-loss-function}
L\left( \eta,\eta_\lambda \right) = \frac{1}{N}\sum_{i=1}^N \frac{1}{N}\sum_{j=1}^{m_i}  \left( \frac{\mu\left(t_{ij}\right)}{\mu_\lambda\left(t_{ij}\right)} - \frac{\mu_\lambda \left(t_{ij}\right)}{\mu\left(t_{ij}\right)} - 2\right).
\end{equation}


\noindent The ideal smoothing parameters are those which minimize (\ref{eq:gamma-SKL-loss-function}). The performance-oriented iteration operates on a alternative expression of the symmetrized Kullback-Leibler loss. The mean value theorem gives us that (\ref{eq:gamma-SKL-loss-function}) can be written

\begin{equation}\label{eq:gamma-SKL-loss-function-mvt}
L_\omega\left( \eta,\eta_\lambda \right) = L\left( \eta,\eta_\lambda \right) = \frac{1}{N}\sum_{i=1}^N \frac{1}{N}\sum_{j=1}^{m_i} \omega^*\left(t_{ij}\right)  \left( \eta\left(t_{ij}\right) - \eta_\lambda\left(t_{ij}\right)\right)^2,
\end{equation}

\noindent
where $\omega^*\left(t_{ij}\right) = b''\left(\eta^*\left(t_{ij}\right)\right)$ and $\eta^*\left(t_{ij}\right)$ is a convex combination of  $\eta\left(t_{ij}\right)$ and $\eta_\lambda\left(t_{ij}\right)$. One can construct an unbiased risk estimate under the weighted loss, $L_\omega$, using re-weighted observations. Letting ${Z_{i}}_\omega = W_i Z_i$, where $W_i$ is the $m_i \times m_i$ diagonal matrix having diagonal entries $\omega^*\left(t_{i1}\right), \dots, \omega^*\left(t_{i,m_i}\right)$, an unbiased estimate of relative loss is given by 

\begin{equation}\label{eq:weighted-unbiased-risk-estimate}
U_\omega\left( \lambda \right) = L\left( \eta,\eta_\lambda \right) = \frac{1}{N}\sum_{i=1}^N \frac{1}{N}\sum_{j=1}^{m_i} \omega^*\left(t_{ij}\right)  \left( \eta\left(t_{ij}\right) - \eta_\lambda\left(t_{ij}\right)\right)^2.
\end{equation}

\noindent
See \cite{gu2013smoothing}, Theorem 5.2. To find the optimal value of the smoothing parameter, the performance-oriented iteration tracks loss $L\left(\eta, \eta_\lambda \right)$ indirectly, simultaneously updating $\lambda, \theta_\beta$. Since it does not explicitly keep track of $L\left(\eta, \eta\lambda\right)$ itself, it may not be the most effective way to search for the optimal smoothing parameters, but it is numerically efficient. The performance-oriented iteration works on (\ref{eq:gamma-SKL-loss-function}) and updates the smoothing parameters updated according to $U_\omega\left( \lambda \right)$. Instead of fixing smoothing parameters and moving according to a particular Newton update, one chooses an update from among a family of Newton updates that is perceived to be better performing according to $U_\omega\left(\lambda\right)$. If the smoothing parameters stabilize at, say, $\left(\lambda^*,\theta^*_\beta\right)$ and the corresponding Newton iteration converges at $\eta^*$, then it is clear that $\eta^* = \eta_{\lambda^*}$ is the minimizer. In a neighborhood of $\eta^*$ where the corresponding values of (\ref{eq:penalized-weighted-sums-of-squares}) closely approximate the penalized likelihood functional (\ref{eq:penalized-likelihood-functional}) for smoothing parameters close to $\left( \lambda^*, \theta^*_\beta \right)$, then the $\eta_{\lambda, \eta^*}$s are, in turn, hopefully close approximations to the $\eta_\lambda$s. Thus, through indirect comparison $\eta^*$ is perceived to be better performing among the other $\eta_\lambda$s in the neighborhood.

\bigskip

An alternative to the performance-oriented iteration is to choose the optimal smoothing parameters by comparing candidate $\eta_\lambda$s directly; the generalized approximate cross validation (GACV) score \cite{xiang1996generalized} keeps track of $L\left(\eta, \eta_\lambda\right)$, approximating the score which is analogous to the generalized cross validation score (GCV) in the usual penalized regression setting (see \citep{wahba1990spline}). We refer the reader to the aforementioned sources for extensive discussion; for the same reason that we utilized the LosoCV criterion rather than leave-one-out or generalized cross validation for smoothing parameter selection when estimating $\phi$, we did not explore using GACV for model selection for the innovation variance function.


%\bibliography{../Master}
%
%\end{document}



\chapter{A Reproducing Kernel Hilbert Space Framework for Covariance Estimation} \label{SSANOVA-chapter}


We propose an alternate route for estimating the Cholesky decomposition of a covariance matrix when the data are unbalanced. In this chapter, we present a functional varying coefficient model to extend model \eqref{eq:mcd-ar-model}. The functional varying coefficient model serves as a flexible alternative to parametric models for the GARPs and accommodates unbalanced data without the need for imputation. We propose a blueprint for the construction of an estimator of a covariance matrix for longitudinal data by modeling $T$ as smooth two-dimensional surface, and present a reproducing kernel Hilbert space framework for estimating the functional components of the Cholesky decomposition. Chapter~\ref{psplines-chapter} demonstrates multidimensional smoothing with penalized B-splines as a flexible and computationally convenient alternative to the Hilbert space methods.

\bigskip
%A predominant difficulty in the estimation of covariance matrices is the potentially high dimensionality of the problem, as the number of unknown elements in the covariance matrix grows quadratically with the size of the matrix. It is well-known that the sample covariance matrix can be unstable in high dimensions; ways for controlling the complexity of estimates is highly desirable for improving stability of estimates. In the longitudinal-data literature, it is a common practice to use parametric models for the covariance structure.  Many have specified parsimonious parametric models for $\phi_{ijk}$ to overcome the issue of dimensionality.  
%
%\bigskip
%
%We naturally accommodate irregularly spaced data and unequal sample sizes between subjects by defining the autoregressive parameters as the values of a smooth function evaluated at within-subject pairs of observed time points.  Furthermore, by viewing $\phi\left(t,s\right)$ as a smooth \emph{bivariate} function, we can utilize the information across the subdiagonals of $T$ to inform the fit, rather than treating each subdiagonal separately.  As in the classical nonparametric function estimation setting, we assume $\phi$ to vary in a high-dimensional (possibly infinite) function space. We propose two representations of $\phi\left(\cdot, \cdot\right)$ and $\sigma\left(\cdot, \cdot\right)$: approximation by smoothing splines and approximation by B-spline basis expansion. 
%
%We assume $Y\left(t\right)$ has covariance function $G\left(t,s\right)$ and that $\epsilon\left(t\right)$ follows a zero mean Gaussian white noise process with unit variance. Under mild assumptions regarding the behaviour of $Y$, then $G\left(t,s\right)$ satisfies some smoothness conditions, where smoothness is defined in terms of square integrability of certain derivatives. We view the entries of $\Sigma$ as values of $G$ evaluated at the distinct pairs of within-subject observed time points. 
%\bigskip

There has been substantial interest recently in the use of varying coefficient models for extending parametric models for longitudinal data \citep{noh2010sparse,csenturk2013modeling,csenturk2008generalized,chiang2001smoothing,hoover1998nonparametric,fan1999statistical}. Given a sample of repeated measurements on $N$ independent subjects, it is convenient to model the observed data collected on an individual as sampled from a realization of a continuous-time stochastic process $Y\left(t\right)$. Allowing individual-specific observation times, let $t_{i} = \left\{t_{i1} <  \dots < t_{i,p_i}\right\}$ denote the time points at which the sequence of measurements on the $i^{th}$ subject were taken, and let
\[
Y_i = \left(y_{i1}, \dots, y_{i,p_i}\right)'
\]
\noindent
denote the corresponding measurements, $i = 1, \dots, N$. We assume that measurement times are drawn from some distribution having compact domain $\mathcal{T}$; without loss of generality, we take $\mathcal{T} = \left[0,1\right]$. We use varying coefficient models to extend the linear model corresponding to the Cholesky decomposition \eqref{eq:mcd-ar-model}. Consider the following model as a generalization of \eqref{eq:mcd-ar-model}: 
\begin{equation}  \label{eq:cholesky-regression-model-1} 
y\left(t_{ij} \right)  = \sum_{k < j} \tilde{\phi}\left(t_{ij} ,t_{ik}\right) y\left(t_{ik}\right) + \epsilon\left({t_{ij}}\right), \quad \begin{array}{l} i = 1, \dots, N\\ j = 1, \dots, p_i,\end{array}
\end{equation}
\noindent
where the prediction errors $\epsilon\left(t\right)$ follow a mean zero Gaussian process, with variance function $\sigma^2\left(t\right)$. The coefficient associated with regressing the measurement taken at time $t$ on the measurement taken at time $s$ is given by the value of the autoregressive coefficient function evaluated at $\left(t,s\right)$:
\[
\tilde{\phi}\left(t,s\right), \quad 0 \le s < t \le 1.
\]

Under Model~\eqref{eq:cholesky-regression-model-1}, the negative log likelihood satisfies 
\begin{equation} \label{eq:full-joint-likelihood}
-2\ell\left(\phi, \sigma^2 \vert Y_1,\dots, Y_N \right) = \sum_{i=1}^N \sum_{j=2}^{p_i} \log \sigma_{ij}^2+  \sum_{i=1}^N \sum_{j=2}^{p_i} \frac{1}{\sigma^{2}_{ij}}\left( y_{ij} - \sum_{k<j} \phi\left(\bfv_{ijk}\right) y_{ik}  \right)^2,
\end{equation}
\noindent
where $\sigma_{ij}^2 = \sigma^2\left(t_{ij}\right)$.
\bigskip

A number of methods have been developed for estimating the varying coefficients for the mean trajectory of repeated measurements. \cite{wu2003nonparametric} and \cite{dahlhaus1997fitting} have explored one dimensional varying coefficient models for the Cholesky decomposition \eqref{eq:one-dimensional-mcd-vc-model} for balanced longitudinal data. Writing the varying coefficient as a bivariate function, we can model unbalanced longitudinal data, and even accommodate longitudinal data for which there is no associated fixed set of observation times. Our goal is to use bivariate smoothing to estimate $\tilde{\phi}\left(t,s\right)$ for $0 \le s < t \le 1$. In similar fashion, we estimate the innovation variance function $\sigma^2\left(t \right)$, $0 \le t \le 1$ by smoothing the squared prediction residuals as a function of $t$. 

\bigskip

This model formulation grants access to the abundance of regularization techniques that are accessible in the usual function estimation setting. Nonparametric models are often used for ``checking'' or eliciting parametric models; see \cite{cox1988testing} and \cite{liu2004hypothesis}. In this light, it is convenient to parameterize $\tilde{\phi}$ so that the fitted function can easily be used as diagnostic tools or for suggesting parsimonious or structured models for the Cholesky decomposition. Given the prevalence of stationary covariance models in the applied literature, including those specifying the elements of $T$ as a function of the lag between observations, we take a convenient parameterization of the varying coefficient function with inputs
\begin{align}\label{eq:l-m-transformation}
l = t - s \mbox{ and } m = \frac{t + s}{2}, 
\end{align}
\noindent
and model 
\begin{equation} \label{eq:phi-to-tilde-phi} 
\phi\left(l,m\right) = {\phi}\left(t-s, \frac{1}{2}\left(s+t\right)\right) = \tilde{\phi}\left(t,s\right),
\end{equation}
\noindent
where $l$ is the continuous analogue of the usual discrete lag between time points $t$ and $s$, and $m$ is its orthogonal direction. Stationary covariance models specify that the covariance between a pair of measurements taken at times $t$ and $s$ can be written as a function of $\vert t - s\vert $ only, so that
\begin{equation*}
Cov\left(y\left( t \right),y\left( s \right)\right) = G\left( \vert t - s\vert  \right)
\end{equation*}
\noindent
for some positive definite function $G$. Model~\eqref{eq:cholesky-regression-model-1} corresponds to a stationary process when $\phi$ can be written as a function of $l$ only and the innovation variances are constant in $t$. Taking stationarity as a form of simplicity or parsimony in covariance models, our approach is to regularize nonstationarity in the autoregressive varying coefficient and the innovation variance function so that simultaneous application of heavy penalization to both functions results in models that are close to stationary covariance matrices.%Imposing regularization so that functions $\phi$ which incur large penalty correspond to functions of $l$ only allows us to model nonstationarity characterized by the autoregressive varying coefficient \eqref{eq:l-m-transformation} in a fully data-driven way. One can employ a similar approach to regularizing the innovation variance function, so that application of heavy penalization yields fitted functions $\log \sigma^2$ which are constant in $t$,


\bigskip

For estimation of $\phi$, we employ the smoothing spline framework which can naturally incorporate structural differences in the functional components into modeling (see \cite{kimeldorf1971some} and \cite{wahba1990spline} for comprehensive presentation). To enhance the statistical interpretability of model parameters, we decompose $\phi$ into functional components similar to the notion of the main effect and the interaction terms in classical analysis of variance. We adopt the smoothing spline analogue of the classical ANOVA model proposed by \cite{gu2013smoothing}, and estimation is achieved through similar computational strategies.


%%%%%%%%%%%%%%%%%%%%%%%%%%%%%%%%%%%%%%%%%%%%%%%%%%%%%%%%%%%%%%%%%%%%%%%%%%%%%%%%%%%%%%
%%%%%%%%%%%%%%%%%%%%%%%%%%%%%%%%%%%%%%%%%%%%%%%%%%%%%%%%%%%%%%%%%%%%%%%%%%%%%%%%%%%%%%

\section{The Function Space for Smoothing Spline ANOVA Models} \label{SSANOVA-function-space}

Smoothing spline ANOVA models \citep{gu2002smoothing} are a versatile family of smoothing methods that are applicable for both univariate and multivariate problems. These models are rooted in the theory of reproducing kernel Hilbert spaces and have been studied extensively for nonparametric function estimation (see \cite{aronszajn1950theory}, \cite{wahba1990spline}, and \cite{berlinet2011reproducing} for detailed examinations).  However, to our knowledge, they have received little attention in the context of covariance modeling. Before we demonstrate the estimation of $\phi$ using a smoothing spline ANOVA model, we first must establish some notation and review the relevant mathematical details of reproducing kernel Hilbert spaces. 



\subsection{Properties of Reproducing Kernel Hilbert Spaces}

A Hilbert space $\hilbert$ of functions on a set $\mathcal{V}$ with inner product $\langle \cdot, \cdot\rangle_\hilbert$ is defined as a complete inner product linear space. For each $\bfv \in \mathcal{V}$, let $\left[\bfv \right]$ map $f \in \hilbert$ to $f\left(\bfv\right) \in \Re$, which is known as the evaluation functional at $\bfv$. A Hilbert space is called a reproducing kernel Hilbert space if the evaluation functional $\left[\bfv\right]f = f\left(\bfv\right)$ is continuous in $\hilbert$ for all $\bfv \in \mathcal{V}$. The Reisz Representation Theorem gives that there exists $K_{_\bfv} \in \hilbert$, the representer of the evaluation functional $\left[\bfv\right]\left(\cdot\right)$, such that $\langle K_{_\bfv}, f \rangle_\hilbert = f\left(\bfv\right)$ for all $f \in \mathcal{H}$. See Theorem 2.2 in \cite{gu2013smoothing}.

\bigskip

The symmetric, bivariate function $K\left(\bfv_1, \bfv_2 \right) = K_{_{\bfv_2 }}\left(\bfv_1\right) = \langle K_{_{\bfv_1}}, K_{_{\bfv_2}} \rangle_\hilbert$ is called the reproducing kernel (RK) of $\hilbert$. The RK satisfies that for every $\bfv \in \mathcal{V}$ and $f \in \mathcal{H}$,

\begin{enumerate}
\item $K\left(\cdot, \bfv \right) \in \hilbert$ 
\item $f\left(\bfv\right) = \langle f, K\left(\cdot, \bfv\right)\rangle_\hilbert$\label{rkhs-reproducing-property}
\end{enumerate}
\noindent
The second property is called the reproducing property of $K$. Every reproducing kernel uniquely determines the RKHS, and in turn, every RKHS has unique reproducing kernel. See Theorem 2.3 in \cite{gu2013smoothing}. The kernel satisfies that for any $\left\{\bfv_1,\dots, \bfv_{n_1}\right\}$, $\left\{{\bfu}_1,\dots, {\bfu}_{n_2}\right\} \in \mathcal{V}$ and $\left\{a_1,\dots, a_{n_1}\right\}$, $\left\{b_1,\dots, b_{n_2}\right\} \in \Re$,
\begin{equation}
 \langle\sum_{i = 1}^{n_1} a_i K\left(\cdot, \bfv_i\right), \sum_{j = 1}^{n_2} b_j K\left(\cdot, {\bfu}_j\right) \rangle_\hilbert = \sum_i \sum_j a_ib_j K\left(\bfv_i ,\bfu_j\right).
\end{equation}
\noindent
The representer of any bounded linear functional can be obtained from the reproducing kernel $K$. 

%%%%%%%%%%%%%%%%%%%%%%%%%%%%%%%%%%%%%%%%%%%%%%%%%%%%%%%%%%%%%%%%%%%%%%%%%%%%%%%%%%%%%%

\subsubsection{The Smoothing Spline Model Space}

Suppose that $J\left(f\right)$ is a penalty functional defined on $\hilbert$ measuring the roughness of $f$. When $J\left(f\right)$ is in the form of a squared semi-norm, it induces an orthogonal decomposition of $\hilbert$. Let $\mathcal{N}_J = \left\{ f:\; J\left(f\right) = 0\right\}$ denote the null space of $J$, and consider the decomposition
\[
\hilbert = \hilbert_0 \oplus \hilbert_1,
\]

\noindent
where $\hilbert_1$ is the subspace of $\hilbert$ with $J\left(f\right)$ as its squared norm. For the cubic smoothing spline, the roughness penalty corresponds to 
\begin{equation} \label{eq:SS-penalty-functional}
J\left(f\right) = \int_0^1  \left(f^{\left(2\right)}\left(x\right)\right)^2\;dx.
\end{equation}
\noindent
The penalty on the squared second derivative induces a decomposition of the function space
\[
C^{\left(2\right)}\left[0,1\right] = \left\{f: \int \limits_{0}^1 \left(f^{\left(2\right)}\left(x\right)\right)^2\;dx < \infty \right\}
\]
\noindent %with $d_\Upsilon = 2$, 
which is a Hilbert space if equipped with inner product
\begin{align} \label{eq:SS-RKHS-inner-product}
%\begin{split}
%\langle f,g\rangle &= \langle f,g\rangle_0 + \langle f,g\rangle_1\\
%\langle f,g \rangle= \sum_{i=0}^{1}M_{i} f M_{i} g + \int_0^1 f^{\left( 2 \right)}\left(x\right)g^{\left( 2 \right)}\left(x\right)dx,% \quad i = 1,2
\langle f,g \rangle_\hilbert=M_{0} f M_{0} g + M_{1} f M_{1} g + \int_0^1 f^{\left( 2 \right)}\left(x\right)g^{\left( 2 \right)}\left(x\right)dx,% \quad i = 1,2
%\end{split}
\end{align}
\noindent
where the $i^{th}$ order differential operator $M_i$ is given by $M_i f = \int_0^1 f^{\left( i \right)}\left(x\right) dx$. 

%Denote the norm corresponding to this inner product by
%\[
%\vert \vert f \vert \vert^2 = \left< f,f\right> = \left< f,f\right>_0 + \left< f,f\right>_1 = \vert \vert P_0 f \vert \vert^2 + \vert \vert P_1 f \vert \vert^2
%\]
\noindent


\bigskip
 
Given inner product \eqref{eq:SS-RKHS-inner-product}, the reproducing kernel $K$ can be expressed in terms of the scaled Bernoulli polynomials $\left\{ k_j\left(v\right) = \frac{1}{j!}B_j\left(x\right) \right\}$, $x \in \left[0,1\right]$, where $B_j$ is defined according to:
\begin{align*}
B_0\left(v\right) &= 1\\
\frac{d}{dv} B_j\left(v\right) &= jB_{j-1}\left(v\right), \;j = 1, 2, \dots
\end{align*}
\noindent
One can verify that $\int \limits_0^1 k_i^j\left(x \right)dx = \delta_{ij}$ for $i,j= 0,1$, where $\delta_{ij}$ is the Kronecker delta. This implies that  $\left\{k_0, k_1\right\}$ form an orthonormal basis for 
\[
\hilbert_{0} = \left\{ f: f^{\left( 2 \right)} = 0 \right\}
\] 
\noindent %\sum_{i=0}^{1}M_{i} f M_{i} g,% \quad i = 1, 2,  
under the inner product $\langle f,g\rangle_0 =  M_0 f M_0 g + M_1 f M_1 g$ and that 
\[
K_{0}\left(x,y\right) =  k_0\left(x\right)  k_0\left(y\right) +  k_1\left(x\right)  k_1\left(y\right) 
\]
\noindent
is the reproducing kernel for $\hilbert_{0}$. One can further decompose $\hilbert_0$ into the tensor sum of the subspaces spanned by $k_0$ and $k_1$:
\begin{align}\label{eq:SS-RKHS-null-space-tensor-sum}
%\begin{split} 
\hilbert_0 &=  \hilbert_{00} \oplus \hilbert_{01}= \left\{f: \;f \propto 1 \right\} \oplus \left\{f: \;f \propto k_1\right\}
%\end{split}
\end{align}
\noindent
where the corresponding reproducing kernels for each subspace are given by $1$ and $k_1\left(x\right)k_1\left(y\right)$, respectively. The subspaces of $\hilbert$ which are orthogonal to $\hilbert_0$ are comprised of functions $f$ satisfying 
\[
\hilbert_{1} = \lbrace f: M_0 f = M_1 f =  0,\;\;\;\int\limits_{0}^1 \left(f''\left(x\right)\right)^2;dx < \infty \rbrace, %\quad i = 1,2.
\]
One can show that the representer for the evaluation functional $\left[x\right] \left(\cdot\right)$ in $\hilbert_{1}$ with squared norm $\langle f,g\rangle_{\hilbert1}= \int_0^1 f^{\left(2\right)}\left(x\right)g^{\left(2\right)}\left(x\right)\;dx$ is given by the function
\begin{equation}
{{K }_{1}}\left(x,y\right) = k_{2}\left(x\right)k_{2}\left(y\right) - k_{4}\left(x-y \right).
\end{equation}
\noindent
See Example 2.3.3 in \cite{gu2002smoothing} for proof. It is obvious that $\hilbert_0 \bigcap \hilbert_1 = \left\{0\right\}$, so the converse of Theorem 2.5 in \cite{gu2013smoothing} gives us that the reproducing kernel for the full space
\begin{align}\label{eq:SS-RKHS-as-tensor-sum}
%\begin{split} 
\hilbert = \hilbert_0  + \hilbert_1,
%\end{split}
\end{align}
\noindent
is given by $K = K_0 + K_1$. Using the decomposition of $\hilbert_0$ into the constant and linear subspaces in \eqref{eq:SS-RKHS-null-space-tensor-sum}, we can further decompose $\hilbert$ into
\begin{equation}\label{eq:RKHS-ANOVA-decomposition}
\hilbert = \hilbert_{00}  +  \hilbert_{01} + \hilbert_1, 
\end{equation}
\noindent
where $ \hilbert_{01} \oplus \hilbert_1$ forms the contrast in a one-way ANOVA decomposition with averaging operator $\mathcal{A}f = \int_0^1 f\left(x\right)\;dx$. 
The reproducing kernel $K = K_{00} + K_{01} + K_1$ can be defined in terns of the corresponding reproducing kernels 
\begin{align}
\begin{split} \label{eq:cubic-spline-hilbert-space-rks}
K_{00}\left(x,y\right) &= 1,\\
K_{01}\left(x,y\right) &= k_1\left(x\right)k_1\left(y\right), \mbox{ and}\\
K_{1}\left(x,y\right) &= k_2\left(x\right)k_2\left(y\right) - k_4\left(x-y\right).
\end{split}
\end{align}
\noindent
The kernel $K_{00}$ generates the ``mean'' space. Together, the kernels $K_{01}$ and $K_{1}$ generate the ``contrast'' space, with $K_{01}$ contributing to the ``parametric contrast'' and $K_{1}$ to the ``nonparametric contrast.''

\subsubsection{The Tensor Product Smoothing Spline Model Space}

To estimate a bivariate function using the ANOVA decomposition given in \eqref{eq:RKHS-ANOVA-decomposition}, one may construct a tensor product reproducing kernel Hilbert space. The space can be constructed through the reproducing kernel, which is constructed using the reproducing kernels on each of the marginal domains. One-way ANOVA decompositions on the marginal domains naturally induce an ANOVA decomposition on the product domain. It can be shown that the products of reproducing kernels on the marginal domains form reproducing kernels on the product domain; see Theorem 2.6 in \cite{gu2013smoothing}.

\bigskip

Let $\hilbert_{\left[1\right]}$ and $\hilbert_{\left[2\right]}$ denote reproducing kernel Hilbert spaces on marginal domains $\left[0, 1\right]$ equipped with corresponding reproducing kernels $K_{\left[1\right]}$ and $K_{\left[2\right]}$, each defined as in \eqref{eq:cubic-spline-hilbert-space-rks}. The RKHS corresponding to the tensor product smoothing spline is given by
\[
\hilbert = \hilbert_{\left[1\right]} \otimes \hilbert_{\left[2\right]}
\]
\noindent
and has reproducing kernel 
\[
K\left(\bfx,\bfy\right) = K_{\left[1\right]}\left(x_{\left[1\right]},y_{\left[1\right]}\right) K_{\left[2\right]}\left(x_{\left[2\right]},y_{\left[2\right]}\right),
\]
\noindent 
where $\bfx = \left(x_1, x_2\right)$ and $\bfy = \left(y_1, y_2\right)$.

\bigskip

The tensor product space can be constructed with nine tensor sum terms, which are defined by the decomposition of the marginal subspaces
\[
\hilbert_{\left[i\right]} = \hilbert_{00\left[1\right]}  +  \hilbert_{01\left[i\right]} + \hilbert_{1\left[i\right]}, \quad i = 1,2.
\]
\noindent
Table~\ref{table:tensor-product-cubic-spline-RKHS-table} gives the tensor sum terms defining the decomposition of $\hilbert$ and the functional components corresponding to each subspace. The reproducing kernels for each of the subspaces are given in Table~\ref{table:tensor-product-cubic-spline-RK-table}.

%\begin{table}[H]
%\centering % used for centering table
%\begin{tabular}{r|c|c|c|} % centered columns (4 columns)
%\multicolumn{1}{c}{} & \multicolumn{1}{c}{	$\hilbert_{00\left[2\right]}$}	&	\multicolumn{1}{c}{$\hilbert_{01\left[2\right]}$}	&\multicolumn{1}{c}{ $\hilbert_{1\left[2\right]}$}\\ [1.5ex] 
%\cline{2-4}  % inserts single horizontal line\\
%$\hilbert_{00\left[1\right]}$		& $\hilbert_{00\left[1\right]}\otimes \hilbert_{00\left[2\right]}$ 	&	$\hilbert_{00\left[1\right]}	\otimes \hilbert_{01\left[2\right]} $	&	$\hilbert_{00\left[1\right]}	\otimes \hilbert_{1\left[2\right]}$   \\ [1.5ex] 
%$\hilbert_{01\left[1\right]}$		& $\hilbert_{01\left[1\right]} \otimes \hilbert_{00\left[2\right]}$			& 	$\hilbert_{01\left[1\right]} \otimes \hilbert_{01\left[2\right]}$   &   $\hilbert_{01\left[1\right]} \otimes \hilbert_{1\left[2\right]}$\\ [1.5ex] 
% $\hilbert_{1\left[1\right]}$	& 	 $\hilbert_{1\left[1\right]} \otimes \hilbert_{00\left[2\right]}$	&	$\hilbert_{1\left[1\right]} \otimes \hilbert_{01\left[2\right]}$ 	&	$\hilbert_{1\left[1\right]} \otimes \hilbert_{1\left[2\right]}$ \\ [1.5ex] 
%\cline{2-4}
%\end{tabular}
%\caption{\textit{Construction of the tensor product cubic spline subspace from marginal subspaces $\hilbert_{\left[1\right]}$, $\hilbert_{\left[2\right]}$}} % title of Table
%\label{table:tensor-product-cubic-spline-RKHS-table}
%\end{table}

\begin{table}[H]
\begin{center}% used for centering table
\begin{tabular}{r|c|c|c|} % centered columns (4 columns)
\multicolumn{1}{c}{} & \multicolumn{1}{c}{	$\hilbert_{00\left[2\right]}$}	&	\multicolumn{1}{c}{$\hilbert_{01\left[2\right]} $}	&\multicolumn{1}{c}{ $\hilbert_{1\left[2\right]}$}\\ [1.5ex] 
\cline{2-4}  % inserts single horizontal line\\
$\hilbert_{00\left[1\right]} $		& $\hilbert_{00\left[1\right]}\otimes \hilbert_{00\left[2\right]}$ 	&	$\hilbert_{00\left[1\right]}	\otimes \hilbert_{01\left[2\right]} $	&	$\hilbert_{00\left[1\right]}	\otimes \hilbert_{1\left[2\right]}$   \\ [1.5ex] 
$\hilbert_{01\left[1\right]}$		& $\hilbert_{01\left[1\right]} \otimes \hilbert_{00\left[2\right]}$			& 	$\hilbert_{01\left[1\right]} \otimes \hilbert_{01\left[2\right]}$   &   $\hilbert_{01\left[1\right]} \otimes \hilbert_{1\left[2\right]}$\\ [1.5ex] 
 $\hilbert_{1\left[1\right]}$	& 	 $\hilbert_{1\left[1\right]} \otimes \hilbert_{00\left[2\right]}$	&	$\hilbert_{1\left[1\right]} \otimes \hilbert_{01\left[2\right]}$ 	&	$\hilbert_{1\left[1\right]} \otimes \hilbert_{1\left[2\right]}$ \\ [1.5ex] 
\cline{2-4}
\end{tabular}
\end{center}
\hfill
\hfill
\begin{center}
\begin{tabular}{r|c|c|c|} % centered columns (4 columns)
\multicolumn{1}{c}{} & \multicolumn{1}{c}{	$\left\{1\right\}$}	&	\multicolumn{1}{c}{$ \left\{k_1\right\}$}	&\multicolumn{1}{c}{ $\hilbert_{1\left[2\right]}$}\\ [1.5ex] 
\cline{2-4}  % inserts single horizontal line\\
$ \left\{1\right\}$		& mean	&	$p$-main effect	&	$np$-main effect  \\ [1.5ex] 
$ \left\{k_1\right\}$	& 	$p$-main effect	& 	$p\times p$-interaction   & $p \times np$-interaction  \\ [1.5ex] 
 $\hilbert_{1\left[1\right]}$	& 	$np$-main effect 	&  $np\times p$-interaction	&	$np \times np$-interaction \\ [1.5ex] 
\cline{2-4}
\end{tabular}
\end{center}
\caption{\textit{Construction of the tensor product cubic spline subspace from marginal subspaces} $\hilbert_{\left[1\right]}$, $\hilbert_{\left[2\right]}$ \textit{and the corresponding functional components, where ``n'' and ``p'' mean ``parametric'' and ``nonparametric,'' respectively.}} % title of Table
\label{table:tensor-product-cubic-spline-RKHS-table}
\end{table}


%\begin{landscape}
\begin{table}[H]
\centering % used for centering table
\begin{tabular}{lll} % centered columns (4 columns)
\hline 
\hline %inserts double horizontal lines
Subspace 	& 		Reproducing kernel 		 \\
\hline % inserts single horizontal line
$\hilbert_{00\left[1\right]} \otimes \hilbert_{00\left[2\right]}$ & 	$1$	\\ [1ex] 
$\hilbert_{01\left[1\right]} \otimes \hilbert_{00\left[2\right]} $& 	$k_1\left(x_1\right)k_1\left(y_1\right)$	\\ [1ex] 
$\hilbert_{01\left[1\right]} \otimes \hilbert_{01\left[2\right]}$ & 	$k_1\left(x_1\right)k_1\left(y_1\right)k_1\left(y_1\right)k_1\left(y_2\right)$ \\ [1ex] 
$\hilbert_{1\left[1\right]} \otimes \hilbert_{00\left[2\right]}$  	& 	$k_2\left(x_1\right)k_2\left(y_1\right) - k_4\left(x_1 - y_1\right)$	\\ [1ex] 
$\hilbert_{1\left[1\right]} \otimes \hilbert_{01\left[2\right]}$ 	& 	$\left[k_2\left(x_1\right)k_2\left(y_1\right) - k_4\left(x_1 - y_1\right)\right]k_1\left(x_2\right)k_1\left(y_2\right)$ \\ [1ex]  
$\hilbert_{1\left[1\right]} \otimes \hilbert_{1\left[2\right]}$  & $\left[k_2\left(x_1\right)k_2\left(y_1\right) - k_4\left(x_1 - y_1\right)\right]\left[k_2\left(x_2\right)k_2\left(y_2\right) - k_4\left(x_2 - y_2\right)\right]$	\\ [1ex]  
\hline %inserts single line
\hline %inserts single line
\end{tabular}
\caption{\textit{Reproducing kernels corresponding to the subspaces for the cubic tensor product smoothing spline given in Table~\ref{table:tensor-product-cubic-spline-RKHS-table}.}} % title of Table
\label{table:tensor-product-cubic-spline-RK-table}
\end{table}
%\end{landscape}


The penalty functional driving the ANOVA decomposition of the marginal subspaces can be generalized to penalize the $m^{th}$ order derivative by letting
\[
J\left(f\right) = \int_0^1  \left(f^{\left(m\right)}\left(x\right)\right)^2\;dx.
\]
\noindent
For example, letting $m = 1$ corresponds to the space for a linear smoothing spline, where the null space of the penalty functional is spanned by constant functions. For detailed derivations of the smoothing spline ANOVA decomposition with arbitrary penalty order $m$, we refer the reader to Chapter 2 in \cite{gu2013smoothing}. 


\subsubsection{A General Form for Multiple-Term Reproducing Kernel Hilbert Spaces}

The previous construction of the RKHS for the tensor product cubic spline contains multiple tensor sum terms. We can write 
\begin{equation} \label{eq:multi-term-RKHS}
\hilbert = \bigoplus\limits_{\beta} \hilbert_\beta, 
\end{equation}
\noindent
where $\beta$ is a generic index. The subspaces $\hilbert_\beta$ have reproducing kernels $K_\beta$ and corresponding inner products $\langle f_\beta,g_\beta {\rangle_{\hilbert}}_\beta$, where $f_\beta = P_\beta f$ denotes the projection of $f$ into the subspace $\hilbert_\beta$.  For example, one can write the RKHS for the tensor product smoothing spline according to \eqref{eq:multi-term-RKHS} using the subspaces given in Table~\ref{table:tensor-product-cubic-spline-RK-table}. 

\bigskip

The subspaces $\hilbert_\beta$ are independent modules, and the inner products $\langle f_\beta,g\beta {\rangle_{\hilbert}}_\beta$ are not necessarily comparable between subspaces. To standardize across the subspaces, an inner product in $\hilbert$ can be specified via
\begin{equation} \label{eq:multi-term-inner-product}
\langle f,g \rangle_\hilbert = \sum_\beta \theta_\beta^{-1} \langle f,g {\rangle_{\hilbert_\beta}}.
\end{equation}
\noindent
where $\theta_\beta \in \left(0,\infty\right)$ are additional smoothing parameters. The corresponding reproducing kernel for $\hilbert$ is given by 
\begin{equation} \label{eq:multi-term-RK}
K = \sum_\beta \theta_\beta K_\beta,
\end{equation}
\noindent
which can be used to specify the penalty $J\left(f\right) = \langle f,f \rangle_\hilbert$. Subspaces which don't contribute to $J\left(f\right)$ form $\hilbert_0 = \left\{f: J\left(f\right) = 0\right\}$, the null space of $J\left(f\right)$. The subspaces contributing to $J\left(f\right)$ form the space $\hilbert_1 = \hilbert \ominus \hilbert_0$, in which $J\left(f\right)$ is a full inner product. For this specification, denote the penalty constructed as such by 
\begin{equation}\label{eq:multi-term-penalty}
J\left(\phi\right) =  \sum_{\beta} \theta^{-1}_\beta  \langle \phi_\beta,\phi_\beta {\rangle_{\hilbert}}_\beta = \sum_{\beta} \theta^{-1}_\beta J_\beta \left( \phi_\beta \right).
\end{equation}
\noindent
The $\left \{ \theta_\beta \right\}$ are implicit in notation henceforth to permit ease of exposition.




%%%%%%%%%%%%%%%%%%%%%%%%%%%%%%%%%%%%%%%%%%%%%%%%%%%%%%%%%%%%%%%%%%%%%%%%%%%%%%%%%%%%%%

\section{A Reproducing Kernel Hilbert Space Framework for the Generalized Autoregressive Varying Coefficient} \label{RKHS-for-phi}

We can construct the model space for the generalized autoregressive varying coefficient $\phi$ using the previous recipe for constructing a tensor product RKHS. Let $\hilbert_{\left[l\right]}$ denote the RKHS for the domain of $l \in \left[0,1\right]$ with reproducing kernel $K_{\left[l\right]}$, and similarly, let $\hilbert_{\left[m\right]}$ denote the RKHS for the domain of $m \in \left[0,1\right]$ with reproducing kernel $K_{\left[m\right]}$. The function space for $\phi\left(t,s\right) \in \hilbert$
\begin{align*}
\hilbert &= \hilbert_{\left[l\right]} \otimes \hilbert_{\left[m\right]} \\
&= \hilbert_0 \oplus \hilbert_1
\end{align*}
\noindent
is obtained as in Section~\ref{SSANOVA-function-space}, with reproducing kernel
\[
K = K_{\left[l\right]}K_{\left[m\right]}.
\] 

%\[
%\hilbert_1 = \left[\right] \oplus  \left[\hilbert_{1\left[1\right]} \otimes \hilbert_{00\left[2\right]}\right] \oplus \left[\hilbert_{1\left[1\right]} \otimes \hilbert_{01\left[2\right]}\right] \oplus \left[\hilbert_{1\left[1\right]} \otimes \hilbert_{1\left[2\right]}\right]
%\]

Let $\bfv_{ijk} = \left(t_{ij} - t_{ik}, \frac{1}{2}\left(t_{ij} + t_{ik}\right)\right) = \left(l_{ijk}, m_{ijk}\right)$ denote the tuple corresponding to the transformed pair of observation times. Fixing the innovation variances $\sigma_{ij}^2 = \sigma^2\left(t_{ij}\right)$ in \eqref{eq:full-joint-likelihood}, the negative log likelihood satsifies
\begin{equation}\label{eq:negative-log-likelihood-given-sigma}
-2\ell\left(\phi \vert Y_1,\dots, Y_N, \sigma^2\right) = \sum_{i=1}^N \sum_{j=2}^{p_i} \frac{1}{\sigma^{2}_{ij}}\left( y_{ij} - \sum_{k<j} \phi\left(\bfv_{ijk}\right) y_{ik}  \right)^2.
\end{equation}
\noindent
The roughness penalty associated with reproducing kernel $K$ can be written as $J\left(\phi\right) =\vert \vert P_1 \phi \vert\vert^2$, the squared norm of the projection of $\phi$ onto $\hilbert_1$. Appending this to \eqref{eq:negative-log-likelihood-given-sigma}, the penalized negative log likelihood may be written
 \begin{equation} \label{eq:phi-penalized-sums-of-squares}
-2\ell\left(\phi \vert Y_1,\dots, Y_N, \sigma^2\right) + \lambda J\left(\phi\right) = \sum_{i=1}^N \sum_{j=2}^{p_i} \frac{1}{\sigma^{2}_{ij}}\left( y_{ij} - \sum_{k<j}\left( L_{_{ijk}}\phi\right) y_{ik}  \right)^2 + \lambda \vert\vert P_1 \phi \vert \vert^2, 
\end{equation} 
\noindent
where $L_{ijk} = \left[\bfv_{ijk}\right]\phi$ denotes the evaluation functional at $\bfv_{ijk}$.

%\bigskip
%\noindent
%Then one may write $\psi\left( \bfv_{ijk} \right)$ as the inner product of itself with the reproducing kernel:
%\begin{equation} \label{eq:representer-as-inner-product}
%\psi_{ijk}\left( \bfv \right) = \langle \psi_{_{ijk}}, K_{_{\bfv}} \rangle = L_{_{ijk}} K_{_{\bfv}} = L_{_{ijk,\left(\cdot\right)}} K \left(\bfv,\cdot\right)
%\end{equation}
% \noindent
% where the notation $L_{_{ijk,\left(\cdot\right)}}$ indicates that $L_{ijk}$ is applied to what immediately follows as a function of $\left( \cdot \right)$, so that one can obtain $\psi_{_{ijk}}\left(\bfv\right)$ by applying $L_{_{ijk}}$ to $K\left(\bfv, \bfv^*\right)$, considered as a function of $\bfv^*$. 

% \begin{equation} \label{eq:phi-penalized-sums-of-squares}
% -2\ell_\phi + \lambda J\left(\phi\right) = \sum_{i=1}^N \sum_{j=2}^{p_i} \frac{1}{\sigma^{2}_{ij}}\left( y_{ij} - \sum_{k<j} \phi\left(\bfv_{ijk}\right) y_{ik}  \right)^2 + \lambda J\left( \phi \right),
% \end{equation}
%\noindent
% which can now be written
% \begin{equation} \label{eq:phi-penalized-sums-of-squares-RK-norm}
%-2\ell_\phi + \lambda J\left(\phi\right) = \sum_{i=1}^N \sum_{j=2}^{p_i} \frac{1}{\sigma^{2}_{ij}}\left( y_{ij} - \sum_{k<j}\left( L_{_{ijk}}\phi\right) y_{ik}  \right)^2 + \lambda \vert\vert P_J \phi \vert \vert_\hilbert^2, 
%\end{equation} 
%\noindent
%where $P_J$ is the projection operator which projects $\phi$ onto the subspace $\hilbert_J$, and $L_{_{ijk}}$ denotes the evaluation functional $\left[\bfv_{ijk}\right] \phi$. 


\subsection{A Representer Theorem}

\cite{wahba1990spline} established an explicit form for the minimizer of the penalized sums of squares in the usual function estimation setting. The following theorem establishes the form for the minimizer of \eqref{eq:phi-penalized-sums-of-squares}, the penalized sums of squares for the varying coefficient model \eqref{eq:cholesky-regression-model-1}. Define  
\[
V = \bigcup\limits_{i,j,k} \left\{\bfv_{ijk} \right\} \equiv \left\{ \bfv_1,\dots,\bfv_{\vert V \vert} \right\}
\]
\noindent
as the set of unique within-subject pairs of observation times. Let $\psi_{ijk}$ denote the representer of  $L_{ijk}$, i.e. $\psi_{ijk}$ satisfies
\[
\langle \psi_{ijk}, \phi \rangle = L_{ijk}\phi, \quad \phi \in \hilbert.
\]

 \begin{theorem} \label{theorem:finite-dimensional-minimizer}
 Let $\left\{\nu_1,\dots, \nu_{d_\Upsilon}\right\}$ span $\hilbert_0$, the null space of $J$. Let $B$ denote the $\vert V \vert \times d_\Upsilon$ matrix having $i^{th}$ column equal to $\nu_i$ evaluated at the observed $\bfv \in V$, and assume that $B$ has full column rank. Then the minimizer $\phi_\lambda$ of \eqref{eq:phi-penalized-sums-of-squares} is given by
 \begin{equation} \label{eq:form-of-the-minimizer-phi}
\phi_\lambda\left(\bfv\right) = \sum_{i = 1}^{d_\Upsilon} d_i \nu_{i}\left(\bfv\right) + \sum_{j = 1}^{\vert V \vert} c_j K_1\left(\bfv_j, \bfv\right),
\end{equation}
\noindent
where $K_1\left(\bfv_j, \bfv \right) = {K_1}_{\bfv_j}\left(\bfv_j \bfv \right)$ denotes the reproducing kernel for $\hilbert_1$ evaluated at ${\bfv_j}$, the $j^{th}$ element of $V$, viewed as a function of $\bfv$.
\end{theorem}
\vspace{0.5cm}
\noindent
The proof, which is similar in spirit to the proof of Theorem 1.3.1 in \cite{wahba1990spline} can be found in Appendix~\ref{chapter-2-appendix}.


%%%%%%%%%%%%%%%%%%%%%%%%%%%%%%%%%%%%%%%%%%%%%%%%%%%%%%%%%%%%%%%%%%%%%%%%%%%%%%%%%%%%%%%%%%%
\subsection{Model Fitting}

Let $Y$ denote the vector of length $n_y= \sum_{i} p_i - N$  constructed by stacking the $N$ observed response vectors $Y_1,\dots, Y_N$ less their first element $y_{i1}$ one on top of each other:
\begin{align}\label{eq:stacked-response-vector}
Y &= \left( Y'_1, Y'_2, \dots, Y'_{N} \right)'\\
 &= \left( y_{12}, y_{13},\dots, y_{1,p_1}, \dots, y_{N2}, y_{N3},\dots, y_{Np_N} \right)'.
\end{align}
\noindent
Define $X_i$ to be the $p_i \times \vert V \vert$ matrix containing the covariates necessary for regressing each measurement $y_{i2}, \dots, y_{i,p_i}$ on its predecessors as in Model~\eqref{eq:cholesky-regression-model-1}, and stack these on top of one another to obtain
\begin{equation} \label{eq:ar-design-matrix-1}
X = \begin{bmatrix}
X_1 \\
X_2\\
\vdots \\
X_N
\end{bmatrix},
\end{equation}
\noindent
which has dimension $n_y \times \vert V \vert$. The penalized negative log likelihood in \eqref{eq:phi-penalized-sums-of-squares} can be expressed as
\begin{equation} \label{eq:penalized-likelihood-vectorized}
-2\ell\left(\phi \vert Y_1,\dots, Y_N, \sigma^2\right) + \lambda J\left(\phi\right) = \vert \vert D^{-1/2}\left( Y - X \left( Bd + K_nc \right) \right) \vert \vert^2  + \lambda c^\prime K_n c 
\end{equation}
\noindent
where the $\left(i,j\right)$ entry of the $\vert V \vert \times \vert V \vert$ matrix $K_n$ is given by $\langle P_1 \xi_i,  P_1 \xi_j \rangle_\hilbert$. The $\vert V \vert \times d_\Upsilon$ matrix $B$ has $\left(i,j\right)$ element equal to $\nu_j\left(\bfv_i\right)$, and we assume $B$ to be full column rank.  The diagonal matrix $D$ holds the $n_Y \times n_Y$  innovation variances $\sigma^2_{ijk}$. The following examples demonstrate how to construct the subject-specific design matrices $X_1,\dots, X_N$ when observation times are common across all subjects and when observation times are subject-specific.

\begin{example}{Construction of $X_i$ with complete data} \label{example:construction-of-X}

\vspace{.3cm} 

Construction of the autoregressive design matrix $X_i$ is straightforward in the case that there are an equal number of measurements on each subject at a common set of measurement times $t_1,\dots, t_p$. When complete data are available for measurement times $t_1, \dots, t_p$, 
\begin{equation}
X_i =  \begin{bmatrix} 
y_{i1} & 0 & 0 & 0 &0&\dots & 0 \\
 0 & y_{i 1} &  y_{i2}&0 &0& \dots & 0 \\
 \vdots &&&&&&\\
 0 & 0 & \dots &0 & y_{i1} & \dots &  y_{i,{p-1}}
\end{bmatrix}
\end{equation}
\noindent
for all $i = 1,\dots, N$. Note that this design matrix specification does not require that measurement times be regularly spaced.  
\end{example}

\begin{example}{Construction of $X_i$ with incomplete data}

\vspace{.3cm} 

We demonstrate the construction of the autoregressive design matrices when subjects do not share a universal set of observation times for $N = 2$; the construction extends naturally for an arbitrary number of trajectories. Let subjects have corresponding sample sizes $p_1 = 4$, $p_2 = 4$, with measurements on subject 1 taken at $t_{11} = 0, t_{12} = 0.2, t_{13} = 0.5, t_{14} = 0.9$ and on subject 2 taken at $t_{21} = 0, t_{22} = 0.1, t_{23} = 0.5, t_{24} = 0.7$.  Then the unique within-subject pairs of observation times $\left(t,s\right)$ such that $0 \le s < t \le 1$ are given by 
\begin{table}[H]
\centering
\begin{tabular}{l|r;{2pt/2pt}r;{2pt/2pt}r;{2pt/2pt}r;{2pt/2pt}r;{2pt/2pt}r;{2pt/2pt}r;{2pt/2pt}r;{2pt/2pt}r;{2pt/2pt}r;{2pt/2pt}r}
$i$ & $2$ & $1$ & $1,2$ & $2$ & $1$ & $2$ & $2$ & $2$ & $1$ & $1$ & $1$\\ 
\hline
$t$ & $0.1$ & $0.2$ & $0.5$ & $0.5$ & $0.5$ & $0.7$ & $0.7$ & $0.7$ &$ 0.9$ &$ 0.9$ & $0.9$ \\ 
 $s$ & $0.0$& $0.0$ & $0.0$ & $0.1$ & $0.2$ & $0.0$ & $0.1$ & $0.5$ & $0.0$ & $0.2$ & $0.5$ \\
\end{tabular}
\end{table}
where the top row indicates which subject was observed at each pair $\left(t,s\right)$. This gives that $V =  \left\{\bfv_{121},\dots, \bfv_{143}  \right\} \bigcup \left\{\bfv_{221},\dots, \bfv_{243}  \right\} = \left\{\bfv_1,\dots, \bfv_{11} \right\}$, where the distinct observed $\bfv = \left(l, m\right)$ are 

\begin{table}[H]
\centering
\begin{tabular}{l|r;{2pt/2pt}r;{2pt/2pt}r;{2pt/2pt}r;{2pt/2pt}r;{2pt/2pt}r;{2pt/2pt}r;{2pt/2pt}r;{2pt/2pt}r;{2pt/2pt}r;{2pt/2pt}r}
$l$ & $0.10$&$0.20$&$0.50$&$0.40$&$0.30$&$0.70$&$0.60$&$0.20$&$0.90$&$0.70$&$0.40$ \\ 
  $m$ & $0.05  $&$0.10$&$0.25$&$0.30$&$0.35$&$0.35$&$0.40$&$0.60$&$0.45$&$0.55$&$0.70$ \\ 
\end{tabular}
\end{table}
\noindent
Then a potential construction of the autoregressive design matrix for subject is given by:
\[
X_1 =  \begin{bmatrix} 
0   & y_{11}  &0  &  0 &   0  &  0 & 0 & 0 & 0  & 0  & 0 \\
0   &	  0   &	y_{11}  &    0   & y_{12}   &  0 & 0 & 0 & 0  & 0 & 0 \\
 0   &  0         &     0       &    0   &    0        & 0  & 0 &0 &  y_{11} & y_{12} & y_{13} 
\end{bmatrix}
\]
\noindent
and similarly, for subject 2:
\[
X_2 =  \begin{bmatrix} 
y_{21}    & 	0  &	  0           &      0            &  0 &   0   & 0 & 0 & 0 & 0 & 0  \\
0   	      &  	0  &	y_{21}     &  y_{22} &  0 &  &    0   &  0 & 0 & 0 & 0 & 0 \\
 0   	      &        0  &    0           &      0        & 0    &  y_{21}    & y_{22}& y_{23} &    0   & 0  & 0
\end{bmatrix}
\]
\end{example}

Defining $\tildeY = D^{-1/2} Y$, $\tildeB = D^{-1/2} X B $, and $\tildeK_n = D^{-1/2} X K_n$, the penalized negative log likelihood \eqref{eq:penalized-likelihood-vectorized} may be written
\begin{equation}\label{eq:penalized-loglik-tilde-vectorized}
-2\ell \left(c, d \right) + \lambda J\left( \phi \right) = \bigg[ \tildeY - \tildeB d - \tildeK_n c\bigg]'\bigg[ \tildeY - \tildeB d - \tildeK_n c\bigg] + \lambda c'Kc.
\end{equation}
\noindent
Taking partial derivatives with respect to $d$ and $c$ and setting them equal to zero yields normal equations: 
\begin{align}
\begin{split}
\tildeB'\tildeB d + \tildeB'\tildeK_n c &= \tildeB' \tildeY \\
\tildeK_n'\tildeB d + \tildeK_n'\tildeK_n c + \lambda K_n c &= \tildeK_n' \tildeY.
\end{split}
\end{align}
\noindent
Thus, for fixed smoothing parameters, the solution $\phi$ is obtained by finding $c$ and $d$ which satisfy
\begin{equation} \label{eq:vectorized-normal-equations}
\begin{bmatrix}
\tildeB'\tildeB & \tildeB'\tildeK_n \\
\tildeK_n'\tildeB & \tildeK_n'\tildeK_n + \lambda K_n\\
\end{bmatrix}
\begin{bmatrix}
d\\
c\\
\end{bmatrix}
= \begin{bmatrix}
\tildeB'\tildeY \\
 \tildeK_n'\tildeY\\
\end{bmatrix}.
\end{equation}

Fixing smoothing parameters $\lambda$ and $\theta_\beta$ (hidden in $K_n$ and $\tildeK_n$ if present), assuming that $\tildeK_n$ is full column rank, \eqref{eq:vectorized-normal-equations} can be solved by the Cholesky decomposition of the $\left( \vert V \vert + d_\Upsilon \right) \times \left( \vert V \vert + d_\Upsilon \right)$ matrix, followed by forward and backward substitution. See \cite{golub2012matrix}. Singularity of $\tildeK_n$ demands special consideration. Write the Cholesky decomposition
\begin{equation} \label{eq:normal-equation-cholesky}
\begin{bmatrix}
\tildeB'\tildeB & \tildeB'\tildeK_n \\
\tildeK_n'\tildeB & \tildeK_n'\tildeK_n + \lambda K_n\\
\end{bmatrix}
= \begin{bmatrix}
C'_1 & 0 \\
C'_2  & C'_3 
\end{bmatrix}
\begin{bmatrix}
C_1 & C_2 \\
0  & C_3 
\end{bmatrix}
\end{equation}
\noindent
where $\tildeB'\tildeB = C'_1 C_1$, $C_2 = \left(C'_1\right)^{-1} \tildeB' \tildeK_n$, and $C'_3 C_3 = \lambda K_n +  \tildeK_n'\left( I - \tildeB\left( \tildeB' \tildeB \right)^{-1} \tildeB' \right)\tildeK_n$. Using an exchange of indices known as pivoting, one may write 
\begin{equation*}
C_3 = \begin{bmatrix} H_1 & H_2 \\ 0 & 0 \end{bmatrix} = \begin{bmatrix} H \\  0 \end{bmatrix},
\end{equation*}
\noindent
where $H_1$ is nonsingular. Define
\begin{equation} \label{eq:cholesky-factor-mod}
\tilde{C}_3 = \begin{bmatrix}
H_1 & H_2 \\
0  & \delta I 
\end{bmatrix}, \;\;
\tilde{C} = \begin{bmatrix}
C_1 & C_2 \\
0  & \tilde{C}_3 
\end{bmatrix};
\end{equation}
\noindent
then
\begin{equation} \label{eq:cholesky-factor-mod-inverse}
\tilde{C}^{-1} = \begin{bmatrix}
C_1^{-1} & -C_1^{-1} C_2 \tilde{C}_3^{-1} \\
0  & \tilde{C}_3^{-1}
\end{bmatrix}.
\end{equation}

Premultiplying \eqref{eq:normal-equation-cholesky} by $(\tilde{C}')^{-1}$, straightforward algebra gives 
\begin{equation} \label{eq:vectorized-normal-equations-cholesky}
\begin{bmatrix}
I & 0 \\
0 & (\tilde{C}'_3)^{-1} C'_3 C_3 \tilde{C}_3^{-1}\\
\end{bmatrix}
\begin{bmatrix}
\tilde{d}\\
\tilde{c}\\
\end{bmatrix}
= \begin{bmatrix}
(C'_1)^{-1} \tildeB'\tildeY \\
(\tilde{C}'_3)^{-1} \tildeK_n'\left( I - \tildeB\left( \tildeB' \tildeB \right)^{-1} \tildeB' \right) \tildeY\\
\end{bmatrix}
\end{equation}
\noindent
where $\left( \tilde{d}'\;\;\tilde{c}' \right)' =  \tilde{C}' \left( d\;\;c \right)'$. Partition $\tilde{C}_3 = \begin{bmatrix} F &  L\end{bmatrix}$; then $HF = I$ and $HL = 0$. So
\begin{align*}
(\tilde{C}'_3)^{-1} C'_3 C_3 \tilde{C}_3^{-1} &= \begin{bmatrix} F' \\ L' \end{bmatrix} C'_3C_3 \begin{bmatrix} F &  L\end{bmatrix} \\
&= \begin{bmatrix} F' \\ L' \end{bmatrix} H'H \begin{bmatrix} F &  L\end{bmatrix} \\
&= \begin{bmatrix} I & 0 \\ 0 & 0 \end{bmatrix}.
\end{align*}
\noindent
If $L'C'_3 C_3 L = 0$, then $L'\tildeK_n'\left( I - \tildeB\left( \tildeB' \tildeB \right)^{-1} \tildeB' \right)\tildeK_n L = 0$, so $L'\tildeK_n'\left( I - \tildeB\left( \tildeB' \tildeB \right)^{-1} \tildeB' \right) \tildeY = 0$. Thus, the linear system has form
\begin{equation} \label{eq:vectorized-normal-equations-cholesky-2}
\begin{bmatrix}
I & 0 & 0\\
0 & I & 0 \\
0 & 0 & 0 \\
\end{bmatrix}
\begin{bmatrix}
\tilde{d}\\
\tilde{c}_1\\
\tilde{c}_2
\end{bmatrix}
= \begin{bmatrix}
* \\
* \\
0
\end{bmatrix},
\end{equation}
\noindent
which can be solved, but with $\tilde{c}_2$ arbitrary. One may perform the Cholesky decomposition of \eqref{eq:vectorized-normal-equations} with pivoting, replace the trailing $0$ with $\delta I$ for appropriate value of $\delta$, and proceed as if $\tildeK_n$ were of full rank. 

\bigskip


Solving for the coefficients gives
\begin{equation} \label{eq:d-c-hat}
\begin{bmatrix} \hat{d} \\ \hat{c} \end{bmatrix} = \tilde{C}^{-1} (\tilde{C}')^{-1} \begin{bmatrix} \tildeB' \\ \tildeK_n' \end{bmatrix} \tildeY. 
\end{equation} 
\noindent
It follows that
\begin{equation} \label{eq:tildeY-hat-equals-tildeA-tildeY}
\widehat{\tildeY} = \tildeB \hat{d} + \tildeK_n \hat{c} = \begin{bmatrix} \tildeB & \tildeK_n \end{bmatrix} \tilde{C}^{-1} (\tilde{C}')^{-1} \begin{bmatrix} \tildeB' \\ \tildeK_n' \end{bmatrix} \tildeY = \tildeA_{\lambda,\bftheta} \tildeY,
\end{equation} 
\noindent
where
\begin{align}
\begin{split} \label{eq:smoothing-matrix-A-tilde}
\tildeA_{\lambda,\bftheta} &= \begin{bmatrix} \tildeB & \tildeK_n \end{bmatrix} \tilde{C}^{-1} (\tilde{C}')^{-1} \begin{bmatrix} \tildeB' \\ \tildeK_n' \end{bmatrix}  \\
&= G + \left(I - G\right) \tildeK_n \left[\tildeK_n'\left( I - G \right)\tildeK_n + \lambda K_n\right]^{-1} \tildeK_n'\left(I - G\right),
\end{split}
\end{align} 
\noindent
for $G = \tildeB\left(\tildeB' \tildeB \right)^{-1}\tildeB'$.


%%%%%%%%%%%%%%%%%%%%%%%%%%%%%%%%%%%%%%%%%%%%%%%%%%%%%%%%%%%%%%%%%%%%%%%%%%%%%%%%%%%%%%%%%%%



\subsection{Smoothing Parameter Selection} \label{gaussian-unbiased-risk-estimate}

By varying smoothing parameters $\lambda$ and $\theta_\beta$, the minimizer $\phi_\lambda$ of \eqref{eq:vectorized-normal-equations} defines a family of potential estimates. In practice, we need to choose a specific estimate from the family, which requires effective methods for smoothing parameter selection. We consider two criteria that are commonly used for smoothing parameter selection in the context of smoothing spline models for longitudinal data. The first score is an unbiased estimate of a relative loss and assumes known variances $\sigma_t^2$. The unbiased risk estimate has attractive asymptotic properties; see \cite{gu2013smoothing} for a comprehensive examination. The second score, the leave-one-subject-out cross validation (LosoCV) score, provides an estimate of the same loss without assuming a known variance function. We review a computationally convenient approximation of the LosoCV score proposed by \cite{xu2012asymptotic}, who demonstrate the shortcut score's asymptotic optimality. To simplify notation for the initial presentation, we only make explicit the dependence of estimates and their components on $\lambda$ and conceal any dependence on $\theta_\beta$. 


\subsubsection{Unbiased Risk Estimate}

Define  $\tildeY$, $\tildeB$, and $\tildeK$ as before. Let $\tildeepsilon = D^{-1/2} \epsilon$ denote the vector of length $n_Y = \sum_{i = 1}^N p_i - N$ containing the standardized prediction errors $\epsilon_{ij} \sim N\left(0,1\right)$. Write the mean of $\tildeY$ as
\begin{align}
\tilde{\mu} = D^{-1/2} X \left[ Bd + K_nc \right] = D^{-1/2} X \Phi.
\end{align}
\noindent
We can assess $\hat{\tildeY}_\lambda$, an estimate of the mean of $\tildeY$ based on observed data $y_{ij}$, $i = 1,\dots, N$, $j = 1,\dots, p_i$, using the loss function
\begin{align}
\begin{split}
L\left(\lambda\right) &= \sum_{i = 1}^N \sum_{j = 1}^{p_i} \left(\hat{\tildey}_{ij} - E\left[\tildey_{ij}\right] \right)^2\\
&= \vert \vert \tildeY - \tilde{\mu} \vert \vert^2
\end{split}
\end{align}
\noindent
Then straightforward algebra yields that 
\begin{align} 
L\left(\lambda\right) = \mu'\left( I - \tildeA_{\lambda,\bftheta} \right)^2\mu - 2\mu'\left( I - \tildeA_{\lambda,\bftheta} \right)^2 \tildeA_{\lambda,\bftheta} \tildeepsilon + \tildeepsilon' \tildeA_{\lambda,\bftheta}^2 \tildeepsilon
\end{align}
Define the unbiased risk estimate
\begin{equation*} 
%U\left(\lambda\right) = \frac{1}{N}\tildeY'\left( I - \tildeA_{\lambda,\bftheta} \right)^2\tildeY + \frac{2}{N}\mbox{tr}\tildeA_{\lambda,\bftheta}
U\left(\lambda\right) = \tildeY'\left( I - \tildeA_{\lambda,\bftheta} \right)^2\tildeY + {2}\mbox{tr}\tildeA_{\lambda,\bftheta}.
\end{equation*}
 \noindent
Adding $\mu$ to and subtracting $\mu$ from the quadratic terms, one can verify with straightforward algebra that
\begin{align*}
%\begin{split}
U\left(\lambda\right) &= \left( \tildeY - \mu + \mu - \tildeA_{\lambda,\bftheta} \tildeY \right)'\left( \tildeY - \mu + \mu - \tildeA_{\lambda,\bftheta} \tildeY \right) + 2\mbox{ tr}\tildeA_{\lambda,\bftheta} \\
&= \left(\tildeA_{\lambda,\bftheta} \tildeY - \mu \right)'\left( \tildeA_{\lambda,\bftheta} \tildeY - \mu \right) + \tildeepsilon'\tildeepsilon + 2\tildeepsilon' \left( I- \tildeA_{\lambda,\bftheta}\right)\mu- 2\left( \tildeepsilon'\tildeA_{\lambda,\bftheta} \tildeepsilon -  \mbox{tr}\tildeA_{\lambda,\bftheta}\right).
%\end{split}
\end{align*}
\noindent
This gives
\begin{equation*} 
U\left(\lambda\right) - L\left(\lambda\right) - \tildeepsilon'\tildeepsilon  =  2\tildeepsilon' \left( I- \tildeA_{\lambda,\bftheta}\right)\mu- 2\left( \tildeepsilon'\tildeA_{\lambda,\bftheta} \tildeepsilon -  \mbox{tr}\tildeA_{\lambda,\bftheta}\right), 
\end{equation*}
 \noindent
where $\tilde{\epsilon}_i = \tildeY_i - D_i^{-1/2}X_i \Phi$ denotes the vector of standardized residuals for subject $i$. This shows that $U\left(\lambda\right)$ is unbiased for the relative loss $L\left(\lambda\right) + \tildeepsilon'\tildeepsilon$.  Under mild conditions on the risk function
  \[
 R\left(\lambda\right) = E\left[L\left(\lambda\right)\right],
 \]
\noindent
one can establish that $U$ is also a consistent estimator. See Chapter 3 of \cite{gu2013smoothing} for a formal theorem and proof.


\subsubsection{Leave-one-subject-out Cross Validation}  
The conditions under which the the cross validation and generalized cross validation scores traditionally used for smoothing parameter selection yield desirable properties generally do not hold when the data are clustered or longitudinal in nature. Instead, the leave-one-subject-out (LosoCV) cross validation score has been widely used for smoothing parameter selection for semiparametric and nonparametric models for longitudinal or functional data. The LosoCV criterion is defined as

\begin{equation} \label{eq:LOSOCV}
V_{loso}\left(\lambda\right) = \frac{1}{N}\sum_{i=1}^N \left( \tildeY_i - \widehat{\tilde{\mu}}^{\left[-i\right]}_{i}\right)'\left( \tildeY_i -  \widehat{\tilde{\mu}}^{\left[-i\right]}_{i}\right)
\end{equation}
\noindent
where $\widehat{\tilde{\mu}}^{\left[-i\right]}_{i}$ is the estimate of $E\left[ \tildeY_i \right]$ based on the data when $\tildeY_i$ is omitted. Intuitively, the LosoCV score is appealing because it preserves any within-subject dependence by leaving out all observations from the same subject together in the cross-validation.  However, despite its prevalent use, theoretical justifications for its use have not been established. In their seminal work, \cite{rice1991estimating} were the first to present a heuristic justification of LosoCV by demonstrating that it mimics the mean squared prediction error. Consider new observations $\tildeY^*_i = \left(\tilde{y}_{i1}^*, \tilde{y}_{i1}^*, \dots, \tilde{y}_{i, p_i}^*\right)$. We may write the mean squared prediction error for the new observations as follows:  
\bigskip 
\begin{align}
\begin{split}\label{eq:MSPE}
MSPE &= \frac{1}{N}\sum_{i=1}^N E\left[ \vert \vert \tildeY^*_i - \widehat{\tilde{\mu}}_{i} \vert \vert^2 \right]\\
&=  \frac{1}{N}\sum_{i=1}^N E\left[ \vert \vert \tildeY^*_i -  \tilde{\mu}_{i}+ \tilde{\mu}_{i} - \widehat{\tilde{\mu}}_{i} \vert \vert^2 \right]\\
&=  \frac{1}{N}\sum_{i=1}^N \left\{p_i + E\left[ \vert \vert \tilde{\mu}_{i} - \widehat{\tilde{\mu}}_{i} \vert \vert^2 \right] \right\}
\end{split}
\end{align}
\noindent
When $\left\{ \sigma^2\left(t\right)\right\}$ is known, $\tilde{\epsilon}_i$ is a mean zero multivariate normal vector with $Cov\left(\tilde{\epsilon}_i\right) = I_{p_i}$, which gives the last equality. Since $\tildeY_i$ and $ \widehat{\tilde{\mu}}_{i} $ are independent, the expected LosoCV score can be written
\begin{equation} \label{eq:MSPE_LOSOCV}
E\left[V_{loso}\left(\lambda\right) \right] =  \frac{1}{N}\sum_{i=1}^N\left\{ p_i +  E\left[ \vert \vert \widehat{\tilde{\mu}}^{\left[ -i \right]}_{i} - \tilde{\mu}_{i} \vert \vert^2 \right] \right\}. 
\end{equation}
\noindent
When $N$ is large, we expect that $\widehat{\tilde{\mu}}_{i}$ should be close to $\widehat{\tilde{\mu}}^{\left[ -i \right]}_{i}$, so $E\left[V_{loso}\left(\lambda\right) \right]$ should be a good approximation to the mean-squared prediction error. For a formal proof of consistency, see \cite{xu2012asymptotic}.

\bigskip

The definition of $V_{loso}$ would lead one to initially believe that calculation of the score requires solving $N$ separate minimization problems. However, \cite{xu2012asymptotic} established a computational shortcut that requires solving only one minimization problem that involves all data. 
%  \subsubsection{Computation of the LosoCV score}
  
  \begin{lemma}[Shortcut formula for LosoCV] \label{lemma:losocv-shortcut}
  The LosoCV score satisfies the following identity:
  \begin{equation*}
 V_{loso}\left( \lambda \right) = \frac{1}{N} \sum_{i = 1}^N \left(\tildeY_i - \widehat{\tildeY_i}\right)' \left(\left(I_{p_i} - \tildeA_{ii}\right)^{-1}\right)'\left(I_{p_i} - \tildeA_{ii}\right)^{-1}\left(\tildeY_i - \widehat{\tildeY_i}\right),
  \end{equation*}
  \noindent
  where $\tildeA_{ii}$ is the diagonal block of smoothing matrix $\tildeA_{\lambda,\bftheta}$ corresponding to the observations on subject $i$, and $I_{p_i}$ is a $p_i \times p_i$ identity matrix.
\end{lemma}

A detailed presentation and proof can be found in \cite{xu2012asymptotic} and supplementary materials \cite{xuasymptotic}.  The authors additionally proposed an approximation to the LosoCV score to further reduce the computational cost of evaluating $V_{loso}$, which can be expensive due to the inversion of the $I_{p_i} - \tildeA_{ii}$. Using the Taylor expansion of $\left(I_{p_i} - \tildeA_{ii}\right)^{-1} \approx I_{p_i} + \tildeA_{ii}$, we can use the following to approximate $V_{loso}$:
\begin{equation} \label{eq:approx-losocv}
V_{loso}^*\left( \lambda \right) = \frac{1}{N} \vert \vert \left(I - \tildeA_{\lambda,\bftheta}\right)\tildeY \vert \vert^2 + \frac{2}{N} \sum_{i = 1}^N \hat{\tilde{e}}'_{i}\tildeA_{ii}\hat{\tilde{e}}_i,
\end{equation}
\noindent
where $\hat{\tilde{e}}_i$ is the portion of the vector of prediction errors $\left(I - \tildeA_{\lambda,\bftheta}\right)\tildeY$ corresponding to subject $i$. They show that under mild conditions, and for fixed, nonrandom $\lambda$, the approximate LosoCV score $V_{loso}^*$ and the true LosoCV score $V_{loso}$ are asymptotically equivalent. See Theorem 3.1 of \cite{xu2012asymptotic}.
  

\subsubsection{Selection of Multiple Smoothing Parameters}

With the definition of the unbiased risk estimate and the leave-one-subject-out criteria, the expression of the smoothing matrix in \eqref{eq:smoothing-matrix-A-tilde} permits straightforward evaluation of both scores $U\left(\lambda, \bftheta \right)$ and $V_{loso}^*\left(\lambda, \bftheta \right)$, where $\bftheta = \left(\theta_1,\dots, \theta_q\right)'$ denotes the vector of smoothing parameters associated with each RK.  In this section, we discuss an algorithm to minimize the unbiased risk estimate $U\left(\lambda, \bftheta\right)$ with respect to $\lambda$ and $\bftheta$ hidden in $K = \sum_{\beta} \theta_\beta K_\beta$, where the $\left(i,j\right)$ entry of $K_\beta$ is given by $K_\beta\left(\bfv_i,\bfv_j\right)$, for $\bfv_i,\bfv_j \in V$.  We present minimization of the unbiased risk estimate explicitly, but the mechanics of the optimization are very similar to those necessary for optimizing the leave-one-subject-out cross validation criterion. The details of a procedure for explicitly minimizing the alternative criterion are presented in \cite{xu2012asymptotic}, which is based on the algorithms of \cite{gu1991minimizing}, \cite{kim2004smoothing} and \cite{wood2004stable}. The algorithm in \cite{kim2004smoothing} is the basis for the following algorithm. The key difference between the minimization of $U$ and the minimization of $V^*_{loso}$ lies in the calculation of the gradient and the Hessian matrix in the Newton update. To minimize the unbiased risk estimate,

\begin{enumerate}
\item Fix $\bftheta$; minimize $U\left(\lambda \vert \bftheta\right)$ with respect to $\lambda$.\label{step-1}
\item Update $\bftheta$ using the current estimate of $\lambda$. \label{step-2}
\end{enumerate}

\noindent
Executing \ref{step-1} follows immediately from the expression for the smoothing matrix. Performing the update in \ref{step-2} requires approximating the gradient and the Hessian of $U\left( \bftheta \vert \lambda \right)$ with respect to $\bfkappa = \log\left(\bftheta\right)$. Optimizing with respect to $\bfkappa$ rather than on the original scale is motivated by two driving factors. First, $\bfkappa$ is invariant to scale transformations because the derivatives of $U\left(\cdot\right)$ with respect to $\bfkappa$ are invariant to such transformations, while the derivatives with respect to $\bftheta$ are not. The second motivation for optimizing with respect to $\bfkappa$ is that it converts a constrained optimization ($\theta_\beta \ge 0$) problem to an unconstrained one.

\subsubsection{Algorithms}

The following presents the main algorithm for minimizing $U\left(\lambda, \bftheta \right)$ and its key components are presented in the section to follow. The minimization of $U$ is done via two nested loops. Fixing tuning parameter $\lambda$, the outer loop minimizes $U$ with respect to smoothing parameters $\theta_\beta$ via quasi-Newton iteration of \cite{dennis1996numerical}, as implemented in the \texttt{nlm} function in \texttt{R}. The inner loop then minimizes $-2\ell + \lambda J\left(\phi\right)$ with fixed tuning parameters via Newton iteration. Fixing the $\theta_\beta$s in $J \left(\phi\right) = \sum_\beta \theta^{-1}_\beta J_\beta \left(\phi_\beta\right)$, the outer loop with a single $\lambda$ is straightforward. 
 

\begin{algorithm}[H]
\caption{Selection of multiple smoothing parameters for the SSANOVA model.}\label{alg:SSANOVA-algorithm}
\begin{algorithmic}[1]
\State \textbf{Initialization:} 
	\State Set $\Delta \bfkappa := 0$; \;$\bfkappa_{-}:=\bfkappa_{0}$; \;$U_- = \infty$;

\State \textbf{Iteration:} 
	\While{not converged}
		\State For current value $\bfkappa^* = \bfkappa_- + \Delta \bfkappa$, compute $K^*_\theta = \sum_{\beta = 1}^g \theta^*_\beta K_\beta$, scale so that $\mbox{tr}\left(K_\beta\right)$ is fixed. \label{begin-loop}
		\State Compute $\tildeA\left(\lambda \vert \bftheta^* \right) = \tildeA\left(\lambda, \exp\left({\bfkappa^*} \right)\right)$.
		\State Minimize $U\left(\lambda \vert \bfkappa^* \right) =  \tildeY'\left( I - \tildeA_{\lambda,\bftheta} \right)^2\tildeY + 2\mbox{tr}\tildeA_{\lambda,\bftheta} $
		\State Set $U_* := \min \limits_\lambda U\left( \lambda \vert \bfkappa^* \right) $
		\If{$U^* > U_-$ }
		 		\State Set $\Delta \bfkappa := \Delta \bfkappa/2$
		 		\State Go to \ref{begin-loop}.
		\Else
		\State Continue
		\EndIf
		\State Evaluate the approximation of gradient $\mathbf{g} = \left(\partial /\partial \bfkappa\right) U\left(\bfkappa \vert \lambda\right)$
		\State Evaluate the approximation of Hessian $H = \left(\partial^2 /\partial \bfkappa\partial \bfkappa' \right) U\left(\bfkappa \vert \lambda\right)$.
		\State Calculate step $\Delta \bfkappa$:
			\If{$H$ positive definite}  
				\State $\Delta \bfkappa := -H^{-1} \mathbf{g}$
			\Else
				\State $\Delta \bfkappa := -\tilde{H}^{-1} \mathbf{g}$, where $\tilde{H} = \textup{diag}\left(e\right)$ is positive definite. \label{ensure-hessian-PD}
			\EndIf
	\EndWhile
\State \textbf{Calculate optimal model:} 
	\If{$\Delta \kappa_\beta < -\gamma$, for $\gamma$ large}
		\State Set $\kappa^*_{\beta} := -\infty$
	\EndIf
	\State Compute $K^*_\theta = \sum_{\beta = 1}^g \theta^*_{\beta} K_\beta$;
	\State Calculate $\begin{bmatrix} d \\ c \end{bmatrix} = \tilde{C}^{-1} \left(\tilde{C}'\right)^{-1} \begin{bmatrix} \tildeB' \\ {\tildeK^*_\theta}' \end{bmatrix} \tildeY$ as in \eqref{eq:d-c-hat}
	\end{algorithmic}
\end{algorithm}

\cite{gu1991minimizing} present details on convergence criteria based on those suggested in \cite{gill1981practical}. \cite{gill1981practical} provide detailed discussion of the Newton method based on the Cholesky decomposition necessary for calculating the update direction for $\bfkappa$. Algorithm step (\ref{ensure-hessian-PD}) returns a descent direction even when $H$ is not positive definite by adding positive mass $e$ to the diagonal elements of $H$ if necessary to produce $\tilde{H} = G'G$ where $G$ is upper triangular. See Chapter 4 in \cite{gill1981practical} for details. 

\bigskip

The unbiased risk estimate $U\left(\lambda, \bftheta\right)$ is fully parameterized by $\left\{\lambda_\beta \right\} = \left\{\lambda \theta^{-1}_\beta \right\}$, so the smoothing parameters $\left(\lambda, \left\{\theta^{-1}_\beta \right\}\right)$ over-parameterize the score, which is the reason for scaling the trace of $K_\beta$. The starting values for the $\theta$ quasi-Newton iteration are obtained with two passes of the fixed-$\theta$ outer loop as follows:
\begin{enumerate}
\item Set $\breve{\theta}_\beta^{-1} \propto \mbox{tr}\left( K_\beta \right)$, minimize $U\left(\lambda\right)$ with respect to $\lambda$ to obtain $\breve{\phi}$. \label{theta-starting-values-1}
\item Set $\bar{\theta}_\beta^{-1} \propto  J_\beta\left(\breve{\phi}_\beta \right)$, minimize $U\left(\lambda\right)$ with respect to $\lambda$ to obtain $\bar{\phi}$. \label{theta-starting-values-2}
\end{enumerate}
\noindent
The first pass allows equal opportunity for each penalty to contribute to $U$, allowing for arbitrary scaling of $J_\beta \left(\phi_\beta\right)$. The second pass grants greater allowance to terms exhibiting strength in the first pass. The following $\theta$ iteration fixes $\lambda$ and starts from $\check{\theta}_\beta$. These are the starting values adopted by \cite{gu1991minimizing}; the starting values for the first pass loop are arbitrary, but are invariant to scalings of the $\theta_\beta$. The starting values in \ref{theta-starting-values-2} for the second pass of the outer are based on more involved assumptions derived from the background formulation of the smoothing problem. After the first pass, the initial fit $\breve{\phi}$ reveals where the structure in the true $\phi$ lies in terms of the components of the subspaces $\hilbert_\beta$. Less penalty should be applied to terms exhibiting strong signal.   %: the penalty is defined as in \eqref{eq:multi-term-penalty} having form
%\[
%J\left(\phi\right) = \sum_{\beta} \theta^{-1}_\beta J_\beta \left( \phi_\beta \right).
%\]
%\noindent



%%%%%%%%%%%%%%%%%%%%%%%%%%%%%%%%%%%%%%%%%%%%%%%%%%%%%%%%%%%%%%%%%%%%%%%%%%%%%%%%%%%%%%%%%%%
%%%%%%%%%%%%%%%%%%%%%%%%%%%%%%%%%%%%%%%%%%%%%%%%%%%%%%%%%%%%%%%%%%%%%%%%%%%%%%%%%%%%%%%%%%%

\section{A Reproducing Kernel Hilbert Space Framework for the Innovation Variance Function} \label{chapter-3-IV-modeling-section}

Fixing $\phi$ in \eqref{eq:full-joint-likelihood}, the negative log likelihood of the data $Y_1,\dots, Y_N$ satisfies
\begin{equation} \label{eq:penalized-joint-loglik-given-phi-2}
-2\ell\left(\sigma^2 \vert Y_1,\dots, Y_N ,\phi \right) =  \sum_{i = 1}^N \sum_{j = 1}^{p_i} \log \sigma^2_{ij}  + \sum_{i = 1}^N \sum_{j = 1}^{p_i} \frac {\epsilon_{ij}^2}{\sigma^2_{ij}};
\end{equation}
\noindent
where $\epsilon_{ij} =  y_{ij} - \sum_{k<j} \phi_{ijk} y_{ik}$. Let 
\begin{equation}
\mbox{RSS}\left( t \right) = \sum_{i,j:t_{ij}= t} \left( y_{ij} - \sum_{k<j} \phi_{ijk} y_{ik}\right)^2
\end{equation}
\noindent
denote the squared innovations for the observations $y_{ij}$ having corresponding measurement time $t = t_{ij}$. Then $\mbox{RSS}\left( t \right)/\sigma^2\left(t\right) \sim \chi^2_{df_t}$, where the degrees of freedom $df_{t}$ corresponds to the number of observations $y_{ij}$ having corresponding measurement time $t$. In this light, for fixed $\phi$, the penalized likelihood \eqref{eq:penalized-joint-loglik-given-phi-2} is that of a variance model with the $\epsilon_{ij}^2$ serving as the response.  This corresponds to a generalized linear model with Gamma errors and known scale parameter equal to 2. Let $z_{ij} = \epsilon_{ij}^2$, and let $Z_{i} = \left(z_{i1},\dots, z_{i,p_i} \right)'$ denote the vector of squared residuals for the $i^{th}$ observed trajectory. 

\bigskip

The Gamma distribution is parameterized by shape parameter $\alpha$ and scale parameter $\beta$. Let $\sigma^2$ denote the mean of the distribution given by $\alpha \beta$. For a single observation $Z$, reparameterizing the Gamma likelihood in terms of $\left(\alpha, \sigma^2 \right)$ and dropping terms that don't involve $\sigma^2\left(\cdot\right)$ gives  
\begin{align}
\begin{split}
-\ell \left(\sigma^2, \alpha \vert z \right) &\propto \alpha\left[\frac{z}{\sigma^2} + \log \sigma^2\right]  \\ 
&= \alpha\left[ze^{-\eta} + \eta\right],\label{eq:gamma-iv-likelihood-canonical-link}
\end{split}
\end{align}
\noindent
where $\alpha^{-1}$ is the dispersion parameter and $\eta = \log \sigma^2$. The log likelihood of the squared working residuals $Z_1,\dots, Z_N$ becomes 
\begin{equation} \label{eq:penalized-joint-loglik-given-phi-3}
-2\ell\left(  \sigma^2 \vert Z_1,\dots, Z_N \right) =  \sum_{i = 1}^N \sum_{j = 1}^{p_i} \eta_{ij}  + \sum_{i = 1}^N \sum_{j = 1}^{p_i} z_{ij}e^{-\eta_{ij}},
\end{equation}
\noindent
which coincides with a Gamma distribution with scale parameter $\alpha = 2$. Smoothing spline ANOVA models for exponential families have been studied extensively; see \cite{wahba1995smoothing}, \cite{wang1997grkpack}, and \cite{gu2013smoothing}. Fixing $\phi$, we take the estimator of $\eta\left(t\right) = \log\sigma^2\left(t\right)$ to be the minimizer of the penalized negative log likelihood:
\begin{equation} \label{eq:penalized-joint-loglik-given-phi}
-2\ell\left( \eta \vert Z_1,\dots, Z_N, \right) +\lambda J \left(\eta\right) =  \sum_{i = 1}^N \sum_{j = 1}^{p_i} \eta\left(t_{ij}\right)  + \sum_{i = 1}^N \sum_{j = 1}^{p_i} z_{ij} e^{-\eta\left(t_{ij}\right)} + \lambda J\left(\eta\right),  
\end{equation}
\noindent
for $\eta \in \hilbert$, where the penalty $J$ can be written as a square norm and decomposed as in \eqref{eq:multi-term-penalty}, with
\begin{equation*} 
J\left(\eta \right) = \sum_{\beta} \theta_\beta^{-1}\langle \eta,\eta \rangle_{\hilbert_\beta}.
\end{equation*}
\noindent 
The first term in \eqref{eq:penalized-joint-loglik-given-phi-3} serves as a measure of the goodness of fit of $\eta$ to the data, and only depends on $\eta$ through the evaluation functional $\left[t_{ij}\right]\eta$, so the argument justifying the form of the minimizer in \eqref{eq:form-of-the-minimizer-phi} applies to $\eta$. Let $\mathcal{T} = \bigcup_{i,j} \left\{t_{ij}\right\}$ denote the unique values of the observations times pooled across subjects. The minimizer of the penalized likelihood \eqref{eq:penalized-joint-loglik-given-phi} has the form 
\begin{equation} \label{eq:form-of-smoothing-spline-solution-kappa}
\eta\left( t \right) = \sum_{i = 1}^{d_\Upsilon} d_i \nu_i\left( t \right) + \sum_{j = 1}^{\vert \mathcal{T} \vert} c_j K_1\left(t_j,t\right),
\end{equation}  
\noindent
where $\left\{\nu_i \right\}$ form a basis for the null space $\hilbert_0$ and $K_1\left(t_j,t\right)$ is the reproducing kernel for $\hilbert_1$ evalutated at ${t_j}$, the $j^{th}$ element of $\mathcal{T}$, viewed as a function of $t$.
%%%%%%%%%%%%%%%%%%%%%%%%%%%%%%%%%%%%%%%%%%%%%%%%%%%%%%%%%%%%%%%%%%%%%%%%%%%%%%%%%%%%%%%%%%%


\subsection{Model Fitting}

The Gamma penalized negative log likelihood \eqref{eq:form-of-smoothing-spline-solution-kappa} is non-quadratic, so $\eta_\lambda$ must be computed using iteration even for fixed smoothing parameters. Standard theory for exponential families gives us that the functional 
\begin{align}  \label{eq:penalized-likelihood-functional}
%\begin{split}
L\left( \eta \right) &= \sum_{i = 1}^N \sum_{j = 1}^{p_i} \eta\left(t_{ij}\right)  + \sum_{i = 1}^N \sum_{j = 1}^{p_i} z_{ij} e^{-\eta\left(t_{ij}\right)} 
%\end{split}
\end{align}
\noindent
is continuous and convex in $\eta \in \hilbert$. We assume that the $\vert V \vert \times d_\Upsilon$ matrix $B$ which has $\left(i,j\right)$ element $\nu_j\left(t_i\right)$ is full column rank, so that $L\left(\eta \right)$ is strictly convex in $\hilbert$ and the minimizer of \eqref{eq:penalized-joint-loglik-given-phi} uniquely exists. See \cite{wahba1995smoothing}. 

\bigskip

The minimizer can be computed via Newton iteration using a quadratic approximation of \eqref{eq:penalized-likelihood-functional} at a point $\tilde{\eta}$. Letting $\tilde{u}_{ij} = -z_{ij}e^{-\tilde{\eta}_{ij}}$, the Newton iteration uses the minimizer of the penalized weighted sums of squares
\begin{equation} \label{eq:penalized-weighted-sums-of-squares}
\sum_{i=1}^N\sum_{j=1}^{p_i} \left(\tilde{z}_{ij} - \eta\left(t_{ij}\right)  \right)^2 + \lambda J\left(\eta\right)
\end{equation}
\noindent
to update $\tilde{\eta}$, where $\tilde{z}_{ij} = \tilde{\eta}\left(t_{ij}\right) - \tilde{u}_{ij}$.



%%%%%%%%%%%%%%%%%%%%%%%%%%%%%%%%%%%%%%%%%%%%%%%%%%%%%%%%%%%%%%%%%%%%%%%%%%%%%%%%%%%%%%%%%%%

\subsection{Smoothing Parameter Selection for Exponential Families} \label{smoothing-parameter-selection-exponential-families}
Performance-oriented is a typical choice for method of smoothing parameter selection when data are generated from a distribution belonging to exponential families. This section provides a brief overview of the the performance-oriented iteration, specifically for selecting the optimal degree of smoothing for $\sigma^2$. This approach is just one of many in the inventory of model selection techniques for penalized regression with exponential families. We refer the reader desiring detailed examination to \cite{zhang2006component}, \cite{xiang1996generalized}, \cite{wahba1995smoothing},  \cite{wood2004stable}, and \cite{wood2017generalized}. 

\bigskip

A measure of the discrepancy between distributions belonging to an exponential family having densities of the form $p\left(z\right) = \exp\left\{\left(z \eta - b\left(\eta\right)\right)/a\left(\phi\right) + c\left(z,\phi\right) \right\}$ is the Kullback-Leibler distance
\begin{align}
\begin{split} \label{eq:kl-distance-definition}
\mbox{KL}\left(\eta, \eta_\lambda\right) &= E_\lambda\left[Z \left(\eta - \eta_\lambda \right) - \left(b\left(\eta\right)- b\left(\eta_\lambda\right) \right)\right]/a\left(\phi\right)\\
&=\left[ b'\left(\eta\right) \left(\eta - \eta_\lambda \right) - \left(b\left(\eta\right)- b\left(\eta_\lambda\right) \right)\right]/a\left(\phi\right).
\end{split}
\end{align}
\noindent
For the Gamma distribution, the KL distance simplifies to
\[
\mbox{KL}\left(\eta, \eta_\lambda\right) = -\sigma^2\left( e^{-\eta} - e^{-{\eta_\lambda}}\right) - \left(\eta-{\eta_\lambda}\right),
\]
\noindent
which is not symmetric. Thus, a natural choice of loss function for measuring the performance of an estimator $\eta_\lambda\left(t\right)$ of $\eta \left(t\right)$ is the symmeterized Kullback-Leibler distance averaged over the observed time points $t_{11}, \dots ,  t_{N,p_N}$. For the Gamma distribution, this is given by 
\begin{equation}\label{eq:gamma-SKL-loss-function}
 L\left( \eta,\eta_\lambda \right) \equiv  = \frac{1}{N}\sum_{i=1}^N \left[\frac{1}{p_i}\sum_{j=1}^{p_i}  \left( \frac{\sigma^2\left(t_{ij}\right)}{\sigma^2_\lambda\left(t_{ij}\right)} - \frac{\sigma^2_\lambda \left(t_{ij}\right)}{\sigma^2\left(t_{ij}\right)} - 2\right)\right].
\end{equation}
\noindent
The ideal smoothing parameters are those which minimize \eqref{eq:gamma-SKL-loss-function}. One can derive an unbiased risk estimate $U$ for the Gamma distribution as in Section~\ref{gaussian-unbiased-risk-estimate} for the Gaussian case. Theorem 5.2 in \cite{gu2013smoothing} gives that the minimizer of $U$, which relies only on the data, approximately minimizes and quadratic approximation to \eqref{eq:gamma-SKL-loss-function}. To find the optimal value of the smoothing parameter, the performance-oriented iteration tracks loss $L\left( \eta,\eta_\lambda \right)$ (through $U$) indirectly, simultaneously updating $\lambda, \theta_\beta$. Since it does not explicitly keep track of $L\left( \eta,\eta_\lambda \right)$ itself, it may not be the most effective way to search for the optimal smoothing parameters, but it is numerically efficient. Instead of fixing smoothing parameters and moving according to a particular Newton update, one chooses an update from among a family of Newton updates that is perceived to be better performing according to $L\left( \eta,\eta_\lambda \right)$. If the smoothing parameters stabilize at, say, $\left(\lambda^*,\theta^*_\beta\right)$ and the corresponding Newton iteration converges at $\eta^*$, then it is clear that $\eta^* = \eta_{\lambda^*}$ is the minimizer. In a neighborhood of $\eta^*$ where the corresponding values of the quadratic approximation of $L$ closely approximate the penalized likelihood functional \eqref{eq:penalized-likelihood-functional} for smoothing parameters close to $\left( \lambda^*, \theta^*_\beta \right)$, then the $\eta_{\lambda, \eta^*}$s are, in turn, hopefully close approximations to the $\eta_\lambda$s. Thus, through indirect comparison $\eta^*$ is perceived to be better performing among the other $\eta_\lambda$s in the neighborhood. See Chapter 5, Section 2 of \cite{gu2013smoothing} for thorough discussion.


%The performance-oriented iteration operates on an alternative expression of the symmeterized Kullback-Leibler loss. 
%The mean value theorem gives us that \eqref{eq:gamma-SKL-loss-function} can be written
%\begin{equation}\label{eq:gamma-SKL-loss-function-mvt}
% L\left( \lambda \right) = \frac{1}{N}\sum_{i=1}^N \left[ \frac{1}{p_i}\sum_{j=1}^{p_i} \left( \eta\left(t_{ij}\right) - \eta_\lambda\left(t_{ij}\right)\right)^2\right],
%\end{equation}
%\noindent
%where $\eta^*\left(t_{ij}\right)$ is a convex combination of  $\eta\left(t_{ij}\right)$ and $\eta_\lambda\left(t_{ij}\right)$. 

\bigskip

An alternative to the performance-oriented iteration is to choose the optimal smoothing parameters by comparing candidate $\eta_\lambda$s directly. The generalized approximate cross validation (GACV) score in \cite{xiang1996generalized} keeps track of $L\left( \eta,\eta_\lambda \right)$, approximating the score which is analogous to the generalized cross validation score (GCV) in the usual penalized regression setting \citep{wahba1990spline}. We refer the reader to the aforementioned sources for extensive discussion. For the same reason that we utilized the LosoCV criterion rather than leave-one-out or generalized cross validation for smoothing parameter selection when estimating $\phi$, we did not explore using GACV for model selection for the innovation variance function.

\bigskip

To jointly estimate the autoregressive coefficient function and the innovation variance function, we adopt an iterative approach in the spirit of \cite{huang2006covariance}, \cite{huang2007estimation}, and \cite{pourahmadi2000maximum}. A procedure for minimizing 
\[
-2\ell\left(\phi \vert Y_1, \dots, Y_N , \eta \right) + \lambda_\phi  J_\phi\left(\phi\right) + \lambda_\eta  J_\eta\left(\eta\right)
\]
starts with initializing $\left\{ e^{\eta_{ij}} = \sigma^2_{ij}\right\} = 1$ for $i = 1,\dots, N$, $j = 1,\dots, p_i$.  For fixed $\eta$, we take $\phi^*$ to minimize the penalized negative log likelihood 
\[
-2\ell\left(\phi, \eta\vert Y_1, \dots, Y_N\right) + \lambda_\phi  J_\phi\left(\phi\right).
\]
\noindent
Given $\phi^*$ and setting $\phi = \phi^*$, we update our estimate of $\eta$ by taking $\eta^*$ to minimize the penalized negative log likelihood of the working residuals  
\[
-2\ell\left( \eta \vert Z_1,\dots, Z_N, \phi^* \right) + \lambda_\eta  J_\eta\left(\eta\right).
\]
This process of iteratively updating $\phi^*$ and ${\eta}^*$ is repeated until convergence. 


%%%%%%%%%%%%%%%%%%%%%%%%%%%%%%%%%%%%%%%%%%%%%%%%%%%%%%%%%%%%%%%%%%%%%%%%%%%%%%%%%%%%%%%%%%%
%%%%%%%%%%%%%%%%%%%%%%%%%%%%%%%%%%%%%%%%%%%%%%%%%%%%%%%%%%%%%%%%%%%%%%%%%%%%%%%%%%%%%%%%%%%


 
%==============================================================================================================================================
%==============================================================================================================================================

\chapter{A P-spline Model for the Cholesky Decomposition} \label{psplines-chapter}

%\subsubsection{A tensor product P-spline model for the generalized autoregressive coefficients}


In this chapter, we demonstrate multidimensional smoothing with penalized B-splines, or \textit{P-splines}, as a flexible and computationally convenient alternative to the Hilbert space methods presented in Chapter~\ref{SSANOVA-chapter}. P-spline models are an extension of (generalized) linear regression models. They exploit the attractive properties of the B-spline basis along with the use of computationally convenient difference penalties. The formulation of the penalty is independent of the basis, which provides added modeling flexibility due to the ease with which one can employ various types of regularization. The B-spline functions have compact support, making them more attractive than the smoothing spline basis when the function to be estimated exhibits compact support as well, such as covariance matrices having banded Cholesky factor. Despite their flexibility, fitting P-spline models are only as computationally intensive as in regression modeling.  

\section{Tensor Product B-splines for Multidimensional Smoothing}

Splines are piecewise polynomial functions, where the piecewise polynomials are joined at certain values of the domain called knots. B-splines are a basis for splines. Given a set of knots, B-splines can be easily computed recursively for any polynomial degree (see \cite{de1978practical} and \cite{dierckx1995curve}). The smoothness of a fitted curve can be controlled by the number of B-splines used in the basis expansion used to approximate the curve. Fewer knots (thus, fewer basis functions) lead to smoother fits, and there is an extensive body of research focused on the choice of knot placement. Some authors have proposed adaptive smoothing techniques which attempt to automatically optimize the number and the positions of the knots; see \cite{friedman1989flexible}, \cite{kooperberg1991study}. However, this problem is nontrivial and requires nonlinear optimization, and is still an open problem today. However, limiting the number of B-splines is not the only approach to controlling the complexity of the fitted function.

\bigskip

Instead, \cite{eilers1996flexible} propose alternative an approach to nonparametric smoothing based on finite difference penalties. The difference penalties are trivial to compute and can be done so independently of the basis, unlike the smoothing spline penalty functional \eqref{eq:SS-penalty-functional}. Their approach circumvents the choice of knot specification. They achieve smoothness in fitted functions by purposefully overfitting the smooth coefficient vectors using a B-spline basis with a large number of equally spaced knots.  Augmenting the log likelihood with the difference penalty prevents overfitting and accommodates a potentially ill-conditioned fitting procedure. 

\bigskip

Analogous to the smoothing spline representation \eqref{eq:form-of-the-minimizer-phi}, we can represent $\phi$ using a B-spline basis. But first, in order to illustrate the ideas in the sections to follow, it is pragmatic to first review some basic properties of B-splines. For an exhaustive and more formal mathematical review, see  \cite{de1978practical} and \cite{dierckx1995curve}. A B-spline is a function constructed from piecewise polynomial functions which are connected in a very particular way. Their values can be computed recursively; for a non-decreasing sequence of knots $\left\{t_i\right\}$, the value of the $i^{th}$ B-spline of order $k$ can be defined using

\begin{align} 
\begin{split} \label{eq:bspline-recursive-relation}
B_{i1}\left(x\right) &= \left\{ \begin{array}{ll}
1, & t_i \le x < t_{i+1}\\
0, & otherwise
\end{array} \right.
\\
B_{ik}\left(x\right) &= \frac{x-t_i}{t_{i+k-1}-t_i}B_{i,k-1}\left(x\right) + \frac{t_{i+k}-x}{t_{i+k}-t_{i+1}}B_{i+1,k-1}\left(x\right). 
\end{split}
\end{align}

Figure~\ref{fig:overlapping-linear-cubic-bsplines} shows two sets of B-splines; the top facet displays linear B-splines and the bottom displays B-splines of degree 2. A single isolated B-spline is shown on the left side of the axis in each panel. In Figure~\ref{fig:overlapping-linear-bsplines}, the single B-spline of degree 1 consists of two linear pieces: one piece from $x_3$ to $x_4$, and the other from $x_4$ to $x_5$, which are the knots that define its support. In the right part of Figure~\ref{fig:overlapping-linear-bsplines}, three more B-splines of degree 1 are shown. Each one based on three knots. Comparing these with the overlapping quadratic B-splines in Figure~\ref{fig:overlapping-cubic-bsplines}, we can see that the extent to which neighboring B-splines overlap depends on the polynomial degree of the basis. 

\begin{figure}[H]
 \begin{center}
 \begin{subfigure}[t]{\textwidth}
  \centering
   \includegraphics[width=0.75\textwidth]{img/uni_linear_bsplines}
 \caption{\textit{B-splines of degree 1} }\label{fig:overlapping-linear-bsplines}
  \end{subfigure}
   \end{center}
  \hfill
  \begin{center}
 \begin{subfigure}[t]{0.75\textwidth}
\includegraphics[width = \textwidth]{img/uni_cubic_bsplines}
 \caption{\textit{B-splines of degree 2}}
\label{fig:overlapping-cubic-bsplines}
 \end{subfigure}
 \end{center}
   \caption{\textit{ On the left: a single, isolated B-spline basis function, and on the right: several overlapping B-splines.  }}\label{fig:overlapping-linear-cubic-bsplines}
\end{figure}

B-splines make attractive basis functions for nonparametric regression; a linear combination of B-spline basis functions gives a smooth curve. Once a B-spline basis is computed, their application is no more difficult than polynomial regression, and extension to two-dimensional smoothing is available with the use of tensor products. To construct a B-spline representation for $\phi$, we need to equip the $l$ and $m$ axes each with a B-spline basis: let

\[
{B_l}_{_1}\left(l\right),\dots, {B_l}_{_{k_l}}\left(l\right)  \mbox{ and } {B_m}_{_1}\left(m\right),\dots, {B_m}_{_{k_m}}\left(m\right)
\]
\noindent
denote the B-spline bases for $l$ and $m$, each having a set of equally spaced knots along their respective domain. It is worth noting that one is free to specify a different basis for each dimension either by using different order B-spline or using different numbers of knots. Order of the basis will be indicated only when necessary and otherwise suppressed to maintain simplicity of notation. The tensor product basis functions
\begin{equation*}
T_{kk'}\left(l,m\right) = {B_l}_{_k}\left(l\right){B_m}_{_{k'}}\left(m\right)
\end{equation*}
\noindent
carve the $l$-$m$ domain into rectangles. Figure~\ref{fig:bicubic-bspline-function} shows a single $T_{kk'}$, where the marginal B-spline bases are of degree 2. For a given knot grid, we can approximate $\phi$ by
\begin{equation} \label{eq:tensor-product-bspline-expansion-phi}
\phi\left(l,m\right) = \sum_{r=1}^{k_l} \sum_{c=1}^{k_m} \theta_{rc} {B_l}_{_r}\left(l\right) {B_m}_{_c}\left(m\right). 
\end{equation}
\noindent
%Substituting this expression for $\phi$ in the negative log likelihood \eqref{eq:full-joint-likelihood} gives
%\begin{align}
%\begin{split} \label{eq:full-joint-likelihood-BS}
%-2\ell\left(\theta_{11},\dots, \theta_{k_l,k_m} \vert Y_1,\dots, Y_N, \sigma^2\right)  = \sum_{i=1}^N \sum_{j=2}^{p_i} \log\sigma_{ij}^2 +  \sum_{i=1}^N \sum_{j=2}^{p_i} \frac{1}{\sigma^{2}_{ij}}\left( y_{ij} - \sum_{k<j}\left[ \sum_{r=1}^{k_l} \sum_{c=1}^{k_m} \theta_{i'j'} {B_l}_{_{r}}\left(l_{ijk}\right) {B_m}_{_{c}}\left(m_{ijk}\right)\right] y_{ik}  \right)^2.
%\end{split}
%\end{align}
% For fixed $\sigma^2$ as in \eqref{eq:penalized-joint-loglik-given-sigma}, we define the estimator of $\phi$ as the minimizer of 
%\begin{equation} 
%-2\ell\left(\phi \vert Y_1,\dots, Y_N\right) + \lambda J_{\lambda}\left(\phi\right) = \sum_{i=1}^N \sum_{j=2}^{p_i} \sigma^{-2}_{ij}\left( y_{ij} - \sum_{k<j} \phi\left(\bfv_{ijk}\right) y_{ik}  \right)^2 + \lambda J\left( \phi \right),
%\end{equation} 
%\noindent
%where $J\left(\phi\right)$ is a penalty on the roughness of the fitted function. 
%
%
\begin{figure}[H]
    \begin{center}
 \begin{subfigure}[t]{0.48\textwidth}
  \centering
  \includegraphics[width=\textwidth]{img/bicubic_basis_function}
  \end{subfigure}
 \begin{subfigure}[t]{.48\textwidth}
  \centering
\includegraphics[width=\textwidth]{img/bicubic_bspline_contour}
 \end{subfigure}
  \caption{\textit{ Tensor product of two quadratic B-splines }}\label{fig:bicubic-bspline-function}
 \end{center}
\end{figure}

%%%%%%%%%%%%%%%%%%%%%%%%%%%%%%%%%%%%%%%%%%%%%%%%%%%%%%%%%%%%%%%%%%%%%%%%%%%%%%%%%%%%%%%%%%%
%%%%%%%%%%%%%%%%%%%%%%%%%%%%%%%%%%%%%%%%%%%%%%%%%%%%%%%%%%%%%%%%%%%%%%%%%%%%%%%%%%%%%%%%%%%
%%%%%%%%%%%%%%%%%%%%%%%%%%%%%%%%%%%%%%%%%%%%%%%%%%%%%%%%%%%%%%%%%%%%%%%%%%%%%%%%%%%%%%%%%%%
%%%%%%%%%%%%%%%%%%%%%%%%%%%%%%%%%%%%%%%%%%%%%%%%%%%%%%%%%%%%%%%%%%%%%%%%%%%%%%%%%%%%%%%%%%%

\section{Difference Penalties}

The specification of a P-spline model provides a simple way to avoid the issue of optimal knot selection for the $l$ and $m$ bases. This is done by constructing the marginal B-spline bases with a large number of knots - more than necessary. Application of a smoothness penalty prevents overfitting. A penalty based on discrete differences of the B-spline coefficients controls the fit in a way much like the classical second derivative penalty \eqref{eq:SS-penalty-functional} does. However, its construction is trivial even when the number of basis functions is very large. Using the properties of B-splines, it is straightforward to show that the difference penalty of order $d$ approximates the integrated square of the $d^{th}$ derivative well, so little is lost by using it in place of the derivative-based penalty. \cite{o1986statistical} established that for $f\left(x\right) = \sum\limits_{j=1}^k \theta_j B_j\left(x\right)$, one can derive a banded matrix $P$ using the properties of B-splines such that $J\left(f\right) = \int_0^1 \left( f^{\prime \prime}\left(x\right)\right)^2$ can be written
 \[
J\left(f\right) = \theta^\prime P \theta
 \] 
\noindent
where $\theta = \left(\theta_1,\dots, \theta_k\right)$ denotes the vector of B-spline basis coefficients. The $\left(i,j\right)$ element of the penalty matrix $P$ is given by
 \[
 p_{ij} = \int_0^1 B_i^{\prime \prime}\left(x\right) B_j^{\prime \prime}\left(x\right).
 \]
\cite{wand2008semiparametric} extend the work in \cite{o1986statistical} to higher order derivatives for general degree B-splines and derive an exact matrix algebraic expression for the penalty matrices.  The computation of $P$ is nontrivial and becomes very tedious when the third and fourth derivative are used as the roughness measure. In the cubic case, the expression is a result of the application of Simpson's Rule applied to the inter-knot differences since each $B_i^{\prime \prime} B_j^{\prime \prime}$ is a piecewise quadratic function. The penalty may be written
 \[
 P = \left(B^{\prime \prime}\right)^\prime \textup{diag}\left(\omega \right) B^{\prime \prime}, 
 \]
 \noindent
 where $B^{\prime \prime}$ is the $3\left( k + 7 \right) \times \left( k + 4 \right)$ matrix with $i$-$j^{th}$ entry given by $B_j^{\prime \prime} \left(x_i^*\right)$, $x^*_i$ is the $i^{th}$ element of 
\[
\left( \theta_1,\frac{\theta_1+\theta_2}{2},\theta_2,\theta_2,\frac{\theta_2+\theta_3}{2},\theta_3,\dots,\theta_{k+7},\frac{\theta_{k+7}+\theta_{k+8}}{2},\theta_{k+8} \right),
\]
 \noindent
 and $\omega$ is the $3\left(k+7\right) \times 1$ vector given by
\begin{align*}
\omega &= \left( \frac{1}{6}\left(\Delta \theta \right)_1,\frac{4}{6}\left(\Delta \theta \right)_1, \frac{1}{6}\left(\Delta \theta \right)_1,\frac{1}{6}\left(\Delta \theta \right)_2, \frac{4}{6}\left(\Delta \theta \right)_2,  \right. \\
&\qquad   \left. {} \frac{1}{6}\left(\Delta \theta \right)_2 , \dots , \frac{1}{6}\left(\Delta \theta \right)_{n+7}, \frac{4}{6}\left(\Delta \theta \right)_{k+7}, \frac{1}{6}\left(\Delta \theta \right)_{k+7}  \right) \\
\end{align*}
\noindent
where $\left(\Delta \theta \right)_j = \theta_{j+1}-\theta_j$. They generalize this to the case of any order penalty and present a table of formulas for constructing any arbitrary penalty matrix, $P$.  

\bigskip

Alternatively, \cite{eilers1996flexible} replace the curvature penalty \eqref{eq:SS-penalty-functional} with a finite difference penalty on the B-spline coefficients. They suggest enforcing smoothness of fitted functions $f\left(x\right) = \sum\limits_{j=1}^k \theta_i B_j\left(x\right)$ using the $d^{th}$ order difference penalty:
\begin{equation} \label{eq:bspline-difference-penalty-1}
J_d\left( f \right) = \sum_{j=d}^k \left(\Delta^d \theta_j\right)^2.
\end{equation} 
\noindent
where $\Delta \theta_j = \theta_j - \theta_{j-1}$, and $\Delta^2 \theta_j = \Delta\left(\Delta \theta_j\right) = \theta_j - 2\theta_{j-1} + \theta_{j-2}$. In general, $\Delta^d \theta_j = \Delta\left(\Delta^{d-1} \theta_j \right)$. Let $D_d$ denote the differencing operator:
\[
D_d\theta = \Delta^d \theta.
\]
\noindent
Then, \eqref{eq:bspline-difference-penalty-1} can be written in terms of the squared norm of the difference operator applied to the vector of B-spline coefficients:
\begin{align} 
\begin{split} \label{eq:bspline-difference-penalty-2}
J_d\left( f \right) &= \vert \vert D_d\theta \vert \vert^2 \\
&= \theta^\prime P_d \theta
\end{split}
\end{align}
\noindent
where $P_d = D_d^\prime D_d$.  The connection between the second-derivative penalty to the penalty on second-order differences of the B-spline coefficients can be established with straightforward calculus and the recursive property of the B-spline basis functions.  The derivative properties of B-splines permits the traditional smoothness penalty applied to $f$ to be written 
%\begin{equation*} 
%\int_0^1 \left( f^{\prime \prime}\left(x\right)\right)^2\;dx = \int_{0}^{1} \left\{ \sum\limits_{j=1}^k  \theta_j B_{j3}^{\prime \prime}\left(x\right) \right\}^2\; dx.
%\end{equation*}
%\noindent
\begin{equation*} \label{eq:second-derivative-bspline-penalty}
\int_0^1 \left( f^{\prime \prime}\left(x\right)\right)^2\;dx =  \int_{0}^{1}  \bigg[ \sum\limits_{i=1}^k \sum\limits_{j=1}^k \Delta^2 \theta_i \Delta^2 \theta_j B_{i,1}\left(x\right)B_{j,1}\left(x\right) \;dx\bigg]. 
\end{equation*}
\noindent
where $B_{j,1}\left(x\right)$ is the $j^{th}$ B-spline of order 1. Most of the cross products of $B_{i,1}$ and $B_{j,1}$ vanish since B-splines of degree 1 only overlap when $j$ is $i-1$, $i$, or $i+1$. Thus, we have that
\begin{align}
\begin{split}
\int_0^1 \left( f^{\prime \prime}\left(x\right)\right)^2\;dx  = {} &  \int_0^1 \bigg[ \left( \sum\limits_{j} \Delta^2 \theta_j  B_{j,1}\left(x\right) \right)^2  + 2 \sum_{j}\Delta^2 \theta_j\Delta^2 \theta_{j-1}B_{j,1}B_{j-1,1}\left(x\right) \bigg]\;dx\\ 
= {} & \sum \limits_{j}  \left( \Delta^2\theta_j \right)^2 \int_0^1 B_{j,1}^2\left(x\right)\ + 2 \sum\limits_{j} \Delta^2 \theta_j\Delta^2 \theta_{j-1,1} \int_0^1 B_{j,1}\left(x\right)B_{j-1,1}\left(x\right)\;dx. 
\end{split}
\end{align}
\noindent
This can be written as
\begin{equation} \label{eq:derivative-penalty-difference-penalty-connection}
\int_0^1 \left( f^{\prime \prime}\left(x\right)\right)^2  = c_1 \sum\limits_{j} \left( \Delta^2 \theta_j\right)^2 + c_2 \sum\limits_{j} \Delta^2 \theta_j\Delta^2 \theta_{j-1}.
\end{equation}
\noindent
Given a set of equidistant knots, the constants $c_1$ and $c_2$ are given by
\begin{equation}
\begin{split}
c_1 & =   \int_0^1 \left(B_{j,1}\left(x\right)\right)^2\;dx \mbox{ and}\\
c_2 & = \int_0^1 B_{j,1}\left(x\right)B_{j-1,1}\left(x\right) \;dx.
\end{split}
\end{equation}
This establishes that traditional smoothness penalty on the squared second derivative can be written as a linear combination of a penalty on the second-order differences of the B-spline coefficients \eqref{eq:bspline-difference-penalty-1} and the sum of the cross products of neighboring second differences. The second term in \eqref{eq:derivative-penalty-difference-penalty-connection} leads to a complex objective function when minimizing the penalized likelihood, where seven adjacent spline coefficients occur, as opposed to five if only the first term in \eqref{eq:derivative-penalty-difference-penalty-connection} is used in the penalty. The added complexity is a consequence of overlapping B-splines, which quickly increases when using higher order differences and higher order B-splines. 

\bigskip

A smoother sequence of coefficients leads to a smoother curve, as illustrated in Figure~\ref{fig:increasing-lambda-pspline-fits}.  The relationship between P-spline curves and their coefficients is easily characterized if we consider the coefficients as the skeleton of the function, and draping the B-splines over them puts the flesh on the bones, so to speak. As long as the coefficient sequence is smooth, the number of basis functions (and coefficients) is unimportant since the penalty ensures the smoothness of the skeleton and that the fitting procedure is well-conditioned.  

\begin{figure}[H] 
\begin{subfigure}{.4\textwidth}
  \centering
   \graphicspath{{img/}}
  \includegraphics[scale=0.4]{pspline_pord2_xsmall_lambda.png}
  \label{fig:pspline_small_lambda}
\end{subfigure}
\begin{subfigure}{.5\textwidth}
  \centering
   \graphicspath{{img/}}
  \includegraphics[scale=0.4]{pspline_pord2_small_lambda.png}
  \label{fig:pspline_small_lambda}
\end{subfigure}
\begin{subfigure}{.5\textwidth}
  \centering
   \graphicspath{{img/}}
  \includegraphics[scale=0.4]{pspline_pord2_medium_lambda.png}
  \label{fig:pspline_small_lambda}
\end{subfigure}
\begin{subfigure}{.5\textwidth}
  \centering
   \graphicspath{{img/}}
  \includegraphics[scale=0.4]{pspline_pord2_large_lambda.png}
  \label{fig:pspline_small_lambda}
\end{subfigure}
\caption{\textit{Illustration of the impact of the second order difference penalty. The number of B-splines used is the same in each plot, with the value of the penalty parameter increasing from left to right and top to bottom across each plot. The red circles are the values of each of the B-spline coefficients; as the penalty increases, they form as smoother sequence as we move across the four plots, which results in a smoother fitted function. As the penalty parameter approaches infinity, the fit approaches a linear function as shown in the bottom right plot.}}
\label{fig:increasing-lambda-pspline-fits}
\end{figure}
The limiting P-spline fit approaches a polynomial as the smoothing parameter tends to infinity. Under a difference penalty of order $d$, the fitted function will approach a polynomial of degree $\left(d-1\right)$ for large values of the smoothing parameter as long as the degree of the B-splines is greater than or equal to $k$. Figure~\ref{fig:PS-difference-order-demo} demonstrates the impact of the order of the penalty on the fitted function as the smoothing parameter increases. To verify this mathematically, we need to use the relationship between the differenced coefficient sequence and the derivative of a B-spline. See Appendix~\ref{psplines-appendix}. Consider using the second-order difference penalty. When $\lambda$ is large, the penalty dominates the penalized likelihood, so that the minimizer $\theta$ must be such that $\sum\limits_{j}\left(\Delta^2\theta_j\right)^2$ is close to zero. Consequently, each of the individual second differences must also be nearly zero, and thus the second derivative of the fitted function must be close to zero over the entire domain.
\begin{figure}[H]
\begin{subfigure}{.5\textwidth}
  \centering
   \graphicspath{{img/}}
  \includegraphics[scale=0.45]{PS_penalty_section_figure_6_order_0.png}
  %\label{fig:pspline_small_lambda}
\caption{$d=0$ }
\end{subfigure}
\begin{subfigure}{.5\textwidth}
  \centering
   \graphicspath{{img/}}
  \includegraphics[scale=0.45]{PS_penalty_section_figure_6_order_1.png}
 % \label{fig:pspline_small_lambda}
\caption{$d=1$}
\end{subfigure}
\begin{subfigure}{.5\textwidth}
  \centering
   \graphicspath{{img/}}
  \includegraphics[scale=0.45]{PS_penalty_section_figure_6_order_2.png}
  %\label{fig:pspline_small_lambda}
\caption{$d=2$}
\end{subfigure}
\begin{subfigure}{.5\textwidth}
  \centering
   \graphicspath{{img/}}
  \includegraphics[scale=0.45]{PS_penalty_section_figure_6_order_3.png}
  %\label{fig:pspline_small_lambda}
\caption{$d=3$}
\end{subfigure}
\caption{\textit{Illustration of the impact of the order of the difference penalty. The number of B-splines used is the same in each plot, with the penalty parameter varying from across the same grid of values. The fitted curves in the upper left plot correspond to the difference penalty of order $0$, where $\vert D_0 \theta \vert^2 = \sum_{i} \theta_i^2$, analogous to ridge regression using the B-spline basis as regression covariates. The fitted curves approach polynomials of degree $d-1$ as $\lambda \rightarrow \infty$.}} \label{fig:PS-difference-order-demo}
\end{figure}


%%%%%%%%%%%%%%%%%%%%%%%%%%%%%%%%%%%%%%%%%%%%%%%%%%%%%%%%%%%%%%%%%%%%%%%%%%%%%%%%%%%%%%%%%%%
%%%%%%%%%%%%%%%%%%%%%%%%%%%%%%%%%%%%%%%%%%%%%%%%%%%%%%%%%%%%%%%%%%%%%%%%%%%%%%%%%%%%%%%%%%%
%%%%%%%%%%%%%%%%%%%%%%%%%%%%%%%%%%%%%%%%%%%%%%%%%%%%%%%%%%%%%%%%%%%%%%%%%%%%%%%%%%%%%%%%%%%
%%%%%%%%%%%%%%%%%%%%%%%%%%%%%%%%%%%%%%%%%%%%%%%%%%%%%%%%%%%%%%%%%%%%%%%%%%%%%%%%%%%%%%%%%%%
%%%%%%%%%%%%%%%%%%%%%%%%%%%%%%%%%%%%%%%%%%%%%%%%%%%%%%%%%%%%%%%%%%%%%%%%%%%%%%%%%%%%%%%%%%%
%%%%%%%%%%%%%%%%%%%%%%%%%%%%%%%%%%%%%%%%%%%%%%%%%%%%%%%%%%%%%%%%%%%%%%%%%%%%%%%%%%%%%%%%%%%
%%%%%%%%%%%%%%%%%%%%%%%%%%%%%%%%%%%%%%%%%%%%%%%%%%%%%%%%%%%%%%%%%%%%%%%%%%%%%%%%%%%%%%%%%%%
%%%%%%%%%%%%%%%%%%%%%%%%%%%%%%%%%%%%%%%%%%%%%%%%%%%%%%%%%%%%%%%%%%%%%%%%%%%%%%%%%%%%%%%%%%%

\section{The P-spline Estimator of the Generalized Autoregressive Varying Coefficient}

To extend the use of the difference penalty \eqref{eq:bspline-difference-penalty-1} to the bivariate setting, the only necessary modification to the one-dimensional differencing procedure is the addition of a second difference penalty so that there is a penalty for each variable, $l$ and $m$. Let $\Theta$ denote the $k_l \times k_m$ matrix of basis coefficients $\left\{\theta_{rc}\right\}$. For given $\Theta$, the fitted value $\phi\left(l,m\right)$ may be written 
\[
\sum_{r=1}^{k_l} \sum_{c=1}^{k_m} \theta_{rc} {B_l}_{_{r}}\left(l\right){B_m}_{_{c}}\left(m\right).
\]
\noindent
Let $\lambda = \left(\lambda_l, \lambda_m\right)$ denote the tuple of smoothing parameters for the $l$ and $m$ dimensions, respectively. We take $\phi_\lambda$ to be the minimizer of 
\begin{align} 
\begin{split}\label{eq:folded-difference-penalty-log-likelihood}
&-2\ell\left(\phi \vert Y_1,\dots, Y_N, \sigma^2\right) + J_{\lambda}\left(\phi\right) = \sum_{i=1}^N \sum_{j=2}^{p_i} \frac{1}{\sigma^{2}\left({t_{ij}}\right)} \left(y_{ij} - \sum_{k=1}^{j-1} \left( \sum_{r=1}^{k_l} \sum_{c=1}^{k_m} \theta_{rc} B_r\left(l_{ijk}\right)B_c\left(m_{ijk}\right)\right)y_{ik} \right)^2 \\ 
&\phantom{{} - 2\ell\left(\phi \vert Y_1,\dots, Y_N\right) + J_{\lambda}\left(\phi\right) =  } + \lambda_l \sum_{r=1}^{k_l} \vert\vert \theta_{r \cdot}  D_{d_{\ms l}} \vert\vert^2 + \lambda_m \sum_{c=1}^{k_m}\vert \vert  D_{d_{\ms m}} \theta_{\cdot c} \vert\vert^2,
\end{split}
\end{align}
\noindent
where $\theta_{r \cdot}$ and $\theta_{\cdot c}$ denote the $r^{th}$ row and $c^{th}$ column of $\Theta$, respectively. The second term in \eqref{eq:folded-difference-penalty-log-likelihood} imposes a difference penalty of order $d_{\ms l}$ on the rows of the coefficient matrix, while the third term places a difference penalty of order $d_{\ms m}$ on the columns. 

\bigskip

The penalized log likelihood is quadratic in $\theta = \left(\theta_{11}, \dots, \theta_{k_l, k_m} \right)'$. Demonstration of computation is simple if we express the coefficient matrix $\Theta$ in ``unfolded'' notation so that we can write the mean of the stacked response vector $Y$ as defined in \eqref{eq:stacked-response-vector} as in the usual multiple regression form
\begin{equation*}
E\left[Y \right] = X\mbox{vec}\left\{\phi\left( \bfv \right)\right\} = X B \theta,
\end{equation*}
\noindent
where $\theta = \mbox{vec}\left( \Theta \right)$ denotes the vectorized coefficient matrix constructed by stacking the columns of $\Theta$. The $\vert V \vert \times k_l k_m$ tensor product basis $B$ is constructed from the tensor product of the marginal B-spline bases defined in \citet{eilers2006fast} as the \textit{row-wise Kronecker product} of the individual bases:
\begin{equation} \label{eq:rowwise-kronecker-product}
B = B_m \square B_l = \left( B_m \otimes 1^\prime_{k_m} \right) \odot \left(1^\prime_{k_l} \otimes  B_l  \right).
\end{equation}
\noindent
The operator $\odot$ denotes the element-wise matrix product; $1_{k_l}$ ($1_{k_m}$) denotes the column vector of ones having length $k_l$ ($k_m$.) The operations in \eqref{eq:rowwise-kronecker-product} construct $B$ such that the $i^{th}$ row of $B_m\square B_l$ is the Kronecker product of the corresponding rows of $B_m$ and $B_l$. We can compactly express the penalty in \eqref{eq:folded-difference-penalty-log-likelihood} by writing
\begin{equation*} \label{eq:tensor-product-penalty}
\lambda_l \vert \vert P_l \theta \vert \vert^2 + \lambda_m \vert \vert P_m \theta \vert\vert^2
\end{equation*}
\noindent
where $P_l = I_{k_m} \otimes D'_{d_{\ms l}} D_{d_{\ms l}} $ and $P_m =  D'_{d_{\ms m}} D_{d_{\ms m}} \otimes I_{k_l}$. The $n_Y \times k_lk_m$  matrix $X$ is defined as before \eqref{eq:ar-design-matrix-1}. Then the log likelihood \eqref{eq:folded-difference-penalty-log-likelihood} can be written as
\begin{equation} \label{eq:tensor-pspline-objective-function}
-2\ell\left(\phi \vert Y_1,\dots, Y_N\right) + J_{\lambda}\left(\phi\right) = \left( Y - XB\theta\right)^\prime D^{-1}\left( Y - XB\theta\right)  + \lambda_l\vert\vert P_l \theta \vert\vert^2 + \lambda_m \vert\vert P_m \theta\vert \vert^2.
\end{equation}
\noindent
Taking derivatives and setting equal to zero gives normal equations:
\begin{equation} \label{eq:tensor-pspline-normal-equations}
\left[ \left(XB\right)^\prime D^{-1} XB +  \lambda_l P_l+ \lambda_m P_m\right]\theta = \left(X B\right)^\prime D^{-1}Y.
\end{equation}
\noindent
The solution $\phi_\lambda$ is given by $\phi_\lambda\left(\bfv\right) = \sum_{r=1}^{k_l} \sum_{c=1}^{k_m} \hat{\theta}_{rc} {B_l}_{_{r}}\left(l\right){B_m}_{_{c}}\left(m\right)$, where
\begin{equation} 
\hat{\theta} = \left[ \left(XB\right)^\prime D^{-1} XB +  \lambda_l P_l+ \lambda_m P_m\right]^{-1} \left(X B\right)^\prime D^{-1}Y.
\end{equation}
\noindent
We note that the size of the system of equations \eqref{eq:tensor-pspline-normal-equations} which determine the basis coefficients remains fixed at $k_l k_m$, even as the number of observations increases. The grid of regression coefficients can be recovered by arranging the elements of $\hat{\theta}$ into a matrix of $k_l$ columns having length $k_m$. The vector of fitted values is given by 
\begin{align}
\begin{split} \label{eq:pspline-smoothing-matrix}
\hat{Y} = AY  = X \left[ \left(XB\right)^\prime D^{-1} XB +  \lambda_l P_l+ \lambda_m P_m\right]^{-1} \left(X B\right)^\prime D^{-1}Y,
\end{split}
\end{align}
\noindent
where $A = X \left[ \left(XB\right)' D^{-1} XB +  \lambda_l P_l+ \lambda_m P_m\right] \left(X B\right)' D^{-1}$ is the ``smoothing'' matrix, analogous to the smoothing matrix $\tilde{A}$ \eqref{eq:smoothing-matrix-A-tilde} for the smoothing spline estimator in Chapter~\ref{SSANOVA-chapter}. Its use in smoothing parameter selection and model tuning is similar to the reproducing kernel Hilbert space framework, which we will discuss in the next section.

\bigskip

It is important to note that the construction of the tensor product basis $B$ and penalty matrix $P$ requires special care in this setting, where the domain of ${\phi}\left(l,m\right)$ is restricted to the region satisfying $0 \le s < t \le 1$, which is shown in Figure~\ref{fig:triangle-domain}.

\begin{figure}[H]
    \graphicspath{{img/}}
 \includegraphics[scale=0.2]{knot-removal-on-triangle-domain.png}
 \caption{$\frac{l}{2} < m < 1 - \frac{l}{2}, \quad 0 < l < 1.$} \label{fig:triangle-domain}
 \end{figure}

When the tensor product basis is constructed on the regular grid defined by the cartesian product of the knots of the marginal bases $B_l$ and $B_m$, a large number of basis functions anchored are at knots near which we have no data, so there is little information about the corresponding basis coefficient. As a result, the resulting tensor product matrix can be ill-conditioned and solving \eqref{eq:tensor-pspline-normal-equations} results in singularities. In this case, the quality of the estimator can suffer terribly. To correct for this instability, one can simply remove the knots corresponding to tensor products functions which do not overlap with the function domain from the basis, $B$, and trimming the penalty matrices $P_l$ and $P_m$ as needed. With the trimmed basis and penalties, optimization can be carried out as previously discussed. 

\bigskip

Triangular B-splines are a new tool for modeling complex objects with nonrectangular topology. The scheme is based on blending functions and control points, and lets us model piecewise pol
Bivariate B-splines are useful for smoothing over arbitrary domains, making them a natural alternative for the construction of a basis for $\bfv = \left(l,m\right)$. Multidimensional B-splines are well-developed by mathematicians, but they are rarely used in the statistical community. To smooth over non-rectangular domains, the domain is approximated by a set of triangles, or a triangulation, where each triangle is defined by its three vertices. The B-splines are defined according to a set of triples which correspond to set of knots over the bivariate domain. See \citet{dahmen1992blossoming} and \citet{seidel1991symmetric} for details. 

%%%%%%%%%%%%%%%%%%%%%%%%%%%%%%%%%%%%%%%%%%%%%%%%%%%%%%%%%%%%%%%%%%%%%%%%%%%%%%%%%%%%%%%%%%%
%%%%%%%%%%%%%%%%%%%%%%%%%%%%%%%%%%%%%%%%%%%%%%%%%%%%%%%%%%%%%%%%%%%%%%%%%%%%%%%%%%%%%%%%%%%
%%%%%%%%%%%%%%%%%%%%%%%%%%%%%%%%%%%%%%%%%%%%%%%%%%%%%%%%%%%%%%%%%%%%%%%%%%%%%%%%%%%%%%%%%%%
%%%%%%%%%%%%%%%%%%%%%%%%%%%%%%%%%%%%%%%%%%%%%%%%%%%%%%%%%%%%%%%%%%%%%%%%%%%%%%%%%%%%%%%%%%%
%%%%%%%%%%%%%%%%%%%%%%%%%%%%%%%%%%%%%%%%%%%%%%%%%%%%%%%%%%%%%%%%%%%%%%%%%%%%%%%%%%%%%%%%%%%
%%%%%%%%%%%%%%%%%%%%%%%%%%%%%%%%%%%%%%%%%%%%%%%%%%%%%%%%%%%%%%%%%%%%%%%%%%%%%%%%%%%%%%%%%%%
%%%%%%%%%%%%%%%%%%%%%%%%%%%%%%%%%%%%%%%%%%%%%%%%%%%%%%%%%%%%%%%%%%%%%%%%%%%%%%%%%%%%%%%%%%%
%%%%%%%%%%%%%%%%%%%%%%%%%%%%%%%%%%%%%%%%%%%%%%%%%%%%%%%%%%%%%%%%%%%%%%%%%%%%%%%%%%%%%%%%%%%


%%==============================================================================================================================================
%%==============================================================================================================================================
%%==============================================================================================================================================
%%==============================================================================================================================================
%%==============================================================================================================================================
%%%==============================================================================================================================================


%%==============================================================================================================================================
%%==============================================================================================================================================
%%==============================================================================================================================================
%%==============================================================================================================================================
%%==============================================================================================================================================
%%==============================================================================================================================================

%%==============================================================================================================================================
%%==============================================================================================================================================
%%==============================================================================================================================================
%%==============================================================================================================================================
%%==============================================================================================================================================
%%==============================================================================================================================================

%\subsection{The reguarlized MLE for $\phi$ via tensor product P-splines}
%% \subfile{chapter-3-subfiles/chapter-3-tensor-product-pspline-MLE}
%To extend the P-spline framework to allow estimation of a bivariate function, $\phi$, we simply need to equip the $l$ and $m$ axes each with a B-spline basis. A basis for the varying coefficient function is constructed taking the tensor product of the two marginal bases. Let 
%\[
%B_{1}\left(l\right),\dots, B_{K}\left(l\right)  \mbox{ and } B_{1}\left(m\right),\dots, B_{L}\left(m\right)
%\]
%denote the B-spline bases for $l$ and $m$, each having a set of equally spaced knots along their respective domain. It is worth noting that while we have chosen not to distinguish between $\left\{ B_k \right\}$ and $\left\{ {B}_l \right\}$ for the sake of brevity, one is free to specify a different basis for each dimension either by using different order B-spline or, of course, using different numbers of knots, and hence entirely different knot sequences since P-splines rely on bases with equally spaced knots. The tensor product basis functions
%\begin{equation*}
%T_{jk}\left(l,m\right) = B_j\left(l\right){B}_k\left(m\right)
%\end{equation*}
%\noindent
%carve the $l$-$m$ domain into rectangles.  Figure~\ref{fig:sparse_bicubic_BS_basis} shows a thinned tensor product basis $\left\{ T_{kl} \right\}$; a portion of the basis was omitted to eliminate overlapping of the basis functions so that the reader can identify individual tensor products. Each ``hill'' in Figure~\ref{fig:sparse_bicubic_BS_basis} is associated with an unknown coefficient $\theta_{ij}$ which determines the height of the hill. For a given knot grid, we can approximate a surface by
%
%\begin{equation} \label{eq:varying-coefficient-tensor-product-expansion}
%\phi\left(l,m\right) = \sum_{i=1}^K \sum_{j=1}^L \theta_{ij} B_{i}\left(l\right) B_{j}\left(m\right), 
%\end{equation}
%\noindent
%and the function evaluated at the observed $\left(l_{ijk}, m_{ijk}\right)$ may be written 
%\begin{equation*} 
%\vphistar = B_m \Theta B_l^\prime
%\end{equation*}
%\noindent 
%where $\Theta$ denotes the $K \times L$ matrix of tensor product coefficients, with elements $\theta_{ij}$.
%
%\begin{figure}[H]
%\centering
% \graphicspath{{img/}}
%  \includegraphics[width=4in, height=4in]{bicubic_basis_function.png}
% 
%% 
% \subsection{Regularization with difference penalties} \label{subsection:univariate-psplines}
%
%The minimizer of \eqref{eq:loglikelihood} honors the fidelity to the data, so to balance the complexity of the fitted function with the goodness of fit to the data, we can append a penalty to the negative log likelihood to control the fitted function. By using rich B-spline bases for $l$ and $m$ alongside discrete difference penalties on the spline coefficients, we can achieve as much smoothness of the fitted function in both the $l$ and $m$ dimensions as desired. \cite{o1986statistical} was the first to propose using a rich B-spline basis and using a penalty to restrict the flexibility of the fitted curve, like \cite{wahba1990spline} applying a penalty to the second derivative of the fitted curve:
%\[
%J = \int_0^1 \left[ f^{\prime \prime}\left(l\right)\right]^2\;dx.
%\]
%
%For a B-spline of the form
%\[
%f\left(x\right) = \sum\limits_{j=1}^n \theta_i B_j\left(x\right),
%\]
%one can derive a banded matrix $P$ using the properties of B-splines such that 
% \[
% J = \theta^\prime P \theta
% \] 
% \noindent
% where $\theta = \left(\theta_1,\dots, \theta_n\right)$. The $i$-$j^{th}$ element of $P$ is given by
% \[
% p_{ij} = \int_0^1 B_i^{\prime \prime} \left( x \right)B_j^{\prime \prime} \left( x \right)\;dx.
% \]

%As discussed in Section 2, we can define an entire class of functional autoregressive models using only the $l$ direction, and additionally, as discussed in Section 3, there is a natural expectation about the functional form of the autoregressive coefficient function (and hence covariance) as a function of $l$. The use of smoothing splines to estimate $\phi$ outlined in Section~\ref{} yields smooth null models, but smoothness of the elements of the Cholesky factor alone may not lead to desirable structure in the inverse covariance matrix.  

%
%These approaches implicitly adopt different notions of sparsity. Like \cite{huang2006covariance} and \cite{levina2008sparse}, our aim is to regularize the inverse of the covariance matrix through the Cholesky factor. Expressing the varying coefficient function using a tensor product basis expansion builds the foundation for a flexible estimation framework within which employing multiple notions of smoothness is simple and straightforward. 
%
%
%In some applications, it is useful to work with third and fourth order differences, since for large values of $\lambda$, the fitted curve approaches a parametric polynomial model. This may be of particular interest in the context of estimating the elements of the Cholesky factor, as many have proposed simple parametric functions of lag only for $\phi$, such as low order polynomials. See \cite{pourahmadi1999joint}. However, with the use of higher order derivatives, the computation of $P$ is nontrivial and becomes very tedious. \cite{eilers1996flexible} were the first to propose P-splines, or \emph{penalized B-splines}, as an approach to nonparametric regression. P-splines circumvent complexity associated with constructing such penalty matrices by omitting derivatives and integrals altogether, replacing them with finite differences and sums. 
%
%Instead, flexibility of the fitted function is controlled by using a discrete penalty matrix based on finite difference formulas. Smoothness of the fitted function is achieved by first using a rich B-spline basis with equally spaced knots to purposefully overfit the smooth coefficient vectors; this eliminates the difficulty of choosing the optimal set of knots. Then by attaching a difference penalty to the goodness of fit measure, one may prevent overfitting and make a potentially ill-conditioned fitting procedure a well-conditioned one. The finite difference penalty is simple to compute and can be handled mechanically for any order of the differences. Since it is easily introduced into regression equations, it is feasible to evaluate the impact of different orders of the differences on the fitted model.  Using the properties of B-splines, it is straightforward to show that the difference penalty of order $d$ is a good discrete approximation to the integrated square of the $d^{th}$ derivative, so little is lost by replacing the derivative-based penalty with
%
%\begin{equation} \label{eq:bspline-difference-penalty}
%J_d\left( f \right) = \sum_{j=d}^n \left(\Delta^d \theta_j\right)^2
%\end{equation} 
%\noindent
%where $\theta = \left( \theta_1,\dots,\theta_n \right)$. Let $D_d$ denote the matrix difference operator: $D_d\theta = \Delta^d \theta$, where
%
% \begin{align*}
% \Delta \theta_j &= \theta_j - \theta_{j-1}, \quad  \Delta^2 \theta_j = \Delta\left(\Delta \theta_j\right) = \theta_j - 2\theta_{j-1} + \theta_{j-2}
% \end{align*}
%\noindent 
%In general,
%\begin{equation*}
%\Delta^d \theta_j = \Delta\left(\Delta^{d-1} \theta_j \right).
%\end{equation*} 
%\noindent
%Then, \eqref{eq:bspline-difference-penalty} can be written in terms of the squared norm of the difference operator applied to the vector of B-spline coefficients:
%\begin{align} 
%\begin{split} \label{eq:bspline-difference-penalty-vector-form}
%J_d\left( f \right) &= \vert \vert D_d\theta \vert \vert^2 \\
%&= \theta^\prime P_d \theta
%\end{split}
%\end{align}
%\noindent
%where $P_d = D_d^\prime D_d$.  To examine the connection between the second-derivative penalty to the penalty on second-order differences of the B-spline coefficients, we only need to employ straightforward calculus and exploit the recursive property of the B-spline basis functions:
%
%\begin{equation*} 
%\int_0^1 \left[ f^{\prime \prime}\left(x\right)\right]^2\;dx = \int_{0}^{1} \left\{ \sum\limits_{j=1}^n  \theta_j B_{j3}^{\prime \prime} \left(l\right) \right\}^2\; dl.
%\end{equation*}
%\noindent
%The derivative properties of B-splines permits this to be written as 
%\begin{equation*} \label{eq:second-derivative-bspline-penalty}
%\int_0^1 \left[ f^{\prime \prime}\left(x\right)\right]^2\;dx =  \int_{0}^{1}  \bigg[ \sum\limits_{j=1}^n \sum\limits_{k=1}^n \Delta^2 \theta_j \Delta^2 \theta_k B_{j1}\left(l\right)B_{k,1}\left(l\right)\bigg]\; dl  . 
%\end{equation*}
%\noindent
%Most of the cross products of $B_{j1}\left(x\right)$ and $B_{k,1}\left( x \right)$ vanish since B-splines of degree 1 only overlap when $j$ is $k-1$, $k$, or $k+1$. Thus, we have that
%\begin{align}
%\begin{split}
%\int_0^1 \left[ f^{\prime \prime}\left(x\right)\right]^2\;dx  = {} &  \int_0^1 \bigg[ \left\{ \sum\limits_{j=1}^n   \Delta^2 \theta_j  B_j\left(l,1\right)  \right\}^2  + 2 \sum_{j}\Delta^2 \theta_j\Delta^2 \theta_{j-1}B_j\left(l,1\right)B_{j-1}\left(l,1\right) \bigg]\; dl \\ 
%= {} & \sum \limits_{j=1}^n  \left( \Delta^2\theta_j \right)^2 \int_0^1 B_j^2\left(l,1\right)\;dl \\
%   &{} \;\;\;\;\;\;\;\;\;\;\;\;\;\;\;\;\;\; + 2 \sum\limits_{j=1}^n \Delta^2 \theta_j\Delta^2 \theta_{j-1} \int_0^1 B_j\left(l,1\right)B_{j-1}\left(l,1\right)\;dl 
%\end{split}
%\end{align}
%\noindent
%which can be written as
%\begin{equation} \label{eq:derivative-penalty-difference-penalty-connection}
%\int_0^1 \left[ f^{\prime \prime}\left(x\right)\right]^2\;dx  = c_1 \sum\limits_{j=2}^n \left( \Delta^2 \theta_j\right)^2 + c_2 \sum\limits_{j=3}^n \Delta^2 \theta_j\Delta^2 \theta_{j-1}
%\end{equation}
%\noindent
%Given a set of equidistant knots, the constants $c_1$ and $c_2$ are given by
%\begin{equation}
%\begin{split}
%c_1 & =   \int_0^1 B_{j1}^2\left(x\right) dx\\
%c_2 & = \int_0^1 B_{j1}\left(x\right)B_{j-1,1}\left(x\right) dx.
%\end{split}
%\end{equation}
%
%This gives us that the traditional smoothness penalty on the squared second derivative can be written as a linear combination of a penalty on the second-order differences of the B-spline coefficients \eqref{eq:bspline-difference-penalty} and the sum of the cross products of neighboring second differences. The second term in \eqref{eq:derivative-penalty-difference-penalty-connection} leads to a complex objective function when minimizing the penalized likelihood, where seven adjacent spline coefficients occur, as opposed to five if only the first term in \eqref{eq:derivative-penalty-difference-penalty-connection} is used in the penalty. The added complexity is a consequence of overlapping B-splines, which quickly increases when using higher order differences and higher order B-splines. We can seamlessly augment the likelihood with the difference penalty to achieve smooth fitted functions without the complexity posed by the derivative-based penalty.
%%citet{chen2011efficient}, citet{pourahmadi1999joint}, and citet{pourahmadi2002dynamic} have elicited parametric models for the generalized autoregressive coefficients, letting the GARPs depend only on the distance between two time points.
%
%A smoother sequence of coefficients leads to a smoother curve, as illustrated in Figure~\ref{fig:second_ord_PS_pen_SML_lambda}.  The relationship between P-spline curves and their coefficients is easily characterized if we consider the coefficients as the skeleton of the function, and draping the B-splines over them puts the flesh on the bones. As long as the coefficient sequence is smooth, the number of basis functions (and coefficients) is unimportant since the penalty ensures the smoothness of the skeleton and that the fitting procedure is well-conditioned. Figure~\ref{fig:overcomplete_basis_pspline} illustrates this utility of the penalty for simulated data; there are $m=10$ observations and $60$ cubic B-splines. This property of P-splines cannot be overly appreciated because it frees us from the concern of choosing the optimal set of knots. Unless computational constraints are of concern, which is possible with large models, it is prudent to use even more B-splines. Figure~\ref{fig:PS_penalty_section_figure_2} shows how the fitted function changes as the tuning parameter varies when the data are sparsely sampled. P-splines enjoy a number of additional advantageous properties, many of which are inherited from the attractive properties of B-splines. See \cite{eilers1996flexible}  for a detailed presentation. 
%
%\begin{figure}[H] \label{fig:PS-smoothing-figure-1}
%\begin{subfigure}{.5\textwidth}
%  \centering
%   \graphicspath{{img/}}
%  \includegraphics[scale=0.5]{pspline_pord2_xsmall_lambda.png}
%  \label{fig:pspline_small_lambda}
%\end{subfigure}
%\begin{subfigure}{.5\textwidth}
%  \centering
%   \graphicspath{{img/}}
%  \includegraphics[scale=0.5]{pspline_pord2_small_lambda.png}
%  \label{fig:pspline_small_lambda}
%\end{subfigure}
%\begin{subfigure}{.5\textwidth}
%  \centering
%   \graphicspath{{img/}}
%  \includegraphics[scale=0.5]{pspline_pord2_medium_lambda.png}
%  \label{fig:pspline_small_lambda}
%\end{subfigure}
%\begin{subfigure}{.5\textwidth}
%  \centering
%   \graphicspath{{img/}}
%  \includegraphics[scale=0.5]{pspline_pord2_large_lambda.png}
%  \label{fig:pspline_small_lambda}
%\end{subfigure}
%\caption{\textit{Illustration of the impact of the second order difference penalty. The number of B-splines used is the same in each plot, with the value of the penalty parameter increasing from left to right and top to bottom across each plot. The fitted curve in the upper left plot is the most ``wiggly'' of any of the fits, as the penalty plays the weakest roll in the fitted coefficients there. The red circles are the values of each of the B-spline coefficients; as the penalty increases, they form as smoother sequence as we move across the four plots, which results in a smoother fitted function. As the penalty parameter approaches infinity, the fit approaches a linear function as shown in the bottom right plot.}}
%\label{fig:second-ord-PS-pen-SML-lambda}
%\end{figure}
%%==============================================================================================================================================
%%==============================================================================================================================================
%%==============================================================================================================================================
%%==============================================================================================================================================
%%==============================================================================================================================================
%%==============================================================================================================================================

% \begin{columns}
%\begin{column}{0.5\textwidth}
%Equip $l$ and $m$ with
%\begin{align*}
%B_{1}\left(l\right),\dots, B_{K}\left(l\right),\\
%B_{1}\left(m\right),\dots, B_{L}\left(m\right)
%\end{align*}
%to build
%\begin{equation*}
%T_{jk}\left(l,m\right) = B_j\left(l\right){B}_k\left(m\right)
%\end{equation*}
%  \end{column}
%\begin{column}{0.5\textwidth}  %%<--- here
%    \begin{center}
%    \begin{figure}
%    \graphicspath{{img/}}
% \includegraphics[width=4cm]{sparse_bicubic_basis}
% \caption{A ``thinned'' tensor product basis}
% \end{figure}
%     \end{center}
%\end{column}
%\end{columns}
%\vspace{0.3cm}
%\begin{equation*}
%\phi\left(l,m\right) = \sum_{i=1}^K \sum_{j=1}^L \theta_{ij} B_{i}\left(l\right) B_{j}\left(m\right)
%\end{equation*}
%
 
%The parameters of the functional autoregressive model given by \eqref{eq:MyModel} define the elements of the precision matrix $\Omega$, rather than the elements of $\Sigma$ itself. It is well known that if we let $Y = \left(Y_1, \dots, Y_m\right)^\prime$ denote the random vector having joint distribution with mean zero and covariance matrix $\Sigma$, then the elements of $\Sigma^{-1}=\Omega$, $\left\{ \omega_{ij} \right\}$ may be interpreted as partial covariances between the elements of $Y$. This suggests shrinking $\phi$ to zero for large values of $l$. One can show that if $T$ has $k$ non-zero diagonals, then the middle $k$ diagonals of $\Sigma^{-1}$ are non-zero.  

%For ease of exposition, we first focus our attention on the estimation of $\phi$ assume that $\sigma^2\left(t\right)$ is fixed and known. Estimation of the innovation variance function is presented in Section~\ref{section:variance-estimation}. In the case that subjects share a common set of observation times $t_1 < \dots < t_m$,  it is well known that the MLE for $\Sigma$, $S = \sum_{i=1}^N y_i y_i^\prime$ is highly unstable in high dimensions, a condition that is potentially worsened when one or more subjects has at least one observation time that is unique from the set of observation times common across subjects. To mitigate instability due to high dimensionality and simultaneously permit the estimation of $\phi\left(\cdot,\cdot\right)$ as a smooth bivariate function, we obtain a covariance estimator by applying bivariate smoothing of the elements of the Cholesky factor. 

%%====================================================================================

\section{Smoothing Parameter Selection}

%\subfile{chapter-3-subfiles/chapter-3-tensor-product-pspline-model-selection}
As with the RKHS framework and accompanying smoothing spline representation, the smoothing matrix  
\begin{equation*}\label{eq:pspline-smoothing-matrix}
A_\lambda = X \big( \left(X B\right)' D^{-1} XB +  \lambda_l P_l+ \lambda_m P_m \big)^{-1}\left(X B\right)' D^{-1}
\end{equation*}
\noindent
and its properties play an integral role in selecting the optimal smoothing parameter in any regularized regression, including the P-spline framework. We discussed the leave-one-subject-out cross validation score \eqref{eq:LOSOCV} and its computationally efficient approximation \eqref{eq:approx-losocv} in Chapter~\ref{SSANOVA-chapter}, which rely directly on the smoothing matrix for calculation. The model selection criteria discussed in Section~\ref{gaussian-unbiased-risk-estimate}  can be calculated as in the smoothing spline setting by replacing $\tildeA_{\lambda,\bftheta}$ with $A_\lambda$. For detailed discussion of P-spline model selection with respect to multiple smoothing parameters, see \cite{wood2017generalized}.

\bigskip

%Summarizing the complexity of a fitted P-spline is non-trivial; it requires the simultaneous consideration of the smoothing parameters, the number of basis functions in the B-spline basis, and the order of the difference penalties. To assess model complexity, \cite{eilers1996flexible} follow \cite{hastie1990generalized}, who use the trace of the smoothing matrix as an approximation of the \textit{effective dimension} (ED) of a linear smoother. The effective (model) dimension is defined as 
%
%%\begin{align}
%\begin{equation} \label{eq:trace-of-the-smoothing-matrix}
%\textup{ED} = \textup{tr}\left( A_\lambda \right) = \textup{tr}\bigg( X\left( \left(XB\right)^\prime D^{-1}XB +  \lambda_l P_l+ \lambda_m P_m\right)^{-1} \left(X B\right)^\prime D^{-1}  \bigg)
%\end{equation}
%%\end{align}
%
%\noindent
%The ED combines the effect of the smoothing parameter, the number of basis functions, and the differencing order, and it is easy to compute. When the number of basis functions is significantly smaller than the sample size, it is advantageous to use the cyclic property of the trace: 
%
%\begin{align*}
%\textup{tr}\left( A_\lambda \right) &= \textup{tr}\bigg( X\left( \left(XB\right)^\prime D^{-1}XB +  \lambda_l P_l+ \lambda_m P_m\right)^{-1} \left(X B\right)^\prime D^{-1}  \bigg)\\
%&= \textup{tr}\bigg( \left(X B\right)^\prime D^{-1}  X\left( \left(XB\right)^\prime D^{-1}XB +  \lambda_l P_l+ \lambda_m P_m\right)^{-1} \bigg),
%\end{align*}
%
%\noindent
%which requires computing the trace of a $k_lk_m \times k_lk_m$ matrix, which is computationally more economical when the total number of basis functions is smaller than the total number of observations. This approach to approximating the effective model dimension is also consistent with \cite{ye1998measuring}, who constructed a generalization of the concept of a model's degrees of freedom using the idea that the degrees of freedom can also be interpreted as the sum of the sensitivity of each fitted value with respect to the corresponding observed value.
%
%\bigskip
%
%Using the eigenstructure of the smoothing matrix, one can show that as the smoothing parameters tend to infinity, the effective dimension approaches $d_l + d_m$, the sum of the order of the differencing operators for $l$ and $m$. Let
%
%\begin{equation*}
%Q = \left(X B\right)^\prime D^{-1} XB \qquad \mbox{and} \qquad Q_\lambda = \lambda_l P_l + \lambda_m P_m.
%\end{equation*}
%
%Again using cyclic property of the trace, we can write
%\begin{align*}
%%\begin{split}
%\mbox{tr}\left(A_\lambda \right) &= \mbox{tr}\bigg[ \left(Q + Q_\lambda \right)^{-1}Q \bigg]\\
%&=\mbox{tr}\bigg[ Q^{1/2}\left(Q + Q_\lambda \right)^{-1}Q^{1/2} \bigg] \\
%&=\mbox{tr}\bigg[\left(I + Q^{-{1/2}}Q_\lambda Q^{-{1/2}} \right)^{-1} \bigg]
%%\end{split}
%\end{align*}
%
%\noindent
%Finally we have that
%\begin{equation*}
%\mbox{tr}\left(A_\lambda \right) = \mbox{tr}\bigg[\left(I + \lambda Q^{-{1/2}}Q_\lambda Q^{-{1/2}}  \right)^{-1} \bigg] = \sum_{j=1}^{k_lk_m} \frac{1}{1 + \lambda \gamma_j},
%\end{equation*}
%
%\noindent
%where $\gamma_j$, $j=1,\dots,k_lk_m$ are the eigenvalues of $Q^{-{1/2}}Q_\lambda Q^{-{1/2}}$. The matrix constructed from the sum of the penalty terms, $Q_\lambda$, has exactly $d_l + d_m$ eigenvalues equal to zero. Hence, $Q^{-{1/2}}Q_\lambda Q^{-{1/2}} $ has $d_l + d_m$ eigenvalues equal to zero, so for large $\lambda$, only the $d_l + d_m$ terms with $\gamma_j=0$ contribute to the sum which gives the trace of $A_\lambda$. Then
% 
% \[
%\lim_{\lambda \rightarrow \infty  } \mbox{tr}\left(A_\lambda\right) = d_l + d_m.
% \]
%
%%A further simplification of \eqref{eq:hat-matrix-trace}
%%
%%\begin{align*} 
%%\left(B^T B + \lambda D^T D \right)^{-1} B^T B &= \left(B^T B + \lambda D^T D \right)^{-1} \left( B^T B + \lambda D^T D - \lambda D^T D\right) \nonumber \\
%%&= I - \lambda\left(B^T B + \lambda D^T D \right)^{-1} D^T D \label{eq:cyclic_hat_matrix_simplification}
%%\end{align*}
%%
%%\begin{align*} 
%%\left[\left(WB\right)^\prime D^{-1}WB +  \lambda_l P_l+ \lambda_m P_m\right]^{-1} \left(W B \right)^\prime D^{-1} WB  &= \left[\left(WB\right)^\prime D^{-1}WB +  \lambda_l P_l+ \lambda_m P_m\right]^{-1}\left(W B \right)^\prime D^{-1} \times \\
%%&\mbox{\;\;\;\;\;\;\;\;\;\;\;\;\;\;\;\;\;\;\;\;\;} \left[WB + \lambda_l P_l+ \lambda_m P_m - \left(\lambda_l P_l+ \lambda_m P_m\right) \right] \\
%%&= I - \lambda\left(B^T B + \lambda D^T D \right)^{-1} D^T D \label{eq:cyclic_hat_matrix_simplification}
%%\end{align*}
%
%This clearly shows that the effective dimension is always less than $k_lk_m$, the number of B-spline used in the regression basis; further, the effective dimension is always smaller than $\min\left(\sum_{i=1}^N p_i - N, k_lk_m\right)$. This is visually demonstrated in Figure~\ref{fig:pspline-limiting-effective-dimension}, which displays impact of the smoothing parameter on the effective dimension of the P-spline fit to the simulated data shown in Figure~\ref{fig:overcomplete-pspline-basis}. As $\lambda$ increases, the effective dimension approaches the order of the difference penalty. Even for small $\lambda$, the effective dimension never exceeds the number of observations, so there are no issues when fitting P-splines with many more basis functions than observations. 
%
%\begin{figure}[H]
%\begin{center}
%\graphicspath{{img/}}
% \includegraphics[width=0.7\textwidth]{pspline-limiting-effective-dimension}
%\caption{\textit{The limiting behaviour of the trace of the smoothing matrix $A_\lambda$ as the smoothing parameter increases for the P-spline fit to the 10 observations using 60 B-spline basis functions, shown in Figure~\ref{fig:overcomplete-pspline-basis}. For weakly enforced smoothing, the effective dimension is equal to the number of observations, and as $\lambda \rightarrow \infty$, the effective dimension approaches the order of the difference penalty.}} \label{fig:pspline-limiting-effective-dimension}
%\end{center}
%\end{figure}
%
%The effective model dimension is closely connected to model selection criteria; \cite{eilers1996flexible} propose the use of the Akaike information criterion (AIC) for selecting the optimal value of the smoothing parameters $\lambda = \left(\lambda_l, \lambda_m\right)$, which is equivalent to the unbiased risk estimator discussed in Chapter~\ref{SSANOVA-chapter} under a Gaussian likelihood. For a detailed discussion of the connection between the unbiased risk estimate and AIC in the non-gaussian case, see \cite{wood2017generalized}, Chapter 4, Section 5. The same reference provides a detailed discussion of computational methods for minimizing the unbiased risk estimate with respect to multiple smoothing parameters. The algorithm shares the same basic structure as the one outlined in Section~\ref{gaussian-unbiased-risk-estimate}, with modifications on the derivatives and the Hessians to account for the fact that the P-spline basis and penalty are constructed independently of one another.
%


\section{The P-spline Estimator for the Innovation Variance Function}

The P-spline estimator for the log innovation variance function is constructed via penalized similarly to the smoothing spline estimator in Section~\ref{chapter-3-IV-modeling-section}.  Fixing $\phi = \phi^*$ at an estimate $\phi^*$ of $\phi$, the the log likelihood of the squared working residuals can be written as in \eqref{eq:penalized-joint-loglik-given-phi-2}
\[
-2\ell\left( \sigma^2  \vert Z_1,\dots, Z_N, \phi,  \right) =  \sum_{i = 1}^N \sum_{j = 1}^{p_i} \log \sigma^2_{ij}  + \sum_{i = 1}^N \sum_{j = 1}^{p_i} \frac {z_{ij}}{\sigma^2_{ij}},
\]
\noindent
where $\epsilon_{ij} =  y_{ij} - \sum\limits_{k<j} \phi^*_{ijk} y_{ik}$, and $z_{ij} = \epsilon_{ij}^2$. We can approximate $\eta \left(t\right) = \log \sigma^2\left(t\right)$ using a B-spline basis expansion, letting
\[
\eta\left(t\right) = \sum\limits_{j = 1}^{k_t} \theta_j B_{j}\left(t\right).
\] 
\noindent
Model estimation and smoothing parameter selection can be carried out using performance-oriented iteration as described in Section~\ref{smoothing-parameter-selection-exponential-families}, substituting the above expansion for $\eta$ and trading the smoothing spline penalty for the discrete difference penalty \eqref{eq:bspline-difference-penalty-1}. For detailed presentation of optimization procedures, see \cite{marx1999generalized}. 

   
\chapter{Simulation studies} \label{simulation-studies-chapter}


In this section we compare bivariate spline estimators of the Cholesky factor to other methods of covariance estimation. Our primary comparisons are that with the parametric polynomial estimator proposed by \cite{pourahmadi1999joint},  \cite{pan2003modelling}, and \cite{pourahmadi2002dynamic}, which is also based on the modified Cholesky decomposition, and with the oracle estimator, which effectively gives a lower bound on the risk for given covariance structure. As a benchmark, we also include the sample covariance matrix, and two regularized variants of it: the tapered sample covariance matrix \citep{cai2010optimal} and the soft thresholding estimator \citep{rothman2009generalized}, which does not rely on a natural ordering among the variables. In the simulations, the smoothing spline estimator of the modified Cholesky decomposition was constructed using the framework of a tensor product cubic smoothing spline. For each covariance matrix used for simulation, the P-spline estimator was constructed so that the order of the difference penalties for $l$ and $m$ are treated as additional tuning parameters.

\bigskip

Simulations were carried out for five covariance structures: the diagonal covariance with homogenous variances, a heterogenous autoregressive process with linear varying coefficient function, the same heterogeneous process but truncated to zero to band the inverse covariance matrix, the rational quadratic covariance model, and the compound symmetric model. The two-dimensional surfaces corresponding to each of these are shown left to right in Figure~\ref{fig:true-covariance-heatmaps}. The first row of image plots display the surface which coincides with the appropriate discrete covariance matrix, and in the second row are the surface maps of the corresponding Cholesky factors. Precise models used for simulations are defined in Table~\ref{table:simulation-model-list}. 

\begin{table}[H]
\centering
\caption{\textit{Covariance models used for data generation in the simulation study.}}
\begin{tabular}{p{7cm}p{7cm}}
\hline
 \multicolumn{2} {l} {I: Mutual independence} \\[0.3cm]
 $\Sigma = \mathrm{I}$ & $\begin{aligned}
\phi\left(t,s\right) &= 0, \quad 0 \le s < t \le 1,\\[0.15cm] 
\sigma^2\left(t\right) &= 1, \quad 0 \le t \le 1.
\end{aligned}$ \\[0.2cm]
\\
\hline
\\
 \multicolumn{2} {l} {II: Linear varying coefficient function, constant innovation variances} \\[0.3cm]
$\Sigma = T^{-1} D {T'}^{-1}$ & $\begin{aligned}
\phi\left(t,s\right) &= t - \frac{1}{2},  \quad 0 \le t \le 1, \\[0.15cm]
\sigma^2\left(t\right) &= 0.1^2,  \quad 0 \le t \le 1.
\end{aligned}$ \\
\\
\hline
\\
\multicolumn{2} {l} {III: Banded linear varying coefficient function, constant innovation variances} \\[0.3cm]
 $\Sigma = T^{-1} D {T'}^{-1}$ &$ \begin{aligned}
\phi\left(t,s\right) &= \left\{\begin{array}{ll} t - \frac{1}{2}, & t - s \le 0.5\\ 
0, & t - s > 0.5\end{array}\right.,\\[0.15cm]
\sigma^2\left(t\right) &= 0.1^2, \quad 0 \le t \le 1.
\end{aligned}$  \\
\\
\hline
\\
\multicolumn{2} {l} {IV: Rational quadratic covariance} \\[0.3cm]
 $\Sigma = \left[\sigma_{ij}\right]$ &  $\begin{aligned}\sigma_{ij} &=\begin{array}{ll} \left(1 + \frac{\left(t_i - t_j\right)^2}{2\alpha k^2}\right)^{-\alpha}, & 0 < t_i, t_j < 1\end{array}\\
                                                                      k &= 0.6,\;\;\alpha = 1\end{aligned}$ \\
 \\
\hline
\\
\multicolumn{2} {l} {V: Compound symmetry} \\[0.3cm]
 $\begin{array}{l}\Sigma = \sigma^2\left(\rho \mathrm{J} + \left(1-\rho\right)\mathrm{I}\right),  \\
 \\
  \rho=0.7, \quad \sigma^2=1 \end{array}$  &  $\begin{aligned}
\phi_{ts} &= \frac{\rho}{1 + \left(t-2\right)\rho}, \begin{array}{l} t = 2, \dots, M,\\ 
				 s = 1, \dots, t-1 \end{array}\\
\sigma_t^2 &= \left\{\begin{array}{ll} 1, & t = 1\\ 1 -\frac{\left(t-2\right)\rho^2}{1 + \left(t-2\right)\rho}, & t = 2, \dots, M \end{array}\right.
\end{aligned}$ \\
 \\
\hline
\end{tabular} \label{table:simulation-model-list}
\end{table}


\bigskip

Figure~\ref{fig:true-covariance-heatmaps} displays a two dimensional representation of each covariance matrix $\Sigma$ and it's corresponding Cholesky factor $T$ used in the simulation study. The smallest elements of each matrix correspond to dark green pixels, while the light pink (white) pixels correspond to the large (largest) elements of the matrix. Comparison of the covariance matrices with the generalized autoregressive coefficient function which defines lower triangular surface in the second row demonstrates that covariance structures exhibiting sparsity or parsimony do not necessarily exhibit the same simplicity in the components of the Cholesky decomposition. The Cholesky factor for Model III, the truncated linear varying coefficient AR model, is sparse, with elements on the outer half of the subdiagonals equal to zero. While this corresponds to a banded inverse covariance structure, $\Sigma$ itself is not sparse.  The compound symmetric model has simple structure and is parsimonious; its dependence parameters can be expressed as the evaluation of a function which is constant in time $t$. However, the elements of the Cholesky factor and diagonal matrix of innovation variances $D = T \Sigma T'$ do not exhibit such elementary structure, the elements of which are nonlinear in $t$. 

%\subfile{chapter-4-subfiles/chapter-4-true-covariance-heatmaps}
\begin{figure}[H] 
\begin{center}
  \includegraphics[width = \textwidth]{img/chapter-4/cov-cholesky-grid}%}
\caption{\textit{Heatmaps of the true covariance matrices (row 1) under simulation Model I - Model V (see Table~\ref{table:simulation-model-list}) and the function $\phi$ defining the corresponding Cholesky factor $T$ (row 2).} } \label{fig:true-covariance-heatmaps}
\end{center}
\end{figure}


\bigskip

%\subfile{chapter-4-subfiles/chapter-4-true-covariance-functions}

%%%%%%%%%%%%%%%%%%%%%%%%%%%%%%%%%%%%%%%%%%%%%%%%%%%%%%%%%%%%%%%%%%%%%%%%%%%%%%%%%%%%%%%%%%%%%%%%
%%%%%%%%%%%%%%%%%%%%%%%%%%%%%%%%%%%%%%%%%%%%%%%%%%%%%%%%%%%%%%%%%%%%%%%%%%%%%%%%%%%%%%%%%%%%%%%%
%%%%%%%%%%%%%%%%%%%%%%%%%%%%%%%%%%%%%%%%%%%%%%%%%%%%%%%%%%%%%%%%%%%%%%%%%%%%%%%%%%%%%%%%%%%%%%%%
%%%%%%%%%%%%%%%%%%%%%%%%%%%%%%%%%%%%%%%%%%%%%%%%%%%%%%%%%%%%%%%%%%%%%%%%%%%%%%%%%%%%%%%%%%%%%%%%
%%%%%%%%%%%%%%%%%%%%%%%%%%%%%%%%%%%%%%%%%%%%%%%%%%%%%%%%%%%%%%%%%%%%%%%%%%%%%%%%%%%%%%%%%%%%%%%%

\section{Loss functions and corresponding risk measures}
Let $\hat{\Sigma}$ be an estimator of the true $M \times M$ covariance matrix $\Sigma$. To assess performance of an estimator $\hat{\Sigma}$, we consider two commonly loss functions:
\begin{equation} \label{eq:quad-loss}
\Delta_1\left(\Sigma,\hat{\Sigma} \right) = tr\left(\left( \Sigma^{-1} \hat{\Sigma} - \mathrm{I}\right)^2 \right),
\end{equation}
\noindent
\begin{equation} \label{eq:entropy-loss}
\Delta_2\left(\Sigma,\hat{\Sigma}\right) = tr\left( \Sigma^{-1} \hat{\Sigma} \right) - log \vert \Sigma^{-1} \hat{\Sigma} \vert - M.
\end{equation}
\noindent
$\Sigma$ denotes the true covariance matrix and $\hat{\Sigma}$ is an $M \times M$ positive definite matrix. Each of these loss functions is $0$ when $\hat{\Sigma} = \Sigma$ and is positive when $\hat{\Sigma} \ne \Sigma$. Both measures of loss are scale invariant. If we let random vector $Y$ have covariance matrix $\Sigma$, and define the $Z$ as some linear transformation of $Y$:

\[
Z = CY. 
\]
\noindent
for some $M \times M$ matrix $C$,  then $Z$ has covariance matrix $\Sigma_Z = C \Sigma C'$. Given an estimator $\hat{\Sigma}$ of $\Sigma$, one immediately obtains an estimator for $\Sigma_Z$, $\hat{\Sigma}_Z = C \hat{\Sigma} C'$. If $C$ is invertible, then the loss functions $\Delta_1$ and $\Delta_2$ satisfy
\[
\Delta_i\left(\Sigma,\hat{\Sigma}\right) = \Delta_i\left(C \Sigma C', C \hat{\Sigma}C' \right). 
\]
\noindent
The first loss $\Delta_1$, or the quadratic loss, measures the discrepancy between $\left(\Sigma^{-1} \hat{\Sigma}\right)$ and the identity matrix with the squared Frobenius norm. The Frobenius norm of a matrix $A$ is given by 

\[
\vert \vert A \vert \vert_F^2 = \mbox{tr}\left(A A'\right).
\]
\noindent
The second loss $\Delta_2$ is commonly referred to as the entropy loss; it gives the Kullback-Leibler divergence of two multivariate Normal densities with the same mean and the two corresponding covariance matrices. The quadratic loss penalizes overestimates more than underestimates, so ``smaller'' estimates are favored more under $\Delta_1$ than $\Delta_2$. For example, among the class of estimators comprised of scalar multiples $cS$ of the sample covariance matrix, \cite{haff1980empirical} established that $S$ is optimal under $\Delta_2$, while the smaller estimator $\frac{NS}{N+M+1}$ is optimal under $\Delta_1$. 

\bigskip

Given $\Sigma$, the corresponding values of the risk functions are obtained by taking expectations:

\begin{equation*}
R_i \left(\Sigma,\hat{\Sigma}\right) = E_\Sigma\left[\Delta_i\left(\Sigma,\hat{\Sigma}\right)\right], \quad i = 1,2.
\end{equation*}
\noindent
We prefer an estimator $\hat{\Sigma}$ with smaller risk.  Given $\Sigma$, we can estimate the risk of an estimator via Monte Carlo approximation. 

%%%%%%%%%%%%%%%%%%%%%%%%%%%%%%%%%%%%%%%%%%%%%%%%%%%%%%%%%%%%%%%%%%%%%%%%%%%%%%%%%%%%%%%%%%%%%%%%
%%%%%%%%%%%%%%%%%%%%%%%%%%%%%%%%%%%%%%%%%%%%%%%%%%%%%%%%%%%%%%%%%%%%%%%%%%%%%%%%%%%%%%%%%%%%%%%%
%%%%%%%%%%%%%%%%%%%%%%%%%%%%%%%%%%%%%%%%%%%%%%%%%%%%%%%%%%%%%%%%%%%%%%%%%%%%%%%%%%%%%%%%%%%%%%%%
%%%%%%%%%%%%%%%%%%%%%%%%%%%%%%%%%%%%%%%%%%%%%%%%%%%%%%%%%%%%%%%%%%%%%%%%%%%%%%%%%%%%%%%%%%%%%%%%
%%%%%%%%%%%%%%%%%%%%%%%%%%%%%%%%%%%%%%%%%%%%%%%%%%%%%%%%%%%%%%%%%%%%%%%%%%%%%%%%%%%%%%%%%%%%%%%%

\section{Alternative estimators}
%\subfile{chapter-4-subfiles/chapter-4-benchmark-estimators}
The following estimators serve as benchmarks for performance under the five simulation settings outlined above: the MCD polynomial estimator $\hat{\Sigma}_{poly}$, the sample covariance matrix $S$, the soft thresholding estimator $S^\lambda$, and the tapering estimator $S^\omega$. We will review the general definitions of these, but for detailed discussion of the construction and properties of these estimators, see Sections~\ref{elementwise-shrinkage-estimators} and \ref{chapter-1-matrix-decompositions}.

\bigskip

In the spirit of the GLM, the MCD polynomial estimator is a particular case of estimators which model the components of the Cholesky decomposition using covariates. The polynomial estimator takes the GARPs and IVs to be polynomials of lag and time, respectively:

\begin{align*}
\begin{split}  \label{eq:GARP-IV-parametric-model}
\phi_{jk} &= z'_{jk} \gamma \\
\log \sigma^2_{j} &= z'_{j}\lambda, 
\end{split}
\end{align*}
\noindent
for $j = 1,\dots, M$, $k = 1,\dots, j-1$. The vectors $z_j$ and $z_{jk}$ are of dimension $q \times 1$ and $p \times 1$  which hold covariates

\begin{align}
\begin{split} 
z'_{jk} &= \left(1, t_j - t_k, \left(t_j - t_k\right)^2,\dots, \left(t_j - t_k\right)^{p-1}\right)', \\
z'_{j}  &= \left(1, t_j, \dots, t_j^{q-1}\right)'.
\end{split}
\end{align} \label{eq:mcd-polynomial-model}
\noindent
where the orders of the polynomials, $p$ and $q$, are chosen by BIC. 

\bigskip

\cite{rothman2009generalized} presented a class of generalized thresholding estimators, including the soft-thresholding estimator given by

\[
S^{\lambda}=   \begin{bmatrix} \mbox{sign}\left(s_{ij}\right) \left(s_{ij} - \lambda\right)_+ \end{bmatrix},
\]
\noindent 
where $\sigma^*_{ij}$ denotes the $i$-$j^{th}$ entry of the sample covariance matrix, and $\lambda$ is a penalty parameter controlling the amount of shrinkage applied to the empirical estimator. 

\bigskip

The tapering estimator proposed by \cite{cai2010optimal} is given by
\[
S^{\omega} =  \begin{bmatrix} \omega_{ij}^k s_{ij} \end{bmatrix},
\]
\noindent
where the $\omega_{ij}^k$ are given by 
\begin{equation*}
\omega^k_{ij} = k_h^{-1} \left[ \left( k - \vert i-j\vert\right)_+ - \left(k_h - \vert i-j\vert\right)_+ \right],
\end{equation*}
\noindent
The weights $\omega^k_{ij}$ are controlled by a tuning parameter, $k$,  which can take integer values between 0 and $M$. Without loss of generality,  we assume that $k_h = k/2$ is even. The weights may be rewritten as
\begin{align*}
\omega_{ij} = \left\{\begin{array}{ll} 1, & \vert i -j \vert \le k_h \\
                             2 - \frac{i - j}{k_h}, & k_h < \vert  i -j \vert \le k, \\
                             0, & \mbox{otherwise}  \end{array} \right.
\end{align*}
\noindent

\bigskip

Tuning parameter selection for the regularized versions of the sample covariance matrix was performed using cross validation. Under certain conditions pertaining to the ratio of sample sizes of the training and validation datasets, the $K$-fold cross validation criterion is a consistent estimator of the Frobenius norm risk. It is defined 

\begin{equation} \label{eq:K-fold-matrix--cv}
\mbox{CV}_F\left(\lambda \right) = \argmin{\lambda} K^{-1} \sum_{k = 1}^K  \vert \vert\hat{\Sigma}^{\left(-k\right)} - \tilde{\Sigma}^{\left(k\right)}  \vert \vert_F^2, 
\end{equation}
\noindent
There is little established about the optimal method for tuning parameter selection in for the class of estimators based on element-wise shrinkage of the sample covariance matrix.  However, based on the results of an extensive simulation study presented in \cite{fang2016tuning}, we use $K = 10$-fold cross validation to select the tuning parameters for both the tapering estimator $S^\omega$ and the soft thresholding estimator $S^{\lambda}$. They authors implement cross validation for a number of element-wise shrinkage estimators for covariance matrices in the \cite{CVTuningCov} R package, which was used to calculate the risk estimates for $S^{\omega}$ and $S^{\lambda}$. 

\bigskip

As discussed in Chapter 1, in the limit, soft thresholding produces a positive definite estimator with probability tending to 1 (\cite{rothman2009generalized}), however element-wise shrinkage estimators of the covariance matrix, including the soft thresholding estimator, are not guaranteed to be positive definite. We observed simulations runs which yielded a soft thresholding estimator that was indeed not positive definite.  In this case, the estimate has at least one eigenvalue less than or equal to zero, and the evaluation of the entropy loss \ref{eq:entropy-loss} is undefined. To enable the evaluation of the entropy loss, we coerced these estimates to the ``nearest'' positive definite estimate via application of the technique presented in \cite{cheng1998modified}.  For a symmetric matrix $A$, which is not positive definite,  a modified Cholesky algorithm produces a symmetric perturbation matrix $E$ such that $A + E$ is positive definite.

\bigskip

\cite{pan2003modelling} present an iterative procedure for estimating coefficient vectors $\lambda$, $\gamma$ of the polynomial model \ref{eq:mcd-polynomial-model}. Their algorithm uses a quasi-Newton step for computing the MLE under the multivariate normal likelihood. Their work is  is implemented in the JMCM package for \textsf{R}, which we used to compute the polynomial MCD estimates.  For implementation details, see \cite{pan2017jmcm}. 	 

\bigskip

In addition to these estimators, we include risk estimates for the oracle estimator for each of the simulation models in Table~\ref{table:simulation-model-list}, which serves as a practical lower bound for the risk under each generating model. For the case of mutual independence with constant variance, the oracle estimator of the covariance matrix is a diagonal matrix with the diagonal elements given by $\hat{\sigma^2}$, which is an estimate of the variance based on all of the data, $y_{ij}$, $i = 1,\dots, N$, $j = 1,\dots, m_i$. The oracle estimator for Model II is obtained by fitting the model

\begin{equation} \label{eq:model-2-oracle-model}
y\left(t_{ij}\right) = \sum_{k < j} \left(\beta_0 + \beta_1 t_{ij}\right) y\left( t_{ik} \right) + \epsilon_{ij},
\end{equation}
\noindent 
where $\epsilon_{ij}$ are independent mean zero Normal random variables with common variance $\sigma^2$. The estimator of $\bfbeta = \left(\beta_0, \beta_1\right)'$, $\hat{\bfbeta}$ is taken to be 

\begin{equation}\label{eq:model-2-oracle-estimator}
\argmin{\beta} \vert \vert Y - X B \beta \vert \vert^2, 
\end{equation}
\noindent
where $X$ denotes the matrix of autoregressive covariates as defined in (\ref{eq:ar-design-matrix-1}) and Example~\ref{example:construction-of-X}, and the matrix $B$ contains the basis for a linear function of $t$:

\[
\begin{bmatrix}
1 & t_{11} \\
1 & t_{12} \\
\vdots & \vdots \\
1 & t_{1,m_1} \\
\vdots & \vdots \\
1 & t_{N,1} \\
\vdots & \vdots \\
1 & t_{N,m_N} \\
\end{bmatrix}.
\]
\noindent
The estimator for $\sigma^2\left(t\right)$ is then the mean of the squared residuals:

\[
\hat{\sigma^2\left( t \right)} = \frac{1}{N}\sum_{i = 1}^N\frac{1}{m_i - 1} \sum_{j = 1}^{m_i} e_{ij}^2,
\]
\noindent
where $e_{i1} = y_{i1}$, $i = 1,\dots, N$. The oracle estimator for Model III is obtained in the same fashion, but $y\left(t_{ij}\right)$ is regressed only on its predecessors such that $t_{ij} - t_{ik} < 0.5$:

\begin{equation} \label{eq:model-3-oracle-model}
y\left(t_j\right) = \sum_{ t_j - t_k < 0.5} \left(\beta_0 + \beta_1 t_j\right) y\left( t_k \right) + \epsilon. 
\end{equation}

The oracle estimator under Model IV, the rational quadratic covariance model, assumes that $Y_1, \dots, Y_N$ is a random sample from a mean zero multivariate normal distribution with covariance matrix $\Sigma = \begin{bmatrix} \sigma_{ij} \end{bmatrix}$, where the elements of the covariance matrix are defined according to the parametric function given in Table~\ref{table:simulation-model-list}.

\bigskip

The compound symmetric covariance model (V) can be written as a simple random effects model:

\begin{equation} \label{eq:model-5-oracle-model}
Y_i = Z_i b_i + \bfepsilon_i,
\end{equation}
\noindent
where $\bfepsilon_i$ is a vector of residuals from a $N\left(0,\sigma_\epsilon^2\right)$ distribution, and the $b_i$ are independent $N\left(0,\sigma_b^2 \mathrm{I}\right)$ random vectors, the elements of which are mutually independent of the elements of $\epsilon_{i}$. The matrix of covariates corresponding to the random effects contains only an intercept term:

\[
Z_i = \begin{bmatrix}  1 \\ 1 \\ \vdots \\1 \end{bmatrix}.
\]
\noindent
Under this model, the within-subject covariance structure is given by
\[
Cov\left(Y_i\right) = \sigma_\epsilon^2 \mathrm{I} +  \sigma_b^2 1 1'. 
\]
\noindent
The oracle estimator can be obtained using restricted maximum likelihood estimation under a Normal likelihood with this covariance structure.

%\subfile{chapter-4-subfiles/chapter-4-benchmark-study-discussion}

%%%%%%%%%%%%%%%%%%%%%%%%%%%%%%%%%%%%%%%%%%%%%%%%%%%%%%%%%%%%%%%%%%%%%%%%%%%%%%%%%%%%%%%%%%%%%%%%
%%%%%%%%%%%%%%%%%%%%%%%%%%%%%%%%%%%%%%%%%%%%%%%%%%%%%%%%%%%%%%%%%%%%%%%%%%%%%%%%%%%%%%%%%%%%%%%%
%%%%%%%%%%%%%%%%%%%%%%%%%%%%%%%%%%%%%%%%%%%%%%%%%%%%%%%%%%%%%%%%%%%%%%%%%%%%%%%%%%%%%%%%%%%%%%%%
%%%%%%%%%%%%%%%%%%%%%%%%%%%%%%%%%%%%%%%%%%%%%%%%%%%%%%%%%%%%%%%%%%%%%%%%%%%%%%%%%%%%%%%%%%%%%%%%
%%%%%%%%%%%%%%%%%%%%%%%%%%%%%%%%%%%%%%%%%%%%%%%%%%%%%%%%%%%%%%%%%%%%%%%%%%%%%%%%%%%%%%%%%%%%%%%%
\section{Data generation procedures}

For each of the covariance models, we generated a set of observations of sample size $N = 50, 100$ from a multivariate normal distribution for each of three different values of within-subject sample size $M = 10, 20, 30$. To generate data according to Models II and III, which are parameterized in terms of the components of the Cholesky decomposition, the Cholesky factor $T$ and diagonal innovation variance matrix $D$ are constructed by evaluating $\phi$ and $\sigma^2$ at the fixed observation times. The data are then sampled according to the multivariate normal distribution with covariance matrix $\Sigma = T^{-1} D {T'}^{-1}$. Given covariance matrix $\Sigma$, risk estimates are obtained from$N_{sim} = 100$ samples from an $M$-dimensional multivariate Normal distribution with mean zero and the same covariance.  Since construction of the sample covariance matrix $S$, $S^\omega$, and $S^\lambda$ rely on having an equal number of regularly-spaced observations on each subject, simulations comparing performance across estimators were conducted using complete data with common measurement times across all $N$ subjects. The observation times, which are equally spaced, are mapped from the integers $1,2, \dots, M$ to the unit interval for estimation.

\bigskip

Our second concern in evaluation of our methods is how performance changes when the data exhibit varying degrees of sparsity. We fix the number of sampled trajectories $N$ and vary $M$, the size of the set  of possible measurement times

\[
t_1,\dots, t_M.
\]
\noindent
We generate irregular data by first generating a complete dataset as we did for the first simulation study:

\begin{align*}
Y_1 &= \left(y_1\left(t_1\right), y_1\left(t_2\right), \dots, y_1\left(t_M\right)\right)' \\
Y_2 &= \left(y_2\left(t_1\right), y_2\left(t_2\right), \dots, y_2\left(t_M\right)\right)' \\
&\vdots \\
Y_N &= \left(y_N\left(t_1\right), y_N\left(t_2\right), \dots, y_N\left(t_M\right)\right)',
\end{align*}
\noindent
where $Y_1,\dots, Y_N$ are independently and identically distributed according to an $M$-dimensional multivariate Normal distribution with mean zero and having covariance structure identical to one of Models I - V in \ref{table:simulation-model-list}. To induce sparsity, we subsample from the complete data $\left\{y_i\left(t_j\right) \right\}$, $i = 1,\dots, N$, $j = 1,\dots, M$, randomly omitting an observation $y_i\left(t_j\right)$ with probability $0.1$, $0.2$, and $0.3$. For both sets of simulations, the smoothing parameters for the smoothing spline and P-spline estimators were selected using both leave-one-subject-out cross validation $\mbox{losoCV}\left(\lambda\right)$ and unbiased risk estimate $\mbox{U}\left(\lambda\right)$. Given the selected values of the tuning parameters, we computed the estimated covariance matrix and compared it to the true covariance matrix via entropy loss and quadratic loss. 

%%%%%%%%%%%%%%%%%%%%%%%%%%%%%%%%%%%%%%%%%%%%%%%%%%%%%%%%%%%%%%%%%%%%%%%%%%%%%%%%%%%%%%%%%%%%%%%%
%%%%%%%%%%%%%%%%%%%%%%%%%%%%%%%%%%%%%%%%%%%%%%%%%%%%%%%%%%%%%%%%%%%%%%%%%%%%%%%%%%%%%%%%%%%%%%%%
%%%%%%%%%%%%%%%%%%%%%%%%%%%%%%%%%%%%%%%%%%%%%%%%%%%%%%%%%%%%%%%%%%%%%%%%%%%%%%%%%%%%%%%%%%%%%%%%
%%%%%%%%%%%%%%%%%%%%%%%%%%%%%%%%%%%%%%%%%%%%%%%%%%%%%%%%%%%%%%%%%%%%%%%%%%%%%%%%%%%%%%%%%%%%%%%%
%%%%%%%%%%%%%%%%%%%%%%%%%%%%%%%%%%%%%%%%%%%%%%%%%%%%%%%%%%%%%%%%%%%%%%%%%%%%%%%%%%%%%%%%%%%%%%%%

\section{Results}

\subsection{Simulations with complete data}
Figure~\ref{fig:cov-estimate-lattice} provides a visual summary of the qualitative differences between the estimates resulting from each of the eight methods of estimation for the five covariance structures used for simulation. The first row in the grid shows the surface plot of each of the true covariance structures, and each row thereafter corresponds to the five covariance estimates for the given estimation method. The surface plots of the oracle estimate in the second row serve as a point of reference for the `gold standard` in each scenario, since the oracle estimates were constructed assuming that the functional form of the covariance is known (either the full covariance structure or the components of the Cholesky decomposition.) The corresponding estimates of the Cholesky factor $T$ for the estimators based on the modified Cholesky decomposition are shown in Figure~\ref{fig:chol-estimate-lattice}, and the decomposition of the $\hat{T}$ corresponding to the smoothing spline ANOVA estimator $\hat{\Sigma}_{SS}$ into functional components is displayed in Figure~\ref{fig:ssanova-component-lattice}

%\subfile{chapter-4-subfiles/chapter-4-cov-lattice-ggplot}

\begin{figure}[H] 
\centering
\caption{\textit{Covariance Model I - Model V (see Table~\ref{table:simulation-model-list}) used for simulation and corresponding estimates. The columns in the grid correspond to each simulation model. The first row of shows the true covariance structure, and each row beneath corresponds to each of the estimators.}}
  \includegraphics[width = 1\textwidth]{img/chapter-4/cov-estimate-lattice}\label{fig:cov-estimate-lattice}
\end{figure}

\begin{figure}[H] 
\centering
\caption{\textit{The generalized autoregressive coefficient function $\phi$ which defines the elements of the true lower triangle of Cholesky factor $T$ corresponding to Model I - Model V and estimates of the same surface for estimators based on the modified Cholesky decomposition. The true covariance structure is displayed across the top row.}}
  \includegraphics[width = 1\textwidth]{img/chapter-4/cholesky-estimate-lattice}
  \label{fig:chol-estimate-lattice}
\end{figure}

%\subfile{chapter-4-subfiles/chapter-4-ssanova-lattice-ggplot}

\begin{figure}[H] 
\caption{\textit{Estimated functional components of the smoothing spline ANOVA decomposition $\phi = \phi_1 + \phi_2 + \phi_{12}$ for $\hat{\Sigma}_{SS}$ under each simulation model I - V.}}
  \includegraphics[width = \textwidth]{img/chapter-4/ssanova-estimate-lattice} \label{fig:ssanova-component-lattice}
\end{figure}

\bigskip
The results of the simulations for complete data under entropy loss are presented in Tables~\ref{table:simulation-1-entropy-loss-sigma-1} - \ref{table:simulation-1-entropy-loss-sigma-5}, where the smoothing parameters for our smoothing spline estimator $\hat{\Sigma}_{SS}$ and P-spline estimator $\hat{\Sigma}_{PS}$ are chosen using the unbiased risk estimate. Performance of the estimator when the smoothing parameter is chosen using leave-one-subject-out cross validation is comparable; these results are left to Appendix~\ref{simulation-studies-appendix}. Risk estimates under quadratic loss, while there is not agreement between results every time, qualitatively, they are similar in nature to those with entropy loss and are also presented in Appendix~\ref{simulation-studies-appendix}, Tables~\ref{table:simulation-1-quad-loss-sigma-1}-\ref{table:simulation-1-quad-loss-sigma-5}. Since both loss functions are not standardized, they cannot be compared across dimensions $M$.

\bigskip

In general, our estimators outperform the alternative estimators across the five covariance structures. This is not surprising; the soft thresholding estimator assumes no ordering of the variables of the random vector, which all but one of the generating structures exhibit. The tapering estimator assumes that the absolute value of the covariance decays as $l$ increases; only model IV satisfies this. The parametric estimator based on the modified Cholesky decomposition assumes that $\phi$ can be modeled as a univariate function of $l$, which does not hold for any of the models, save model IV.

\bigskip

The smoothing spline estimator outperforms the P-spline estimator in cases where the underlying covariance structure cannot be modeled as a multiplicative function of $l$ and $m$ - namely, model II. The P-spline estimator outperforms the smoothing spline estimator under model IV, likely due to the advantage of trivially change the order of the difference penalty. When the difference order is specified so that the generating model belongs to the null space $\hilbert_0$, the search for the optimal set of smoothing parameters is a much easier task.   
 
\bigskip
%%%%%%%%%%%%%%%%%%%%%%%%%%%%%%%%%%%%%%%%%%%%%%%%%%%%%%%%%%%%%%%%%%%%%%%%%%%%%%%%%%%%%%%%%%%%%%%%
%%%%%%%%%%%%%%%%%%%%%%%%%%%%%%%%%%%%%%%%%%%%%%%%%%%%%%%%%%%%%%%%%%%%%%%%%%%%%%%%%%%%%%%%%%%%%%%%
\setlength{\dashlinedash}{0.5pt}
\setlength{\dashlinegap}{1pt}
\setlength{\arrayrulewidth}{0.2pt}
%\subfile{chapter-4-subfiles/simulation-study-1-entropy-table-model-1}
% latex table generated in R 3.4.3 by xtable 1.8-2 package
% Tue Mar  6 10:27:03 2018
\begin{table}[H]
\centering
\caption{\textit{Multivariate normal simulations for Model I. Estimated entropy risk is reported for our smoothing spline ANOVA estimator and P-spline estimator, the oracle estimator for each covariance structure, the parametric polynomial estimator of Pan and MacKenzie (2003), the sample covariance matrix, the tapered sample covariance matrix, and the soft thresholding estimator.}}
\begin{tabular}{lrrrrrrrr}
 & $M$ & $\hat{\Sigma}_{oracle}$& $\hat{\Sigma}_{SS}$& $\hat{\Sigma}_{PS}$ & $\hat{\Sigma}_{poly}$ & $S$ &$S^\omega$& $S^\lambda$ \\ 
  \hline
$N = 50$ & 10 &0.0135 & 0.0685 & 0.1261 &  0.1102 & 1.2047 & 0.5369 & 1.1742 \\ 
   & $20$ & 0.0229 & 0.0834 & 0.1713 &  0.1096 & 4.9850 & 1.3957 & 4.7796 \\ 
   & $30$ & 0.0196 & 0.1102 & 0.1969 &  0.1127 & 12.5517 & 2.8019 & 11.3175 \\ 
 $N = 100$ & $10$ & 0.0105 & 0.0451 & 0.0671 & 0.0531 & 0.5685 & 0.2045 & 0.5236 \\ 
   & $20$ & 0.0105 &0.0425 & 0.0965 &  0.0512 & 2.2831 & 0.5724 & 2.1358 \\ 
   & $30$ &0.0139 & 0.0431 & 0.1148 &  0.0472 & 5.2770 & 1.2430 & 4.9126 \\ 
   \hline
\end{tabular}
\label{table:simulation-1-entropy-loss-sigma-1}
\end{table}


%%%%%%%%%%%%%%%%%%%%%%%%%%%%%%%%%%%%%%%%%%%%%%%%%%%%%%%%%%%%%%%%%%%%%%%%%%%%%%%%%%%%%%%%%%%%%%%%
%%%%%%%%%%%%%%%%%%%%%%%%%%%%%%%%%%%%%%%%%%%%%%%%%%%%%%%%%%%%%%%%%%%%%%%%%%%%%%%%%%%%%%%%%%%%%%%%
% latex table generated in R 3.4.3 by xtable 1.8-2 package
% Tue Mar  6 10:27:30 2018

\begin{table}[H]
\centering
\caption{\textit{Multivariate normal simulations for model II.}}
\begin{tabular}{lrrrrrrrr}
 & $M$ &$\hat{\Sigma}_{oracle}$& $\hat{\Sigma}_{SS}$& $\hat{\Sigma}_{PS}$ & $\hat{\Sigma}_{poly}$ & $S$ &$S^\omega$& $S^\lambda$ \\ 
  \hline
   $N = 50$ & $10$ & 0.0581 &  0.0689 & 0.3423 &4.7673 & 1.2832 & 1.4644 & 1.1770 \\ 
   & $20$ &0.0439 & 0.0581 & 1.3640 &  97.2334 & 5.1665 & 21.6407 & 39.3522 \\ 
     & $30$ & 0.0627 & 0.0811 & 2.6485 &  153.9665 & 12.3582 & 55.3674 & 133.9980 \\ 
  $N = 100$ & 10 &  0.0386 & 0.0457 & 0.2945 & 4.7911 & 0.5812 & 0.8335 & 0.5628 \\ 
    & $20$ & 0.0269 & 0.0416 & 1.2875 &  98.1989 & 2.3364 & 10.1841 & 10.0864 \\ 
    & $30$ &  0.0288 & 0.0367 & 2.4365 & 158.2480 & 5.2389 & 33.5207 & 62.5030 \\ 
   \hline
\end{tabular}
\label{table:simulation-1-entropy-loss-sigma-2}
\end{table}


%%%%%%%%%%%%%%%%%%%%%%%%%%%%%%%%%%%%%%%%%%%%%%%%%%%%%%%%%%%%%%%%%%%%%%%%%%%%%%%%%%%%%%%%%%%%%%%%
%%%%%%%%%%%%%%%%%%%%%%%%%%%%%%%%%%%%%%%%%%%%%%%%%%%%%%%%%%%%%%%%%%%%%%%%%%%%%%%%%%%%%%%%%%%%%%%%
%\subfile{chapter-4-subfiles/simulation-study-1-entropy-table-model-2}
%\subfile{chapter-4-subfiles/simulation-study-1-entropy-table-model-3}
% latex table generated in R 3.4.3 by xtable 1.8-2 package
% Tue Mar  6 10:27:35 2018


\begin{table}[H]
\centering
\caption{\textit{Multivariate normal simulations for model III.} }
\begin{tabular}{lrrrrrrrr}
 & $M$ &$\hat{\Sigma}_{oracle}$&  $\hat{\Sigma}_{SS}$& $\hat{\Sigma}_{PS}$ &$\hat{\Sigma}_{poly}$ & $S$ &$S^\omega$& $S^\lambda$ \\ 
   \hline
 $N = 50$ & 10 & 0.0619 & 0.3296 & 0.1065 & 3.0108 & 1.2030 & 1.1460 & 1.1467 \\ 
      & $20$ &0.0695 & 1.1100 & 0.2555 &  62.7522 & 4.9824 & 17.2244 & 14.9189 \\ 
    & $30$ &0.0576 & 2.3215 & 0.6242 &  1091.1933 & 12.4792 & 49.9135 & 121.7795 \\ 
     $N = 100$ & $10$ & 0.0268 &  0.2904 & 0.0579 &3.0383 & 0.5699 & 0.5545 & 0.5371 \\ 
     & $20$ & 0.0275 & 1.1963 & 0.2011 & 62.8960 & 2.2700 & 11.8274 & 9.5217 \\ 
    & $30$ &  0.0221 & 2.2811 & 0.3845 &1105.0449 & 5.2234 & 29.1693 & 60.3529 \\ 
   \hline
\end{tabular}
\label{table:simulation-1-entropy-loss-sigma-3}
\end{table}

%%%%%%%%%%%%%%%%%%%%%%%%%%%%%%%%%%%%%%%%%%%%%%%%%%%%%%%%%%%%%%%%%%%%%%%%%%%%%%%%%%%%%%%%%%%%%%%%
%%%%%%%%%%%%%%%%%%%%%%%%%%%%%%%%%%%%%%%%%%%%%%%%%%%%%%%%%%%%%%%%%%%%%%%%%%%%%%%%%%%%%%%%%%%%%%%%
%\subfile{chapter-4-subfiles/simulation-study-1-entropy-table-model-4}
% latex table generated in R 3.4.3 by xtable 1.8-2 package
% Tue Mar  6 10:27:39 2018
\begin{table}[H]
\centering
\caption{\textit{Multivariate normal simulations for model IV.}}
\begin{tabular}{lrrrrrrrr}
 & $M$ & $\hat{\Sigma}_{oracle}$ & $\hat{\Sigma}_{SS}$& $\hat{\Sigma}_{PS}$ & $\hat{\Sigma}_{poly}$ & $S$ &$S^\omega$& $S^\lambda$ \\ 
  \hline
 $N = 50$ & $10$ & 0.0217 & 0.3348 & 0.1966 & 0.7144 & 1.2218 & 0.7397 & 1.1921 \\ 
   & $20$ &0.0286 & 0.9177 & 0.3499 &  1.4588 & 4.9091 & 1.9786 & 4.9206 \\ 
       & $30$ &  0.0283 &1.5992 & 0.5100 & 2.2173 & 12.6114 & 3.7440 & 12.1489 \\ 
     $N = 100$ & 10 & 0.0125 & 0.3047 & 0.2237 &  0.6958 & 0.5570 & 0.3168 & 0.5515 \\ 
       & $20$ &0.0105 & 0.8911 & 0.3704 &  1.4813 & 2.2659 & 0.9365 & 2.2474 \\ 
       & $30$ & 0.0134 & 1.5213 & 0.5282 & 2.2228 & 5.2106 & 1.9312 & 5.2111 \\ 
   \hline
\end{tabular} 
\label{table:simulation-1-entropy-loss-sigma-4}
\end{table}

%%%%%%%%%%%%%%%%%%%%%%%%%%%%%%%%%%%%%%%%%%%%%%%%%%%%%%%%%%%%%%%%%%%%%%%%%%%%%%%%%%%%%%%%%%%%%%%%
%%%%%%%%%%%%%%%%%%%%%%%%%%%%%%%%%%%%%%%%%%%%%%%%%%%%%%%%%%%%%%%%%%%%%%%%%%%%%%%%%%%%%%%%%%%%%%%%
%\subfile{chapter-4-subfiles/simulation-study-1-entropy-table-model-5}
% latex table generated in R 3.4.3 by xtable 1.8-2 package
% Tue Mar  6 10:27:42 2018
\begin{table}[H]
\centering
\caption{\textit{Multivariate normal simulations for model V.}}
\begin{tabular}{lrrrrrrrr}
 & $M$ &$\hat{\Sigma}_{oracle}$&$\hat{\Sigma}_{SS}$& $\hat{\Sigma}_{PS}$ & $\hat{\Sigma}_{poly}$ & $S$ &$S^\omega$& $S^\lambda$ \\ 
  \hline
 $N = 50$ & $10$ &  0.0986 &0.2769 & 0.2464 & 1.2420 & 1.2023 & 18.5222 & 2.9824 \\ 
  & $20$ &0.2512 & 0.7514 & 0.8772 &  2.8557 & 5.0195 & 34.6618 & 13.8690 \\ 
  & $30$ &  0.2641& 1.1776 & 0.9791  & 4.5791 & 12.3460 & 46.5437 & 26.1364 \\ 
 $N = 100$ & 10 &0.0520 & 0.2416 & 0.1722 &  1.1491 & 0.5821 & 16.4081 & 1.7397 \\ 
  & $20$ & 0.0827 & 0.7286 & 0.2965 &  2.9080 & 2.2918 & 32.5295 & 5.4649 \\ 
   & $30$ &  0.1799 & 1.1813 & 0.4291 & 4.4402 & 5.2197 & 39.2914 & 15.4295 \\ 
   \hline
\end{tabular}\label{table:simulation-1-entropy-loss-sigma-5}
\end{table}


%\subfile{chapter-4-subfiles/simulation-study-1-entropy-single-table}
%%%%%%%%%%%%%%%%%%%%%%%%%%%%%%%%%%%%%%%%%%%%%%%%%%%%%%%%%%%%%%%%%%%%%%%%%%%%%%%%%%%%%%%%%%%%%%%%
%%%%%%%%%%%%%%%%%%%%%%%%%%%%%%%%%%%%%%%%%%%%%%%%%%%%%%%%%%%%%%%%%%%%%%%%%%%%%%%%%%%%%%%%%%%%%%%%
%%%%%%%%%%%%%%%%%%%%%%%%%%%%%%%%%%%%%%%%%%%%%%%%%%%%%%%%%%%%%%%%%%%%%%%%%%%%%%%%%%%%%%%%%%%%%%%%
%%%%%%%%%%%%%%%%%%%%%%%%%%%%%%%%%%%%%%%%%%%%%%%%%%%%%%%%%%%%%%%%%%%%%%%%%%%%%%%%%%%%%%%%%%%%%%%%
%%%%%%%%%%%%%%%%%%%%%%%%%%%%%%%%%%%%%%%%%%%%%%%%%%%%%%%%%%%%%%%%%%%%%%%%%%%%%%%%%%%%%%%%%%%%%%%%
\subsection{Performance with irregularly sampled data}
%%%%%%%%%%%%%%%%%%%%%%%%%%%%%%%%%%%%%%%%%%%%%%%%%%%%%%%%%%%%%%%%%%%%%%%%%%%%%%%%%%%%%%%%%%%%%%%%
%%%%%%%%%%%%%%%%%%%%%%%%%%%%%%%%%%%%%%%%%%%%%%%%%%%%%%%%%%%%%%%%%%%%%%%%%%%%%%%%%%%%%%%%%%%%%%%%
%%%%%%%%%%%%%%%%%%%%%%%%%%%%%%%%%%%%%%%%%%%%%%%%%%%%%%%%%%%%%%%%%%%%%%%%%%%%%%%%%%%%%%%%%%%%%%%%
%%%%%%%%%%%%%%%%%%%%%%%%%%%%%%%%%%%%%%%%%%%%%%%%%%%%%%%%%%%%%%%%%%%%%%%%%%%%%%%%%%%%%%%%%%%%%%%%
%%%%%%%%%%%%%%%%%%%%%%%%%%%%%%%%%%%%%%%%%%%%%%%%%%%%%%%%%%%%%%%%%%%%%%%%%%%%%%%%%%%%%%%%%%%%%%%%
%\subfile{chapter-4-subfiles/chapter-4-missing-data-study-discussion}

Estimated risk under entropy loss is given in Tables~\ref{table:simulation-study-2-entropy-risk-model-1} - \ref{table:simulation-study-2-entropy-risk-model-5}.  Risk estimates under quadratic loss echo in sentiment and are left to Appendix~\ref{simulation-studies-appendix}, Tables~\ref{table:simulation-study-2-quad-risk-model-1} - \ref{table:simulation-study-2-quad-risk-model-5}. Neither model selection perform better than the other across all of the simulation settings. This might suggest that when the estimated innovation variances are close to the true variances of the prediction residuals, using the unbiased risk estimate with the working residuals as substitute for the relative error is a reasonable approach to modeling. Performance degradation of the estimator in the presence of missing data is highly dependent on the underlying structure of the Cholesky factor of the inverse covariance matrix. For Models I and IV, the identity matrix and the rational quadratic covariance model, performance remains fairly stable as the proportion of missing data increases. The estimator exhibits similar degrees of performance degradation under Models II, III, and V.  Interestingly, these models (with the exception of Model III, which is a special case) have true varying coefficient functions which are naturally parameterized as functions of $t$, while the models under which the performance remain stable across increasing proportions of missing data are naturally parameterized in terms of $l$. 

%%%%%%%%%%%%%%%%%%%%%%%%%%%%%%%%%%%%%%%%%%%%%%%%%%%%%%%%%%%%%%%%%%%%%%%%%%%%%%%%%%%%%%%%%%%%%%%%
%%%%%%%%%%%%%%%%%%%%%%%%%%%%%%%%%%%%%%%%%%%%%%%%%%%%%%%%%%%%%%%%%%%%%%%%%%%%%%%%%%%%%%%%%%%%%%%%
%%%%%%%%%%%%%%%%%%%%%%%%%%%%%%%%%%%%%%%%%%%%%%%%%%%%%%%%%%%%%%%%%%%%%%%%%%%%%%%%%%%%%%%%%%%%%%%%
%%%%%%%%%%%%%%%%%%%%%%%%%%%%%%%%%%%%%%%%%%%%%%%%%%%%%%%%%%%%%%%%%%%%%%%%%%%%%%%%%%%%%%%%%%%%%%%%
%%%%%%%%%%%%%%%%%%%%%%%%%%%%%%%%%%%%%%%%%%%%%%%%%%%%%%%%%%%%%%%%%%%%%%%%%%%%%%%%%%%%%%%%%%%%%%%%
\bigskip
\setlength{\dashlinedash}{0.5pt}
\setlength{\dashlinegap}{1pt}
\setlength{\arrayrulewidth}{0.2pt}

% latex table generated in R 3.4.3 by xtable 1.8-2 package
% Wed Mar 21 09:46:43 2018
\begin{table}[H]
\centering
\caption{\textit{Model 1: Entropy risk estimates and corresponding standard errors 
                            for the MCD smoothing spline ANOVA estimator via 100 simulated multivariate
                            normal samples of size $N = 50$
                            when 0\%, 10\%, 20\%, and 30\% of the data are missing for each subject. Risk is reported for the estimator constructed using
                            the unbiased risk estimate and leave-one-subject-out cross validation for smoothing parameter selection.} }
\label{table:simulation-study-2-entropy-risk-model-1}
\begin{tabular}{lrrlrl}
   $M$ & \% missing & \multicolumn{2} {c} {$\Delta_2(\hat{\Sigma}^{U}_{SS})$} & \multicolumn{2} {c} {$\Delta_2(\hat{\Sigma}^{V^*}_{SS})$}\\ \hline
10 & 0.0 & 0.06854186 & (0.0065) & 0.0822183 & (0.0075) \\ 
   & 0.1 & 0.08895763 & (0.0080) & 0.0997540 & (0.0083) \\ 
   & 0.2 & 0.08474403 & (0.0069) & 0.1257789 & (0.0110) \\ 
   & 0.3 & 0.14281452 & (0.0114) & 0.1552415 & (0.0142) \\ 
   \hline
20 & 0.0 & 0.08337738 & (0.0056) & 0.0924326 & (0.0167) \\ 
   & 0.1 & 0.10467926 & (0.0072) & 0.3019903 & (0.1922) \\ 
   & 0.2 & 0.13920223 & (0.0076) & 0.2099852 & (0.0308) \\ 
   & 0.3 & 0.17160295 & (0.0088) & 0.3784635 & (0.1054) \\ 
   \hline
\end{tabular}
\end{table}


%%%%%%%%%%%%%%%%%%%%%%%%%%%%%%%%%%%%%%%%%%%%%%%%%%%%%%%%%%%%%%%%%%%%%%%%%%%%%%%%%%%%%%%%%%%%%%%%
%%%%%%%%%%%%%%%%%%%%%%%%%%%%%%%%%%%%%%%%%%%%%%%%%%%%%%%%%%%%%%%%%%%%%%%%%%%%%%%%%%%%%%%%%%%%%%%%
% latex table generated in R 3.4.3 by xtable 1.8-2 package
% Wed Mar 21 09:46:43 2018
\begin{table}[H]
\centering
\caption{\textit{Model 2: Entropy risk estimates and corresponding standard errors.} }
\label{table:simulation-study-2-entropy-risk-model-2}
\begin{tabular}{lrrlrl}
   $M$ & \% missing & \multicolumn{2} {c} {$\Delta_2(\hat{\Sigma}^{U}_{SS})$} & \multicolumn{2} {c} {$\Delta_2(\hat{\Sigma}^{V^*}_{SS})$}\\ \hline
10 & 0.0 & 0.0689091 & (0.0057) & 0.0863937 & (0.0070) \\ 
   & 0.1 & 0.0961388 & (0.0066) & 0.1396364 & (0.0119) \\ 
   & 0.2 & 0.2089429 & (0.0140) & 0.1988000 & (0.0173) \\ 
   & 0.3 & 0.2947206 & (0.0212) & 0.3247143 & (0.0297) \\ 
   \hline
20 & 0.0 & 0.0580730 & (0.0042) & 0.0851086 & (0.0061) \\ 
   & 0.1 & 0.6508269 & (0.0437) & 0.6936141 & (0.0366) \\ 
   & 0.2 & 3.9959421 & (0.2127) & 7.9307772 & (2.6348) \\ 
   & 0.3 & 16.4362761 & (1.3678) & 24.4878411 & (1.5554) \\ 
   \hline
\end{tabular}
\end{table}


%%%%%%%%%%%%%%%%%%%%%%%%%%%%%%%%%%%%%%%%%%%%%%%%%%%%%%%%%%%%%%%%%%%%%%%%%%%%%%%%%%%%%%%%%%%%%%%%
%%%%%%%%%%%%%%%%%%%%%%%%%%%%%%%%%%%%%%%%%%%%%%%%%%%%%%%%%%%%%%%%%%%%%%%%%%%%%%%%%%%%%%%%%%%%%%%%
% latex table generated in R 3.4.3 by xtable 1.8-2 package
% Wed Mar 21 09:46:43 2018
\begin{table}[H]
\centering
\caption{\textit{Model 3: Entropy risk estimates and corresponding standard errors.} }
\label{table:simulation-study-2-entropy-risk-model-3}
\begin{tabular}{lrrlrl}
   $M$ & \% missing & \multicolumn{2} {c} {$\Delta_2(\hat{\Sigma}^{U}_{SS})$} & \multicolumn{2} {c} {$\Delta_2(\hat{\Sigma}^{V^*}_{SS})$}\\ \hline
10 & 0.0 & 0.3295884 & (0.0063) & 0.3463639 & (0.0093) \\ 
   & 0.1 & 0.3442326 & (0.0079) & 0.3555080 & (0.0097) \\ 
   & 0.2 & 0.3922506 & (0.0098) & 0.4231472 & (0.0138) \\ 
   & 0.3 & 0.4518739 & (0.0187) & 0.5270384 & (0.0237) \\ 
   \hline
20 & 0.0 & 1.1100351 & (0.0107) & 1.1312420 & (0.0089) \\ 
   & 0.1 & 1.3867351 & (0.0384) & 1.5369483 & (0.0360) \\ 
   & 0.2 & 4.4685998 & (0.2608) & 4.4221240 & (0.2856) \\ 
   & 0.3 & 13.9195476 & (1.3110) & 16.5667952 & (1.1101) \\ 
   \hline
\end{tabular}
\end{table}

%%%%%%%%%%%%%%%%%%%%%%%%%%%%%%%%%%%%%%%%%%%%%%%%%%%%%%%%%%%%%%%%%%%%%%%%%%%%%%%%%%%%%%%%%%%%%%%%
%%%%%%%%%%%%%%%%%%%%%%%%%%%%%%%%%%%%%%%%%%%%%%%%%%%%%%%%%%%%%%%%%%%%%%%%%%%%%%%%%%%%%%%%%%%%%%%%
% latex table generated in R 3.4.3 by xtable 1.8-2 package
% Wed Mar 21 09:46:43 2018
\begin{table}[H]
\centering
\caption{\textit{Model 4: Entropy risk estimates and corresponding standard errors.} }
\label{table:simulation-study-2-entropy-risk-model-4}
\begin{tabular}{lrrlrl}
   $M$ & \% missing & \multicolumn{2} {c} {$\Delta_2(\hat{\Sigma}^{U}_{SS})$} & \multicolumn{2} {c} {$\Delta_2(\hat{\Sigma}^{V^*}_{SS})$}\\ \hline
10 & 0.0 & 0.3347516 & (0.0056) & 0.3420091 & (0.0063) \\ 
   & 0.1 & 0.3561451 & (0.0076) & 0.3536609 & (0.0079) \\ 
   & 0.2 & 0.3901020 & (0.0111) & 0.3884112 & (0.0098) \\ 
   & 0.3 & 0.4395183 & (0.0139) & 0.4399004 & (0.0162) \\ 
   \hline
20 & 0.0 & 0.9176583 & (0.0083) & 0.9345338 & (0.0074) \\ 
   & 0.1 & 0.9316105 & (0.0101) & 0.9592996 & (0.0116) \\ 
   & 0.2 & 0.9620128 & (0.0090) & 1.0192813 & (0.0201) \\ 
   & 0.3 & 1.0339355 & (0.0123) & 1.0986877 & (0.0680) \\ 
   \hline
\end{tabular}
\end{table}


%%%%%%%%%%%%%%%%%%%%%%%%%%%%%%%%%%%%%%%%%%%%%%%%%%%%%%%%%%%%%%%%%%%%%%%%%%%%%%%%%%%%%%%%%%%%%%%%
%%%%%%%%%%%%%%%%%%%%%%%%%%%%%%%%%%%%%%%%%%%%%%%%%%%%%%%%%%%%%%%%%%%%%%%%%%%%%%%%%%%%%%%%%%%%%%%%
% latex table generated in R 3.4.3 by xtable 1.8-2 package
% Wed Mar 21 09:46:43 2018
\begin{table}[H]
\centering
\caption{\textit{Model 5: Entropy risk estimates and corresponding standard errors.} }
\label{table:simulation-study-2-entropy-risk-model-5}
\begin{tabular}{lrrlrl}
   $M$ & \% missing & \multicolumn{2} {c} {$\Delta_2(\hat{\Sigma}^{U}_{SS})$} & \multicolumn{2} {c} {$\Delta_2(\hat{\Sigma}^{V^*}_{SS})$}\\ \hline
10 & 0.0 & 0.2768874 & (0.0054) & 0.2855551 & (0.0090) \\ 
   & 0.1 & 0.4139307 & (0.0160) & 0.4290270 & (0.0161) \\ 
   & 0.2 & 0.8698641 & (0.0448) & 0.9289941 & (0.0586) \\ 
   & 0.3 & 1.8588993 & (0.1172) & 2.1368920 & (0.1284) \\ 
   \hline
20 & 0.0 & 0.7514261 & (0.0053) & 0.7609570 & (0.0063) \\ 
   & 0.1 & 1.2295533 & (0.0522) & 1.1317517 & (0.0294) \\ 
   & 0.2 & 2.5715989 & (0.0976) & 2.4974678 & (0.1081) \\ 
   & 0.3 & 7.4723499 & (0.3235) & 6.8275522 & (0.3006) \\ 
   \hline
\end{tabular}
\end{table}


%%%%%%%%%%%%%%%%%%%%%%%%%%%%%%%%%%%%%%%%%%%%%%%%%%%%%%%%%%%%%%%%%%%%%%%%%%%%%%%%%%%%%%%%%%%%%%%%
%%%%%%%%%%%%%%%%%%%%%%%%%%%%%%%%%%%%%%%%%%%%%%%%%%%%%%%%%%%%%%%%%%%%%%%%%%%%%%%%%%%%%%%%%%%%%%%%

%\subfile{chapter-4-subfiles/chapter-4-numerical-discussion-2}
%%%%%%%%%%%%%%%%%%%%%%%%%%%%%%%%%%%%%%%%%%%%%%%%%%%%%%%%%%%%%%%%%%%%%%%%%%%%%%%%%%%%%%%%%%%%%%%%
%%%%%%%%%%%%%%%%%%%%%%%%%%%%%%%%%%%%%%%%%%%%%%%%%%%%%%%%%%%%%%%%%%%%%%%%%%%%%%%%%%%%%%%%%%%%%%%%
%%%%%%%%%%%%%%%%%%%%%%%%%%%%%%%%%%%%%%%%%%%%%%%%%%%%%%%%%%%%%%%%%%%%%%%%%%%%%%%%%%%%%%%%%%%%%%%%
%%%%%%%%%%%%%%%%%%%%%%%%%%%%%%%%%%%%%%%%%%%%%%%%%%%%%%%%%%%%%%%%%%%%%%%%%%%%%%%%%%%%%%%%%%%%%%%%
%%%%%%%%%%%%%%%%%%%%%%%%%%%%%%%%%%%%%%%%%%%%%%%%%%%%%%%%%%%%%%%%%%%%%%%%%%%%%%%%%%%%%%%%%%%%%%%%


%\bibliography{../Master}
%\end{document}
       

\chapter{Data analysis} \label{data-analysis-chapter}
\subsubsection{Kenward cattle weight data}

\cite{kenward1987method} reported an experiment designed to investigate the impact of the control of intestinal parasites in cattle. The grazing season runs from spring to autumn, during which cattle can potentially ingest roundworm larvae which develop from eggs deposited around the pasture from feces of previously infected cattle. Once infected, the animal is deprived of nutrients and immune resistance to disease is suppressed which can significantly impact animal growth. Monitoring the effect of a treatment for the disease requires repeated weight measurements on animals over the grazing season. 

\bigskip

To compare two methods for controlling the disease, say treatment A and treatment B, each of 60 cattle were assigned randomly to two groups, each of size 30. Animal subjects were put out to pasture at the start of grazing season, with each member of the groups receiving one of the two treatments. Animals were weighed $m = 11$ times over a 133-day period; the first 10 measurements on each animal were made at two-week intervals and the final measurement was made one week later. Weights were recorded to the nearest kilogram. The measurement times were common across animals and were rescaled to $t = 1, 2, \dots, 10, 10.5$. The longitudinal dataset is balanced, as there were no missing observations for any of the experimental units. Observed weights are shown in Figure~\ref{fig:cattle-weights-by-trt}.
  
\begin{figure}[h] 
\begin{center}
\includegraphics[width = \textwidth]{img/cattle/cattle-weights-vs-time-by-trt}
\caption{Subject-specific weight curves over time for treatment groups A and B.}\label{fig:cattle-weights-by-trt}
\end{center}
\end{figure} 

We see an upward trend in weights over time, with variance in weights increasing over time for both groups. Treatment group B demonstrates a sharp decrease in the final weight measurement. The analysis of the same dataset provided by \cite{zimmerman1997structured} rejected equality of the two covariance matrices corresponding to treatment group using the classical likelihood ratio test, making it reasonable to study each treatment group's covariance matrix separately. Following \cite{pan2017jmcm}, \cite{zhang2015joint}, \cite{pourahmadi1999joint}, and \cite{pan2006regression}, we analyze the data from the $N = 30$ cattle assigned to treatment group A, which we assume share a common $11 \times 11$ covariance matrix $\Sigma$. The profile plot in Figure~\ref{fig:cattleA-weights-vs-time} of the weights for units in treatment group A shows a clear upward trend in weights;  variances appear to increase over time, suggesting that the covariance structure is nonstationary.

\bigskip

Before modeling the covariance structure, it is necessary to construct an adequate estimate of the mean weight trajectories. After centering the data using the fitted mean, the residuals serve as the data reserved for estimating the functions defining the Cholesky factor and innovation variances.  To account for any between-subject variability, we adopt an approach akin to the dynamical conditionally linear mixed model proposed by Pourahmadi and Daniels; see \cite{pourahmadi2002dynamic} for a detailed discussion. We model

\begin{equation} \label{eq:cattleA-dynamic-cond-mixed-model-1}
r\left(t_{ij}\right) = \sum_{k < j} \phi\left( t_{ij}, t_{ik} \right) r\left(t_{ij}\right) + \epsilon\left(t_{ij}\right)
\end{equation}
\noindent
where
\begin{equation} \label{eq:cattleA-dynamic-cond-mixed-model-2}
r\left(t_{ij}\right) = y\left(t_{ij}\right) - \left(f\left(t_{ij} \right) + \alpha_{i}\right).
\end{equation}
\noindent
The subject-specific random effects $\left\{ \alpha_i \right\}$ are assumed to be mutually independent and independent of $ \epsilon\left(t_{ij}\right)$ for all $i,\;j$, with

\[
\alpha_i \sim N\left( 0, \sigma_\alpha^2 \right).
\]
\noindent
We assume that the subject trajectories share a common mean function $f \in $

\[
\mathcal{C}^2 = \left\{f: \; f,\;f' \mbox{ absolutely continuous, } \int\left(f''\left(x\right)\right)^2 \;dx < \infty  \right\}.
\]
\noindent
Figure~\ref{fig:cattleA-weights-vs-time} displays the he observed weight trajectories over time. Figure~\ref{fig:cattleA-smoothed-weights-vs-time} shows the corresponding fitted mean curves.

\begin{figure}[H] 
\begin{center}
    \includegraphics[width=\textwidth]{img/cattle/cattleA-weights-vs-time}
\end{center}
 \caption{Weight trajectories over the observation period for experimental units in treatment group A.}\label{fig:cattleA-weights-vs-time}
 \end{figure}

\begin{figure}[H] 
\caption{Fitted mean weight curve for cattle in treatment group A. }
\begin{center}
\includegraphics[width = \textwidth]{img/cattle/cattleA-weights-vs-time-mean-fit}
\end{center}\label{fig:cattleA-smoothed-weights-vs-time}
\end{figure} 

The nonstationarity suggested in Figure~\ref{fig:cattle-weights-by-trt} is also supported by the sample correlations given in Table~\ref{table:cattleA-sample-correlations}; correlations within the subdiagonals are not constant and increase over time, a secondary indication that a stationary covariance is not appropriate for the data.  Table~\ref{table:sample-regressogram-garps} gives the sample generalised autoregressive parameters and the innovation variances, which are plotted in Figure~\ref{fig:cattleA-regressogram} and Figure~\ref{fig:cattleA-innovation-variogram} respectively. 

\begin{table}[H] 
\begin{center}
\begin{tabular}{r|rrrrrrrrrrr}
& \multicolumn{11}{c}{day}\\
&&&&&&&&&&\\
& 0 & 14 & 28 & 42 & 56 & 70 & 84 & 98& 112& 126 &133\\
  \hline\noalign{\smallskip} 
0 & 1.00 & 0.82 & 0.76 & 0.65 & 0.63 & 0.58 & 0.51 & 0.52 & 0.51 & 0.46 & 0.46 \\ 
  14 & 0.82 & 1.00 & 0.91 & 0.86 & 0.83 & 0.75 & 0.64 & 0.68 & 0.61 & 0.59 & 0.56 \\ 
  28 & 0.76 & 0.91 & 1.00 & 0.93 & 0.89 & 0.85 & 0.75 & 0.77 & 0.71 & 0.69 & 0.67 \\ 
  42 & 0.65 & 0.86 & 0.93 & 1.00 & 0.93 & 0.90 & 0.80 & 0.82 & 0.74 & 0.70 & 0.67 \\ 
  56 & 0.63 & 0.83 & 0.89 & 0.93 & 1.00 & 0.94 & 0.85 & 0.88 & 0.81 & 0.77 & 0.74 \\ 
  70 & 0.58 & 0.75 & 0.85 & 0.90 & 0.94 & 1.00 & 0.92 & 0.93 & 0.89 & 0.85 & 0.81 \\ 
  84 & 0.51 & 0.64 & 0.75 & 0.80 & 0.85 & 0.92 & 1.00 & 0.92 & 0.92 & 0.86 & 0.84 \\ 
  98 & 0.52 & 0.68 & 0.77 & 0.82 & 0.88 & 0.93 & 0.92 & 1.00 & 0.96 & 0.94 & 0.91 \\ 
  112 & 0.51 & 0.61 & 0.71 & 0.74 & 0.81 & 0.89 & 0.92 & 0.96 & 1.00 & 0.96 & 0.95 \\ 
  120 & 0.46 & 0.59 & 0.69 & 0.70 & 0.77 & 0.85 & 0.86 & 0.94 & 0.96 & 1.00 & 0.98 \\ 
  133 & 0.46 & 0.56 & 0.67 & 0.67 & 0.74 & 0.81 & 0.84 & 0.91 & 0.95 & 0.98 & 1.00 \\ 
   \hline
\end{tabular}
\caption{Cattle data: treatment group A sample correlations.}\label{table:cattleA-sample-correlations}
\end{center}
\end{table}


\begin{table}[H] 
\begin{center}
\begin{tabular}{lc|ccccccccccc|cr}
 \multicolumn{14}{c}{day} \\
&&&&&&&&&&&&\\
& &  0 & 14 & 28 & 42 & 56 & 70 & 84 & 98 & 112 & 126 & 133  \\ 
  \cline{2-13}\noalign{\smallskip}  
&0 & 1 & &&&&&&&&& & 4.673& \\ 
&  14& 1.00 & $\ddots$&&&&&&&&&& 3.939 &\\ 
&  28 & 0.04 & 0.90 &  &&&&&&&&& 3.370&\\ 
&  42 & -0.25 & 0.25 & 0.88 &  &&&&&&&&3.000& \\ 
&  56 & -0.02 & 0.07 & 0.12 & 0.90 & &&&&&&& 3.299&\\ 
day &  70 & 0.04 & -0.28 & 0.11 & 0.37 & 0.82  &&&&&&& 3.363 & $\log\left(\sigma^2_t\right)$\\ 
 & 84 & 0.12 & -0.23 & 0.04 & -0.16 & 0.08 & 1.03  &&&&&& 3.610\\ 
 & 98 & -0.06 & 0.05 & 0.02 & -0.27 & 0.23 & 0.61 & 0.42 &&&&& 3.403&\\ 
 & 112 & 0.18 & -0.10 & 0.05 & -0.26 & -0.10 & 0.03 & 0.30 & 0.93&&&& 2.780&  \\ 
 & 126 & -0.26 & 0.15 & 0.45 & -0.33 & -0.19 & 0.01 & -0.18 & 0.37 & 0.94 & $\ddots$&&3.280& \\ 
 & 133 & 0.13 & -0.26 & 0.08 & 0.28 & 0.04 & -0.36 & -0.05 & -0.07 & 0.37 & 0.85 & 1  &2.262&\\ 
\end{tabular} 
\caption{Cattle data: treatment group A sample generalized autoregressive parameters (below the main diagonal) and log sample innovation variances (rightmost column.)}\label{table:sample-regressogram-garps}
\end{center}
\end{table}

\begin{figure}[H]
 \begin{subfigure}[t]{.48\textwidth}
  \centering
    \includegraphics[width=\textwidth]{img/cattle/cattleA-innovation-variogram}
 \caption{Sample innovation variances $\hat{\sigma}_t^2$} \label{fig:cattleA-innovation-variogram}
 \end{subfigure}
 \begin{subfigure}[t]{.48\textwidth}
  \centering
\includegraphics[width = \textwidth]{img/cattle/cattleA-regressogram}
 \caption{Sample generalized autoregressive parameters $\hat{\phi}_{ij}$.}
\label{fig:cattleA-regressogram}
 \end{subfigure}
 \caption{\textit{Empirical estimates of the parameters of the Cholesky decomposition.}} \label{fig:cattleA-innovation-variogram-and-regressogram}
\end{figure}

%
%\begin{figure}[H] \label{fig:cattleA-innovation-variogram}
%\begin{center}
%    \includegraphics[width=\textwidth]{img/cattle/cattleA-innovation-variogram}
%\end{center}
% \caption{Sample estimates of innovation variances $\sigma_t^2$ obtained by applying the modified Cholesky decomposition to the sample covariance matrix.}
% \end{figure}
%
%\begin{figure}[H] \label{fig:cattleA-regressogram}
%\begin{center}
%\includegraphics[width = \textwidth]{img/cattle/cattleA-regressogram}
%\end{center}
% \caption{Sample estimates of the generalized autoregressive parameters $\phi_{ij}$ obtained by applying the modified Cholesky decomposition to the sample covariance matrix.}
%\end{figure} 


Analyzing the sample regressogram (Figure~\ref{fig:cattleA-regressogram}) and sample innovation variogram (Figure~\ref{fig:cattleA-innovation-variogram}), \cite{pourahmadi1999joint}suggested that both sample generalised autoregressive parameters and the logarithms of the innovation variances can be characterized in terms of cubic functions of the lag only. They model 

\begin{align}
\begin{split} \label{eq:pourahmadi-cubic-model}
\phi_{ts} = x'_{ts}\gamma, \\
\log\left(\sigma_t^2\right) = z'_{t}\xi, 
\end{split}
\end{align}
\noindent
for $t = 2,\dots, 11$ where 

\begin{align*}
x'_{ts} = \begin{bmatrix} 1 & t - s& \left(t - s\right)^2 & \left(t - s\right)^3 \end{bmatrix},\; \mbox{and } z'_{t} = \begin{bmatrix} 1 & t& t^2& t^3 \end{bmatrix}.
\end{align*}
\noindent
They estimate of $\gamma$ and $\xi$ via maximum likelihood.  Figure~\ref{fig:cattleA-smoothed-regressogram-variogram} shows the estimated cubic polynomials corresponding to Model~\ref{eq:pourahmadi-cubic-model}. 

%\begin{figure}[H]
%\centering
%\subfloat[The sample regressogram for the cattle data from treatment group A, overlaid with a cubic polynomial smooth.]{
%  \includegraphics[width = .45\textwidth]{img/cattle/cattleA-regressogram-with-cubic-smooth}\label{fig:cattleA-regressogram-cubic-smooth}
%} 
%\hfill
%\subfloat[The sample variogram for the cattle data from treatment group A, overlaid with a cubic polynomial smooth.]{
%  \includegraphics[width = .45\textwidth]{img/cattle/cattleA-innovariogram-with-cubic-smooth}\label{fig:cattleA-innovariogram-cubic-smooth}
%} 
%\end{figure}


\begin{figure}[H]
 \begin{subfigure}{.48\textwidth}
  \centering
\includegraphics[width = \textwidth]{img/cattle/cattleA-regressogram-with-cubic-smooth}
 \caption{Smoothed sample regressogram.} 
 \label{fig:cattleA-innovariogram-cubic-smooth}
 \end{subfigure}
 \begin{subfigure}{.48\textwidth}
  \centering
\includegraphics[width = \textwidth]{img/cattle/cattleA-innovariogram-with-cubic-smooth}
 \caption{Smoothed sample innovation variances.} 
\label{fig:cattleA-innovariogram-cubic-smooth}
 \end{subfigure}
 \caption{\textit{Cubic polynomomials fitted to the sample regressogram and innovation variances for the cattle data from treatment group A.}} \label{fig:cattleA-smoothed-regressogram-variogram}
\end{figure}

Choice of penalty is critical for convergence of the iterative estimation of $\phi$ and $\log\left(\sigma_2 \right)$. \cite{pan2017jmcm} concluded that the regressogram of empirical estimates of $\phi_{t,t-l}$ show consistent behaviour over $l$ for each value of $t$, indicating a lack of a strong functional component of $m$. To facilitate a making this modeling decision in an entirely data-driven manner, we let $\phi \in \hilbert = \hilbert_{\left[1\right]} \otimes \hilbert_{\left[2\right]}$, where 

\begin{align*} 
\hilbert_{\left[1\right]} &= \bigg\{ \phi: \ddot{\phi} = 0 \bigg\} \oplus \left\{\phi: \phi\left(0\right) = \dot{\phi}\left(0\right) = 0; \;\; \int\limits_0^1 \ddot{\phi}^2 \;dx < \infty \right\} \\
\hilbert_{\left[2\right]} &= \bigg\{ \phi: \phi \propto 1 \bigg\} \oplus \left\{ \phi: \int\limits_0^1 \phi \;dx = 0, \;\; \dot{f} \in \mathcal{L}_2\left[0,1\right]  \right\} 
\end{align*} 
\noindent
This decomposition leads to a null space comprised of functions of $l$ only, which is attractive because it coincides with the modeling assumptions made by $\phi$ \cite{pan2017jmcm}, \cite{huang2006covariance}, and \cite{wu2003nonparametric} for the same data set.  Figure~\ref{fig:fitted-cholesky-decomposition-cattle-date} shows the estimated Cholesky surface $\phi\left( t,s\right)$ and innovation variance function $\sigma^2\left(t\right)$ evaluated at $t = 1, 2, \dots, 10, 10.5$ and the corresponding pairs of observation times $\left(t,s\right)$, $1 \le s < t \le 10.5$. Figure~\ref{fig:cattle-fitted-cholesky-ssanova} shows $\hat{\phi}$ decomposed into the functional components of its ANOVA decomposition.

%\begin{figure}[H]
% \begin{subfigure}{.33\textwidth}
%  \centering
%  \includegraphics[width = \textwidth]{img/chapter-5/cattle-cholesky-estimate-ggplot}
% \caption{Estimated Cholesky factor $\hat{T}$}
% \end{subfigure}
% \begin{subfigure}{.33\textwidth}
%  \centering
%  \includegraphics[width = \textwidth]{img/chapter-5/cattle-D-estimate-ggplot}
% \caption{Estimated innovation variances \newline $\hat{D} = diag\left( \sigma^2\left(t_1\right),\dots, \sigma^2\left(t_{11}\right) \right)$}
% \end{subfigure}
% \begin{subfigure}{.33\textwidth}
%  \centering
%  \includegraphics[width = \textwidth]{img/chapter-5/cattle-cov-estimate-ggplot}
% \caption{Estimated covariance matrix $\hat{\Sigma} = \hat{T}^{-1} \hat{D} {\hat{T}'}^{-1}$}
% \end{subfigure}
%\caption{Components of the fitted modified Cholesky decomposition for the cattle weight data.} \label{fig:fitted-cholesky-decomposition-cattle-date}
%\end{figure}
%

\begin{figure}[H]
  \centering
  \includegraphics[width = \textwidth]{img/chapter-5/cattle-cholesky-estimate-ggplot}
\caption{Components of the fitted modified Cholesky decomposition for the cattle weight data.} \label{fig:fitted-cholesky-decomposition-cattle-date}
\end{figure}


%\subfile{chapter-5-subfiles/cattle-cholesky-ssanova-ggplot}
\begin{figure}[H] 
\centering
\caption{Components of the SSANOVA decomposition of the estimated generalized autoregressive coefficient function $\phi$ evaluated on the grid defined by the observed time points.}
  \includegraphics[width = 0.6\textwidth]{img/chapter-5/cattle-ssanova-estimate-lattice} \label{fig:cattle-fitted-cholesky-ssanova}
\end{figure}

{\needsparaphrased{TO DO: compare with the likelihood evaluated at the models suggested in \cite{pan2017jmcm}.}}

Evaluating the normal likelihood at the fitted model gives $\hat{\ell} = -818.5323$.


%\bibliography{../Master}
%
%\end{document}
 

\chapter{Concluding Remarks and Future Work}\label{concluding-remarks-chapter}


%%%%%%%%%%%%%%%%%%%%%%%%%%%%%%%%%%%%%%%%%%%%%%%%%%%


The previous discussion proposes a flexible framework for estimating the covariance matrix for longitudinal data. By modeling the Cholesky decomposition of the covariance matrix, we reframe covariance estimation as the estimation of a varying coefficient model, which allows for unconstrained estimation as well as a statistically intuitive interpretation of the elements of a covariance matrix. The varying coefficient model for the Cholesky decomposition naturally accommodates irregularly-spaced longitudinal data and allows varying within-subject sample sizes without requiring imputation of missing observations. The overall framework inherits the flexibility of the varying coefficient model, which allows us to leverage any of the tools classically used for nonparametric regression problems in the context of covariance estimation.

\bigskip
%%%%%%%%%%%%%%%%%%%%%%%%%%%%%%%%%%%%%%%%%%%%%%%%%%%

Estimation of the varying coefficient model is performed using bivariate smoothing using penalties which are motivated by the prevalent tendency to specify stationary models for the covariance matrix. Penalties enforce regularization of the fitted function so that under heavy penalization, the fitted components of the Cholesky factor correspond to covariance matrices which are close to stationary. We demonstrate the estimation procedure with two proposed representations of the varying coefficient function and the innovation variance function. A smoothing spline ANOVA model for the generalized autoregressive varying coefficient and the innovation variance function allow the fitted functions to be decomposed into their stationary and nonstationary functional components. We propose an alternative functional representation for $\left(\phi, \log\sigma^2\right)$ using tensor product B-splines; smoothness is achieved by applying penalties to discrete differences of the vector of basis coefficients. The discrete penalties, which are constructed independently of the basis, offer flexibility over the smoothing spline penalties and require little computational complexity to implement. 

\bigskip
The choice of basis is important when the unknown functions parameterizing the varying coefficient model are better represented by one or the other. Simulation studies reveal the advantages and disadvantages of our smoothing spline estimator and our P-spline estimator. The simulations illustrate the relative performance of both estimators compared to alternative estimators proposed in the longitudinal data literature. 

\bigskip

We apply our method to data generated from a longitudinal experiment examining the effectiveness of two treatments for intestinal parasites in cattle as measured by subject body weight over time. For a single treatment group, our nonparametric estimator echoes some of the modeling assumptions made to specify parametric models in previous analysis of the same data.

%%%%%%%%%%%%%%%%%%%%%%%%%%%%%%%%%%%%%%%%%%%%%%%%%%%

\bigskip

Minimizing computational demand is an obvious motivator for future extensions of our work. The smoothing spline estimator circumvents the need for knot selection since it is constructed using a basis function for each of the unique within-subject pairs of measurement times. This is suitable when there is a fixed set of measurement times and unbalanced date arise due to missing observations. For the case that there is little overlap in measurement times across subjects so that these times are ``nearly'' unique for each subject, the size of the set $\vert V \vert $ can be so large that the dimension of the kernel matrix $K_n$ as defined in (\ref{eq:penalized-likelihood-vectorized}) presents serious computational problems. An infinite dimensional Hilbert space is not necessarily required for representing the unknown function to be estimated, since the penalty effectively enforces a low dimensional model space. Efficient approximation can be carried by using a subset of the elements in $V$ to represent the unknown function. Algorithm~\ref{alg:SSANOVA-algorithm} can directly accommodate such a low dimensional representation. \cite{kim2004smoothing} provide detailed discussion of the efficient approximation of $\hilbert$.

\bigskip 

The versatility of this framework leaves many paths open for further modeling exploration. In particular, an obvious example is the classification of observations into one of $K$ groups by quadratic discriminant analysis. A lot of recent attention has been directed toward the problem of estimating separate $p\times p$ covariance matrices $\Sigma_1,\dots, \Sigma_K$ for $K$ separate groups, where the number of groups $K$ and the dimension $p$ are both potentially large. Often, there is not enough data to estimate separate $\Sigma_i$ well for each group. For example, problems in financial management including portfolio selection can be reduced to the prediction of a sequence of large $p \times p$ covariance matrices \citep{tsay2005analysis}. Our procedure to covariance estimation encourages exploration of this problem; the regression model associated with the Cholesky decomposition \eqref{eq:mcd-ar-model} can incorporate additional group-specific covariates. 

\bigskip

The construction of the penalty for the P-spline estimator is convenient and easily appended to the log likelihood. This allows us to easily use both shrinkage and smoothing for covariance estimation, and combining shrinkage and smoothing may produce better estimates than using shrinkage and smoothing alone. While adding additional penalty parameters can introduce added computational requirements, the connection between nonparametric regression modes and mixed models presents a way to mitigate this complexity. Smoothing parameters are interpreted as variances of random effects, so model estimation and smoothing parameter selection can be performed simultaneously using the stable and efficient algorithms and software that are available for mixed models. Restricted maximum likelihood (REML) has proven to be very useful as a model selection tool, often producing smoother fits than generalized cross validation due to its better resistance to over-fitting \citep{wand2003semiparametric}. 


\bigskip
Recently \cite{eilers1999discussion} pointed how to interpret P-splines as a mixed model. \cite{lee2011p} proposed the use of P-splines within a mixed modeling framework to estimate multidimensional functions which can be decomposed into their functional components as with smoothing spline ANOVA models. This approach adds attractiveness to the interpretability of the models, and it allows for computationally convenient model fitting and selection. Application of the mixed model framework presented in \cite{lee2011p} to estimation of $\phi$ is attractive, because it not only provides an avenue for stable smoothing parameter selection, but it also permits the decomposition of the tensor product into functional components as in the SSANOVA model presented in Chapter~\ref{SSANOVA-chapter}. Direct application of this approach, however, is inaccessible due to the deconstruction of the marginal B-spline bases to adjust for the triangular domain of the autoregressive varying coefficient. Figure~\ref{fig:triangle-domain} illustrates how to ``trim'' the pairs of B-splines that don't overlap with the domain of $\phi$, which lies on the triangle $0 < s < t< 1$. Trimming the basis inhibits identifiability of functional components, though in the case that an additive model is appropriate, this trimming is unnecessary and REML may be employed for model estimation.

\bigskip

Alternatively, bivariate B-splines inherit several of the appealing properties of univariate B-splines and are applicable in various modeling problems, particularly for those involving non-rectangular domains. They have been used extensively in the field of graphics for the construction of smooth surfaces over irregular domains, but thus have far received little attention in the field of statistics. However, a recent paper by \cite{zhou2014principal} employs a mixed effects model for the functional principal components as bivariate splines on triangulations for data observed on an irregular grid. The application of their ideas to covariance estimation presents a promising approach to estimation of the Cholesky decomposition via bivariate smoothing.  


       

%
% If you have appendices in your dissertation, you will need the
% following, else keep it commented. The following appendices are in
% files called ``app1.tex'', and ``app2.tex'', and they
% look just like any chapter.
%

\appendix
\include{SSANOVA-appendix}
\chapter{Chapter~\ref{psplines-chapter}} \label{psplines-appendix}

\section{Connecting the finite difference penalty to B-spline derivatives}

The evaluation of the $i^{th}$ B-spline using the recursive relation (\ref{eq:bspline-recursive-relation}) can be derived from their definition as divided differences of truncated power functions.  

\begin{definition} \label{definition:order_k_Bspline}
Let $t= \left\{ t_i \right\}$ denote a non-decreasing sequence. The $i^{th}$ B-spline of order $k$ which corresponds to the knot sequence $t$ is defined by 
\begin{equation} \label{eq:bspline_definition}
B_{i,k,t}\left(x\right) = \left(t_{i+k}-t_i\right)\left[t_i,\dots,t_{i+k}\right]\left(\cdot -x\right)_+^{k-1}
\end{equation}
\end{definition}

The placeholder notation, $\left(\cdot - x\right)_+^{k-1}$, is used to indicate that the $k^{th}$ divided difference of the truncated power function $g\left(t \right) = \left(t-x\right)^{k-1}_+$ is obtained by fixing $x$ and applying the divided difference to $g\left(t \right)$ as a function of $t$ alone. Henceforth, we will write $B_{ik}$ rather than $B_{i,k,t}$ when the knot sequence can be inferred from surrounding context.

\bigskip

The definition of $B_i$ as a divided difference is necessary to bridge the expression for its derivative to the differences of its coefficients. The derivative of the truncated power function $g\left(x\right) = \left(t-x\right)_+^{k-1}$ is given by 
\[
\frac{\partial}{\partial x}g\left(x\right) = \frac{\partial}{\partial x} \left(t-x\right)_+^{k-1} = -\left(k-1\right)\left(t-x\right)_+^{k-2}.
\]

\noindent
Substituting \ref{eq:bspline_definition} into the recursive relation (\ref{eq:bspline-recursive-relation}), we may write the derivative of the $i^{th}$ B-spline of order $k$ as follows:
\begin{align*}
 B'_{i,k}\left(x\right) &= \bigg[ \left[ t_{i+1},\dots,t_{i+k} \right] -\left[ t_{i},\dots,t_{i+k-1} \right] \bigg] \frac{\partial}{\partial x} \left(\cdot - x\right)_+^{k-1}\\
&= -\left(k-1\right) \bigg[ \left[ t_{i+1},\dots,t_{i+k} \right] -\left[ t_{i},\dots,t_{i+k-1} \right] \bigg] \left( \cdot - x \right)^{k-2}_+ \\
&= -\left(k-1\right) \bigg[ -\frac{B_{i+1,k-1}\left(x\right)}{\left(t_{i+k} - t_{i+1} \right)}  + \frac{B_{i,k-1}\left(x\right)}{\left(t_{i+k-1} - t_{i} \right)} \bigg]  
\end{align*}

\noindent
This allows us to write 
\begin{align} 
\frac{\partial}{\partial x} \bigg[ \sum_{i} \theta_i B_i \bigg] &=  \sum_i \theta_i B'_{i,k}\nonumber \\
&= \sum_i \left(k-1\right) \frac{\theta_i - \theta_{i-1}}{t_{i+k-1}-t_i}B_{i,k-1}. \label{eq:bspline-derivative}
\end{align}  

Note that the limits on the previous summation in \ref{eq:bspline-derivative} are left unspecified; the formula is written for bi-infinite sums, and their application to finite sums is accessible after they are written formally as bi-infinite sums by augmenting the appropriate zero terms. However, if we are interested in a particular interval over the domain, say $\left[t_r,t_s\right]$, then for $x \in \left[t_r,t_s\right]$, then


\[
\frac{\partial}{\partial x} \bigg[ \sum_i \theta_i B_{i,k}\left(x\right) \bigg]  = \sum_{r-k+2}^{s-1}\left(k-1\right) \frac{\theta_i-\theta_{i-1}}{t_{i+k-1} - t_i}B_{i,k-1}\left(x\right)
\]

\noindent
since $B_{i,k-1}\left(x\right)=0$ for all $i \not \in \left\{r-k+2,\dots, s-1 \right\}$ when $t_r \le x\le t_s$. Applying \ref{eq:bspline-derivative} $j$ times gives us that the $j^{th}$ derivative of $f = \sum_{i} \theta_i B_{ik}$ has form


\begin{align}
\frac{\partial^j}{\partial x^j} \bigg[ \sum_i \theta_i B_{i,k}\left(x\right) \bigg] =  \sum_i \theta_i^{\left(j+1\right)} B_{i,k-j} \label{eq:BS_jth_deriv_a} \\
\theta_i^{\left(j+1\right)} \equiv \left\{ \begin{array}{cl} \theta_i, & j = 0 \\
						\frac{\theta_i^{\left(j\right)} - \theta_{i-1}^{\left(j\right)} }{\left( t_{i+k-j}-t_{i}\right)/\left(k-j\right)}, & j \ge 1 
  \end{array} \right.\label{eq:BS_jth_deriv_b}
\end{align}

\begin{proof}
We proceed by induction on $j$. We have already shown the case for $j=1$ in the derivation of \ref{eq:bspline-derivative}. Assume that the statement holds for some $j^* >1$, so that we have
\[
\frac{\partial^{j^*}}{\partial x^{j^*}} \bigg[ \sum_i \theta_i B_{i,k}\left(x\right) \bigg] = \sum_i \frac{\theta_i^{\left(j^*\right)} - \theta_{i-1}^{\left(j^*\right)} }{\left( t_{i+k-{j^*}}-t_{i}\right)/\left(k-{j^*}\right)} B_{i,k-j^*}\left(x\right).
\]

\noindent
Then the $\left( j^* + 1 \right)^{st}$ derivative is given by 
\begin{align*}
\frac{\partial^{j^*+1}}{\partial x^{j^*+1}} \bigg[ \sum_i \theta_i B_{i,k} \bigg] &= \sum_i \frac{\theta_i^{\left(j^*\right)} - \theta_{i-1}^{\left(j^*\right)} }{\left( t_{i+k-{j^*}}-t_{i}\right)/\left(k-{j^*}\right)}  B'_{i,k-j^*} \\
&= \sum_i \theta_i^{\left(j^*\right)}  B'_{i,k-j^*} \\
&= \sum_i \theta_i^{\left(j^*\right)} \left(k-\left(j^*+1\right)\right)\bigg[ \frac{B_{i,k-\left(j^*+1\right)}}{t_{i+k-\left({j^*}+1\right)}-t_{i}} - \frac{ B_{i+1,k-\left(j^*+1\right)} }{ t_{i+k-\left({j^*}+1\right)+1}-t_{i+1}} \bigg] \\
&= \sum_i \frac{\theta_i^{\left(j^*\right)} - \theta_{i-1}^{\left(j^*\right)}}{\left(t_{i+k-\left(j^* + 1\right)}-t_i\right)/\left(k-\left(j^*+1\right)\right)}B_{i,k-\left(j^* + 1\right)}\\
&= \sum_i \theta_i^{\left(j^*+1\right)}  B_{i,k-\left(j^* + 1\right)}
\end{align*}
\end{proof}
The choice to write $k-j$ as a divisor in the denominator lends to the interpretation of \ref{eq:BS_jth_deriv_a} as a difference quotient, with the quantity
\[
\frac{t_{i+k-j} - t_i}{k-j}
\]
representing a mean mesh length of sorts on the interval $\left[t_i,t_{i+k-j}\right]$. We note that the case where $t$ contains replicated knots leads to division by zero. This is, however, a trivial situation, since for $t_i = t_{i+k-j}$, we have $B_i = 0$, and we take $\frac{0}{0} = 0$.



\chapter{Chapter~\ref{simulation-studies-chapter} Appendix} \label{simulation-studies-appendix}

 
%%-----------------------------------------------------------------------------------------------------------------------------------------------------------------------------------------------------------------------------------
%%-----------------------------------------------------------------------------------------------------------------------------------------------------------------------------------------------------------------------------------
%%-----------------------------------------------------------------------------------------------------------------------------------------------------------------------------------------------------------------------------------
%%-----------------------------------------------------------------------------------------------------------------------------------------------------------------------------------------------------------------------------------
%%-----------------------------------------------------------------------------------------------------------------------------------------------------------------------------------------------------------------------------------
%%-----------------------------------------------------------------------------------------------------------------------------------------------------------------------------------------------------------------------------------
%%-----------------------------------------------------------------------------------------------------------------------------------------------------------------------------------------------------------------------------------
%%-----------------------------------------------------------------------------------------------------------------------------------------------------------------------------------------------------------------------------------
%%-----------------------------------------------------------------------------------------------------------------------------------------------------------------------------------------------------------------------------------
%%-----------------------------------------------------------------------------------------------------------------------------------------------------------------------------------------------------------------------------------
%%-----------------------------------------------------------------------------------------------------------------------------------------------------------------------------------------------------------------------------------
%%-----------------------------------------------------------------------------------------------------------------------------------------------------------------------------------------------------------------------------------

\section{Quadratic Risk Estimates for Simulation with Complete Data}
%%-----------------------------------------------------------------------------------------------------------------------------------------------------------------------------------------------------------------------------------
%%-----------------------------------------------------------------------------------------------------------------------------------------------------------------------------------------------------------------------------------
%%-----------------------------------------------------------------------------------------------------------------------------------------------------------------------------------------------------------------------------------
%%-----------------------------------------------------------------------------------------------------------------------------------------------------------------------------------------------------------------------------------
%\subfile{chapter-4-subfiles/simulation-study-1-quad-table-model-1}
\begin{table}[H]
\centering
\caption{\textit{Multivariate normal simulations for model I. Estimated quadratic risk is reported for our smoothing spline ANOVA estimator and P-spline estimator, the oracle estimator for each covariance structure, the parametric polynomial estimator of Pan and MacKenzie (2003), the sample covariance matrix, the tapered sample covariance matrix,
                                    and the soft thresholding estimator.}}
\begin{tabular}{lrrrrrrrr}
& $p$ &$\hat{\Sigma}_{oracle}$&  $\hat{\Sigma}_{SS}$& $\hat{\Sigma}_{PS}$ &$\hat{\Sigma}_{poly}$ & $S$ &$S^\omega$& $S^\lambda$ \\ 
  \hline
  $N = 50$ & $10$ &0.00267 & 0.0016 & 0.0052 &  0.0912 & 0.3901 & 0.3864 & 0.3874 \\ 
   		& $20$ &0.00459 & 0.0010 & 0.0043 &  0.0757 & 0.8371 & 0.7710 & 0.7716 \\ 
   		& $30$ & 0.00386 & 0.0026 & 0.0036 &  0.1109 & 1.2857 & 1.1937 & 1.2074 \\ 
 $N = 100$ & 10 &  0.00209 & 0.0005 & 0.0010 &0.0426 & 0.2116 & 0.1676 & 0.1720 \\ 
    		&   $20$ &  0.00212 &0.0003 & 0.0011 & 0.0376 & 0.4255 & 0.3902 & 0.3970 \\ 
    		&   $30$ &0.00276 & 0.0002 & 0.0011 &  0.0313 & 0.5984 & 0.5790 & 0.5842 \\ 
   \hline
\end{tabular} 
\label{table:simulation-1-quad-loss-sigma-1}
\end{table}
%%-----------------------------------------------------------------------------------------------------------------------------------------------------------------------------------------------------------------------------------
%%-----------------------------------------------------------------------------------------------------------------------------------------------------------------------------------------------------------------------------------
%%-----------------------------------------------------------------------------------------------------------------------------------------------------------------------------------------------------------------------------------
%%-----------------------------------------------------------------------------------------------------------------------------------------------------------------------------------------------------------------------------------
% latex table generated in R 3.4.3 by xtable 1.8-2 package
% Tue Mar  6 11:10:18 2018
\begin{table}[H]
\centering
\caption{\textit{Multivariate normal simulation-estimated quadratic risk  for model II.} }
\begin{tabular}{lrrrrrrrr}
& $p$ &$\hat{\Sigma}_{oracle}$&$\hat{\Sigma}_{SS}$& $\hat{\Sigma}_{PS}$ & $\hat{\Sigma}_{poly}$ & $S$ &$S^\omega$& $S^\lambda$ \\ 
  \hline
$N = 50$ & $10$ & 0.0483 & 0.0623 & 0.0792 & 7.0137 & 0.6269 & 0.8108 & 0.5770 \\ 
    & $20$ & 0.4317 & 0.7972 & 1.2456 &  852.2787 & 2.7659 & 30.8197 & 36.1492 \\ 
     & $30$ & 6.7921 & 12.8700 & 7.2129 & 4849.8925 & 21.0228 & 365.0301 & 1804.9695 \\ 
     $N = 100$ & $10$ &0.0280 & 0.0254 & 0.0525 &  7.0482 & 0.2683 & 0.4351 & 0.2665 \\ 
    & $20$ &0.2625 & 0.2877 & 0.8153 &  861.3937 & 1.3347 & 5.5170 & 7.3283 \\ 
     & $30$ &2.6619 & 2.7399 & 6.9793 &  5075.4782 & 8.4769 & 66.9461 & 420.2973 \\ 

   \hline
\end{tabular}
\label{table:simulation-1-quad-loss-sigma-2}
\end{table}
 
%\subfile{chapter-4-subfiles/simulation-study-1-quad-table-model-2}
%%-----------------------------------------------------------------------------------------------------------------------------------------------------------------------------------------------------------------------------------
%%-----------------------------------------------------------------------------------------------------------------------------------------------------------------------------------------------------------------------------------
%%-----------------------------------------------------------------------------------------------------------------------------------------------------------------------------------------------------------------------------------
%%-----------------------------------------------------------------------------------------------------------------------------------------------------------------------------------------------------------------------------------
%\subfile{chapter-4-subfiles/simulation-study-1-quad-table-model-3}
\begin{table}[H]
\centering
\caption{\textit{Multivariate normal simulation-estimated quadratic risk  for model III.} }
\begin{tabular}{lrrrrrrrr}
  & $p$ &$\hat{\Sigma}_{oracle}$& $\hat{\Sigma}_{SS}$& $\hat{\Sigma}_{PS}$ & $\hat{\Sigma}_{poly}$ & $S$ &$S^\omega$& $S^\lambda$ \\ 
  \hline
$N = 50$ & $10$ & 0.0697 &0.0656 & 0.0665 &  3.4849 & 0.4977 & 0.6678 & 0.5858 \\ 
    &    $20$ & 0.4706 &1.0095 & 0.9146 &  426.0848 & 2.0716 & 4.8213 & 8.4099 \\ 
    &    $30$ & 5.3699 & 10.8782 & 8.1124 &  5061.3563 & 16.5536 & 779.2829 & 1181.3770 \\ 
     $N = 100$ & 10 & 0.0328 & 0.0486 & 0.0363 &  3.5437 & 0.2437 & 0.2929 & 0.2791 \\ 
     & $20$ &0.1958 & 0.6260 & 0.3783 &  416.1285 & 1.0193 & 1.5353 & 5.1553 \\ 
     & $30$ &2.2121 & 5.9367 & 3.4576 &  5082.1367 & 7.9582 & 14.2394 & 253.4296 \\ 
       \hline
\end{tabular}
\label{table:simulation-1-quad-loss-sigma-3}
\end{table}
%%-----------------------------------------------------------------------------------------------------------------------------------------------------------------------------------------------------------------------------------
%%-----------------------------------------------------------------------------------------------------------------------------------------------------------------------------------------------------------------------------------
%%-----------------------------------------------------------------------------------------------------------------------------------------------------------------------------------------------------------------------------------
%%-----------------------------------------------------------------------------------------------------------------------------------------------------------------------------------------------------------------------------------
%\subfile{chapter-4-subfiles/simulation-study-1-quad-table-model-4}

\begin{table}[H]
\centering
\caption{\textit{Multivariate normal simulation-estimated quadratic risk  for model IV.} }
\begin{tabular}{lrrrrrrrr}
  & $p$ &$\hat{\Sigma}_{oracle}$ &$\hat{\Sigma}_{SS}$& $\hat{\Sigma}_{PS}$ & $\hat{\Sigma}_{poly}$ & $S$ &$S^\omega$& $S^\lambda$ \\ 
  \hline
$N = 50$ & 10 &0.0053 & 0.0144 & 0.0196 &  0.2575 & 0.4420 & 0.4628 & 0.4620 \\ 
  & $20$ & 0.0073 & 0.0449 & 0.0154 & 0.4384 & 0.7951 & 0.9184 & 0.9177 \\ 
  & $30$ & 0.0072 & 0.0893 & 0.0189 &  0.6539 & 1.3363 & 1.3014 & 1.3013 \\ 
 $N = 100$ & 10 &0.0031 & 0.0112 & 0.0186 &  0.2098 & 0.2136 & 0.2299 & 0.2295 \\ 
    &    $20$ &  0.0027 & 0.0420 & 0.0143 & 0.4877 & 0.4509 & 0.4311 & 0.4307 \\ 
    &    $30$ &0.0035 & 0.0792 & 0.0181 &  0.6616 & 0.6263 & 0.6598 & 0.6589 \\ 
   \hline
\end{tabular}
\label{table:simulation-1-quad-loss-sigma-4}
\end{table}
%%-----------------------------------------------------------------------------------------------------------------------------------------------------------------------------------------------------------------------------------
%%-----------------------------------------------------------------------------------------------------------------------------------------------------------------------------------------------------------------------------------
%%-----------------------------------------------------------------------------------------------------------------------------------------------------------------------------------------------------------------------------------
%%-----------------------------------------------------------------------------------------------------------------------------------------------------------------------------------------------------------------------------------

\begin{table}[H]
\centering
\caption{\textit{Multivariate normal simulation-estimated quadratic risk  for model V.} }
\begin{tabular}{lrrrrrrrr}
 $N$ & $p$ &$\hat{\Sigma}_{oracle}$& $\hat{\Sigma}_{SS}$& $\hat{\Sigma}_{PS}$ & $\hat{\Sigma}_{poly}$ & $S$ &$S^\omega$& $S^\lambda$ \\ 
  \hline
   $N = 50$ & 10   0.1610 & & 0.3621 & 0.2456 & 1.3738 & 0.8484 & 1.6174 & 0.8963 \\ 
     		& $20$ & 0.5236 & 0.9911 & 0.8206 &  2.8419 & 1.7324 & 3.0233 & 1.6375 \\ 
     		& $30$ & 0.4632 & 1.5352 & 1.1507 & 4.1877 & 2.5484 & 5.1546 & 2.6727 \\ 
  $N = 100$ & 10 &  0.0813 & 0.3091 & 0.2678 & 1.2439 & 0.4175 & 1.0431 & 0.4922 \\ 
     		 & $20$ &0.1522 & 0.9734 & 0.4111 &  2.7280 & 0.7896 & 2.1932 & 0.8461 \\ 
     & $30$ &0.3656 & 1.6032 & 0.7701 &  3.8905 & 1.2577 & 3.5722 & 1.3270 \\ 
   \hline
\end{tabular}
\label{table:simulation-1-quad-loss-sigma-5}
\end{table}
%\subfile{chapter-4-subfiles/simulation-study-1-quad-table-model-5}
% latex table generated in R 3.4.3 by xtable 1.8-2 package
% Tue Mar  6 11:10:25 2018
%%-----------------------------------------------------------------------------------------------------------------------------------------------------------------------------------------------------------------------------------
%%-----------------------------------------------------------------------------------------------------------------------------------------------------------------------------------------------------------------------------------
%%-----------------------------------------------------------------------------------------------------------------------------------------------------------------------------------------------------------------------------------
%%-----------------------------------------------------------------------------------------------------------------------------------------------------------------------------------------------------------------------------------


\section{Quadratic Risk Estimates for Simulation with Irregularly Sampled Data}
%%-----------------------------------------------------------------------------------------------------------------------------------------------------------------------------------------------------------------------------------
%%-----------------------------------------------------------------------------------------------------------------------------------------------------------------------------------------------------------------------------------
%%-----------------------------------------------------------------------------------------------------------------------------------------------------------------------------------------------------------------------------------
%%-----------------------------------------------------------------------------------------------------------------------------------------------------------------------------------------------------------------------------------
% latex table generated in R 3.4.3 by xtable 1.8-2 package
% Wed Mar 21 09:47:45 2018
\begin{table}[H]
\centering
\caption{\textit{Model 1: Quadratic risk estimates and corresponding standard errors 
                            for the MCD smoothing spline ANOVA estimator via 100 simulated multivariate
                            normal samples of size $N = 50$
                            when 0\%, 10\%, 20\%, and 30\% of the data are missing for each subject. Risk is reported for the estimator constructed using
                            the unbiased risk estimate and leave-one-subject-out cross validation for smoothing parameter selection.} }
\label{table:simulation-study-2-quad-risk-model-1}
\begin{tabular}{lrrlrl}
   $p$ & \% missing & \multicolumn{2} {c} {$\Delta_1(\hat{\Sigma}^{U}_{SS})$} & \multicolumn{2} {c} {$\Delta_1(\hat{\Sigma}^{V^*}_{SS})$}\\ \hline
10 & 0.0 & 0.001625283 & (3e-040) & 0.00242142 & (5e-040) \\ 
   & 0.1 & 0.002667487 & (4e-040) & 0.00340902 & (6e-040) \\ 
   & 0.2 & 0.002203362 & (4e-040) & 0.00481581 & (7e-040) \\ 
   & 0.3 & 0.005959094 & (9e-040) & 0.00791520 & (0.0016) \\ 
   \hline
20 & 0.0 & 0.000865565 & (1e-040) & 0.00265909 & (0.0018) \\ 
   & 0.1 & 0.001350105 & (2e-040) & 0.24942590 & (0.2471) \\ 
   & 0.2 & 0.002791360 & (3e-040) & 0.01027696 & (0.0032) \\ 
   & 0.3 & 0.004419142 & (6e-040) & 0.09231505 & (0.0516) \\ 
   \hline
\end{tabular}
\end{table}

%%-----------------------------------------------------------------------------------------------------------------------------------------------------------------------------------------------------------------------------------
%%-----------------------------------------------------------------------------------------------------------------------------------------------------------------------------------------------------------------------------------
%%-----------------------------------------------------------------------------------------------------------------------------------------------------------------------------------------------------------------------------------
%%-----------------------------------------------------------------------------------------------------------------------------------------------------------------------------------------------------------------------------------

% latex table generated in R 3.4.3 by xtable 1.8-2 package
% Wed Mar 21 09:47:45 2018
\begin{table}[H]
\centering
\caption{\textit{Model 2: Quadratic risk estimates and corresponding standard errors.} }
\label{table:simulation-study-2-quad-risk-model-2}
\begin{tabular}{lrrlrl}
   $p$ & \% missing & \multicolumn{2} {c} {$\Delta_1(\hat{\Sigma}^{U}_{SS})$} & \multicolumn{2} {c} {$\Delta_1(\hat{\Sigma}^{V^*}_{SS})$}\\ \hline
10 & 0.0 & 0.0450916 & (0.0082) & 0.0601659 & (0.0096) \\ 
   & 0.1 & 0.0696728 & (0.0100) & 0.1512636 & (0.0289) \\ 
   & 0.2 & 0.2300287 & (0.0335) & 0.2343197 & (0.0398) \\ 
   & 0.3 & 0.4409229 & (0.0661) & 0.6346628 & (0.1247) \\ 
   \hline
20 & 0.0 & 0.4590734 & (0.0705) & 0.6819051 & (0.1176) \\ 
   & 0.1 & 19.4089837 & (2.0563) & 20.8552036 & (1.5583) \\ 
   & 0.2 & 268.9477374 & (20.7521) & 3969.3959755 & (3513.7089) \\ 
   & 0.3 & 2437.4762290 & (305.7227) & 5001.5651163 & (603.1301) \\ 
   \hline
\end{tabular}
\end{table}

%%-----------------------------------------------------------------------------------------------------------------------------------------------------------------------------------------------------------------------------------
%%-----------------------------------------------------------------------------------------------------------------------------------------------------------------------------------------------------------------------------------
%%-----------------------------------------------------------------------------------------------------------------------------------------------------------------------------------------------------------------------------------
%%-----------------------------------------------------------------------------------------------------------------------------------------------------------------------------------------------------------------------------------

% latex table generated in R 3.4.3 by xtable 1.8-2 package
% Wed Mar 21 09:47:45 2018
\begin{table}[H]
\centering
\caption{\textit{Model 3: Quadratic risk estimates and corresponding standard errors.} }
\label{table:simulation-study-2-quad-risk-model-3}
\begin{tabular}{lrrlrl}
   $p$ & \% missing & \multicolumn{2} {c} {$\Delta_1(\hat{\Sigma}^{U}_{SS})$} & \multicolumn{2} {c} {$\Delta_1(\hat{\Sigma}^{V^*}_{SS})$}\\ \hline
10 & 0.0 & 0.0650014 & (0.0055) & 0.0682312 & (0.0059) \\ 
   & 0.1 & 0.0770316 & (0.0081) & 0.0892940 & (0.0118) \\ 
   & 0.2 & 0.1140654 & (0.0142) & 0.2008099 & (0.0280) \\ 
   & 0.3 & 0.3315869 & (0.0677) & 0.3268610 & (0.0495) \\ 
   \hline
20 & 0.0 & 1.0422739 & (0.1994) & 1.2132111 & (0.2173) \\ 
   & 0.1 & 11.9788732 & (1.7077) & 18.5305750 & (1.5563) \\ 
   & 0.2 & 232.1002465 & (23.7789) & 280.9434501 & (42.1525) \\ 
   & 0.3 & 1667.1547183 & (263.3001) & 2601.3353420 & (338.6449) \\ 
   \hline
\end{tabular}
\end{table}


%%-----------------------------------------------------------------------------------------------------------------------------------------------------------------------------------------------------------------------------------
%%-----------------------------------------------------------------------------------------------------------------------------------------------------------------------------------------------------------------------------------
%%-----------------------------------------------------------------------------------------------------------------------------------------------------------------------------------------------------------------------------------
%%-----------------------------------------------------------------------------------------------------------------------------------------------------------------------------------------------------------------------------------

% latex table generated in R 3.4.3 by xtable 1.8-2 package
% Wed Mar 21 09:47:45 2018
\begin{table}[H]
\centering
\caption{\textit{Model 4: Quadratic risk estimates and corresponding standard errors.} }
\label{table:simulation-study-2-quad-risk-model-4}
\begin{tabular}{lrrlrl}
   $p$ & \% missing & \multicolumn{2} {c} {$\Delta_1(\hat{\Sigma}^{U}_{SS})$} & \multicolumn{2} {c} {$\Delta_1(\hat{\Sigma}^{V^*}_{SS})$}\\ \hline
10 & 0.0 & 0.01436606 & (7e-040) & 0.01655013 & (0.0013) \\ 
   & 0.1 & 0.01684656 & (8e-040) & 0.01893500 & (0.0022) \\ 
   & 0.2 & 0.02374962 & (0.0023) & 0.02433408 & (0.0020) \\ 
   & 0.3 & 0.03204756 & (0.0028) & 0.03424552 & (0.0044) \\ 
   \hline
20 & 0.0 & 0.04488566 & (9e-040) & 0.04670697 & (9e-040) \\ 
   & 0.1 & 0.04654451 & (0.0012) & 0.05029391 & (0.0015) \\ 
   & 0.2 & 0.05132972 & (0.0013) & 0.06053346 & (0.0038) \\ 
   & 0.3 & 0.06230931 & (0.0021) & 0.10699654 & (0.0459) \\ 
   \hline
\end{tabular}
\end{table}

%%-----------------------------------------------------------------------------------------------------------------------------------------------------------------------------------------------------------------------------------
%%-----------------------------------------------------------------------------------------------------------------------------------------------------------------------------------------------------------------------------------
%%-----------------------------------------------------------------------------------------------------------------------------------------------------------------------------------------------------------------------------------
%%-----------------------------------------------------------------------------------------------------------------------------------------------------------------------------------------------------------------------------------
%%-----------------------------------------------------------------------------------------------------------------------------

% latex table generated in R 3.4.3 by xtable 1.8-2 package
% Wed Mar 21 09:47:45 2018
\begin{table}[H]
\centering
\caption{\textit{Model 5: Quadratic risk estimates and corresponding standard errors.} }
\label{table:simulation-study-2-quad-risk-model-5}
\begin{tabular}{lrrlrl}
   $p$ & \% missing & \multicolumn{2} {c} {$\Delta_1(\hat{\Sigma}^{U}_{SS})$} & \multicolumn{2} {c} {$\Delta_1(\hat{\Sigma}^{V^*}_{SS})$}\\ \hline
10 & 0.0 & 0.3621065 & (0.0091) & 0.3623509 & (0.0128) \\ 
   & 0.1 & 0.6778957 & (0.0457) & 0.7067101 & (0.0426) \\ 
   & 0.2 & 2.1262957 & (0.1590) & 2.4381408 & (0.2292) \\ 
   & 0.3 & 6.8051314 & (0.6256) & 8.2414439 & (0.7087) \\ 
   \hline
20 & 0.0 & 0.9910795 & (0.0138) & 1.0334928 & (0.0099) \\ 
   & 0.1 & 1.7214964 & (0.1028) & 1.5051130 & (0.0577) \\ 
   & 0.2 & 5.3527162 & (0.3290) & 5.1871496 & (0.3852) \\ 
   & 0.3 & 29.6617541 & (2.0158) & 25.1766132 & (1.8094) \\ 
   \hline
\end{tabular}
\end{table}


%%-----------------------------------------------------------------------------------------------------------------------------------------------------------------------------------------------------------------------------------
%%-----------------------------------------------------------------------------------------------------------------------------------------------------------------------------------------------------------------------------------
%%-----------------------------------------------------------------------------------------------------------------------------------------------------------------------------------------------------------------------------------
%%-----------------------------------------------------------------------------------------------------------------------------------------------------------------------------------------------------------------------------------
%%-----------------------------------------------------------------------------------------------------------------------------

\section{Comprehensive Tables for Simulations with Complete Data}
%%-----------------------------------------------------------------------------------------------------------------------------
\begin{landscape}
%\subfile{chapter-4-subfiles/simulation-study-1-master-entropy-risk-table}
%\begin{table}[H]
%\centering
%\begin{footnotesize}
%\begin{tabular}{lllllllllllllllll}
%\Model & $N$ & $p$ & \multicolumn{2}{c}{$\hat{\Sigma}^{ure}_{SS}$} &  \multicolumn{2}{c}{$\hat{\Sigma}^{ure}_{PS}$} &  \multicolumn{2}{c}{$\hat{\Sigma}_{oracle}$}   &  \multicolumn{2}{c}{$\hat{\Sigma}_{poly}$}  &  \multicolumn{2}{c}{$S$}   &  \multicolumn{2}{c}{$S^\omega$}   &  \multicolumn{2}{c}{$S^\lambda$}   \\ 
%  \hline
%I &  & $10$ & 0.0749 & (0.0072) & 0.1261 & (0.0107) & 0.0135 & (0.0023) & 0.1102 & (0.0083) & 1.2047 & (0.0286) & 0.5369 & (0.0563) & 1.1742 & (0.0366) \\ 
%    & 50  & $20$ & 0.0872 & (0.0081) & 0.1713 & (0.0095) & 0.0229 & (0.0041) & 0.1096 & (0.0087) & 4.9850 & (0.0644) & 1.3957 & (0.1859) & 4.7796 & (0.1206) \\ 
%    &  50 & $30$ & 0.1102 & (0.0229) & 0.1969 & (0.0118) & 0.0196 & (0.0034) & 0.1127 & (0.0108) & 12.5517 & (0.1322) & 2.8019 & (0.4332) & 11.3175 & (0.3556) \\ 
%    & $100$ & $10$ & 0.0451 & (0.0035) & 0.0671 & (0.0042) & 0.0105 & (0.0015) & 0.0531 & (0.0038) & 0.5685 & (0.0151) & 0.2045 & (0.0235) & 0.5236 & (0.0176) \\ 
%    &  100 & $20$ & 0.0425 & (0.0062) & 0.0965 & (0.0048) & 0.0105 & (0.0020) & 0.0512 & (0.0031) & 2.2831 & (0.0285) & 0.5724 & (0.0744) & 2.1358 & (0.0606) \\ 
%    & 100  & $30$ & 0.0431 & (0.0044) & 0.1148 & (0.0062) & 0.0139 & (0.0021) & 0.0472 & (0.0033) & 5.2770 & (0.0472) & 1.2430 & (0.1569) & 4.9126 & (0.1204) \\ 
%  II & $50$ & $10$ & 0.0899 & (0.0069) & 0.3423 & (0.0082) & 0.0581 & (0.0055) & 4.7673 & (0.0919) & 1.2832 & (0.0334) & 1.4644 & (0.0475) & 1.1770 & (0.0346) \\ 
%    &  50 & $20$ & 0.0949 & (0.0080) & 1.3640 & (0.0158) & 0.0439 & (0.0051) & 97.2334 & (2.4537) & 5.1665 & (0.0610) & 21.6407 & (1.2914) & 39.3522 & (8.1602) \\ 
%    &  50 & $30$ & 0.0811 & (0.0075) & 2.6485 & (0.0472) & 0.0627 & (0.0063) & 1539.665 & (39.7267) & 12.3582 & (0.1070) & 55.3674 & (3.8362) & 133.9980 & (19.2003) \\ 
%    & $100$ & $10$ & 0.0457 & (0.0050) & 0.2945 & (0.0059) & 0.0386 & (0.0034) & 4.7911 & (0.0638) & 0.5812 & (0.0134) & 0.8335 & (0.0293) & 0.5628 & (0.0154) \\ 
%    &  100 & $20$ & 0.0416 & (0.0038) & 1.2875 & (0.0100) & 0.0269 & (0.0027) & 98.1989 & (2.0835) & 2.3364 & (0.0316) & 10.1841 & (0.8276) & 10.0864 & (1.1183) \\ 
%    &  100 & $30$ & 0.0367 & (0.0033) & 2.4365 & (0.0293) & 0.0288 & (0.0031) & 1582.479 & (36.0484) & 5.2389 & (0.0475) & 33.5207 & (0.9390) & 62.5030 & (14.7791) \\ 
%  III & $50$ & $10$ & 0.3416 & (0.0091) & 0.1065 & (0.0090) & 0.0619 & (0.0079) & 3.0108 & (0.0709) & 1.2030 & (0.0312) & 1.1460 & (0.0472) & 1.1467 & (0.0341) \\ 
%    &  50 & $20$ & 1.1140 & (0.0100) & 0.2555 & (0.0109) & 0.0695 & (0.0075) & 62.7522 & (2.1710) & 4.9824 & (0.0689) & 17.2244 & (0.6234) & 14.9189 & (2.7042) \\ 
%    &  50 & $30$ & 2.3215 & (0.0132) & 0.6242 & (0.0390) & 0.0576 & (0.0071) & 1091.193 & (31.2219) & 12.4792 & (0.1182) & 49.9135 & (7.7026) & 121.7795 & (18.3978) \\ 
%    & $100$ & $10$ & 0.2904 & (0.0045) & 0.0579 & (0.0050) & 0.0268 & (0.0027) & 3.0383 & (0.0559) & 0.5699 & (0.0142) & 0.5545 & (0.0162) & 0.5371 & (0.0130) \\ 
%    &  100 & $20$ & 1.1963 & (0.1239) & 0.2011 & (0.0057) & 0.0275 & (0.0036) & 62.8960 & (1.1460) & 2.2700 & (0.0306) & 11.8274 & (0.7008) & 9.5217 & (1.0164) \\ 
%    & 100  & $30$ & 2.2811 & (0.0079) & 0.3845 & (0.0169) & 0.0221 & (0.0024) & 1105.045 & (21.8998) & 5.2234 & (0.0462) & 29.1693 & (0.6585) & 60.3529 & (14.2471) \\ 
%  IV & $50$ & $10$ & 0.3422 & (0.0085) & 0.1966 & (0.0118) & 0.0217 & (0.0049) & 0.7144 & (0.0141) & 1.2218 & (0.0319) & 0.7397 & (0.0436) & 1.1921 & (0.0317) \\ 
%    & 50  & $20$ & 0.9208 & (0.0054) & 0.3499 & (0.0174) & 0.0286 & (0.0046) & 1.4588 & (0.0179) & 4.9091 & (0.0676) & 1.9786 & (0.1650) & 4.9206 & (0.0612) \\ 
%    &  50 & $30$ & 1.5992 & (0.0154) & 0.5100 & (0.0152) & 0.0283 & (0.0044) & 2.2173 & (0.0238) & 12.6114 & (0.1179) & 3.7440 & (0.3991) & 12.1489 & (0.1908) \\ 
%    & $100$ & $10$ & 0.3047 & (0.0047) & 0.2237 & (0.0125) & 0.0125 & (0.0025) & 0.6958 & (0.0080) & 0.5570 & (0.0130) & 0.3168 & (0.0142) & 0.5515 & (0.0147) \\ 
%    &  100 & $20$ & 0.8911 & (0.0036) & 0.3704 & (0.0185) & 0.0105 & (0.0017) & 1.4813 & (0.0140) & 2.2659 & (0.0305) & 0.9365 & (0.0686) & 2.2474 & (0.0334) \\ 
%    & 100  & $30$ & 1.5213 & (0.0029) & 0.5282 & (0.0163) & 0.0134 & (0.0022) & 2.2228 & (0.0141) & 5.2106 & (0.0473) & 1.9312 & (0.1746) & 5.2111 & (0.0584) \\ 
%  V & $50$ & $10$ & 0.2743 & (0.0068) & 0.2464 & (0.0108) & 0.0986 & (0.0200) & 1.2420 & (0.0294) & 1.2023 & (0.0318) & 18.5222 & (0.6731) & 2.9824 & (0.3820) \\ 
%    &  50 & $20$ & 0.7526 & (0.0042) & 0.8772 & (0.0128) & 0.2512 & (0.0580) & 2.8557 & (0.0646) & 5.0195 & (0.0695) & 34.6618 & (0.6202) & 13.8690 & (0.8916) \\ 
%    & 50  & $30$ & 1.1776 & (0.0051) & 0.9791 & (0.0125) & 0.2641 & (0.0474) & 4.5791 & (0.0914) & 12.3460 & (0.1112) & 46.5437 & (0.7836) & 26.1364 & (0.3248) \\ 
%    & $100$ & $10$ & 0.2416 & (0.0039) & 0.1722 & (0.0049) & 0.0520 & (0.0090) & 1.1491 & (0.0202) & 0.5821 & (0.0111) & 16.4081 & (0.4280) & 1.7397 & (0.0363) \\ 
%    &  100 & $20$ & 0.7286 & (0.0028) & 0.2965 & (0.0046) & 0.0827 & (0.0170) & 2.9080 & (0.0383) & 2.2918 & (0.0244) & 32.5295 & (0.5786) & 5.4649 & (0.5497) \\ 
%    & 100  & $30$ & 1.1813 & (0.0051) & 0.4291 & (0.0065) & 0.1799 & (0.0420) & 4.4402 & (0.0655) & 5.2197 & (0.0465) & 39.2914 & (0.2195) & 15.4295 & (0.8464) \\ 
%   \hline
%\end{tabular}
%\end{footnotesize}
%\caption{Risk estimates under entropy loss and corresponding standard errors based on
%                              100 Monte Carlo simulations.} \label{table:master-entropy-risk-table}
%\end{table}
% latex table generated in R 3.4.3 by xtable 1.8-2 package
% Thu Mar 22 08:42:58 2018
% latex table generated in R 3.4.3 by xtable 1.8-2 package
% Thu Mar 22 09:33:53 2018
\begin{table}[H]
\centering
\begin{scriptsize}
\caption{\textit{Multivariate normal simulations for model V. Estimated entropy risk and standard errors of the loss are reported for our smoothing spline ANOVA estimator and P-spline estimator, the oracle estimator for each covariance structure, the parametric polynomial estimator of Pan and MacKenzie (2003), the sample covariance matrix, the tapered sample covariance matrix, and the soft thresholding estimator.}} 
\label{table:master-entropy-risk-table}
\begin{tabular}{lrrrlrlrlrlrlrlrl}
   Model & $N$ & $p$ & \multicolumn{2}{c}{$\hat{\Sigma}^{ure}_{SS}$} & \multicolumn{2}{c}{$\hat{\Sigma}^{ure}_{PS}$} & \multicolumn{2}{c}{$\hat{\Sigma}_{oracle}$} & \multicolumn{2}{c}{$\hat{\Sigma}_{poly}$} & \multicolumn{2}{c}{$S$} & \multicolumn{2}{c}{$S^\omega$} & \multicolumn{2}{c}{$S^\lambda$}\\ \hline
I & 50 & 10 & 0.0685 & (0.0072) & 0.1261 & (0.0107) & 0.0135 & (0.0023) & 0.1102 & (0.0083) & 1.2047 & (0.0286) & 0.5369 & (0.0563) & 1.1742 & (0.0366) \\ 
    & 50 & 20 & 0.0834 & (0.0081) & 0.1713 & (0.0095) & 0.0229 & (0.0041) & 0.1096 & (0.0087) & 4.9850 & (0.0644) & 1.3957 & (0.1859) & 4.7796 & (0.1206) \\ 
    & 50 & 30 & 0.1102 & (0.0229) & 0.1969 & (0.0118) & 0.0196 & (0.0034) & 0.1127 & (0.0108) & 12.5517 & (0.1322) & 2.8019 & (0.4332) & 11.3175 & (0.3556) \\ 
    & 100 & 10 & 0.0451 & (0.0035) & 0.0671 & (0.0042) & 0.0105 & (0.0015) & 0.0531 & (0.0038) & 0.5685 & (0.0151) & 0.2045 & (0.0235) & 0.5236 & (0.0176) \\ 
    & 100 & 20 & 0.0425 & (0.0062) & 0.0965 & (0.0048) & 0.0105 & (0.0020) & 0.0512 & (0.0031) & 2.2831 & (0.0285) & 0.5724 & (0.0744) & 2.1358 & (0.0606) \\ 
    & 100 & 30 & 0.0431 & (0.0044) & 0.1148 & (0.0062) & 0.0139 & (0.0021) & 0.0472 & (0.0033) & 5.2770 & (0.0472) & 1.2430 & (0.1569) & 4.9126 & (0.1204) \\ 
   \hline
II & 50 & 10 & 0.0689 & (0.0069) & 0.3423 & (0.0082) & 0.0581 & (0.0055) & 4.7673 & (0.0919) & 1.2832 & (0.0334) & 1.4644 & (0.0475) & 1.1770 & (0.0346) \\ 
    & 50 & 20 & 0.0581 & (0.0080) & 1.3640 & (0.0158) & 0.0439 & (0.0051) & 97.2334 & (2.4537) & 5.1665 & (0.0610) & 21.6407 & (1.2914) & 39.3522 & (8.1602) \\ 
    & 50 & 30 & 0.0811 & (0.0075) & 2.6485 & (0.0472) & 0.0627 & (0.0063) & 153.9665 & (7.9453) & 12.3582 & (0.1070) & 55.3674 & (3.8362) & 133.9980 & (19.2003) \\ 
    & 100 & 10 & 0.0457 & (0.0050) & 0.2945 & (0.0059) & 0.0386 & (0.0034) & 4.7911 & (0.0638) & 0.5812 & (0.0134) & 0.8335 & (0.0293) & 0.5628 & (0.0154) \\ 
    & 100 & 20 & 0.0416 & (0.0038) & 1.2875 & (0.0100) & 0.0269 & (0.0027) & 98.1989 & (2.0835) & 2.3364 & (0.0316) & 10.1841 & (0.8276) & 10.0864 & (1.1183) \\ 
    & 100 & 30 & 0.0367 & (0.0033) & 2.4365 & (0.0293) & 0.0288 & (0.0031) & 158.2480 & (7.2097) & 5.2389 & (0.0475) & 33.5207 & (0.9390) & 62.5030 & (14.7791) \\ 
   \hline
III & 50 & 10 & 0.3296 & (0.0091) & 0.1065 & (0.0090) & 0.0619 & (0.0079) & 3.0108 & (0.0709) & 1.2030 & (0.0312) & 1.1460 & (0.0472) & 1.1467 & (0.0341) \\ 
    & 50 & 20 & 1.1100 & (0.0100) & 0.2555 & (0.0109) & 0.0695 & (0.0075) & 62.7522 & (2.1710) & 4.9824 & (0.0689) & 17.2244 & (0.6234) & 14.9189 & (2.7042) \\ 
    & 50 & 30 & 2.3215 & (0.0132) & 0.6242 & (0.0390) & 0.0576 & (0.0071) & 218.2387 & (31.2219) & 12.4792 & (0.1182) & 49.9135 & (7.7026) & 121.7795 & (18.3978) \\ 
    & 100 & 10 & 0.2904 & (0.0045) & 0.0579 & (0.0050) & 0.0268 & (0.0027) & 3.0383 & (0.0559) & 0.5699 & (0.0142) & 0.5545 & (0.0162) & 0.5371 & (0.0130) \\ 
    & 100 & 20 & 1.1963 & (0.1239) & 0.2011 & (0.0057) & 0.0275 & (0.0036) & 62.8960 & (1.1460) & 2.2700 & (0.0306) & 11.8274 & (0.7008) & 9.5217 & (1.0164) \\ 
    & 100 & 30 & 2.2811 & (0.0079) & 0.3845 & (0.0169) & 0.0221 & (0.0024) & 221.0090 & (21.8998) & 5.2234 & (0.0462) & 29.1693 & (0.6585) & 60.3529 & (14.2471) \\ 
   \hline
IV & 50 & 10 & 0.3348 & (0.0085) & 0.1966 & (0.0118) & 0.0217 & (0.0049) & 0.7144 & (0.0141) & 1.2218 & (0.0319) & 0.7397 & (0.0436) & 1.1921 & (0.0317) \\ 
    & 50 & 20 & 0.9177 & (0.0054) & 0.3499 & (0.0174) & 0.0286 & (0.0046) & 1.4588 & (0.0179) & 4.9091 & (0.0676) & 1.9786 & (0.1650) & 4.9206 & (0.0612) \\ 
    & 50 & 30 & 1.5992 & (0.0154) & 0.5100 & (0.0152) & 0.0283 & (0.0044) & 2.2173 & (0.0238) & 12.6114 & (0.1179) & 3.7440 & (0.3991) & 12.1489 & (0.1908) \\ 
    & 100 & 10 & 0.3047 & (0.0047) & 0.2237 & (0.0125) & 0.0125 & (0.0025) & 0.6958 & (0.0080) & 0.5570 & (0.0130) & 0.3168 & (0.0142) & 0.5515 & (0.0147) \\ 
    & 100 & 20 & 0.8911 & (0.0036) & 0.3704 & (0.0185) & 0.0105 & (0.0017) & 1.4813 & (0.0140) & 2.2659 & (0.0305) & 0.9365 & (0.0686) & 2.2474 & (0.0334) \\ 
    & 100 & 30 & 1.5213 & (0.0029) & 0.5282 & (0.0163) & 0.0134 & (0.0022) & 2.2228 & (0.0141) & 5.2106 & (0.0473) & 1.9312 & (0.1746) & 5.2111 & (0.0584) \\ 
   \hline
V & 50 & 10 & 0.2769 & (0.0068) & 0.2464 & (0.0108) & 0.0986 & (0.0200) & 1.2420 & (0.0294) & 1.2023 & (0.0318) & 18.5222 & (0.6731) & 2.9824 & (0.3820) \\ 
    & 50 & 20 & 0.7514 & (0.0042) & 0.8772 & (0.0128) & 0.2512 & (0.0580) & 2.8557 & (0.0646) & 5.0195 & (0.0695) & 34.6618 & (0.6202) & 13.8690 & (0.8916) \\ 
    & 50 & 30 & 1.1776 & (0.0051) & 0.9791 & (0.0125) & 0.2641 & (0.0474) & 4.5791 & (0.0914) & 12.3460 & (0.1112) & 46.5437 & (0.7836) & 26.1364 & (0.3248) \\ 
    & 100 & 10 & 0.2416 & (0.0039) & 0.1722 & (0.0049) & 0.0520 & (0.0090) & 1.1491 & (0.0202) & 0.5821 & (0.0111) & 16.4081 & (0.4280) & 1.7397 & (0.0363) \\ 
    & 100 & 20 & 0.7286 & (0.0028) & 0.2965 & (0.0046) & 0.0827 & (0.0170) & 2.9080 & (0.0383) & 2.2918 & (0.0244) & 32.5295 & (0.5786) & 5.4649 & (0.5497) \\ 
    & 100 & 30 & 1.1813 & (0.0051) & 0.4291 & (0.0065) & 0.1799 & (0.0420) & 4.4402 & (0.0655) & 5.2197 & (0.0465) & 39.2914 & (0.2195) & 15.4295 & (0.8464) \\ 
   \hline
\end{tabular}
\end{scriptsize}
\end{table}
\end{landscape}

%%-----------------------------------------------------------------------------------------------------------------------------

\begin{landscape}
%\subfile{chapter-4-subfiles/simulation-study-1-master-quadratic-risk-table}
%\begin{table}[H]
%\centering
%\begin{footnotesize}
%\begin{tabular}{lllllllllllllllll}
%$Model$ & $N$ & $p$ & \multicolumn{2}{c}{$\hat{\Sigma}^{ure}_{SS}$} &  \multicolumn{2}{c}{$\hat{\Sigma}^{ure}_{PS}$} &  \multicolumn{2}{c}{$\hat{\Sigma}_{oracle}$}   &  \multicolumn{2}{c}{$\hat{\Sigma}_{poly}$}  &  \multicolumn{2}{c}{$S$}   &  \multicolumn{2}{c}{$S^\omega$}   &  \multicolumn{2}{c}{$S^\lambda$}   \\ 
%  \hline
%I & $50$ & $10$ & 0.0015 & (3e-040) & 0.0052 & (0.0010) & 0.0267 & (0.0045) & 0.0912 & (0.0103) & 0.3901 & (0.0247) & 0.3864 & (0.0221) & 0.3874 & (0.0224) \\ 
%    &  50 & $20$ & 0.0010 & (2e-040) & 0.0043 & (6e-040) & 0.0459 & (0.0083) & 0.0757 & (0.0098) & 0.8371 & (0.0325) & 0.7710 & (0.0392) & 0.7716 & (0.0386) \\ 
%    & 50  & $30$ & 0.0026 & (0.0018) & 0.0036 & (6e-040) & 0.0386 & (0.0065) & 0.1109 & (0.0152) & 1.2857 & (0.0498) & 1.1937 & (0.0472) & 1.2074 & (0.0472) \\ 
%    & $100$ & $10$ & 0.0005 & (1e-040) & 0.0010 & (1e-040) & 0.0209 & (0.0031) & 0.0426 & (0.0051) & 0.2116 & (0.0124) & 0.1676 & (0.0090) & 0.1720 & (0.0099) \\ 
%    &  100 & $20$ & 0.0003 & (1e-040) & 0.0011 & (1e-040) & 0.0212 & (0.0042) & 0.0376 & (0.0042) & 0.4255 & (0.0161) & 0.3902 & (0.0164) & 0.3970 & (0.0170) \\ 
%    & 100  & $30$ & 0.0002 & (1e-040) & 0.0011 & (1e-040) & 0.0276 & (0.0041) & 0.0313 & (0.0033) & 0.5984 & (0.0262) & 0.5790 & (0.0211) & 0.5842 & (0.0208) \\ 
%  II & $50$ & $10$ & 0.0483 & (0.0070) & 0.0623 & (0.0043) & 0.0792 & (0.0083) & 7.0137 & (0.3452) & 0.6269 & (0.0363) & 0.8108 & (0.0690) & 0.5770 & (0.0377) \\ 
%    &50   & $20$ & 0.7972 & (0.1388) & 1.2456 & (0.1778) & 0.4317 & (0.0809) & 852.279 & (38.431) & 2.7659 & (0.2037) & 30.820 & (15.7299) & 36.1492 & (9.3235) \\ 
%    &   50& $30$ & 6.7921 & (1.5850) & 12.8700 & (1.4200) & 7.2129 & (1.2710) & 1997.851 & (55.87) & 21.0228 & (2.2821) & 365.030 & (18.7437) & 1804.970 & (435.136) \\ 
%    & $100$ & $10$ & 0.0254 & (0.0044) & 0.0525 & (0.0033) & 0.0580 & (0.0071) & 7.0482 & (0.2405) & 0.2683 & (0.0164) & 0.4351 & (0.0279) & 0.2665 & (0.0166) \\ 
%    &  100 & $20$ & 0.2877 & (0.0477) & 0.8153 & (0.1501) & 0.2625 & (0.0377) & 861.394 & (34.1825) & 1.3347 & (0.1086) & 5.5170 & (0.6241) & 7.3283 & (1.4927) \\ 
%    & 100  & $30$ & 2.7399 & (0.4745) & 6.9793 & (0.9114) & 3.6619 & (0.7715) & 1509.564 & (53.587) & 8.4769 & (0.7058) & 66.9461 & (6.0353) & 420.297 & (119.174) \\ 
%   III & $50$ & $10$ & 0.0656 & (0.0053) & 0.0665 & (0.0033) & 0.0697 & (0.0102) & 3.4849 & (0.2297) & 0.4977 & (0.0265) & 0.6678 & (0.0645) & 0.5858 & (0.0365) \\ 
%    &  50 & $20$ & 1.0095 & (0.1420) & 0.9146 & (0.1113) & 0.4706 & (0.0731) & 426.085 & (26.445) & 2.0716 & (0.1360) & 4.8213 & (1.1130) & 8.4099 & (1.3497) \\ 
%    & 50  & $30$ & 10.8782 & (1.1771) & 8.1124 & (1.2342) & 5.3699 & (0.8475) & 5613.564 & (112.439) & 16.5536 & (1.8098) & 779.283 & (14.9847) & 1181.377 & (327.771) \\ 
%    & $100$ & $10$ & 0.0486 & (0.0040) & 0.0363 & (0.0047) & 0.0328 & (0.0040) & 3.5437 & (0.1839) & 0.2437 & (0.0130) & 0.2929 & (0.0196) & 0.2791 & (0.0170) \\ 
%    & 100  & $20$ & 0.6260 & (0.0200) & 0.3783 & (0.0823) & 0.1958 & (0.0308) & 416.129 & (12.8666) & 1.0193 & (0.0701) & 1.5353 & (0.1560) & 5.1553 & (1.0771) \\ 
%    &  100 & $30$ & 5.9367 & (0.7791) & 3.4576 & (0.7345) & 2.2121 & (0.3658) & 4821.367 & (85.815) & 7.9582 & (0.8381) & 14.239 & (1.7202) & 253.430 & (75.168) \\ 
%  IV & $50$ & $10$ & 0.0153 & (0.0010) & 0.0196 & (0.0039) & 0.0053 & (0.0012) & 0.2575 & (0.0340) & 0.4420 & (0.0293) & 0.4628 & (0.0365) & 0.4620 & (0.0363) \\ 
%    & 50  & $20$ & 0.0450 & (6e-040) & 0.0154 & (0.0024) & 0.0073 & (0.0012) & 0.4384 & (0.0416) & 0.7951 & (0.0447) & 0.9184 & (0.0397) & 0.9177 & (0.0395) \\ 
%    & 50  & $30$ & 0.0893 & (0.0022) & 0.0189 & (0.0030) & 0.0072 & (0.0011) & 0.6539 & (0.0557) & 1.3363 & (0.0485) & 1.3014 & (0.0462) & 1.3013 & (0.0453) \\ 
%    & $100$ & $10$ & 0.0112 & (5e-040) & 0.0186 & (0.0029) & 0.0031 & (6e-040) & 0.2098 & (0.0185) & 0.2136 & (0.0109) & 0.2299 & (0.0134) & 0.2295 & (0.0133) \\ 
%    & 100  & $20$ & 0.0420 & (4e-040) & 0.0143 & (0.0014) & 0.0027 & (4e-040) & 0.4877 & (0.0325) & 0.4509 & (0.0167) & 0.4311 & (0.0159) & 0.4307 & (0.0158) \\ 
%    & 100  & $30$ & 0.0792 & (4e-040) & 0.0181 & (0.0020) & 0.0035 & (6e-040) & 0.6616 & (0.0327) & 0.6263 & (0.0215) & 0.6598 & (0.0207) & 0.6589 & (0.0207) \\ 
%  V & $50$ & $10$ & 0.3659 & (0.0123) & 0.2456 & (0.0206) & 0.1610 & (0.0332) & 1.3738 & (0.0999) & 0.8484 & (0.0549) & 1.6174 & (0.1133) & 0.8963 & (0.0554) \\ 
%    & 50  & $20$ & 1.0146 & (0.0102) & 0.8206 & (0.0213) & 0.5236 & (0.1373) & 2.8419 & (0.1751) & 1.7324 & (0.0802) & 3.0233 & (0.1872) & 1.6375 & (0.0889) \\ 
%    &  50 & $30$ & 1.5352 & (0.0088) & 1.1507 & (0.0176) & 0.4632 & (0.0755) & 4.1877 & (0.2390) & 2.5484 & (0.0975) & 5.1546 & (0.3173) & 2.6727 & (0.1067) \\ 
%    & $100$ & $10$ & 0.3091 & (0.0047) & 0.2678 & (0.0112) & 0.0813 & (0.0133) & 1.2439 & (0.0664) & 0.4175 & (0.0258) & 1.0431 & (0.0556) & 0.4922 & (0.0273) \\ 
%    &  100 & $20$ & 0.9734 & (0.0075) & 0.4111 & (0.0084) & 0.1522 & (0.0331) & 2.7280 & (0.1010) & 0.7896 & (0.0306) & 2.1932 & (0.0929) & 0.8461 & (0.0355) \\ 
%    &  100 & $30$ & 1.6032 & (0.0088) & 0.7701 & (0.0098) & 0.3656 & (0.0968) & 3.8905 & (0.1447) & 1.2577 & (0.0466) & 3.5722 & (0.1457) & 1.3270 & (0.0411) \\ 
%   \hline
%\end{tabular}
%\caption{Risk estimates under quadratic loss and corresponding standard errors based on
%                              100 Monte Carlo simulations.} \label{table:master-quadratic-risk-table}
%\end{footnotesize}
%\end{table}
% latex table generated in R 3.4.3 by xtable 1.8-2 package
% latex table generated in R 3.4.3 by xtable 1.8-2 package
% Thu Mar 22 09:34:08 2018
\begin{table}[H]
\centering
\begin{scriptsize}
\caption{\textit{Multivariate normal simulations for model V. Estimated quadratic risk and standard errors of the loss are reported for our smoothing spline ANOVA estimator and P-spline estimator, the oracle estimator for each covariance structure, the parametric polynomial estimator of Pan and MacKenzie (2003), the sample covariance matrix, the tapered sample covariance matrix,
                        and the soft thresholding estimator.}} 
\label{table:master-quad-risk-table}
\begin{tabular}{lrrrlrlrlrlrlrlrl}
   Model & $N$ & $p$ & \multicolumn{2}{c}{$\hat{\Sigma}^{ure}_{SS}$} & \multicolumn{2}{c}{$\hat{\Sigma}^{ure}_{PS}$} & \multicolumn{2}{c}{$\hat{\Sigma}_{oracle}$} & \multicolumn{2}{c}{$\hat{\Sigma}_{poly}$} & \multicolumn{2}{c}{$S$} & \multicolumn{2}{c}{$S^\omega$} & \multicolumn{2}{c}{$S^\lambda$}\\ \hline
I & 50 & 10 & 0.0016 & (3e-040) & 0.0052 & (0.0010) & 0.0267 & (0.0045) & 0.0912 & (0.0103) & 0.3901 & (0.0247) & 0.3864 & (0.0221) & 0.3874 & (0.0224) \\ 
    & 50 & 20 & 0.0010 & (2e-040) & 0.0043 & (6e-040) & 0.0459 & (0.0083) & 0.0757 & (0.0098) & 0.8371 & (0.0325) & 0.7710 & (0.0392) & 0.7716 & (0.0386) \\ 
    & 50 & 30 & 0.0026 & (0.0018) & 0.0036 & (6e-040) & 0.0386 & (0.0065) & 0.1109 & (0.0152) & 1.2857 & (0.0498) & 1.1937 & (0.0472) & 1.2074 & (0.0472) \\ 
    & 100 & 10 & 0.0005 & (1e-040) & 0.0010 & (1e-040) & 0.0209 & (0.0031) & 0.0426 & (0.0051) & 0.2116 & (0.0124) & 0.1676 & (0.0090) & 0.1720 & (0.0099) \\ 
    & 100 & 20 & 0.0003 & (1e-040) & 0.0011 & (1e-040) & 0.0212 & (0.0042) & 0.0376 & (0.0042) & 0.4255 & (0.0161) & 0.3902 & (0.0164) & 0.3970 & (0.0170) \\ 
    & 100 & 30 & 0.0002 & (1e-040) & 0.0011 & (1e-040) & 0.0276 & (0.0041) & 0.0313 & (0.0033) & 0.5984 & (0.0262) & 0.5790 & (0.0211) & 0.5842 & (0.0208) \\ 
   \hline
II & 50 & 10 & 0.0451 & (0.0070) & 0.0623 & (0.0043) & 0.0792 & (0.0083) & 7.0137 & (0.3452) & 0.6269 & (0.0363) & 0.8108 & (0.0690) & 0.5770 & (0.0377) \\ 
    & 50 & 20 & 0.4591 & (0.1388) & 1.2456 & (0.1778) & 0.4317 & (0.0809) & 852.2787 & (38.4308) & 2.7659 & (0.2037) & 30.8197 & (15.7299) & 36.1492 & (9.3235) \\ 
    & 50 & 30 & 6.7921 & (1.5850) & 12.8700 & (1.4200) & 7.2129 & (1.2710) & 4849.8925 & (901.174) & 21.0228 & (2.2821) & 365.0301 & (178.7437) & 1804.9695 & (435.1357) \\ 
    & 100 & 10 & 0.0254 & (0.0044) & 0.0525 & (0.0033) & 0.0580 & (0.0071) & 7.0482 & (0.2405) & 0.2683 & (0.0164) & 0.4351 & (0.0279) & 0.2665 & (0.0166) \\ 
    & 100 & 20 & 0.2877 & (0.0477) & 0.8153 & (0.1501) & 0.2625 & (0.0377) & 861.3937 & (34.1825) & 1.3347 & (0.1086) & 5.5170 & (0.6241) & 7.3283 & (1.4927) \\ 
    & 100 & 30 & 2.7399 & (0.4745) & 6.9793 & (0.9114) & 3.6619 & (0.7715) & 5075.4782 & (908.7174) & 8.4769 & (0.7058) & 66.9461 & (6.0353) & 420.2973 & (119.1735) \\ 
   \hline
III & 50 & 10 & 0.0650 & (0.0053) & 0.0665 & (0.0033) & 0.0697 & (0.0102) & 3.4849 & (0.2297) & 0.4977 & (0.0265) & 0.6678 & (0.0645) & 0.5858 & (0.0365) \\ 
    & 50 & 20 & 1.0423 & (0.1420) & 0.9146 & (0.1113) & 0.4706 & (0.0731) & 426.0848 & (26.4453) & 2.0716 & (0.1360) & 4.8213 & (1.1130) & 8.4099 & (1.3497) \\ 
    & 50 & 30 & 10.8782 & (1.1771) & 8.1124 & (1.2342) & 5.3699 & (0.8475) &  5061.3563 & (572.4879) & 16.5536 & (1.8098) & 779.2829 & (714.9847) & 1181.3770 & (327.7712) \\ 
    & 100 & 10 & 0.0486 & (0.0040) & 0.0363 & (0.0047) & 0.0328 & (0.0040) & 3.5437 & (0.1839) & 0.2437 & (0.0130) & 0.2929 & (0.0196) & 0.2791 & (0.0170) \\ 
    & 100 & 20 & 0.6260 & (0.0200) & 0.3783 & (0.0823) & 0.1958 & (0.0308) & 416.1285 & (12.8666) & 1.0193 & (0.0701) & 1.5353 & (0.1560) & 5.1553 & (1.0771) \\ 
    & 100 & 30 & 5.9367 & (0.7791) & 3.4576 & (0.7345) & 2.2121 & (0.3658) & 5082.1367 & (377.1631) & 7.9582 & (0.8381) & 14.2394 & (1.7202) & 253.4296 & (75.1683) \\ 
   \hline
IV & 50 & 10 & 0.0144 & (0.0010) & 0.0196 & (0.0039) & 0.0053 & (0.0012) & 0.2575 & (0.0340) & 0.4420 & (0.0293) & 0.4628 & (0.0365) & 0.4620 & (0.0363) \\ 
    & 50 & 20 & 0.0449 & (6e-040) & 0.0154 & (0.0024) & 0.0073 & (0.0012) & 0.4384 & (0.0416) & 0.7951 & (0.0447) & 0.9184 & (0.0397) & 0.9177 & (0.0395) \\ 
    & 50 & 30 & 0.0893 & (0.0022) & 0.0189 & (0.0030) & 0.0072 & (0.0011) & 0.6539 & (0.0557) & 1.3363 & (0.0485) & 1.3014 & (0.0462) & 1.3013 & (0.0453) \\ 
    & 100 & 10 & 0.0112 & (5e-040) & 0.0186 & (0.0029) & 0.0031 & (6e-040) & 0.2098 & (0.0185) & 0.2136 & (0.0109) & 0.2299 & (0.0134) & 0.2295 & (0.0133) \\ 
    & 100 & 20 & 0.0420 & (4e-040) & 0.0143 & (0.0014) & 0.0027 & (4e-040) & 0.4877 & (0.0325) & 0.4509 & (0.0167) & 0.4311 & (0.0159) & 0.4307 & (0.0158) \\ 
    & 100 & 30 & 0.0792 & (4e-040) & 0.0181 & (0.0020) & 0.0035 & (6e-040) & 0.6616 & (0.0327) & 0.6263 & (0.0215) & 0.6598 & (0.0207) & 0.6589 & (0.0207) \\ 
   \hline
V & 50 & 10 & 0.3621 & (0.0123) & 0.2456 & (0.0206) & 0.1610 & (0.0332) & 1.3738 & (0.0999) & 0.8484 & (0.0549) & 1.6174 & (0.1133) & 0.8963 & (0.0554) \\ 
    & 50 & 20 & 0.9911 & (0.0102) & 0.8206 & (0.0213) & 0.5236 & (0.1373) & 2.8419 & (0.1751) & 1.7324 & (0.0802) & 3.0233 & (0.1872) & 1.6375 & (0.0889) \\ 
    & 50 & 30 & 1.5352 & (0.0088) & 1.1507 & (0.0176) & 0.4632 & (0.0755) & 4.1877 & (0.2390) & 2.5484 & (0.0975) & 5.1546 & (0.3173) & 2.6727 & (0.1067) \\ 
    & 100 & 10 & 0.3091 & (0.0047) & 0.2678 & (0.0112) & 0.0813 & (0.0133) & 1.2439 & (0.0664) & 0.4175 & (0.0258) & 1.0431 & (0.0556) & 0.4922 & (0.0273) \\ 
    & 100 & 20 & 0.9734 & (0.0075) & 0.4111 & (0.0084) & 0.1522 & (0.0331) & 2.7280 & (0.1010) & 0.7896 & (0.0306) & 2.1932 & (0.0929) & 0.8461 & (0.0355) \\ 
    & 100 & 30 & 1.6032 & (0.0088) & 0.7701 & (0.0098) & 0.3656 & (0.0968) & 3.8905 & (0.1447) & 1.2577 & (0.0466) & 3.5722 & (0.1457) & 1.3270 & (0.0411) \\ 
   \hline
\end{tabular}
\end{scriptsize}
\end{table}
\end{landscape}



%
% The all important bibliography file at the end of your document!! Use
% the bibstyle you (your department) like in the \bibliographystyle{}
% statement and list the name of your bibliography database file in
% the \bibliography{} statement.  In this example, ``bibfile.bib'' is
% the name of the database.  See the LaTeX manual appendix B for details
% about the bibliography database and BibTeX.
%

\bibliographystyle{chicago}
\bibliography{Master}


\end{document}




