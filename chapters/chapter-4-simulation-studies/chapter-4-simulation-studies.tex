\documentclass[12pt]{article}
\usepackage{graphicx,psfrag,amsfonts,float,mathbbol,xcolor,cleveref}
\usepackage{arydshln}
\usepackage{amsmath}
\usepackage{tikz}
\usepackage[mathscr]{euscript}
\usepackage{enumitem}
\usepackage{accents}
\usepackage{lscape}
\usepackage{framed}
%\usepackage{subcaption}
%\usepackage[hang]{subfigure}
\usepackage{subfig} 
\usepackage{natbib}
\usepackage{mathtools}
\usepackage{IEEEtrantools}
\usepackage{times}
\usepackage{filecontents}
\usepackage{subfiles}
\usepackage{cite}
\usepackage{rotating}
\usepackage{arydshln}
\usepackage{amsthm}
\usepackage[letterpaper, left=1in, top=1in, right=1in, bottom=1in,nohead,includefoot, verbose, ignoremp]{geometry}
\usepackage{booktabs}
\newcommand{\ra}[1]{\renewcommand{\arraystretch}{#1}}
\newcommand\numberthis{\addtocounter{equation}{1}\tag{\theequation}}
\newcommand*\needsparaphrased{\color{red}}
\newcommand*\needscited{\color{orange}}
\newcommand*\needsproof{\color{blue}}
\newcommand*\outlineskeleton{\color{green}}
\newcommand{\PP}{\mathcal{P}}
\newcommand{\hilbert}{\mathcal{H}}
\newcommand{\hilbertl}{\mathcal{H}_{\langle l \rangle}}
\newcommand{\hilbertm}{\mathcal{H}_{\langle m \rangle}}
\newcommand{\hilbertlnull}{\mathcal{H}_{0\langle l \rangle}}
\newcommand{\hilbertmnull}{\mathcal{H}_{0\langle m \rangle}}
\newcommand{\hilbertlpen}{\mathcal{H}_{1\langle l \rangle}}
\newcommand{\hilbertmpen}{\mathcal{H}_{1\langle m \rangle}}

\newcommand{\bfeps}{\mbox{\boldmath $\epsilon$}}
\newcommand{\bfgamma}{\mbox{\boldmath $\gamma$}}
\newcommand{\bflam}{\mbox{\boldmath $\lambda$}}
\newcommand{\bfphi}{\mbox{\boldmath $\phi$}}
\newcommand{\bfsigma}{\mbox{\boldmath $\sigma$}}
\newcommand{\bfbeta}{\mbox{\boldmath $\beta$}}
\newcommand{\bfalpha}{\mbox{\boldmath $\alpha$}}
\newcommand{\bfe}{\mbox{\boldmath $e$}}
\newcommand{\bff}{\mbox{\boldmath $f$}}
\newcommand{\bfone}{\mbox{\boldmath $1$}}
\newcommand{\bft}{\mbox{\boldmath $t$}}
\newcommand{\bfo}{\mbox{\boldmath $0$}}
\newcommand{\bfO}{\mbox{\boldmath $O$}}
\newcommand{\bfx}{\mbox{\boldmath $x$}}
\newcommand{\bfX}{\mbox{\boldmath $X$}}
\newcommand{\bfz}{\mbox{\boldmath $z$}}
\newcommand{\argmin}[1]{\underset{#1}{\operatorname{arg}\,\operatorname{min}}\;}
\DeclareMathAlphabet{\mathpzc}{OT1}{pzc}{m}{it}

\newcommand{\bfm}{\mbox{\boldmath $m}}
\newcommand{\bfy}{\mbox{\boldmath $y$}}
\newcommand{\bfa}{\mbox{\boldmath $a$}}
\newcommand{\bfb}{\mbox{\boldmath $b$}}
\newcommand{\bfY}{\mbox{\boldmath $Y$}}
\newcommand{\bfS}{\mbox{\boldmath $S$}}
\newcommand{\bfZ}{\mbox{\boldmath $Z$}}
\newcommand{\cardT}{\vert \mathcal{T} \vert}
%\newenvironment{theorem}[1][Theorem]{\begin{trivlist}
%\item[\hskip \labelsep {\bfseries #1}]}{\end{trivlist}}
%\newenvironment{corollary}[1][Corollary]{\begin{trivlist}
%\item[\hskip \labelsep {\bfseries #1}]}{\end{trivlist}}
%\newenvironment{proposition}[1][Proposition]{\begin{trivlist}
%\item[\hskip \labelsep {\bfseries #1}]}{\end{trivlist}}
%\newenvironment{definition}[1][Definition]{\begin{trivlist}
%\item[\hskip \labelsep {\bfseries #1}]}{\end{trivlist}}

\newtheorem{theorem}{Theorem}[section]
\newtheorem{lemma}[theorem]{Lemma}
\newtheorem{proposition}[theorem]{Proposition}
\newtheorem{corollary}[theorem]{Corollary}

\theoremstyle{definition}
\newtheorem{definition}{Definition}[section]
\newtheorem{example}{Example}[section]
\def\bL{\mathbf{L}}

\begingroup\lccode`~=`_
\lowercase{\endgroup\def~}#1{_{\scriptscriptstyle#1}}
\AtBeginDocument{\mathcode`_="8000 \catcode`_=12 }
\makeatletter
\newcommand{\clearsubcaptcounter}{\setcounter{sub\@captype}{0}}
\renewcommand{\theenumi}{\Roman{enumi}}
\renewcommand{\labelenumi}{\theenumi.}
\renewcommand{\theenumii}{\Alph{enumii}}
\renewcommand{\labelenumii}{\theenumii.}
\renewcommand{\p@enumii}{\theenumi.}
\makeatother

\begin{document}

%\nocite{*}
\def\bL{\mathbf{L}}
%\usepackage{mathtime}

%%UNCOMMENT following line if you have package


\title{ Nonparametric Covariance Estimation for Longitudinal Data via Penalized Tensor Product Splines}

\author{Tayler A. Blake\thanks{The Ohio State University, 1958 Neil Avenue, Columbus, OH 43201} \and  Yoonkyung Lee\thanks{The Ohio State University, 1958 Neil Avenue, Columbus, OH 43201}}

\bibliographystyle{plainnat}
\maketitle

\section{Performance assessment via simulation study} 
\subsection{Performance benchmarking with complete data}

In this section we compare bivariate spline estimators of the Cholesky factor to other methods of covariance estimation. Our primary comparisons are that with the parametric polynomial estimator proposed by citet{pourahmadi1999joint},  \citet{pan2003modelling}, and \citet{pourahmadi2002dynamic}, which is also based on the modified Cholesky decomposition, and with the oracle estimator, which effectively gives a lower bound on the risk for given covariance structure. As a benchmark, we also include the sample covariance matrix, and two regularized variants of it: the tapered sample covariance matrix (\citet{cai2010optimal}) and the soft thresholding estimator (\citet{rothman2009generalized}), which does not rely on a natural ordering among the variables. 

\bigskip

Simulations were carried out for five covariance structures: the diagonal covariance with homogenous variances, a heterogenous autoregressive process with linear varying coefficient function, the same heterogeneous process but truncated to zero to band the inverse covariance matrix, the rational quadratic covariance model, and the compound symmetric model. The two-dimensional surfaces corresponding to each of these are shown left to right in Figure~\ref{fig:true-covariance-heatmaps}. The first row of image plots display the surface which coincides with the appropriate discrete covariance matrix, and in the second row are the surface maps of the corresponding Cholesky factors. Precise models used for simulations are defined in Table~\ref{simulation-model-list}.

\bigskip

\subfile{chapter-4-subfiles/chapter-4-true-covariance-heatmaps}

\bigskip

Connecting the covariance matrices in first row of Figure~\ref{fig:true-covariance-heatmaps} with their Cholesky factor in the second row, covariance structures exhibiting sparsity or parsimony do not necessarily exhibit the same simplicity in the components of the Cholesky decomposition. The Cholesky factor for Model III, the truncated linear varying coefficient AR model, is sparse, with elements on the outer half of the subdiagonals equal to zero. While this corresponds to a banded inverse covariance structure, $\Sigma$ itself is not sparse.  The compound symmetric model has simple structure and is parsimonious; its dependence parameters can be expressed as the evaluation of a function which is constant in time $t$. However, the elements of the Cholesky factor and diagonal matrix $D = T \Sigma T'$ do not exhibit such elementary structure, the elements of which are nonlinear in $t$. 

\bigskip

\subfile{chapter-4-subfiles/chapter-4-true-covariance-functions}
\bigskip

For each of the covariance models, we generated a set of observations of sample size $N = 50, 100$ from a multivariate normal distribution, and considered three different values of within-subject sample size $M = 10, 20, 30$. The estimators were computed with tuning parameters selected using both leave-one-subject-out cross validation $\mbox{losoCV}\left(\lambda\right)$ and unbiased risk estimate $\mbox{U}\left(\lambda\right)$. Given the selected values of the tuning parameters, we computed the estimated covariance matrix and compared it to the true covariance matrix via entropy loss and quadratic loss. 

\bigskip
\subsubsection{Loss functions and corresponding risk measures}
\subfile{chapter-4-subfiles/chapter-4-loss-risk-functions}

\bigskip

\subsubsection{Alternative estimators}
\subfile{chapter-4-subfiles/chapter-4-benchmark-estimators}

\bigskip

\subfile{chapter-4-subfiles/chapter-4-benchmark-study-discussion}

\subsection{Performance with irregularly sampled data}

\subfile{chapter-4-subfiles/chapter-4-missing-data-study-discussion}

\bigskip
\setlength{\dashlinedash}{0.5pt}
\setlength{\dashlinegap}{1pt}
\setlength{\arrayrulewidth}{0.2pt}
\subfile{chapter-4-subfiles/simulation-study-2-risk-tables-ure}

{\needsparaphrased{TODO: remember to cite the nlme package for fitting the MA(1) and CS oracle models.}}

\subfile{chapter-4-subfiles/chapter-4-numerical-discussion-2}

%%%%%%%%%%%%%%%%%%%%%%%%%%%%%%%%%%%%%%%%%%%%%%%%%%%%%%%
%\subsection{Numerical results}
%\subsubsection{Simulation study 1: complete data} \label{benchmarking-results}

%-------------------------------------------------------------------------------------------------------------------------------------------
\subsection{Appendix}

%\subsection{Quadratic risk estimates for simulation study 1}
%\begin{table}[H]\ref{table:simulation-1-quad-loss-sigma-1}
%\caption{Risk estimates and corresponding standard errors for our proposed estimator under quadratic loss, $\Delta_1$ when the data are generated according to model~\ref{item:cov-type-1}.} 
%\centering
%\begin{tabular}{l|r|rrrrrr}
%&  & \multicolumn{2}{c}{$\hat{\Sigma}_{SS}$} & $S$ & $S^\lambda$ & $S^\omega$ \\ 
%&M & \mbox{LosoCV} & \mbox{URE} &  \\ 
%\hline
%		&    10 & 0.0010 & 0.0013 & 0.4702  & 0.3926 & 0.3871 \\ 
%$N = 50$  &    20 & 0.0007 &  0.0006	& 0.8495 & 0.8301 & 0.8287 \\ 
%  		&    30 & 0.0003 &  0.0004	& 1.1449 & 1.1926 & 1.1924  \\ \hdashline
%		 &    10 & 0.0004 &  0.0004	& 0.2072 &  0.1802 & 0.1777\\ 
%$N = 100$ &    20 & 0.0002 & 0.0002	& 0.3920  & 0.3858 & 0.3817 \\ 
%   &    30 & 0.0001 & 0.0001 &0.5712 & 0.6191 & 0.6109 \\ 
%\end{tabular}
%\end{table}
%
%%-------------------------------------------------------------------------------------------------------------------------------------------
%
%\begin{table}[H]\ref{table:simulation-1-quad-loss-sigma-2}
%\centering
%\caption{Risk estimates and corresponding standard errors for our proposed estimator under quadratic loss, $\Delta_1$ when the data are generated according to model~\ref{item:cov-type-2}.} 
%\begin{tabular}{l|r|rrrrrr}
%&  & \multicolumn{2}{c}{$\hat{\Sigma}_{SS}$} & $S$ & $S^\lambda$ & $S^\omega$ \\ 
%&M & \mbox{LosoCV} & \mbox{URE} &  \\ 
%  \hline
%&    10 & 0.0314 &  0.0411	&0.5726  & 0.5810 & 0.7758\\ 
%$N = 50 $ &    20 & 0.3266 & 0.7265	& 2.3130   & 5.5964 & 2.7545  \\ 
% &    30 & 5.0696 &  4.9073	 &15.1096 & 765.7206 & 28.6820  \\ \hdashline
% &    10 & 0.0156 &  0.0147	& 0.2479  & 0.2501 & 0.3544 \\ 
%$N = 100$ &    20 & 0.1894 &  0.2017	 &1.3177 & 5.1945 & 4.7634 \\ 
%  &    30 & 2.3876 &	1.6465  & 9.8175 & 488.6801 & 85.9508\\ 
%\end{tabular}
%\end{table}
%%-------------------------------------------------------------------------------------------------------------------------------------------
%
%
%\begin{table}[H]\ref{table:simulation-1-quad-loss-sigma-3}
%\centering
%\caption{Risk estimates and corresponding standard errors for our proposed estimator under quadratic loss, $\Delta_1$ when the data are generated according to model~\ref{item:cov-type-3}.} 
%\begin{tabular}{l|r|rrrrrr}
%&  & \multicolumn{2}{c}{$\hat{\Sigma}_{SS}$} & $S$ & $S^\lambda$ & $S^\omega$ \\ 
%&M & \mbox{LosoCV} & \mbox{URE} &  \\ 
%  \hline
% 	          &    10 & 0.0562 &	0.0547 & 0.5237 & 0.5810 & 0.5313 \\ 
% $N = 50$ 	 &     20 & 0.7832 & 0.8934   & 2.1419 & 9.5721 & 9.1421\\ 
%  		  &    30 & 8.2650 & 10.6855  & 15.2842 & 407.3659 & 129.7459\\ \hdashline
%		  &    10 & 0.0376 &0.0449	 & 0.2546  & 0.2556 & 0.2661\\ 
% $N = 100$  &    20 & 0.6260 & 0.5967	 & 1.3751 & 3.3281 & 1.2759\\ 
%   &    30 & 5.7635 &	6.2824 & 7.4750& 203.6710 & 10.0634 \\ 
%\end{tabular}
%\end{table}
%
%%-------------------------------------------------------------------------------------------------------------------------------------------
%
%\begin{table}[H]\ref{table:simulation-1-quad-loss-sigma-4}
%\centering
%\caption{Risk estimates and corresponding standard errors for our proposed estimator under quadratic loss, $\Delta_1$ when the data are generated according to model~\ref{item:cov-type-4}.} 
%\begin{tabular}{l|r|rrrrrr}
%&  & \multicolumn{2}{c}{$\hat{\Sigma}_{SS}$} & $S$ & $S^\lambda$ & $S^\omega$ \\ 
%&M & \mbox{LosoCV} & \mbox{URE} &  \\ 
%  \hline
% &    10 & 0.0134 &  0.0145	& 0.4169 & 0.3987 & 0.3985 \\ 
%$N = 50$ &    20 & 0.0590 & 0.0574 & 0.8810& 0.9078 & 0.9073 \\ 
% &    30 & 0.1351 &  0.1362	& 1.2571  & 1.2570 & 1.2575\\ \hdashline
%     &    10 & 0.0077 &  0.0078 & 0.2263  & 0.2111 & 0.2104 \\ 
%  $N = 100$ &    20 & 0.0549 & 0.0534  & 0.4309 & 0.4127 & 0.4120 \\ 
%   &    30 & 0.1331 & 0.1320 & 0.6819  & 0.6579 & 0.6565 \\\
%\end{tabular}
%\end{table}
%
%
%%-------------------------------------------------------------------------------------------------------------------------------------------
%
%\begin{table}[H]\ref{table:simulation-1-quad-loss-sigma-5}
%\centering
%\caption{Risk estimates and corresponding standard errors for our proposed estimator under quadratic loss, $\Delta_1$ when the data are generated according to model~\ref{item:cov-type-5}.} 

%\begin{tabular}{l|r|rrrrrr}
%&  & \multicolumn{2}{c}{$\hat{\Sigma}_{SS}$} & $S$ & $S^\lambda$ & $S^\omega$ \\ 
%&M & \mbox{LosoCV} & \mbox{URE} &  \\ 
%  \hline
% &    10 & 0.3688 & 0.3599	& 0.7872& 0.8058 & 1.4774 \\ 
%$N = 50$ &    20 & 0.9770 &   0.9954	 & 1.6167& 1.7840 & 3.4516 \\ 
%  &    30 & 1.6067 &	1.6151   &  2.5548 & 2.4837 & 4.9027 \\ \hdashline
%  &    10 & 0.3210 & 0.3168 & 0.3913 & 0.3819 & 0.8958\\ 
%  $N = 100$ &    20 & 0.9793 & 0.9774 &  0.8714 & 0.8479 & 2.2110\\ 
%   &    30 & 1.6177 &  1.6032  & 1.2967  & 1.2293 & 3.4968\\ 
%\end{tabular}
%\end{table}
%\subfile{chapter-4-subfiles/simulation-study-1-short-quad-table}

\subsubsection{Quadratic risk estimates for simulation study 1.}
\subfile{chapter-4-subfiles/simulation-study-1-quad-table-model-1}
\subfile{chapter-4-subfiles/simulation-study-1-quad-table-model-2}
\subfile{chapter-4-subfiles/simulation-study-1-quad-table-model-3}
\subfile{chapter-4-subfiles/simulation-study-1-quad-table-model-4}
\subfile{chapter-4-subfiles/simulation-study-1-quad-table-model-5}
%-----------------------------------------------------------------------------------------------------------------------------------------------------------------------------------------------------------------------

\subfile{chapter-4-subfiles/simulation-study-1-extra-tables}

%%-----------------------------------------------------------------------------------------------------------------------------

\begin{landscape}
\subfile{chapter-4-subfiles/simulation-study-1-master-quadratic-risk-table}
\end{landscape}

%%-----------------------------------------------------------------------------------------------------------------------------
\begin{landscape}
\subfile{chapter-4-subfiles/simulation-study-1-master-entropy-risk-table}
\end{landscape}


\bibliography{../Master}
\end{document}
