
Our second concern in evaluation of our methods is how performance changes when the data exhibit varying degrees of sparsity. We fix the number of sampled trajectories $N$ and vary $M$, the size of the set  of possible measurement times

\[
t_1,\dots, t_M.
\]
\noindent
We generate irregular data by first generating a complete dataset 

\begin{align*}
Y_1 &= \left(y_1\left(t_1\right), y_1\left(t_2\right), \dots, y_1\left(t_M\right)\right)' \\
Y_2 &= \left(y_2\left(t_1\right), y_2\left(t_2\right), \dots, y_2\left(t_M\right)\right)' \\
&\vdots \\
Y_N &= \left(y_N\left(t_1\right), y_N\left(t_2\right), \dots, y_N\left(t_M\right)\right)',
\end{align*}
\noindent
where $Y_1,\dots, Y_N$ are independently and identically distributed according to an $M$-dimensional multivariate Normal distribution with mean zero and having covariance structure identical to one of Models I - V in \ref{simulation-model-list}. To induce sparsity, we subsample from the complete data $\left\{y_i\left(t_j\right) \right\}$, $i = 1,\dots, N$, $j = 1,\dots, M$, randomly omitting an observation $y_i\left(t_j\right)$ with probability $0.05$, $0.07$, and $0.09$.

\bigskip

Results under quadratic loss and entropy loss are given in Section~\ref{sparsity-results}, tables \ref{table:simulation-2-sigma-1} - \ref{table:simulation-2-sigma-5}.  Standard errors of the risk estimates are left to the appendix; see Table~\ref{table:ssanova-estimator-performance-with-se-ure} and Table~\ref{table:ssanova-estimator-performance-with-se-loso}.  Performance degradation of the estimator in the presence of missing data is highly dependent on the underlying structure of the Cholesky factor of the inverse covariance matrix. For the identity matrix and for the non-truncated linear varying coefficient GARP model, we observe little change in estimated entropy risk for within subject sample sizes $M = 10$ and $M = 20$ with downsampling as compared to the estimated risk for both sample sizes in the complete data case. 

\bigskip

Making the same comparison for the banded Cholesky factor having linear varying coefficient function truncated at $t = 0.5$, we see only slight decreases in performance for $M = 10$: an estimated entropy risk of 0.3174  with no missing data versus 0.3451 (0.3498, 0.3437) with $5\%$ ($7\%$, $9\%$) missing data. The degredation is more pointed for the moderate sample size of $M = 20$. The rate of missing observations has the greatest impact for the simulation conducted using the compound symmetric model. This is not surprising, since it corresponds to the Cholesky factor having the most complex structure. While the functions defining the Cholesky factors of Models III and IV do not belong to the null space defined by the cubic smoothing spline penalty, they are both piecewise functions with each piece itself belonging to $\mathcal{H}_0$.

\bigskip

{\needsparaphrased{Should the discussion that immediately follows be moved to after the tables containing non-appendix numerical results?}}

\bigskip

{\needsparaphrased{Should the discussion of study \# 1 be with the tablse for study 1, separate from the discussion + tables for study 2?}}

\bigskip

