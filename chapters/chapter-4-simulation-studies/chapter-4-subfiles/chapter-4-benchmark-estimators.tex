
For the complete data case with common observation times across all subjects, we consider three additional covariance estimators for comparison: the sample covariance matrix $S$, the soft thresholding estimator of \citet{rothman2009generalized},  $S^\lambda$, and the tapering estimator of \citet{cai2010optimal}, $S^\omega$. See Chapter 2, Section~\ref{section:element-wise-shrinkage-estimators} for additional discussion of these estimators and those belonging to similar classes. \citet{rothman2009generalized} presented a class of generalized thresholding estimators, including the soft-thresholding estimator given by

\[
S^{\lambda}=   \begin{bmatrix} \mbox{sign}\left(s_{ij}\right) \left(s_{ij} - \lambda\right)_+ \end{bmatrix},
\]
\noindent 
where $\sigma^*_{ij}$ denotes the $i$-$j^{th}$ entry of the sample covariance matrix, and $\lambda$ is a penalty parameter controlling the amount of shrinkage applied to the empirical estimator. \citet{cai2010optimal} derived optimal rates of convergence under the operator norm for the tapering estimator:
\[
S^{\omega} =  \begin{bmatrix} \omega_{ij}^k s_{ij} \end{bmatrix},
\]
\noindent
where the $\omega_{ij}^k$ are given by 
\begin{equation*}
\omega^k_{ij} = k_h^{-1} \left[ \left( k - \vert i-j\vert\right)_+ - \left(k_h - \vert i-j\vert\right)_+ \right],
\end{equation*}
\noindent
The weights $\omega^k_{ij}$ are indexed with superscript to indicate that they  are controlled by a tuning parameter, $k$,  which can take integer values between 0 and $M$, the dimension of the covariance matrix.  Without loss of generality,  we assume that $k_h = k/2$ is even. The weights may be rewritten as
\begin{align*}
\omega_{ij} = \left\{\begin{array}{ll} 1, & \vert \vert i -j \vert \vert \le k_h \\
                             2 - \frac{i - j}{k_h}, & k_h < \vert \vert i -j \vert \vert \le k, \\
                             0, & \mbox{otherwise}  \end{array} \right.
\end{align*}
\noindent
This expression of the weights makes it clear how the selection of $k$ controls the amount of shrinkage applied to different elements of the sample covariance matrix. The estimator applies no shrinkage to elements of $S$ belonging to the subdiagonals closest to the main diagonal. As one moves away from the main diagonal, shrinkage increases. A shrinkage factor of $2 - \frac{i - j}{k_h}$ is applied to elements belonging to subdiagonals $k_h,\dots,k-1,k$, and elements further than $k$ subdiagonals from the main diagonal are shrunk to zero.   


