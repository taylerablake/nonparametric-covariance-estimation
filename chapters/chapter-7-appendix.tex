\documentclass[12pt]{article}
\usepackage{graphicx,psfrag,amsfonts,float,mathbbol,xcolor,cleveref}
\usepackage{arydshln}
\usepackage{amsmath}
\usepackage{tikz}
\usepackage[mathscr]{euscript}
\usepackage{enumitem}
\usepackage{accents}
\usepackage{framed}
\usepackage{subcaption}
\usepackage{natbib}
\usepackage{mathtools}
\usepackage{IEEEtrantools}
\usepackage{times}
\usepackage{cite}
\usepackage{amsthm}
\usepackage[letterpaper, left=1in, top=1in, right=1in, bottom=1in,nohead,includefoot, verbose, ignoremp]{geometry}
\newcommand{\comment}[1]{\text{\phantom{(#1)}} \tag{#1}}
\newcommand\numberthis{\addtocounter{equation}{1}\tag{\theequation}}
\newcommand*\needsparaphrased{\color{red}}
\newcommand*\needscited{\color{orange}}
\newcommand*\needsproof{\color{blue}}
\newcommand*\outlineskeleton{\color{green}}
\newcommand{\ms}{\scriptscriptstyle}
\newcommand{\PP}{\mathcal{P}}
\newcommand{\vphistar}{\mbox{\boldmath $\phi^*$}}
\newcommand{\vsigmasq}{\mbox{\boldmath $\sigma^2$}}
\newcommand{\bfeps}{\mbox{\boldmath $\epsilon$}}
\newcommand{\bfgamma}{\mbox{\boldmath $\gamma$}}
\newcommand{\bflam}{\mbox{\boldmath $\lambda$}}
\newcommand{\bfphi}{\mbox{\boldmath $\phi$}}
\newcommand{\bfsigma}{\mbox{\boldmath $\sigma$}}
\newcommand{\bfbeta}{\mbox{\boldmath $\beta$}}
\newcommand{\bfalpha}{\mbox{\boldmath $\alpha$}}
\newcommand{\bfe}{\mbox{\boldmath $e$}}
\newcommand{\bff}{\mbox{\boldmath $f$}}
\newcommand{\bfone}{\mbox{\boldmath $1$}}
\newcommand{\bft}{\mbox{\boldmath $t$}}
\newcommand{\bfo}{\mbox{\boldmath $0$}}
\newcommand{\bfO}{\mbox{\boldmath $O$}}
\newcommand{\bfx}{\mbox{\boldmath $x$}}
\newcommand{\bfX}{\mbox{\boldmath $X$}}
\newcommand{\bfz}{\mbox{\boldmath $z$}}


\newcommand{\bfm}{\mbox{\boldmath $m}}
\newcommand{\bfy}{\mbox{\boldmath $y$}}
\newcommand{\bfa}{\mbox{\boldmath $a$}}
\newcommand{\bfb}{\mbox{\boldmath $b$}}
\newcommand{\bfY}{\mbox{\boldmath $Y$}}
\newcommand{\bfS}{\mbox{\boldmath $S$}}
\newcommand{\bfZ}{\mbox{\boldmath $Z$}}
\newcommand{\cardT}{\vert \mathcal{T} \vert}
%\newenvironment{theorem}[1][Theorem]{\begin{trivlist}
%\item[\hskip \labelsep {\bfseries #1}]}{\end{trivlist}}
%\newenvironment{corollary}[1][Corollary]{\begin{trivlist}
%\item[\hskip \labelsep {\bfseries #1}]}{\end{trivlist}}
%\newenvironment{proposition}[1][Proposition]{\begin{trivlist}
%\item[\hskip \labelsep {\bfseries #1}]}{\end{trivlist}}
%\newenvironment{definition}[1][Definition]{\begin{trivlist}
%\item[\hskip \labelsep {\bfseries #1}]}{\end{trivlist}}

\newtheorem{theorem}{Theorem}[section]
\newtheorem{lemma}[theorem]{Lemma}
\newtheorem{proposition}[theorem]{Proposition}
\newtheorem{corollary}[theorem]{Corollary}

\theoremstyle{definition}
\newtheorem{definition}{Definition}[section]
\newtheorem{example}{Example}[section]
\def\bL{\mathbf{L}}

\begingroup\lccode`~=`_
\lowercase{\endgroup\def~}#1{_{\scriptscriptstyle#1}}
\AtBeginDocument{\mathcode`_="8000 \catcode`_=12 }

\makeatletter
\renewcommand{\theenumi}{\Roman{enumi}}
\renewcommand{\labelenumi}{\theenumi.}
\renewcommand{\theenumii}{\Alph{enumii}}
\renewcommand{\labelenumii}{\theenumii.}
\renewcommand{\p@enumii}{\theenumi.}
\makeatother

\begin{document}

%\nocite{*}
\def\bL{\mathbf{L}}
%\usepackage{mathtime}

%%UNCOMMENT following line if you have package


\title{ Nonparametric Covariance Estimation for Longitudinal Data via Penalized Tensor Product Splines}

\author{Tayler A. Blake\thanks{The Ohio State University, 1958 Neil Avenue, Columbus, OH 43201} \and  Yoonkyung Lee\thanks{The Ohio State University, 1958 Neil Avenue, Columbus, OH 43201}}

\bibliographystyle{plainnat}
\maketitle

\section{Appendix}

\subsection{Proof of the smoothing spline expansion representation}
\begin{proof} \label{proof:ss-representer-expansion-thm}

Using the properties of reproducing kernels, we can rewrite $\phi^*$ as an inner product of itself with $R$:
 
\begin{eqnarray*}
\phi^*\left(l_j,m_j \right)  &=& \left< R\left(\left(l_j,m_j\right),\left(\cdot,\cdot\right) \right),\phi^*\left(\cdot,\cdot\right)\right>\\
&=& \left<R_0\left( \left(l_j,m_j\right),\left(\cdot,\cdot\right) \right) + R_1\left(\left(l_j,m_j\right),\left(\cdot,\cdot\right) \right),d_0 + d_1k_1\left(\cdot \right)\right. \\ 
&\mbox{ }&\left. \;\;\;\;\;\;\;\;\;\;\;\;\;\;\;\;\;\;\;\;\;\;\;\;\;\;\;\;\;\;\;\;\;+ \sum_{i=1}^{N_{\phi^*}}  c_i R_1\left( \left(l_i,m_i \right),\left(\cdot,\cdot\right) \right) + \rho\left(\left(\cdot,\cdot \right)\right)\right>\\
&=& \left<R_0\left( \left(l_j,m_j\right),\left(\cdot,\cdot\right) \right) , d_0 + d_1k_1\left(\cdot\right)\right> + \left< R_0\left( \left(l_j,m_j\right),\left(\cdot,\cdot\right) \right),\sum_{i=1}^{N_{\phi^*}}  c_i R_1\left( \left(l_i,m_i \right),\left(\cdot,\cdot\right) \right)\right> \\
&\mbox{ }& + \left<R_0\left( \left(l_j,m_j\right),\left(\cdot,\cdot\right) \right), \rho\left(\left(\cdot,\cdot \right)\right)\right> + \left<R_1\left(\left(l_j,m_j\right),\left(\cdot,\cdot\right) \right), d_0 + d_1k_1\left(\cdot \right)\right> \\
&\mbox{ }& + \left<R_1\left(\left(l_j,m_j\right),\left(\cdot,\cdot\right) \right),\sum_{i=1}^{N_{\phi^*}}  c_i R_1\left( \left(l_i,m_i \right),\left(\cdot,\cdot\right) \right) \right> + \left<R_1\left(\left(l_j,m_j\right),\left(\cdot,\cdot\right) \right), \rho\left(\left(\cdot,\cdot \right)\right)\right>\\
&=& \left<R_0\left( \left(l_j,m_j\right),\left(\cdot,\cdot\right) \right) , d_0 + d_1k_1\left(\cdot\right)\right> + \left<R_1\left(\left(l_j,m_j\right),\left(\cdot,\cdot\right) \right),\sum_{i=1}^{N_{\phi^*}}  c_i R_1\left( \left(l_i,m_i \right),\left(\cdot,\cdot\right) \right) \right> \\
&\mbox{ }& + \underbrace{\left<R_0\left( \left(l_j,m_j\right),\left(\cdot,\cdot\right) \right)  , \rho\left(\cdot,\cdot\right) \right>}_{0} + \underbrace{\left<R_1\left( \left(l_j,m_j\right),\left(\cdot,\cdot\right) \right)  , \rho\left(\cdot,\cdot\right) \right>}_{0}\\
&=& d_0 + d_1k_1\left(\cdot \right) + \sum_{i=1}^{N_{\phi^*}}  c_i R_1\left( \left(l_i,m_i \right),\left(l_j,m_j\right) \right)
\end{eqnarray*}
\noindent


Rewriting the data fit functional, we have that  
 \begin{eqnarray*}
&\mbox{ }&\sum_{i=1}^N \sum_{j=1}^{n_i} \sigma_{ij}^{-2} \left(y\left(t_{ij}\right) - \sum_{k=1}^{j-1} \phi^*\left(t_{ij}, t_{ik}  \right) y\left(t_{ik}\right)  \right)^2  \\ 
&=& \sum_{i=1}^N \sum_{j=1}^{n_i} \sigma_{ij}^{-2} \left(y\left(t_{ij}\right) - \sum_{k=1}^{j-1} \left< R\left(\left(l^i_{jk},m^i_{jk}\right),\left(\cdot,\cdot\right) \right),\phi^*\left(\cdot,\cdot\right)\right> y\left(t_{ik}\right)  \right)^2  \\
 \end{eqnarray*}
\noindent
which is free of $\rho$. Consider the contribution of any nonzero $\rho$ to $J\left(\phi^*\right)$: 
  
 \begin{eqnarray*}
 J\left(\phi^*\right) &=& \vert \vert  P_1\phi^* \vert \vert^2\\
 &=& \left< \sum_{i=1}^{N_{\phi^*}}  c_i R_1\left( \left(l_i,m_i\right),\left(\cdot,\cdot\right) \right) + \rho\left(\cdot,\cdot \right), \sum_{j=1}^{N_{\phi^*}} c_j R_1\left( \left(l_j,m_j\right),\left(\cdot,\cdot\right) \right) + \rho\left(\cdot,\cdot\right)\right> \\
 &=& \vert \vert \sum_{i=1}^{N_{\phi^*}}  c_i R_1\left(\left(l_i,m_i\right),\left(\cdot,\cdot\right) \right) \vert \vert^2 + \vert \vert  \rho \vert \vert^2 
 \end{eqnarray*}
\noindent
Thus, including $\rho$ in $\phi^*$ only increases the penalty without improving (decreasing) the data fit functional, so we indeed have that the minimizer of \eqref{eq:objectivefun} has the form
\begin{equation}
 \phi^*\left(l,m\right) =  d_0 + d_1k_1\left(l\right) + \sum_{i=1}^{N_{\phi^*}} c_i R_1\left( \left(l,m\right) , \left(l_i,m_i \right)\right)
 \label{eq:finite-dim-solution}
 \end{equation}
\end{proof}



\bibliography{Master}
\end{document}

