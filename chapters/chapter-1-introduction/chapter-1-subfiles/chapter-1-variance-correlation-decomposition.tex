The variance-correlation decomposition of $\Sigma$ is perhaps the most familiar of the following three parameterizations, which parameterizes the covariance matrix according to

\begin{equation}\label{eq:}
\Sigma = DRD,
\end{equation}
\noindent
where $D = \mbox{diag}\left(\sqrt{\sigma_{11}},\dots , \sqrt{\sigma_{MM}}\right)$ denotes the diagonal matrix with diagonal entries equal to tje square-roots of those of $\Sigma$, and $R$ is the corresponding correlation matrix. This parameterization enjoys attractive practicality because the standard deviations are on the same scale as the responses, and because the estimation of $D$ and $R$ can be separated by eithering iteratively fixing one sequence of parameters to estimate the other. Moreover, one set of parameters may be more important than the others in some applications; the dynamic correlation model presented in  Engle’s (2002) is actually motivated by the fact that variances (volatilities) of individual assets are more important than their time-varying correlations.
\bigskip

While the diagonal entries of $D$ are constrained to be nonnegative, their logarithms are unconstrained. However, the correlation matrix $R$ is positive-definite constrained to have unit diagonal entires and off-diagonal entries to be less than or equal to 1 in absolute value. Because of these constraints, the variance-correlation decomposition does not lend to modeling its

\bigskip
The MLE of the parameters under (M3) is reviewed in Section 4, (M2)-(M3) are of particular interest in the recent literature of finance. In fact, starting with the variance - correlation
decomposition (13), Bollerslev’s (1990) constant-correlation models assume that the correlation matrices {Rt} are constant, that is Rt ≡ R with p(p − 1)/2 parameters. The MLE
of R turns out to be the sample correlation matrix of the suitably standardized vector of returns. Recognizing that constancy of correlations over time is not often satisfied, Engle
(2002) and Tse and Tsui (2002) have recently introduced a dynamic model for {Rt} with scalar coefficients. In the first-order case it takes the form

\begin{equation}
R_t = \left(1 - \alpha - \beta\right)\bar{R} + \alpha R_{t-1}+ \beta\Psi_{t-1}, \quad t = 1,\dots, M
\end{equation}
\noindent
where $\bar{R}$ is the sample correlation matrix of $y_1, \dots ,y_M$, $\Psi_{t-1}$ is a positive-definite correlation matrix having elements which are functions of the lagged observations, and $\alpha$ and $\beta$ are nonnegative with $\alpha+\beta \le 1$ so that $R_t$ is a weighted average of positive-definite matrices with nonnegative coefficients, which is guaranteed to be positive-definite. Though such models are highly parsimonious relative to the full multivariate GARCH models, they specify that all pairwise correlations satisfy the same simple dynamic, lacking flexibility and practical applicability.
