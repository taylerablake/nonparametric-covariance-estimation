\begin{example}{Example: $m^{th}$ order Sobolev space, $W_m\left(0,1\right)$}

A popular choice of RKHS for the marginal space of $l$ and the marginal space of $m$ is the Sobolev space of order $m$. Let $\mathcal{H} = \mathcal{H}_{0} \oplus \mathcal{H}_{1}$ be the RKHS corresponding to the tensor product of the first-order and second-order Sobolev spaces:

\[
\mathcal{H} = \mathcal{H}_{l} \otimes \mathcal{H}_{m}, \;\; \mathcal{H}_{l} = W_2\left(0,1\right),\;\;\mathcal{H}_{m} = W_1\left(0,1\right)\;\mbox{where }
\]

\[W_m\left(0,1\right) \equiv \lbrace f: \;\;f^\prime, \dots, f^{\left( m-1 \right)} \mbox{absolutely continuous}, \int_0^1 \left(f^{\left( m \right)}\right)^2 dt < \infty \rbrace\]
\noindent
%We seek $\left(\cdot, \cdot \right) \in \mathcal{H}$ which minimizes
%\begin{equation}
%\frac{1}{2}\sum_{i=1}^N \sum_{j=2}^{n_i} {\sigma^{-2}_{ij}}\left( y\left(t_{ij}\right) - \sum_{k=1}^{n_i - 1}\phi\left(l^i_{jk},m^i_{jk} \right)y\left(t_{ik}\right) \right)^2 + \lambda J\left(\phi\right)  
%\label{eq:objectivefun}
%\end{equation}
%\noindent
%where $P_1 \phi$ is the projection of $\phi$ onto $\mathcal{H}_1$, $J\left(\phi\right) = \vert \vert P_1 \phi \vert \vert^2$. 

Define the differential operator $M_\nu f = \int_0^1 f^{\left( m \right)}\left(x\right) dx\;,\;\; \nu = 1, \dots, m$ and endow $W_m\left(0,1\right)$ with inner product
%\begin{equation}
%\left< f,g\right> = \underbrace{\sum_{\nu=0}^{m-1} M_\nu f M_\nu g}_{\left< f,g\right>_0} + \underbrace{\int_0^1 f^{\left( m \right)}\left(x\right)g^{\left( m \right)}\left(x\right)dx}_{\left< f,g\right>_1}
%\end{equation}

\begin{equation}
\left< f,g\right> = \left< f,g\right>_0 + \left< f,g\right>_1 = \sum_{\nu=0}^{m-1} M_\nu f M_\nu g + \int_0^1 f^{\left( m \right)}\left(x\right)g^{\left( m \right)}\left(x\right)dx
\end{equation}
\noindent
which induces norm 
\[
\vert \vert f \vert \vert^2 = \left< f,f\right> = \left< f,f\right>_0 + \left< f,f\right>_1 = \vert \vert P_0 f \vert \vert^2 + \vert \vert P_1 f \vert \vert^2
\]
\noindent
Let $k_j\left(x\right) = B_j\left(x\right)/{j!}$ for $x \in \left[0,1\right]$, where $B_j\left(x\right)$ is the $j^{th}$ Bernoulli polynomial which can be defined according to the recursive relationship:

\[
B_0\left(x\right) = 1,\;\;\;\;\;\; \frac{d}{dx} B_r\left(x\right) = rB_{r-1}\left(x\right)
\]
\noindent
Noting that $M_\nu B_r = \delta_{\nu-r}$, $W_m$ can be written as a direct sum of the $m$ orthogonal subspaces: $\lbrace k_r \rbrace_{r=0}^{m-1}$ and $W_m^1$.   Here, $\lbrace k_r \rbrace$ is the subspace spanned by $k_r$ and $W_m^1$ is the space orthogonal to $W_m^0 \equiv \lbrace 1 \rbrace \oplus \lbrace k_1 \rbrace \oplus \dots \oplus \lbrace k_{m-1} \rbrace$ which satisfies 
\[
W_m^1 = \lbrace f: M_\nu f = 0,\;\; \nu = 0,1,\dots, m-1\rbrace
\]

\noindent
Writing $\mathcal{H}$ as the tensor product of the two decomposed Sobolev spaces, we have

\begin{eqnarray}
\mathcal{H} = \mathcal{H}_l  \otimes \mathcal{H}_m &=& W_2 \otimes W_1 \label{eq:HilbertDecomp} \\ 
&=& \left[ W_2^0 \oplus W_2^1 \right] \otimes \left[ W_1^0 \oplus W_1^1 \right] \nonumber \\ 
&=& \left[ \left[ \lbrace 1 \rbrace \oplus \lbrace k_1 \rbrace \right] \oplus W_2^1 \right] \otimes \left[ \lbrace 1 \rbrace \oplus W_1^1 \right] \nonumber \\ 
&=&\left[ \lbrace 1 \rbrace  \oplus \lbrace k_1 \rbrace \right] \oplus W_2^1 \oplus W_1^1 \oplus  \left[ \lbrace k_1 \rbrace  \otimes  W_1^1 \right]  \oplus  \left[W_2^1 \otimes  W_1^1   \right] \nonumber \\
&\equiv& \left[ \mathcal{H}_{\mu^*} \oplus \mathcal{H}_l^0 \right] \oplus \left[ \mathcal{H}_l^1 \oplus \mathcal{H}_m^1 \oplus \mathcal{H}_{lm}^{01} \oplus \mathcal{H}_{lm}^{11}\right]
\nonumber\\
&=& \mathcal{H}_0 \oplus \mathcal{H}_1
\nonumber
\end{eqnarray} 

\noindent
where the functional components corresponding to $\mathcal{H}_\mu$, $\mathcal{H}_l^0$, $\mathcal{H}_l^1$, $\mathcal{H}_m^1$, and $\left[ \mathcal{H}_{lm}^{01} \oplus \mathcal{H}_{lm}^{11}\right]$ are the overall mean, the nonparametric main effect of $l$, the parametric main effect of $l$, the parametric main effect of $m$, the nonparametric-parametric interaction, and the parametric-parametric interaction (between $l$ and $m$). Given this decomposition of the function space, any $\phi \in \mathcal{H}$ may be written as a sum of components from each of the 

\begin{equation}
\phi\left(l,m\right) = \mu^* + \phi_l^*\left(l\right) + \phi_m^*\left(m\right) + \phi_{lm}^*\left(l,m\right)  \label{eq:ANOVAdecomp}
\end{equation} 
\noindent
where $\int_{0}^1 \phi_{l}\left(l\right)dl = \int_{0}^1 \phi_{m}\left(m\right)dm = 0$, $\int_{0}^1 \phi_{lm}\left(l,m\right)dl = \int_{0}^1 \phi_{lm}\left(l,m\right)dm = 0$. The reproducing kernel (r.k.) for $\lbrace k_r \rbrace$ is $k_r\left(x \right)k_r\left(x^\prime \right)$. It can be verified that the r.k. for $W_m^1$ (Craven and Wahba 1979) is given by $R^1\left(x,x^\prime\right) = k_m\left(x \right)k_m\left(x^\prime \right) + \left( -1 \right)^{m-1}k_{2m}\left(\left[ x-x^\prime \right] \right)$
where $\left[ \alpha \right]$ is the fractional part of $\alpha$. The r.k. for $W_m$ is given by 
\begin{eqnarray*}
R\left(x,x^\prime\right) &=& R^0\left(x,x^\prime\right) + R^1\left(x,x^\prime\right) \\
&=&\left[ \sum_{\nu=1}^{m-1} k_\nu\left(x \right)k_\nu\left(x^\prime \right) \right]+ \left[ k_m\left(x \right)k_m\left(x^\prime \right) + \left( -1 \right)^{m-1}k_{2m}\left(\left[ x-x^\prime \right] \right)\right] \label{eq:RKforH1}
\end{eqnarray*}
\noindent
Using the fact that the r.k. for a tensor product space is the product of the corresponding reproducing kernels, the r.k. for $\mathcal{H}$ is given by 
\begin{eqnarray}
R\left( \left(l,m\right),\left(l^\prime,m^\prime\right)\right) &=&  R_l\left(l,l^\prime\right) \times R_m\left(m,m^\prime\right) \nonumber \\
 &=& \left[  R_l^0\left(l,l^\prime\right) + R_l^1\left(l,l^\prime\right) \right] \times \left[  R_m^0\left(l,l^\prime\right) + R_m^1\left(l,l^\prime\right) \right] \nonumber \\
 &=& R_l^0\left(l,l^\prime\right)R_m^0\left(m,m^\prime\right) + R_l^0\left(l,l^\prime\right) R_m^1\left(m,m^\prime\right) \nonumber \\
&\mbox{ }&\;\;\;\;\;\;\;\;\;\;\;\;\;\;\;\;\;\;\;\;\;\;\;\;\;\;\;\;\;\;\; +  R_l^1\left(l,l^\prime\right) R_m^0\left(m,m^\prime\right)  + R_l^1\left(l,l^\prime\right) R_m^1\left(m,m^\prime\right) \nonumber \\
&=& \left[ k_1\left(l \right)k_1\left(l^\prime \right)\right] + \left[ R_l^1\left(l,l^\prime\right)  + k_1\left(l,l^\prime\right) R_m^1\left(m,m^\prime\right) + R_l^1\left(l,l^\prime\right) R_m^1\left(m,m^\prime\right)\right] \nonumber \\
&=& R^0\left( \left(l,m\right) , \left(l^\prime,m^\prime \right) \right) + R^1\left( \left(l,m\right) , \left(l^\prime,m^\prime \right) \right)
\end{eqnarray}
%
%We must introduce some notation to simplify the following expression of the form of the elements in $\mathcal{H}$. Denote the set of unique pairs of observed within-subject time points and the corresponding set of unique transformed coordinates by $\mathcal{W}$ and $\mathcal{W}^*$, respectively:
%
%\begin{eqnarray*}
%\mathcal{W} &=& \bigcup_{i=1}^N \bigcup_{j>k}\left(t_{ij} ,t_{ik} \right)\\
%\mathcal{W}^* &=& \bigcup_{i=1}^N \bigcup_{j>k}\left(t_{ij}-t_{ik} ,\frac{1}{2}\left( t_{ij}+t_{ik} \right) \right) = \bigcup_{i=1}^N \bigcup_{j>k}\left(l^i_{jk},m^i_{jk} \right)\\
%\end{eqnarray*}
%\noindent
%with $\vert \mathcal{W}\vert = \vert \mathcal{W}^* \vert = N_{\phi}$. For simplicity of presentation, relabel the elements of $\mathcal{W}^*$ so that 
%\[
%\mathcal{W}^* = \lbrace \left( l_1,m_1 \right), \left( l_2,m_2 \right), \dots, \left( l_{N_{\phi}},m_{N_{\phi}} \right)  \rbrace
%\]
%\noindent
%One may verify that any $\phi \in \mathcal{H}$ can be written 
%\begin{equation} \label{eq:smoothing-spline-representer-expansion-1}
%\phi\left(l,m \right) = d_0 + d_1k_1\left(l\right) + \sum_{i=1}^n  c_i R_1\left( \left(l,m\right) , \left(l_i,m_i \right)\right) + \rho\left(l,m\right)
%\end{equation}
%\noindent
%where $\rho \perp \mathcal{H}_0 = \lbrace 1\rbrace \oplus \lbrace k_1\rbrace,\; \textup{span}\lbrace R_1\left(\left(l_i, m_i \right),\cdot \right)  \rbrace$.  It can be shown that the minimizer of \ref{eq:objectivefun} has $\rho = 0$, so that the $\phi \in \mathcal{H}$ minimizing \ref{eq:objectivefun} can be written as a (finite) linear combination of inner products:
%
%\begin{equation}\label{eq:ss-finite-dim-solution}
%\phi\left(l,m \right) = d_0 + d_1k_1\left(l\right) + \sum_{i=1}^n  c_i R_1\left( \left(l,m\right) , \left(l_i,m_i \right)\right) 
%\end{equation}
%\noindent
%
%The proof entails demonstrating that  $\rho$ does not improve the first term in \eqref{eq:objectivefun} (the data fit functional) and only adds to the penalty term, $J\left(\phi\right)$. Details are left to the appendix \ref{chapter-7-appendix}.
%Let $\Phi$ be the $N_{\phi} \times 1$ vector of regression coefficients given by \eqref{eq:MyTransformedModel} corresponding to $\phi$ evaluated at the elements of $\mathcal{W}^*$, $\phi = \left(\phi_1,\phi_2, \dots, \phi_{N_{\phi}}  \right)^T$. Let $d = \left(d_0, d_1\right)^T$, $c = \left(c_1, \dots, c_{N_{\phi}}  \right)^T$, and $b = \left(b_1, \dots, b_{N_m}  \right)^T$.   Define $K_{11}$, $K_{12}$, $K_{22}$, $B_{1}$, and $B_2$ as follows: 

\bigskip
TO DO: insert table of the marginal RK's and how they construct the tensor prodcut RK!

\end{example}

