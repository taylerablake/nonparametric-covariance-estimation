\documentclass[../chapter-2-spline-representation.tex]{subfiles}
\begin{document}

Imposing difference penalties on B-spline basis expansions generalizes and simplifies the approach outlined in the previous section in a way that permits application in any context where regression on B-splines is useful. Penalized B-splines, or \emph{P-splines}, are an alternative an approach to nonparametric smoothing which circumvent any complexity associated with constructing such penalty matrices by ommitting derivatives and integrals altogether. Instead, smoothness is imposed via a discrete penalty matrix based on finite difference formulas which is simple to compute. This approach achieves smoothness in fitted functions in two ways:

\begin{enumerate}
\item To avoid the difficulty of choosing the optimal set of knots, use a B-spline basis with a large number of equally spaced knots, purposefully overfitting the smooth coefficient vectors. 
\item Augment the goodness of fit measure with a difference penalty to prevent overfitting and accomodate a potentially ill-conditioned fitting procedure.
\end{enumerate}  
 
Using the properties of B-splines derived in {\needsparaphrased{B-spline section}}, it is relatively straightforward to show that the simplified penalty is nearly equivalent to the derivative-based penalty and that for second order differences, P-splines are very similar to O'Sullivan's approach. In some applications, it can be useful to use differences of a smaller or higher order in the penalty, and the P-spline framework makes the use of a penalty of any arbitrary order nearly seamless. 
 
Consider the varying intercept-only model defined in \ref{eq:varying_intercept_only_model} for the regression of $M$ data points $\left(t_i,y_i\right)$ on a set of $K$ B-splines, $\left\{B_j\right\}$.  By letting the number of knots, $K$, be relatively large, we allow more variation in fitted curve than the data reasonably justify. To make the result less flexible and avoid overfitting, O'Sullivan imposed a penalty on the second derivative of the fitted curve and appended this to the residual sum of squares, giving way to the objective function

\begin{equation} \label{eq:univariate_bspline_ridge_penalty}
\sum_{i=1}^m \left \{ y_i - \sum_{j=1}^K \beta_j B_j\left(t_i\right) \right \}^2 + \lambda \int_{t_{min}}^{t_{max}} \left\{  \sum_{j=1}^K \beta_j B^{\prime \prime}_j\left(t\right) \right\}^2 \; dt.
\end{equation}

The integral of the square of the second derivative of a fitted function has become common as a smoothness penalty since the seminal work on smoothing splines by Reinsch (1967), though it is useful to note that there is nothing particularly special about the second derivative. One could easily specify higher or lower order derivatives in smoothness penalties. In the context of smoothing splines, the first derivative leads to simple equations and a piecewise linear fit, while higher derivatives lead to systems of equations with a high bandwidth and a very smooth fit. 


Proposed for smoothing curves by cite{whittaker1922new}, difference penalties have been utilized for nearly a century, with more recent applications  outlined in  cite{eilers1991penalized}, cite{eilers1991nonparametric}, and cite{eilers1995indirect}. The finite difference penalty is easily introduced into regression equations, making it feasible to evaluate the impact of different orders of the differences on the fitted model. In some applications, it is useful to work with third and fourth order differences, since for high values of $\lambda$, the fitted curve approaches a parametric polynomial model. Detailed discussion on the effect of the smoothing parameter on fitted functions will follow. Let $D_d$ denote the matrix difference operator; that is, $D_d\beta = \Delta^d \beta$, where

 \begin{eqnarray*}
 \Delta \alpha_j &=& \alpha_j - \alpha_{j-1},\\
 \Delta^2 \alpha_j &=& \Delta\left(\Delta \alpha_j\right) = \alpha_j - 2\alpha_{j-1} + \alpha_{j-2},\\ 
 \end{eqnarray*}
\noindent 
and in general,
\begin{equation*}
\Delta^d \alpha_j = \Delta\left(\Delta^{d-1} \alpha_j \right)
\end{equation*}
The $\left(K - d\right) \times K$ differencing matrix $D_d$ is sparse for reasonably small values of $d$; for example, $D_1$ and $D_2$ for small dimensions are given by 
\[
D_1 = \begin{bmatrix} -1&1&0&0\\ 0&-1&1&0\\ 0&0&-1&1 \end{bmatrix}; \qquad D_2 = \begin{bmatrix} 1&-2&1&0 \\ 0&-&-2&1\end{bmatrix}
\]

cite{eilers1996flexible} propose to base the penalty on (higher-order) finite differences of the coefficients of adjacent B-splines:

 \[
 \lambda\vert D_d\alpha\vert^2 = \lambda \alpha^\prime D^\prime_d D_d \alpha = \lambda \alpha^\prime P\alpha,
 \] 



Replacing O'Sullivan's penalty with the difference penalty, we can control the smoothness of the fitted mean function $\mu = \beta_0\left( t \right) = B\alpha$ by minimizing
\begin{equation*} 
S_\lambda = \vert y- B \alpha \vert^2 + \lambda\vert D_d\alpha \vert^2
\end{equation*}

This approach reduces the dimensionality of the problem to the number of B-splines, $K$ instead of the number of observations, $M$ , as with smoothing splines. The tuning parameter $\lambda$ permits continuous control over smoothness of the fit. We will demonstrate that the difference penalty is a good discrete approximation to the integrated square of the $k^{th}$ derivative, and with this penalty, moments of the data are conserved and polynomial regression models occur as limits for large values of $\lambda$. We will explore the connection between a penalty on second-order differences of the B-spline coefficients and O'Sullivan's choice of a penalty on the second derivative of the fitted function. However, the difference penalty can be handled mechanically for any order of the differences.
cite{o1986statistical} used third-degree B-splines and the following penalty:

\begin{equation} \label{eq:osullivan_univariate_bspline_penalty}
h^2 P = \lambda \int_{t_{min}}^{t_{max}} \left\{ \sum_{j}  \alpha_j B_{j,3}^{\prime \prime} \left(t\right) \right\}^2\; dt
\end{equation}
\noindent
From the derivative properties of B-splines, it follows that
\begin{equation} \label{osullivan_univariate_bspline_penalty_via_deriv}
h^2 P = \lambda \int_{t_{min}}^{t_{max}}  \sum_{j} \sum_{k} \Delta^2 \alpha_j \Delta^2 \alpha_k B_{j,1}\left(t\right)B_{k,1}\left(t\right) dt 
\end{equation}
\noindent
Most of the cross products of $B_{j,1}(t)$ and $B_{k,1}(t)$ vanish since B-splines of degree 1 only overlap when $j$ is $k-1$, $k$, or $k+1$. Thus, we have that
\begin{align}
h^2 P = {} & \lambda \int_{t_{min}}^{t_{max}} \bigg[ \left\{ \sum_{j}  \Delta^2 \alpha_j  B_j\left(t,1\right)  \right\}^2  + 2 \sum_{j}\Delta^2 \alpha_j\Delta^2 \alpha_{j-1}B_j\left(t,1\right)B_{j-1}\left(t,1\right) \bigg] dt \nonumber \\ 
= {} & \lambda \bigg[ \sum_j \left( \Delta^2\alpha_j \right)^2 \int_{t_{min}}^{t_{max}} B_j^2\left(t,1\right)\;dt + 2 \sum_j \Delta^2 \alpha_j\Delta^2 \alpha_{j-1} \bigg]
\end{align}
\noindent
or
\begin{align}
h^2 P = \lambda \sum_j \left( \Delta^2\alpha_j \right)^2 \int_{t_{min}}^{t_{max}} B_{j,1}^2\left(t\right) dt {} & +  2\lambda \sum_j \Delta^2 \alpha_j \Delta^2 \alpha_{j-1}  \nonumber \\ 
{} &+\int_{t_{min}}^{t_{max}} B_{j,1}\left(t\right)B_{j-1,1}\left(t\right) dt
\end{align}
\noindent
which can be written as
\begin{equation} \label{eq:osullivan_penalty_decomp}
h^2 P = \lambda\left\{c_1 \sum_j\left( \Delta^2 \alpha_j\right)^2 + c_2 \sum_j\Delta^2 \alpha_j\Delta^2 \alpha_{j-1} \right\}
\end{equation}
\noindent
where, for given equidistant knots, $c_1$ and $c_2$ are constants given by
\begin{equation}
\begin{split}
c_1 & =   \int_{t_{min}}^{t_{max}} B_{j,1}^2\left(t\right) dt\\
c_2 & = \int_{t_{min}}^{t_{max}} B_{j,1}\left(t\right)B_{j-1,1}\left(t\right) dt
\end{split}
\end{equation}

O'Sullivan's ridge-like B-spline penalty in Equation~\ref{eq:osullivan_univariate_bspline_penalty} can be written as a linear combination of a difference penalty (\ref{eq:univariate_pspline_diff_penalty}) and the sum of the cross products of neighboring second differences. The second term in Equation~\ref{eq:osullivan_penalty_decomp} leads to a complex objective function when minimizing the penalized likelihood, where seven adjacent spline coefficients occur, as opposed to five if only the first term in Equation~\ref{eq:osullivan_penalty_decomp} is used in the penalty. The additional complexity is due to overlapping B-splines, which quickly increases when using higher order differences and higher order B-splines. The use of a difference penalty allows us to sidestep the difficulty of constructing a procedure for incorporating the penalty in the likelihood equations. 

Define $\hat{\alpha} = \left(\hat{\alpha}_1, \hat{\alpha}_2, \dots, \hat{\alpha}_K \right)$ to be the minimizer of $S_\lambda$:

\begin{equation*}  
S_\lambda = \sum_{i=1}^m \left\{ y_i - \sum_{j=1}^K \alpha_j B_j\left(t_i\right) \right\}^2 + \lambda \sum_{j=d+1}^K \left( \Delta^d\alpha_j \right)^2
\end{equation*}
\noindent
In vector notation, this may be written

\begin{align}
\begin{split}
S_\lambda &= \vert y- B\alpha  \vert^2  + \lambda \vert D_d \alpha\vert^2 \\
&=  \left( y- B\alpha  \right)^T \left( y-B \alpha\right) + \lambda \alpha^T P \alpha
\end{split} \label{eq:S_pen_varying_intercept_model}
\end{align}
\noindent
where 
\[
P = D_d^T D_d
\]
\noindent
and the elements of $B$ are given by $b_{ij} = B_j\left(t_i\right)$, as defined in \ref{eq:S_varying_intercept_model}. Taking derivatives on both sides of \ref{eq:S_pen_varying_intercept_model} with respect to $\alpha$ gives

\begin{align}
\frac{\partial}{\partial \alpha}S_\lambda ={} & \frac{\partial}{\partial \alpha}\left(\alpha^TB^TB \alpha -2y^T B^T\alpha+\lambda \alpha^T D_k^T D_k \alpha  \right) \nonumber \\
= {} & 2B^TB \alpha - 2B^T y + 2\lambda D_d^TD_d\alpha \nonumber\\
= {} & \left(B^T B +  \lambda D_d^TD_d\right)\alpha - B^T y \label{eq:dSlambda_dAlpha}
\end{align} 
\noindent

and setting equal to zero yields normal equations:
\begin{equation}\label{eq:S_lambda_normal_eq}
B^T y = \left(B^T B +  \lambda D_d^TD_k\right)\alpha,
\end{equation}
which has explicit solution
\[
\hat{\alpha} = \left(B^T B +  \lambda D_d^TD_d\right)^{-1}B^T y
\]
\noindent
The effective hat matrix is now
\[
H_\lambda - B\left(B^T B +  \lambda D_k^TD_k\right)^{-1}B^T 
\]

When $\lambda = 0$, we have the standard normal equations of linear regression with a B-spline basis, and with $k = 0$ \ref{eq:S_lambda_normal_eq} corresponds to the normal equations under the ridge regression penalty. When $\lambda > 0$, the penalty only influences the main diagonal and $k$ sub-diagonals of the system of equations. The compact support and limited overlap of the B-spline basis functions gives this system a banded structure, though exploiting this structure is of little utility since the number of equations is equal to the number of splines, which is generally moderate by design. 
\end{document}