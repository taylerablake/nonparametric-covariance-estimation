
The mechanics in the previous section rely on apriori knowledge of the number and locations of the knots $\left\{t_j\right\}$, $j=1,\dots,K$. In practice this information is readily available, but has a considerable impact on the behaviour of the estimated coefficient functions, as the smoothness of a fitted curve can be controlled by the number of B-splines used in the basis expansion used to approximate the curve. Fewer knots (thus, fewer basis functions) lead to smoother fits. This choice presents a model selection problem, as too many knots lead to overfitting while too few knots lead to underfitting. Optimal knot placement has been closely examined, with some authors proposing automatic methods for optimizing the number and the positions of the knots (Friedman and Silverman, 1989; Kooperberg and Stone, 1991,1992). This is a difficult numerical problem requiring nonlinear optimization, and is still an open problem today. However, limiting the number of B-splines is not the only approach to controlling the complexity of the fitted function. 

As in chapter {\needsparaphrased{smoothing spline chapter}}, we can append a penalty on the coefficients of the basis functions to the goodness of fit measure, and by optimizing this augmented objective function, we can achieve as much smoothness in the fitted function as desired. cite{o1986statistical} was the first to propose using a rich B-spline basis and applying a discrete penalty to the spline coefficients. 

 He proposed a penalty on the second derivative to restrict the flexibility of the fitted curve, similar to the penalty pioneered for smoothing splines by Reinsch (1967). This penalty has become the standard in much of the spline literature; see Eubank (1988), Wahba (1990) and Green and Silverman (1994). This measure of roughness of a curve is given by 
 
 \[
 J = \int_l^u \left[ f^{\prime \prime}\left(x\right)\right]^2\;dx
 \]
 \noindent
 where $l$ and $u$ are the bounds on the domain of $x$. Using the properties of B-splines, if $f\left(x\right) = \sum_{j} \beta_j B_j\left(x\right)$, one can derive a banded matrix $P$ such that 
 \[
 J = \beta^\prime P \beta
 \] 
 \noindent
 where $\beta = \left(\beta_1,\dots, \beta_n\right)$, and the $i$-$j^{th}$ element of $P$ is given by
 \[
 p_{ij} = \int_l^u B_i^{\prime \prime} \left( x \right)B_j^{\prime \prime} \left( x \right)\;dx.
 \]
 \noindent
 He then proposed minimizing
 \begin{eqnarray*}
 Q\left(\beta, \lambda \right) &=& \sum_{i=1}^m \left(y_i - \sum_{j} \beta_j B_j\left(x_i \right)\right)^2 + \lambda \int_l^u \left[ f^{\prime \prime}\left(x\right)\right]^2\;dx\\
 &=& \vert y - B\beta \vert  \vert^2 + \lambda\beta^\prime P \beta
 \end{eqnarray*}

The computation of $P$ is nontrivial and becomes very tedious when the third and fourth derivative are used as the roughness measure. cite{wand2008semiparametric} extend O'Sullivan's work to higher order derivatives for general degree B-splines and derive an exact matrix algebraic expression for the penalty matrices. In the cubic case, the expression is a result of the application of Simpson's Rule applied to the inter-knot differences since each $B_i^{\prime \prime} B_j^{\prime \prime}$ is a piecewise quadratic function. The penalty may be written
 \[
 P = \left(B^{\prime \prime}\right)^\prime \textup{diag}\left(\omega \right) B^{\prime \prime}, 
 \]
 \noindent
 where $B^{\prime \prime}$ is the $3\left( n + 7 \right) \times \left( n + 4 \right)$ matrix with $i$-$j^{th}$ entry given by $B_j^{\prime \prime} \left(x_i^*\right)$, $x^*_i$ is the $i^{th}$ element of 
 
\[
\left( \phi_1,\frac{\phi_1+\phi_2}{2},\phi_2,\phi_2,\frac{\phi_2+\phi_3}{2},\phi_3,\dots,\phi_{n+7},\frac{\phi_{n+7}+\phi_{n+8}}{2},\phi_{n+8} \right),
\]
 \noindent
 and $\omega$ is the $3\left(n+7\right) \times 1$ vector given by
 
\begin{align*}
\omega &= \left( \frac{1}{6}\left(\Delta \phi \right)_1,\frac{4}{6}\left(\Delta \phi \right)_1, \frac{1}{6}\left(\Delta \phi \right)_1,\frac{1}{6}\left(\Delta \phi \right)_2, \frac{4}{6}\left(\Delta \phi \right)_2,  \right. \\
&\qquad   \left. {} \frac{1}{6}\left(\Delta \phi \right)_2 , \dots , \frac{1}{6}\left(\Delta \phi \right)_{n+7}, \frac{4}{6}\left(\Delta \phi \right)_{n+7}, \frac{1}{6}\left(\Delta \phi \right)_{n+7}  \right) \\
\end{align*}
\noindent
 where $\left(\Delta \phi \right)_j = \phi_{j+1}-\phi_j$. They generalize this to the case of any order penalty and present a table of formulas for constructing any arbitrary penalty matrix, $P$.
 