\documentclass[../chapter-2-spline-representation.tex]{subfiles}
\begin{document}

Convenient construction of a reproducing kernel Hilbert space on a domain
\[
\mathcal{V} = \mathcal{V}_1 \otimes \mathcal{V}_2
\]
\noindent
which can be written as a product domain, is available through the tensor product of the RKHS for each of the marginal domains $\mathcal{V}_1$ and $\mathcal{V}_2$. Without loss of generality, we can let $l,\;m \in \left[0,1\right] = \mathcal{V}_1 = \mathcal{V}_2$. Given Hilbert space for the domain of $l$, $\hilbert_{\left[1\right]}$ with reproducing kernel $Q_1$ and Hilbert space on the domain of $m$, $\hilbert_{\left[2\right]}$ with reproducing kernel $Q_2$, the reproducing kernel $Q = Q_{\left[1\right]}Q_{\left[2\right]}$ corresponds to that of the tensor product space of $\hilbert_{\left[1\right]}$ and $\hilbert_{\left[2\right]}$, denoted
\[
\hilbert = \hilbert_{\left[1\right]} \otimes \hilbert_{\left[2\right]}.
\]
\noindent
See \citet{gu2002smoothing}, Theorem 2.6. Let $\mathcal{A}_1$, $\mathcal{A}_2$ denote the averaging operators defining ANOVA decompositions on $\hilbert_{\left[1\right]}$, $\hilbert_{\left[2\right]}$, respectively, where $\hilbert_{0\left[i\right]}$ has RK $Q_{0\left[i\right]}$, $i = 1, 2$ and $\hilbert_{1\left[i\right]}$ has RK $Q_{1\left[i\right]}$ satisfying $\mathcal{A}_1Q_{\left[1\right]}\left(l,\cdot\right) = \mathcal{A}_2Q_{\left[2\right]}\left(m,\cdot\right) = 0$. Then the tensor product space $\hilbert$ has tensor sum decomposition

\begin{align} 
\begin{split} \label{eq:tensor-sum-decomposition}
\hilbert &= \left[\hilbert_{0\left[1\right]} \oplus \hilbert_{1\left[1\right]} \right] \otimes \left[\hilbert_{0\left[2\right]} \oplus \hilbert_{1\left[2\right]} \right] \\
&= \left[\hilbert_{0\left[1\right]} \otimes  \hilbert_{0\left[2\right]}\right] \oplus \left[\hilbert_{0\left[1\right]} \otimes \hilbert_{1\left[2\right]}\right] \oplus \left[\hilbert_{1\left[1\right]} \otimes  \hilbert_{0\left[2\right]}\right] \oplus \left[\hilbert_{1\left[1\right]} \otimes  \hilbert_{1\left[2\right]}\right] 
\end{split}
\end{align}
\noindent
If $Q_{0\left[i\right]} \propto 1$ for $i = 1,2$, then $\hilbert$ can be further simplified:
\begin{equation}
\hilbert = \hilbert_1 \oplus \hilbert_2,
\end{equation}
\noindent
which has reproducing kernel $Q = Q_{\left[1\right]}Q_{\left[2\right]}$.

\begin{example}{\textbf {Tensor product cubic spline}}\\
\vspace{0.5cm}
Let the marginal domains of $l$ and $m$ correspond to $\hilbert_1$ and $\hilbert_2$ respectively, where
\[
\hilbert_i = \mathcal{C}^{\left(m_i\right)} = \left\{ \phi: \int \limits_{0}^1 \phi^{\left(m_i\right)}\;dv < \infty  \right\},
\]
\noindent
which are equipped with inner product
\begin{align}
\begin{split}
\langle f,g\rangle &= \langle f,g\rangle_0 + \langle f,g\rangle_1\\
 &= \sum_{\nu=0}^{m_i-1}M_{\nu} f M_{\nu} g + \int_0^1 f^{\left( m_i \right)}\left(v\right)g^{\left( m_i \right)}\left(v\right)dv, \quad i = 1,2
\end{split}
\end{align}
\noindent
where the order $i$ differential operator $M_\nu$ is defined $M_\nu \phi = \int_0^1 \phi^{\left( m \right)}\left(v\right) dv\;,\;\; \nu = 1, \dots, m_i$, $i = 1,2$. Denote the norm corresponding to this inner product by

\[
\vert \vert f \vert \vert^2 = \left< f,f\right> = \left< f,f\right>_0 + \left< f,f\right>_1 = \vert \vert P_0 f \vert \vert^2 + \vert \vert P_1 f \vert \vert^2
\]
\noindent
The reproducing kernel $Q$ can be expressed in terms of the scaled Bernoulli polynomials $\left\{ k_j\left(v\right) = \frac{1}{j!}B_j\left(v\right) \right\}$, $v \in \left[0,1\right]$, where $B_j$ is defined according to:

\begin{align*}
B_0\left(x\right) &= 1\\
\frac{d}{dx} B_j\left(x\right) &= jB_{j-1}\left(x\right), \;j = 1, 2, \dots
\end{align*}
\noindent
One can verify that $\int \limits_0^1 k_\mu^\nu dv = \delta_{\mu,\nu}$ for $\nu, \mu= 0,\dots, m_i -1$, where $\delta_{\mu,\nu}$ is the Kronecker delta. This implies that the $k_\nu$, $\nu = 0,\dots, m_i-1$ for an orthonormal basis for $\hilbert_{0\left[i\right]} = \left\{ \phi: \phi^{\left( m_i \right)} = 0 \right\}$ under the inner product
\[
\langle f,g\rangle_0 =  \sum_{\nu=0}^{m_i-1}M_{\nu} f M_{\nu} g, \quad i = 1, 2, 
\]
\noindent
and that 
\[
Q_{0\left[i\right]}\left(v,v'\right) = \sum_{\nu=0}^{m_i-1}  k_\nu\left(v\right)  k_\nu\left(v'\right) 
\]
\noindent
is the reproducing kernel for $\hilbert_{0\left[i\right]}$. The subspaces of $\hilbert_{\left[i\right]}$ which are orthogonal to $\hilbert_{0\left[i\right]}$ are comprised of functions $\phi$ satisfying 
\[
\hilbert_{1\left[i\right]} = \lbrace \phi: M_\nu f = 0,\;\; \nu = 0,1,\dots, m_i-1,\int\limits_{0}^1 \phi^{\left(m_i\right)}\;dv < \infty \rbrace, \quad i = 1,2.
\]
One can show that the representer for the evaluation functional $\left[v\right] \phi$ in $\hilbert_{1\left[i\right]}$ with squared norm $\langle f,g\rangle_1= \int_0^1 f^{\left(m_i\right)}g^{\left(m_i\right)}\;dv$ is given by the function

\begin{equation}
{Q_{\left[i\right]} }_v'\left(v\right) = k_{m_i}\left(v\right)k_{m_i}\left(v'\right) + \left(-1\right)^{m_i-1}k_{2m_i}\left(v' - v\right)
\end{equation}
\noindent
See \citet{gu2002smoothing} Example 2.3.3 for proof. The tensor product smoothing spline results from letting $m_1 = m_2 = 2$, so that the marginal subspaces can be written

\begin{align} \label{eq:cubic-spline-hilbert-space}
\left\{ \phi: \phi'' \in \mathcal{L}_2\left[0,1\right] \right\} = &\left\{ \phi: \phi \propto 1 \right\} \oplus  \left\{ \phi: \phi \propto k_1 \right\} \oplus \left\{ \phi: \int_0^1 \phi dv = \int_0^1 \phi' dv = 0,\; \phi'' \in \mathcal{L}_2\left[0,1\right]  \right\} \\
&= \hilbert_{00} \oplus \hilbert_{01} \oplus \hilbert_1,
\end{align}
\noindent
where $ \hilbert_{01} \oplus \hilbert_1$ forms the contrast in a one-way ANOVA decomposition with averaging operator $\mathcal{A}\phi = \int_0^1 \phi\;dv$. The corresponding reproducing kernels are
\begin{align} \label{eq:cubic-spline-hilbert-space-rks}
Q_{00}\left(v,v'\right) &= 1\\
Q_{01}\left(v,v'\right) &= k_1\left(v\right)k_1\left(v'\right)\\
Q_{1}\left(v,v'\right) &= k_2\left(v\right)k_2\left(v'\right) - k_4\left(v-v'\right).
\end{align}
\noindent
The tensor product space can be constructed with nine tensor sum terms; the construction of the tensor product space from the terms of the tensor sum. The corresponding reproducing kernels and inner products are given in Table~\ref{table:tensor-product-cubic-spline-RKHS-table} and Table~\ref{table:tensor-product-cubic-spline-RK-table}, respectively.

\begin{table}[H]
\centering % used for centering table
\begin{tabular}{r|c|c|c|} % centered columns (4 columns)
\multicolumn{1}{c}{} & \multicolumn{1}{c}{	$\hilbert_{00\left[2\right]}$}	&	\multicolumn{1}{c}{$\hilbert_{01\left[2\right]}$}	&\multicolumn{1}{c}{ $\hilbert_{1\left[2\right]}$}\\ [1.5ex] 
\cline{2-4}  % inserts single horizontal line\\
$\hilbert_{00\left[1\right]}$		& $\hilbert_{00\left[1\right]}\otimes \hilbert_{00\left[2\right]}$ 	&	$\hilbert_{00\left[1\right]}	\otimes \hilbert_{01\left[2\right]} $	&	$\hilbert_{00\left[1\right]}	\otimes \hilbert_{1\left[2\right]}$   \\ [1.5ex] 
$\hilbert_{01\left[1\right]}$		& $\hilbert_{01\left[1\right]} \otimes \hilbert_{00\left[2\right]}$			& 	$\hilbert_{01\left[1\right]} \otimes \hilbert_{01\left[2\right]}$   &   $\hilbert_{01\left[1\right]} \otimes \hilbert_{1\left[2\right]}$\\ [1.5ex] 
 $\hilbert_{1\left[1\right]}$	& 	 $\hilbert_{1\left[1\right]} \otimes \hilbert_{00\left[2\right]}$	&	$\hilbert_{1\left[1\right]} \otimes \hilbert_{01\left[2\right]}$ 	&	$\hilbert_{1\left[1\right]} \otimes \hilbert_{1\left[2\right]}$ \\ [1.5ex] 
\cline{2-4}
\end{tabular}
\caption{Construction of the tensor product cubic spline subspace from marginal subspaces $\hilbert_{\left[1\right]}$, $\hilbert_{\left[2\right]}$} % title of Table
\label{table:tensor-product-cubic-spline-RKHS-table}
\end{table}

\begin{landscape}
\begin{table}[H]
\caption{Tensor product cubic spline subspace reproducing kernels and inner products} % title of Table
\centering % used for centering table
\begin{tabular}{lll} % centered columns (4 columns)
\hline\hline %inserts double horizontal lines
Subspace 	& 		Reproducing kernel 		& 	Inner product \\
\hline % inserts single horizontal line
$\hilbert_{00\left[1\right]} \otimes \hilbert_{00\left[2\right]}$ & 	$1$								     & 	$\left( \int_0^1 \int_0^1 f \right) \left( \int_0^1 \int_0^1 g \right)$ \\ [1ex] 
$\hilbert_{01\left[1\right]} \otimes \hilbert_{00\left[2\right]} $& 	$k_1\left(l\right)k_1\left(l'\right)$						     & 	$\left( \int_0^1 \int_0^1 f'_{\left[1\right]} \right) \left( \int_0^1 \int_0^1 g'_{\left[1\right]} \right)$ \\ [1ex] 
$\hilbert_{01\left[1\right]} \otimes \hilbert_{01\left[2\right]}$ & 	$k_1\left(l\right)k_1\left(l'\right)k_1\left(m\right)k_1\left(m'\right)$ & $\left( \int_0^1 \int_0^1 f''_{\left[12\right]} \right) \left( \int_0^1 \int_0^1 g''_{\left[12\right]} \right)$ \\ [1ex] 
$\hilbert_{1\left[1\right]} \otimes \hilbert_{00\left[2\right]}$  	& 	$k_2\left(l\right)k_2\left(l'\right) - k_4\left(l - l'\right)$	      & $\int_0^1 \left( \int_0^1 f''_{\left[12\right]}\;dl' \right) \left(  \int_0^1 g''_{\left[12\right]} \;dl'\right)\;dl $\\ [1ex] 
$\hilbert_{1\left[1\right]} \otimes \hilbert_{01\left[2\right]}$ 	& 	$\left[k_2\left(l\right)k_2\left(l'\right) - k_4\left(l - l'\right)\right]k_1\left(m\right)k_1\left(m'\right)$ & $\int_0^1 \left( \int_0^1 f^{\left(3\right)}_{\left[112\right]}\;dl' \right) \left(  \int_0^1 g^{\left(3\right)}_{\left[112\right]} \;dl'\right)\;dl$ \\ [1ex]  
$\hilbert_{1\left[1\right]} \otimes \hilbert_{1\left[2\right]}$  		& $\left[k_2\left(l\right)k_2\left(l'\right) - k_4\left(l - l'\right)\right]\left[k_2\left(m\right)k_2\left(m'\right) - k_4\left(m - m'\right)\right]$ & $\int_0^1  \int_0^1 f^{\left(4\right)}_{\left[1122\right]}g^{\left(4\right)}_{\left[1122\right]}$ \\ [1ex]  
\hline %inserts single line
\end{tabular}
\label{table:tensor-product-cubic-spline-RK-table}
\end{table}
\end{landscape}
\end{example}

\end{document}



